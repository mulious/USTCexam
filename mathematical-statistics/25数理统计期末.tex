\documentclass[UTF8]{ctexart}
\usepackage{graphicx,amsfonts,amsmath,mathrsfs,amssymb,amsthm,url,color}
\usepackage{fancyhdr,indentfirst,bm,enumerate,natbib,float,tikz}
\usepackage{caption,subcaption,calligra}
\usepackage{graphicx}
\usepackage{enumitem}
\usepackage{bm}
\usepackage{bbding} 

\title{数理统计25期末}
\author{}
\date{}

\textheight 23cm
\textwidth 16.5cm
\topmargin -1.2cm
\oddsidemargin 0cm
\evensidemargin 0cm

\begin{document}
\maketitle
\noindent 一、填空及判断题(15分)\\
1.设$X_1, \cdots, X_n, X_{n+1} \stackrel{\text{i.i.d.}}{\sim} N(\mu, \sigma^2)$,若
    \[
    \frac{\overline{X_n}-X_{n+1}}{\sqrt{b_n \sum_{i=1}^n X_i^2 - c_n (\overline{X_n})^2}} \sim t_{n-1}
    \]
    其中$\overline{X_n}=\frac{1}{n}\sum_{i=1}^n X_i$,则$b_n=$\underline{\hspace{2cm}},$c_n=$\underline{\hspace{2cm}}。\\
答案:$b_n=\frac{n+1}{n(n-1)} \quad c_n=\frac{n+1}{n-1}$\\
先将分子归一化为$N(0,1)$
\[
\overline{X_n}\sim N\left( \mu,\frac{\sigma^2}{n}\right)  \quad \overline{X_n}-X_{n+1}\sim N\left( 0,\frac{(n+1)\sigma^2}{n}\right) 
\]
则
\[
\frac{\overline{X_n}-X_{n+1}}{\sqrt{\frac{(n+1)\sigma^2}{n}}} \sim N(0,1)
\]
再处理分母
\[
\frac{\sum\limits_{i=1}^{n} \left( X_i-\overline{X_n}\right)^2}{\sigma^2}\sim \chi^2(n-1)
\]
其中
\begin{align*}
	\sum\limits_{i=1}^{n} \left( X_i-\overline{X_n}\right)^2 &=\sum\limits_{i=1}^{n}X_i^2-2\overline{X_n}\sum\limits_{i=1}^{n}X_i+n\overline{X_n}^2  \\
	 & =\sum\limits_{i=1}^{n}X_i^2-n\overline{X_n}^2
\end{align*}
则
\[
\frac{\sum\limits_{i=1}^{n}X_i^2-n\overline{X_n}^2}{(n-1)\sigma^2}\sim \frac{\chi^2(n-1)}{n-1}
\]
也就是
\[
\frac{\frac{\overline{X_n}-X_{n+1}}{\sqrt{\frac{(n+1)\sigma^2}{n}}}}{\sqrt{\frac{\sum\limits_{i=1}^{n}X_i^2-n\overline{X_n}^2}{(n-1)\sigma^2}}}\sim t_{n-1}
\]
其中分子分母的独立性来自正态总体的$\overline{X_n}$与$S^2$独立\\
化简得到
\[
\frac{\overline{X_n}-X_{n+1}}{\sqrt{\frac{n+1}{n(n-1)} \sum_{i=1}^n X_i^2 - \frac{n+1}{n-1} (\overline{X_n})^2}}\sim t_{n-1}
\]
就有
\[
b_n=\frac{n+1}{n(n-1)} \quad c_n=\frac{n+1}{n-1}
\]
2.设$X \sim N(\mu, 2)$,求$\mu$的95\%置信区间长度小于1所需的最小样本量\underline{\hspace{2cm}}。\\
答案:$n \geq 31$\\
已知方差,则置信区间为
\[
\left[\overline{X}-\sqrt{\frac{2}{n}}u_{\alpha/2},\overline{X}+\sqrt{\frac{2}{n}}u_{\alpha/2} \right] 
\]
代入数值即得\\
3.$\times$\\
4.$\times$\\
5.$\times$\\
6.$\checkmark$;$\checkmark$;$\times$\\
7.样本量小导致有些格子计数小于5\\



\noindent 二、(15 分)证明来自总体分布$U(\theta, 2\theta)$的$n$个样本$(X_1,...,X_n)$的统计量$(X_{(1)}, X_{(n)})$是极小充分统计量但不是完全统计量。\\
解:均匀总体下的充分完全统计量.\\
通过因子分解定理说明充分统计量:根据题意有样本联合密度
$$
f(\boldsymbol{x} ; \theta)=\frac{\mathbb{I}_{\left\{\theta<x_i<2 \theta, i=1,2, \ldots, n\right\}}}{\theta^n}=\theta^{-n} \mathbb{I}_{\left\{x_{(n)} / 2<\theta<x_{(1)}\right\}} .
$$
因此由因子分解定理知 $T=\left( X_{(1)}, X_{(n)}\right) $ 是 $\theta$ 的充分统计量.\\
通过定理 2.6.2 说明极小充分统计量:对任意 $\boldsymbol{x}$ 和 $\boldsymbol{y}$ ,
$$
\frac{f(\boldsymbol{x} ; \theta)}{f(\boldsymbol{y} ; \theta)}=\frac{\mathbb{I}_{\left\{x_{(n)} / 2<\theta<x_{(1)}\right\}}}{\mathbb{I}_{\left\{y_{(n)} / 2<\theta<y_{(1)}\right\}}}
$$
要使得上式与 $\theta$ 无关,当且仅当 $$\left(x_{(1)}, x_{(n)}\right)=\left(y_{(1)}, y_{(n)}\right)$$
因此由定理2.6.2知 $T=\left(X_{(1)}, X_{(n)}\right)$ 是 $\theta$的极小充分统计量.\\
构造函数或利用辅助统计量说明不完全性:\\
法一:注意到
$$
\mathbb{E}_\theta\left[X_{(1)}\right]=\theta+\frac{1}{n+1} \theta=\frac{n+2}{n+1} \theta, \quad \mathbb{E}_\theta\left[X_{(n)}\right]=\theta+\frac{n}{n+1} \theta=\frac{2 n+1}{n+1} \theta,
$$
上式来自$n$个独立同分布的$\mathrm{U}(0,1)$的次序统计量满足
\[
U_{(1)}\sim \beta(1,n) \quad U_{(n)}\sim \beta(n,1)
\]
取 $g\left(x_{(1)}, x_{(n)}\right)=(2 n+1) x_{(1)}-(n+2) x_{(n)}$ ,则
$$
\mathbb{E}_\theta\left[g\left(X_{(1)}, X_{(n)}\right)\right]=0, \quad \forall \theta>0,
$$
但函数 $g$ 并不几乎处处为零。因此 $\left(X_{(1)}, X_{(n)}\right)$ 不是 $\theta$ 的完全统计量.\\
法二:注意到 $X_i / \theta \sim \mathrm{U}(1,2)$ ,构造辅助统计量
$$
T=\frac{X_{(n)}}{X_{(1)}}=\frac{X_{(n)} / \theta}{X_{(1)} / \theta},
$$
于是 $T$ 的分布与 $\theta$ 无关,是一个辅助统计量.因此 $\left(X_{(1)}, X_{(n)}\right)$ 不是 $\theta$ 的完全统计量.\\





\noindent 三、(15分)设$X$的概率分布为:
\[
\begin{array}{c|ccc}
X & 1 & 2 & 3 \\ \hline
P(X) & 2\theta & 3\theta & 1-5\theta \\
\end{array}
\]
取出$20$个样本,其中有$6$个取值为$1$,有$8$个取值为$2$,有$6$个取值为$3$\\
(1) 求$\theta$的最大似然估计和矩估计,并判断该估计是否无偏;\\
(2)求$\theta$的一致最小方差无偏估计(UMVUE),并比较其方差与C-R下界。\\
解:(1)矩估计:根据题意有
$$
\mathbb{E}[X]=1 \times 2 \theta+2 \times 3 \theta+3 \times(1-5 \theta)=3-7 \theta .
$$
因此反解得 $\theta$ 的矩估计量为
$$\hat{\theta}_M(\boldsymbol{X})=\frac{3-\overline{X}}{7}$$
由 $\overline{x}=\frac{1 \times 6+2 \times 8+3 \times 6}{20}=2$ 知 $\theta$ 的矩估计值为 
$$\hat{\theta}_M(\boldsymbol{x})=1 / 7$$ 而
$$
\mathbb{E}_\theta\left[\hat{\theta}_M(\boldsymbol{X})\right]=\mathbb{E}_\theta\left(\frac{3-\overline{X}}{7}\right)=\frac{3-\mathbb{E}(\overline{X})}{7}=\theta .
$$
所以矩估计 $\hat{\theta}_M(\boldsymbol{X})$ 是 $\theta$ 的无偏估计.\\
最大似然估计:记 $n_j=\sum_{i=1}^n \mathbb{I}_{\left(X_i=j\right)}, j=1,2,3$ 。由分布列知 $\theta$ 的似然函数为
$$
L(\theta)=(2 \theta)^{n_1}(3 \theta)^{n_2}(1-5 \theta)^{n_3}=2^{n_1} 3^{n_2} \theta^{n-n_3}(1-5 \theta)^{n_3} .
$$
由对数似然方程
$$
\frac{\partial \log L(\theta)}{\partial \theta}=\frac{n-n_3}{\theta}-\frac{5 n_3}{1-5 \theta}=0 \Rightarrow \theta=\frac{n-n_3}{5 n} .
$$
经微分检验其确为似然函数的最大值点,因此 $\theta$ 的最大似然估计为 $$\hat{\theta}_L(\boldsymbol{X})=\frac{1}{5 n} \sum_{i=1}^n \mathbb{I}_{\left(X_i \neq 3\right)}$$
由样本观测值知其估计值为 $$\hat{\theta}_L(\boldsymbol{x})=0.14$$
而
$$
\mathbb{E}_\theta\left[\hat{\theta}_L(\boldsymbol{X})\right]=\mathbb{E}_\theta\left(\frac{1}{5 n} \sum_{i=1}^n \mathbb{I}_{\left(X_i \neq 3\right)}\right) =\frac{1}{5 n} \sum_{i=1}^n(5 \theta)=\theta .
$$
所以最大似然估计 $\hat{\theta}_L(\boldsymbol{X})$ 也是 $\theta$ 的无偏估计.\\
(2)一致最小方差无偏估计:样本联合密度函数为
$$
f(\boldsymbol{x} ; \theta)=(2 \theta)^{n_1}(3 \theta)^{n_2}(1-5 \theta)^{n_3}=2^{n_1} 3^{n_2} \theta^n \exp \left\{n_3 \log \frac{1-5 \theta}{\theta}\right\}
$$
自然参数空间 $\Theta^*=\{\eta:-\infty<\eta<\infty\}$ 有内点,于是 $T(\boldsymbol{X})=n_3$ 是 $\theta$ 的充分完全统计量.从而由 Lehmann-Scheffé 定理知,$\hat{\theta}_L(\boldsymbol{X})$ 作为基于 $n_3$ 的无偏估计是 $\theta$ 的 UMVUE,其方差
$$
\operatorname{Var}_\theta\left[\hat{\theta}_L(\boldsymbol{X})\right]=\frac{\operatorname{Var}_\theta\left(I\left(X_1 \neq 3\right)\right)}{25 n}=\frac{5 \theta(1-5 \theta)}{25 n}=\frac{\theta(1-5 \theta)}{5 n} .
$$
而 $n$个样本的Fisher 信息量
$$
I(\theta)=-\mathbb{E}_\theta\left[\frac{\partial^2 \log L(\theta)}{\partial \theta^2}\right]=\mathbb{E}_\theta\left[\frac{n-n_3}{\theta^2}-\frac{25 n_3}{(1-5 \theta)^2}\right]=\frac{5 n \theta}{\theta^2}-\frac{25 n(1-5 \theta)}{(1-5 \theta)^2}=\frac{5 n}{\theta(1-5 \theta)} .
$$
因而 Cramér-Rao 下界为 
$$1 / I(\theta)=\operatorname{Var}_\theta\left[\hat{\theta}_L(\boldsymbol{X})\right]$$
即一致最小方差无偏估计 $\hat{\theta}_L(\boldsymbol{X})$ 的方差达到 $\theta$ 的无偏估计方差的下界.\\



\noindent 四、(12分)设$\left( X_1,...,X_m\right)$来自某总体分布;$\left( Y_1,...,Y_n\right)$来自另外的总体分布;
设$\overline{X} = 70$,$\overline{Y} = 80$,样本容量$n = m = 26$,样本方差$S_1^2 = 10$,$S_2^2 = 12$。\\
(1)假设总体服从正态分布:判断两总体方差是否相同;判断两总体均值是否相同。\\
(2)若去掉正态性假设,检验两总体均值是否相同\\
解:(1)\\
i.两正态总体方差齐性检验:假设检验问题为
$$
H_0: \sigma_1^2=\sigma_2^2 \longleftrightarrow H_1: \sigma_1^2 \neq \sigma_2^2 .
$$
取检验统计量为
$$
F(\boldsymbol{X}, \boldsymbol{Y})=\frac{S_2^2}{S_1^2} \stackrel{H_0}{\sim} F_{n-1, m-1}
$$
由 $F$ 检验知水平 $\alpha=0.2$ 的检验拒绝域为
$$
D=\left\{(\boldsymbol{X}, \boldsymbol{Y}): F(\boldsymbol{X}, \boldsymbol{Y})>F_{25,25}(0.1) \text { 或 } F(\boldsymbol{X}, \boldsymbol{Y})<F_{25,25}(0.9)=\frac{1}{F_{25,25}(0.1)}\right\} \text {. }
$$
现有 
$$F(\boldsymbol{x}, \boldsymbol{y})=s_2^2 / s_1^2=1.2 \in(1 / 1.68,1.68)$$
所以不能拒绝原假设,可认为方差齐性.\\
ii.两正态总体均值差的检验:假设检验问题为
$$
H_0: \mu_1=\mu_2 \longleftrightarrow H_1: \mu_1 \neq \mu_2 .
$$
由于我们认为方差齐性,所以可取检验统计量为
$$
T(\boldsymbol{X}, \boldsymbol{Y})=\frac{\overline{Y}-\overline{X}}{S_w \sqrt{1 / m+1 / n}} \stackrel{H_0}{\sim} t_{m+n-2}
$$
其中 $(m+n-2) S_w^2=(m-1) S_1^2+(n-1) S_2^2$ .由 $t$ 检验知水平 $\alpha=0.05$ 的检验拒绝域为
$$
D=\left\{(\boldsymbol{X}, \boldsymbol{Y}):|T(\boldsymbol{X}, \boldsymbol{Y})|>t_{50}(0.025)\right\}
$$
现有 
$$T(\boldsymbol{x}, \boldsymbol{y})=\frac{80-70}{\sqrt{11} \times \sqrt{1 / 13}}=10.9>2.01$$
所以拒绝原假设,认为两班学生成绩均值有显著差异.\\
(2)两一般总体均值差的检验:假设检验问题仍为
$$
H_0: \mu_1=\mu_2 \longleftrightarrow H_1: \mu_1 \neq \mu_2 .
$$
此时可用大样本检验,取检验统计量为
$$
Z(\boldsymbol{X}, \boldsymbol{Y})=\frac{\overline{Y}-\overline{X}}{\sqrt{S_1^2 / m+S_2^2 / n}} \xrightarrow{H_0} N(0,1), \quad \text { as } m, n \rightarrow \infty .
$$
由 $z$ 检验知水平 $\alpha=0.05$ 的检验拒绝域为
$$
D=\left\{(\boldsymbol{X}, \boldsymbol{Y}):|Z(\boldsymbol{X}, \boldsymbol{Y})|>u_{0.025}\right\} .
$$
现有 
$$Z(\boldsymbol{x}, \boldsymbol{y})=\frac{80-70}{\sqrt{10 / 26+12 / 26}}=10.9>1.96$$
所以拒绝原假设,认为两班学生成绩均值有显著差异.\\




\noindent 五、(15分)设有$n$个样本$(X_1,...,X_n)$来自$\mathrm{Beta}(2,\beta)$分布。
检验假设:
\[
H_0: \beta = 1 \quad \leftrightarrow \quad H_1: \beta \neq 1
\]
求似然比检验,Wald检验和得分检验。\\
解:伽马总体下的三大检验.\\
似然比检验:由题意知似然函数
$$
L(\beta)=\prod_{i=1}^n \beta^2 x \exp \{-\beta x\}=\beta^{2 n} \exp \left\{-\beta \sum_{i=1}^n x_i\right\} \prod_{i=1}^n x_i, \quad \beta>0
$$
由对数似然方程
$$
\frac{\partial \log L(\beta)}{\partial \beta}=\frac{2 n}{\beta}-\sum_{i=1}^n x_i=0 \Rightarrow \beta=\frac{2 n}{\sum_{i=1}^n x_i}
$$
经微分检验其确为似然函数的最大值点,因此 $\beta$ 的最大似然估计为 $\hat{\beta}=2 / \overline{X}$ .于是似然比检验统计量
$$
\Lambda(\boldsymbol{X})=\frac{\sup _{\beta>0} L(\beta)}{\sup _{\beta=1} L(\beta)}=\frac{L(\hat{\beta})}{L(1)}=\frac{(2 / \overline{X})^{2 n} \exp \{-2 n\}}{\exp \{-\overline{X}\}}=(2 / \overline{X})^{2 n} \exp \{\overline{X}-2 n\}
$$
由似然比检验统计量的极限分布知
$$
2 \log \Lambda(\boldsymbol{X})=4 n \log \frac{2}{\overline{X}}+2 \overline{X}-4 n \xrightarrow{H_0} \chi_1^2, \quad \text { as } n \rightarrow \infty
$$
因此水平为 $\alpha$ 的似然比检验的拒绝域为
$$
D=\left\{\boldsymbol{X}: 4 n \log \frac{2}{\overline{X}}+2 \overline{X}-4 n>\chi_1^2(\alpha)\right\} .
$$
Wald 检验:由似然函数计算$n$个样本的 Fisher 信息量为
$$
I_n(\beta)=-\mathbb{E}_\beta\left[\frac{\partial^2 \log L(\beta)}{\partial \beta^2}\right]=\frac{2 n}{\beta^2}
$$
所以 Wald 检验统计量为
$$
W(\boldsymbol{X})=(\hat{\beta}-1)^2 I_n(\hat{\beta})=2 n(1-1 / \hat{\beta})^2=2 n(1-\overline{X} / 2)^2 \xrightarrow{H_0} \chi_1^2, \quad \text { as } n \rightarrow \infty .
$$
因此水平为 $\alpha$ 的 Wald 检验的拒绝域为 
$$D=\left\{\boldsymbol{X}: 2 n(1-\overline{X} / 2)^2>\chi_1^2(\alpha)\right\}$$
得分检验:得分函数为
$$
U_n(\beta)=\frac{\partial \log L(\beta)}{\partial \beta}=\frac{2 n}{\beta}-\sum_{i=1}^n x_i .
$$
所以得分检验统计量为
$$
S(\boldsymbol{X})=\left[U_n(1)\right]^2 I_n^{-1}(1)=\left(2 n-\sum_{i=1}^n X_i\right)^2 /(2 n)=n(2-\overline{X})^2 / 2 \xrightarrow{H_0} \chi_1^2, \quad \text { as } n \rightarrow \infty .
$$
因此水平为 $\alpha$ 的得分检验的拒绝域为 
$$D=\left\{\boldsymbol{X}: n(2-\overline{X})^2 / 2>\chi_1^2(\alpha)\right\}$$
该形式与 Wald 检验的形式是一致的。\\





\noindent 六、(10分)男女舒张压检测数据如下:共检测男性$16$人,其中舒张压$<60$的有$4$人,$>90$的有$2$人;共检测女性$21$人,其中舒张压$<60$的有$5$人,$>90$的有$2$人。

是否能认为男女舒张压分布相同?\\
解:未合并列的齐一性检验(酌情扣分):首先根据题目描述写出列联表如下:
$$
\begin{tabular}{c|ccc|c}
	\hline 舒张压 & $<60$ & $60 \sim 90$ & $>90$ & 合计 \\
	\hline 男性人数 & 4 & 10 & 2 & 16 \\
	女性人数 & 5 & 14 & 2 & 21 \\
	\hline 合计 & 9 & 24 & 4 & 37 \\
	\hline
\end{tabular}
$$
要检验的假设为 $H_0$ :男女性舒张压分布没有显著差异.取检验统计量为
$$
K(\boldsymbol{X})=n \sum_{i=1}^r \sum_{j=1}^s \frac{\left(n_{i j}-n_{i \cdot} n_{\cdot j} / n\right)^2}{n_{i \cdot} n_{\cdot j}} \xrightarrow{H_0} \chi_{(r-1)(s-1)}^2, \quad \text { as } n \rightarrow \infty .
$$
因此水平 $\alpha$ 的检验拒绝域为 
$$D=\left\{\boldsymbol{X}: K(\boldsymbol{X})>\chi_{(r-1)(s-1)}^2(\alpha)\right\}$$
由样本观测值计算检验统计量值时,可以考虑等价形式
$$
K(\boldsymbol{x})=n\left(\sum_{i=1}^r \sum_{j=1}^s \frac{n_{i j}^2}{n_{i \cdot} n_{\cdot j}}-1\right)=0.1040<\chi_2^2(0.2)=3.22 .
$$
因此不拒绝原假设,认为男女性舒张压没有显著差异.\\
合并列的齐一性检验:由于有格子点计数过小,需要合并首尾两列得列联表如下:
$$
\begin{tabular}{c|ccc}
	\hline 舒张压 & 正常 & 过低或过高 & 合计 \\
	\hline 男性人数 & 10 & 6 & 16 \\
	女性人数 & 14 & 7 & 21 \\
	\hline 合计 & 24 & 13 & 37 \\
	\hline
\end{tabular}
$$
要检验的假设为 $H_0$ :男女性舒张压分布没有显著差异.取检验统计量为
$$
K(\boldsymbol{X})=n \sum_{i=1}^r \sum_{j=1}^s \frac{\left(n_{i j}-n_i \cdot n_{\cdot j} / n\right)^2}{n_i \cdot n_{\cdot j}} \xrightarrow{H_0} \chi_{(r-1)(s-1)}^2, \quad \text { as } n \rightarrow \infty .
$$
因此水平 $\alpha$ 的检验拒绝域为 
$$D=\left\{\boldsymbol{X}: K(\boldsymbol{X})>\chi_{(r-1)(s-1)}^2(\alpha)\right\}$$
由样本观测值计算检验统计量值时,可以考虑 $2 \times 2$ 列联表的等价形式
$$
K(\boldsymbol{x})=\frac{n\left(n_{11} n_{22}-n_{12} n_{21}\right)^2}{n_{1\cdot} n_{\cdot1} n_{2\cdot} n_{\cdot 2}}=0.0692<\chi_1^2(0.2)=1.64 .
$$
因此不拒绝原假设,认为男女性舒张压没有显著差异.\\



\noindent 七、(18分)$n$个样本$(X_1,...,X_n)$来自总体分布$X\sim \Gamma\left(\alpha, \frac{1}{\theta}\right)$,其中$\alpha$已知,先验密度为$\pi(\theta) = \frac{1}{\theta} $ ,
损失函数为 
$$L^2(d,\theta) = \frac{1}{\theta^2}(d-\theta)^2$$

求贝叶斯解并证明其为Minimax解\\
解:后验分布的计算:样本 $\boldsymbol{X}=\left(X_1, \ldots, X_n\right)$ 的联合概率密度函数为
$$
f(\boldsymbol{x} \mid \theta)=\prod_{i=1}^n \frac{1}{\theta^\alpha \Gamma(\alpha)} x_i^{\alpha-1} \exp \left\{-\frac{x_i}{\theta}\right\}=\frac{\theta^{-n \alpha}}{[\Gamma(\alpha)]^n} \exp \left\{-\frac{1}{\theta} \sum_{i=1}^n x_i\right\} \prod_{i=1}^n x_i^{\alpha-1}
$$
因此在无信息先验 $\pi(\theta)=(1 / \theta) I_{(0, \infty)}(\theta)$ 下,$\theta$ 的后验密度为
$$
\pi(\theta \mid \boldsymbol{x}) \propto f(\boldsymbol{x} \mid \theta) \pi(\theta) \propto \theta^{-n \alpha-1} \exp \left\{-\frac{1}{\theta} \sum_{i=1}^n x_i\right\}, \quad \theta>0
$$
添加归一化常数后可知 $\theta$ 的后验分布为
$$
\theta \mid \boldsymbol{X}=\boldsymbol{x} \sim \Gamma^{-1}\left(n \alpha, \sum_{i=1}^n x_i\right)
$$
加权平方损失下的 Bayes 估计:在加权平方损失下,$\theta$ 的 Bayes 估计为
$$
\hat{\theta}_B(\boldsymbol{X})=\frac{\mathbb{E}\left(\theta^{-1} \mid \boldsymbol{X}\right)}{\mathbb{E}\left(\theta^{-2} \mid \boldsymbol{X}\right)}=\frac{\frac{n \alpha}{n \overline{X}}}{\frac{n \alpha(n \alpha+1)}{(n \overline{X})^2}}=\frac{n \overline{X}}{n \alpha+1}
$$
\textbf{RK}:在矩存在的条件下有\\
\[
X \sim \Gamma(\alpha,\beta) \quad \text{有} \forall n \in \mathbb{Z} \quad \mathbb{E}[X^n]=\frac{\Gamma(\alpha+n)}{\Gamma(\alpha)\beta^n}
\]
\[
Y \sim \Gamma^{-1}(\alpha,\beta) \quad \text{有} \forall n \in \mathbb{Z} \quad \mathbb{E}[Y^n]=\frac{\Gamma(\alpha-n)\beta^n}{\Gamma(\alpha)}
\]
事实上,$\Gamma$分布的$n$阶矩就是$\Gamma^{-1}$分布的$-n$阶矩\\
验证该 Bayes 估计为 Minimax 估计:Bayes 估计 $\hat{\theta}_B(\boldsymbol{X})$ 的风险函数,注意$n\overline{X}=\sum\limits_{i=1}^n X_i \sim \Gamma\left(n\alpha,\frac{1}{\theta} \right) $
$$
\begin{aligned}
	R\left(\hat{\theta}_B(\boldsymbol{X}), \theta\right) & =\mathbb{E}\left[\frac{(n \overline{X} /(n \alpha+1)-\theta)^2}{\theta^2}\right] \\
	& =\frac{1}{\theta^2} \mathbb{E}\left(\frac{n \overline{X}+\theta}{n \alpha+1}-\theta-\frac{\theta}{n \alpha+1}\right)^2 \\
	& =\frac{1}{\theta^2}\left[\operatorname{Var}\left(\frac{n \overline{X}+\theta}{n \alpha+1}\right)+\frac{\theta^2}{(n \alpha+1)^2}\right]\\
	&=\frac{1}{n \alpha+1}
\end{aligned}
$$
为常数,所以 $\theta$ 的 Bayes 估计 $\hat{\theta}_B(\boldsymbol{X})=n \overline{X} /(n \alpha+1)$ 是 $\theta$ 的 Minimax 估计.\\


\noindent 八、(20分,附加题)
将Hardy-Weinberg 定律简化如下:$n$个样本$(X_1,...,X_n)$来自总体分布$X$\\
设$X$的概率分布为:
\[
\begin{array}{c|ccc}
X & 1 & 2 & 3 \\ \hline
P(X) & p^2 & 2p(1-p) & (1-p)^2 \\
\end{array}
\]
对于检验问题:
\[
H_0: p \leq p_0 \quad \leftrightarrow \quad H_1: p > p_0
\]
求UMPT。\\
解:先说明样本分布族是单参数指数族:样本联合概率质量函数为
$$
\begin{aligned}
	f(\boldsymbol{x} ; p) & =\left(p^2\right)^{n_0}[2 p(1-p)]^{n_1}\left[(1-p)^2\right]^{n_2} \\
	& =2^{n_1} p^{2 n_0+n_1}(1-p)^{n_1+2 n_2}\\
	&=2^{n_1}(1-p)^{2 n} \exp \left\{\left(2 n_0+n_1\right) \log \frac{p}{1-p}\right\}
\end{aligned}
$$
这是单参数指数族,且 $Q(p)=\log \frac{p}{1-p}$ 为 $p$ 的严格单调增函数,$T=T(\boldsymbol{X})=2 n_0+n_1$ .\\
根据推论 5.4.2 给出检验的 UMPT:注意到原检验问题等价于
$$
H_0: \theta \leq \theta_0 \longleftrightarrow H_1: \theta>\theta_0
$$
其中 $\theta_0=\log \frac{p_0}{1-p_0}$ .由推论 5.4.2 知,离散型要补上随机化常数,假设检验的 UMPT 可取为
$$
\phi(T)= \begin{cases}1, & T>c \\ r, & T=c \\ 0, & T<c\end{cases}
$$
其中常数 $c$ 和 $r$ 满足条件
$$
c=\underset{c^{\prime}}{\arg \min }\left\{\mathrm{P}_{p_0}(T>c) \leq \alpha\right\}, \quad r=\frac{\alpha-\mathrm{P}_{p_0}(T>c)}{\mathrm{P}_{p_0}(T=c)} .
$$
确定检验函数中的常数 $c$ 和 $r$ :记
$$
p_t=\sum_{1 \leq i, j \leq n, i+j \leq n, 2 i+j=t} \frac{n!}{i!j!(n-i-j)!} 2^i p_0^{2 i+j}\left(1-p_0\right)^{j+2(n-i-j)},
$$
则

$$
\mathrm{P}_{p_0}(T=c)=p_c, \quad \mathrm{P}_{p_0}(T>c)=\sum_{t=c+1}^n p_t .
$$





\end{document}
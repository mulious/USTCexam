\documentclass[UTF8,openany]{book}
\usepackage{ctex} 
\usepackage[hypertexnames=true]{hyperref} 
\usepackage{xcolor}
\usepackage{graphicx,amsfonts,amsmath,mathrsfs,amssymb,amsthm,url,color}
\usepackage{fancyhdr,indentfirst,bm,enumerate,natbib,float,tikz}
\usepackage{caption,subcaption,calligra}
\usepackage{afterpage} 
\usepackage{enumitem}



\let\origcleardoublepage\cleardoublepage
\renewcommand{\cleardoublepage}{\clearpage}

\hypersetup{
	colorlinks=true,
	linkcolor=blue,
	filecolor=magenta,      
	urlcolor=cyan,
	pdftitle={数理统计往年卷合集},
	pdfauthor={NULIOUS},
	pdfsubject={主题},
	pdfkeywords={数理统计,往年卷}
}

\title{\Huge 数理统计往年卷合集}
\author{\Large \calligra{NULIOUS}}
\date{}

\textheight 23cm
\textwidth 16.5cm
\topmargin -1.2cm
\oddsidemargin 0cm
\evensidemargin 0cm

\begin{document}
	\maketitle
	\frontmatter 
	\begin{center}
		\Huge 前言
	\end{center}
	
	{\Large 本文档是数理统计(管院)往年卷合集,于25年夏天编写,整理了有试题的往年卷的答案,仅供参考,如文档有错误或者有其他问题请发送至我的邮箱\href{mailto:3589851379@qq.com}{3589851379@qq.com}
	\vspace{5mm}\\
	
	数理统计期末往往考整本书,期中复习的时候可以参考往年期末卷前半本书的试题
	\vspace{5mm}\\

	我的个人主页:\url{https://mulious.github.io/}
	\vspace{5mm}\\
	
	如果试题或答案有误请先参考原卷,其中24,25期末均有答案:\href{https://github.com/mulious/USTCexam/tree/main/mathematical-statistics/original-exam}{数理统计往年题原卷}}
	\vspace{1cm}
	\begin{center}
		\huge 预祝大家在考试前速通成功!
	\end{center}
	\tableofcontents 
	\afterpage{\clearpage}
	\mainmatter
	\chapter{\centering 往年期中试卷}
	\section{\centering 25数理统计期中}
	\noindent 一.填空选择题(每空两分)\\
	(1)设 $X_{1}, X_{2}, \ldots, X_{n}, X_{n+1}$ 为来自同一正态总体的一组简单随机样本,且记 $\overline{X}=$ $\frac{1}{n} \sum_{i=1}^{n} X_{i}$ 及 $S^{2}=\frac{1}{n-1} \sum_{i=1}^{n}\left(X_{i}-\overline{X}\right)^{2}$ 。若统计量 $c_{n}\left(X_{n+1}-\overline{X}\right) / S$ 服从 $t$ 分布,则常数 $c_{n}=$\underline{\hspace{1cm}} ,$t$ 分布的自由度为\underline{\hspace{1cm}},且与 $\sum_{i=1}^{n+1} X_{i}$ 的相关系数为\underline{\hspace{1cm}}\\
	答案:$(\pm) \sqrt{\frac{n}{n+1}} ; n-1 ; 0$\\
	首先对于正态分布,$\overline{X}$与$S^2$是独立的,这说明了$t$分布的分子分母的独立性\\
	由$t$分布的定义:
	\begin{gather}
		t_{n-1}=\frac{N(0,1)}{\sqrt{\frac{\chi^2(n-1)}{n-1}}}
	\end{gather}
	再有\\
	\begin{gather}
		\frac{(n-1) s^2}{\sigma^2} \sim \chi^2(n-1) \\
		X_{n+1} - \overline{X} \sim N\left(\mu, \sigma^2\right) - N\left(\mu, \frac{\sigma^2}{n}\right) = N\left(0, \frac{n+1}{n} \sigma^2\right)
	\end{gather}
	(3)式来自两个变量的独立性\\
	则\\
	\begin{gather}
		\sqrt{\frac{n}{n+1}}\left(X_{n+1}-\overline{X}\right) / S \sim t_{n-1}
	\end{gather}
	特别的,正负号来自$t$分布是对称的(没有负号也没算错)\\
	数理统计涉及独立性的几乎只有$Basu$定理一个,猜测它们独立,即相关系数为$0$\\
	把$\left(X_1, \cdots, X_{n+1}\right)$ 视为样本,则由于指数族的性质,关于 $\lambda$ 有充分完全统计量$T=\sum_{i=1}^{n+1} X_i$\\
	我们不考虑常数部分,则\\
	\begin{gather}
		\frac{\left(X_{n+1}\right)-\left(\overline{X}\right)}{S}=\frac{\left(X_{n+1}-\mu\right)-\left(\overline{X}-\mu\right)}{S}
	\end{gather}
	是与$\mu$无关的统计量(即辅助量),因此由$Basu$定理它们独立,进而相关系数为$0$\\
	(2)设统计量 $\hat{\theta}$ 为总体参数 $\theta$ 的一个点估计,下列说法一般不成立的是\underline{\hspace{1cm}}\\
	(A)若 $\hat{\theta}$ 为 $\theta$ 的矩估计,则 $\hat{\theta}^{2}$ 为 $\theta^{2}$ 的矩估计\\
	(B)若 $\hat{\theta}$ 为 $\theta$ 的最大似然估计,则 $\hat{\theta}^{2}$ 为 $\theta^{2}$ 的最大似然估计\\
	(C)若 $\hat{\theta}$ 为 $\theta$ 的无偏估计,则 $\hat{\theta}^{2}$ 为 $\theta^{2}$ 的无偏估计\\
	(D)若 $\hat{\theta}$ 为 $\theta$ 的相合估计,则 $\hat{\theta}^{2}$ 为 $\theta^{2}$ 的相合估计\\
	答案:C\\
	一般一个随机变量的二阶矩不等于其一阶矩的平方,因此$C$错误\\
	(3)如果极小充分统计量存在,那么充分完全统计量必是极小充分统计量,但是极小充分统计量不一定是完全的。这种说法\underline{\hspace{1cm}}\\
	(A)正确\\
	(B)错误\\
	答案:A\\
	(4)设 $X_{1}, \cdots, X_{n}$ 为来自于正态总体 $N(\mu, 1)$ 的简单随机样本,若要求参数 $\mu$ 的置信系数为 $95 \%$ 的置信区间长度不超过 1 ,则至少需要抽取的样本量 $n$ 为 \underline{\hspace{1cm}}\\
	(A) 14\\
	(B) 16\\
	(C) 18\\
	(D) 20\\
	答案:B\\
	注意方差已知。则置信区间为$\left[\overline{X}-\frac{\sigma}{\sqrt{n}} u_{\frac{\alpha}{2}}, \overline{X}+\frac{\sigma}{\sqrt{n}} u_{\frac{\alpha}{2}}\right]$,带入数值$\sigma=1$即可\\
	(5)在给定一组样本值和先验下,采用后验期望作为感兴趣参数 $\theta$ 的估计,得到估计值 $\hat{\theta}=5$ .下述说法正确的是\underline{\hspace{1cm}}\\
	(A)在重复抽取样本意义下 $\theta$ 的无偏估计值为 1.5\\
	(B)$\hat{\theta}=1.5$ 是 $\theta$ 的有效估计\\
	(C)估计值 1.5 是最小后验均方误差估计\\
	(D)估计值 1.5 是 $\theta$ 的相合估计\\
	答案:C\\
	
	
	\noindent 二。(16 分)随机调查了某保险公司 $n$ 个独立的车险索赔额 $X_{1}, \ldots, X_{n}$(单位:千元),得到如下样本直方图和正态 Q-Q 图。据此回答\\
	(1)该样本来自的总体分布有何特点?可以选择什么分布作为总体分布?给出理由.\\
	(2)试选择合适的参数统计模型,并讨论参数的充分完全统计量.\\
	\begin{figure}[h]  
		\centering  
		\includegraphics[width=0.5\textwidth]{25期中.png}  
		\label{fig:my_label}  
	\end{figure}
	\\
	解:\\
	(1)总体分布为单峰且峰偏左(右偏分布),且正态q-q图在第一象限对角线$y=x$的下方;我们可以选择$\Gamma$分布,卡方分布(也是一种$\Gamma$分布)等符合要求的分布\\
	(2)以总体分布为$\Gamma$分布:$X \sim \Gamma(\alpha,\beta)$为例\\
	样本$(X_1,...,X_n)$有联合密度:
	\begin{gather}
		f(\vec{x};\alpha,\beta)=\frac{\beta^{n\alpha}}{\Gamma(\alpha)^n} \prod_{i=1}^{n} x_i^{\alpha-1} e^{-\beta \sum_{i=1}^{n} x_i}
	\end{gather}
	把密度写成指数族的形式:\\
	\begin{gather}
		f(\vec{x};\alpha,\beta)=\frac{\beta^{n\alpha}}{\Gamma(\alpha)^n} e^{(\alpha-1)\sum_{i=1}^n \ln(x_i)-\beta\sum_{i=1}^n x_i}
	\end{gather}
	自然参数分别为$\alpha-1$,$\beta$,自然空间显然有内点,同时由因子分解定理,有充分完全统计量$T=\left( \sum\limits_{i=1}^n \ln(X_i),\sum\limits_{i=1}^n X_i\right) $\\
	特别的,若选择了卡方分布,则充分完全统计量为$T=\sum\limits_{i=1}^n \ln(X_i)$\\
	
	
	\noindent 三.(20 分)设 $X_{1}, \ldots, X_{n}$ 为来自均匀总体 $\mathrm{U}(\theta, \theta+1)$ 的简单样本,其中 $\theta \in R$ 为未知参数.试\\
	(1)证明 $T=\left(X_{(1)}, X_{(n)}\right)$ 为 $\theta$ 的极小充分统计量但不是完全统计量.\\
	(2)求 $\theta$ 的最大似然估计,并讨论其相合性.\\
	解:\\
	(1)首先要利用因子分解定理证明$(X_{(1)},X_{(n)})$是充分统计量\\
	样本联合密度:\\
	\begin{gather}
		f(\vec{x};\theta) = \mathbb{I}_{\theta < X_{(1)} < X_{(n)} < \theta + 1}
	\end{gather}
	则$h(\boldsymbol{X})=1$,$g(X_{(1)},X_{(n)};\theta)=\mathbb{I}_{\theta < X_{(1)} < X_{(n)} < \theta + 1}$\\
	由因子分解定理可以知道$(X_{(1)},X_{(n)})$是充分统计量\\
	下面再取相同总体中的$n$个样本$(Y_1,\dots,Y_n)$,构造似然比:
	\begin{gather}
		\begin{aligned}
			\frac{f(\vec{x};\theta)}{f(\vec{y};\theta)}=C(\vec{x},\vec{y})& \iff \frac{\mathbb{I}_{\theta < X_{(1)} < X_{(n)} < \theta + 1}}{\mathbb{I}_{\theta < Y_{(1)} < Y_{(n)} < \theta + 1}}=C(\vec{x},\vec{y}) \\
			&\iff (X_{(1)},X_{(n)})=(Y_{(1)},Y_{(n)})
		\end{aligned}
	\end{gather}
	其中$C(\vec{x},\vec{y})$表示仅与$\vec{x}$,$\vec{y}$有关的常数,这就说明了$(X_{(1)},X_{(n)})$是极小充分统计量\\
	下面通过充分统计量来构造辅助量(与$\theta$无关的统计量)来说明$T=(X_{(1)},X_{(n)})$不是完全统计量\\
	设$Z_i=X_i-\theta \sim U(0,1)$,则
	$$Z_{(n)}-Z_{(1)}=((X_{(n)}-\theta)-(X_{(1)}-\theta)) \sim \beta(n-1,2)$$
	与$\theta$无关\\
	上面的结论来自$U(0,1)$的极差分布为$\beta(n-1,2)$\\
	取$a$,$b$使得
	$$P(Z_{(n)}-Z_{(1)}>a)=P(Z_{(n)}-Z_{(1)}<b)>0$$
	再取
	$$\varphi(x)= \begin{cases}1, & x>a \\ 1, & x<b \\ 0, & \text { 其他 }\end{cases}$$
	则$\mathbb{E}[\varphi(T)]=0$但是$\varphi(T)$显然不处处为$0$\\
	则$T$不是完全统计量\\
	\textbf{RK}:对于二元的充分统计量要说明其不是完全的,往往通过相减和相除构造辅助量,再取如$\varphi$这样的函数进行说明\\
	(2)接下来求$\theta$的最大似然估计\\
	由式(8)可以看出\\
	\begin{gather}
		f(\vec{x};\theta) = \begin{cases}
			1, & X_{(n)}-1<\theta<X_{(1)}, \\
			0, & \text{其他}.
		\end{cases}
	\end{gather}
	则$\theta$的最大似然估计$\hat{\theta}_{MLE}$为$(X_{(n)}-1, X_{(1)})$中的任何值\\
	下面利用$Markov$不等式证明其弱相合性\\
	只需说明$tX_{(1)}+(1-t)(X_{(n)}-1)$对于$\theta$的相合性即可,其中$0<t<1$\\
	$$
	\begin{aligned}
		P\Bigl(\bigl|tX_{(1)}+(1-t)(X_{(n)}-1)-\theta\bigr| \geq \epsilon\Bigr) \leq \frac{\mathbb{E}\bigl[\bigl|tX_{(1)}+(1-t)(X_{(n)}-1)-\theta\bigr|\bigr]}{\epsilon}& \\
		\leq \frac{\mathbb{E}\bigl[|tX_{(1)}-t\theta|\bigr] + \mathbb{E}\bigl[|(1-t)(X_{(n)}-1)-(1-t)\theta|\bigr]}{\epsilon}&
	\end{aligned}
	$$
	而
	\begin{gather}
		X_{(1)}-\theta \sim \beta(1,n), \\
		X_{(n)}-\theta \sim \beta(n,1).
	\end{gather}
	则$\mathbb{E}[X_{(1)}] = \theta + \frac{1}{n+1}$,$\mathbb{E}[X_{(n)}] = \theta + \frac{n}{n+1}$\\
	注意关系$\theta < X_{(1)} < X_{(n)} < \theta + 1$\\
	代入$Markov$不等式后令$n\rightarrow \infty$即证弱收敛\\
	
	
	\noindent 四.(25 分)某厂生产的产品分为三个质量等级 $(X=1,2,3)$ ,各等级产品的分布如下\\
	\begin{center}
		\begin{tabular}{c|ccc}
			\hline
			$X$ & 1 & 2 & 3 \\
			\hline
			$P$ & $\theta$ & $2 \theta$ & $1-3 \theta$ \\
			\hline
		\end{tabular}
	\end{center}
	
	其中 $\theta \in(0,1 / 3)$ 未知.为了解该厂产品的质量分布情况,从该厂产品中随机有放回抽取 20 件产品检测后发现一等品有 5 件,二等品有 7 件,三等品有 8 件.试\\
	(1)求 $\theta$ 的矩估计和最大似然估计量,是否都为无偏估计?给出估计值.\\
	(2)求 $\theta$ 的最小方差无偏估计量,其方差是否达到了 Cramér-Rao下界?\\
	解:\\
	(1)$\mathbb{E}[X]=3-4\theta$\\
	则$\theta$的矩估计为$\hat{\theta}_{M}=\frac{3-\overline{X}}{4}$,它自然是无偏的(因为就是拿期望算出来的)\\
	记$n_k=\sum_{i=1}^{n} \mathbb{I}_{X_i=k}$\\
	则$(X_1,...,X_n)$有联合密度$$f(\vec{x};\theta)=\theta^{n_1}(2\theta)^{n_2}(1-3\theta)^{n_3}$$
	则\\
	\begin{gather}
		\ln f(\vec{x};\theta)=n_1 \ln \theta+n_2\ln (2\theta) +n_3\ln(1-3\theta)\\
		\frac{\partial \ln f\left(\vec{x} ; \theta\right)}{\partial \theta}=\frac{n_1}{\theta}+\frac{n_2}{\theta}-\frac{3n_3}{1-3\theta}=0
	\end{gather}
	有$$\hat{\theta}_{MLE}=\frac{n-n_3}{3n}$$
	又$n_3 \sim B(n,1-3\theta)$\\
	则$$\mathbb{E}[\hat{\theta}_{MLE}]=\frac{1}{3}-\frac{\mathbb{E}[n_3]}{3n}=\frac{1}{3}-\frac{n(1-3\theta)}{3n}=\theta$$
	即$\hat{\theta}_{MLE}$为无偏估计\\
	带入数值有$\hat{\theta}_{M}=0.2125$,$\hat{\theta}_{MLE}=0.2$\\
	(2)化为自然指数族的形式\\
	\begin{gather}
		f(\vec{x};\theta)=e^{n_2\ln2}e^{n_1\ln\theta+n_2\ln\theta+n_3\ln(1-3\theta)}\\
		=e^{n_2\ln2}e^{\ln\theta}e^{n_3\ln(\frac{1-3\theta}{\theta})}
	\end{gather}
	其中$h(\boldsymbol{X})=e^{n_2\ln2} \quad C(\theta)=e^{\ln\theta}=\theta$,自然参数为$\ln(\frac{1-3\theta}{\theta})$\\
	又$\theta \in (0,\frac{1}{3})$,则自然参数空间有内点,且由因子分解定理,$T=n_3$为$\theta$的充分完全统计量\\
	同时注意到$\hat{\theta}_{MLE}$无偏且为充分完全统计量的函数,它也就是$\theta$的$UMVUE$\\
	注意$UMVUE$能达到$C-R$下界$\iff$分布为单参数指数族且$UMVUE$为充分完全统计量$T(\vec{x})$的线性函数\\
	所以本题的$UMVUE$能达到$C-R$下界\\
	下面额外给出数值验证:\\
	\begin{gather}
		\text{Var}(\hat{\theta}_{\text{MLE}}) = \frac{\text{Var}(n_3)}{9n^2} = \frac{3n\theta(1-3\theta)}{9n^2} = \frac{\theta(1-3\theta)}{3n}, \\
		\begin{aligned}
			n \text{个样本的Fisher信息量:} I(\theta) &= -\mathbb{E}\left[ \frac{\partial^2}{\partial\theta^2} \ln f(\vec{x};\theta) \right]\\
			&=-\mathbb{E}[-\frac{n_1}{\theta^2}-\frac{n_2}{\theta^2}-\frac{9n_3}{(1-3\theta)^2}]\\
			&=\frac{3n}{\theta(1-3\theta)}
		\end{aligned}
	\end{gather}
	对于$\theta$和$n$个样本的$Fisher$信息量$I(\theta)$,其$C-R$下界为$$\frac{1}{I(\theta)}=\frac{\theta(1-3\theta)}{3n}$$这就说明了$UMVUE$的方差能达到$C-R$下界\\
	\textbf{RK}:对于$n$个样本的$Fisher$信息量$I(\theta)$,$g(\theta)$的$C-R$下界为$\frac{[g'(\theta)]^2}{I(\theta)}$\\
	对于单个样本(总体)的$Fisher$信息量$\widetilde{I(\theta)}$,$g(\theta)$的$C-R$下界为$\frac{[g'(\theta)]^2}{n\widetilde{I(\theta)}}$\\
	这实际上是因为$I(\theta)$是$\widetilde{I(\theta)}$的$n$倍,另外对于$Fisher$信息阵仍有相同的规律\\
	
	
	\noindent 五.(25 分)调查发现人们每天使用手机的时间(单位:分钟)服从正态分布 $N\left(\mu, \sigma^{2}\right)$ ,其中 $\mu \in R, \sigma^{2}>0$ 为未知参数。现随机调查了 25 个人每天使用手机时间,得到样本均值 $\overline{X}=180$ 分钟,样本标准差 $S=20$ 分钟。若取先验分布为 $\pi\left(\mu, \sigma^{2}\right) \propto \sigma^{-2}$ .试\\
	(1)求 $\sigma^{2}$ 的边际后验分布,并给出 $\sigma^{2}$ 的后验期望估计值.\\
	(2)求一个人每天平均使用手机时长 $\mu$ 的 $95 \%$ 置信区间和可信区间,两者的解释有何不同?\\
	解:\\
	(1)先解释一下$\pi\left(\mu, \sigma^{2}\right) \propto \sigma^{-2}$的含义,他表示先验分布与$\mu$成常数倍的关系,但不意味着先验分布给出的\textbf{只有}$\sigma$的信息,因此按照这个先验分布算出来的后验分布是$\mu$与$\sigma$的联合分布,如果要求某一个参数的分布还需要对另一个参数进行积分\\
	样本联合密度为:
	\begin{gather}
		f(\vec{x};\mu,\sigma^2)=(2\pi \sigma^2)^{-\frac{n}{2}}e^{-\frac{\sum_{i=1}^{n} (x_i-\mu)^2}{2\sigma^2}}
	\end{gather}
	后验联合密度为:
	\begin{gather}
		\pi(\mu,\sigma^2|\vec{x}) \propto f(\vec{x};\mu,\sigma^2)\cdot\pi\left(\mu, \sigma^{2}\right) \propto (\sigma^2)^{-\frac{n}{2}-1}e^{-\frac{\sum (x_i-\overline{X})^2 +n(\overline{X}-\mu)^2}{2\sigma^2}}
	\end{gather}
	其中用到了
	$$\sum_{i=1}^{n} (x_i-\mu)^2=\sum_{i=1}^{n} (x_i-\overline{X})^2 +n(\overline{X}-\mu)^2$$
	接下来对参数$\mu$做积分来得到$\sigma^2$的后验密度
	\begin{gather}
		\begin{aligned}
			\pi(\sigma^2|\vec{x})&=\int_{-\infty}^{+\infty} \pi(\mu , \sigma^2|\vec{x})d\mu\\
			&=(\sigma^2)^{-\frac{n}{2}-1}e^{-\frac{\sum_{i=1}^n (x_i-\overline{X})^2}{2\sigma^2}} \int_{-\infty}^{+\infty}e^{-\frac{n(\overline{X}-\mu)^2}{2\sigma^2}}d\mu\\
			&=(\sigma^2)^{-\frac{n}{2}-\frac{1}{2}}e^{-\frac{\sum_{i=1}^n (x_i-\overline{X})^2}{2\sigma^2}} 
		\end{aligned}
	\end{gather}
	式(26)来自$N(\overline{X},\frac{\sigma^2}{n})$的密度的积分为1\\
	即
	\[
	\int_{-\infty}^{+\infty}e^{-\frac{n(\overline{X}-\mu)^2}{2\sigma^2}}d\mu=(\frac{2\pi \sigma^2}{n})^{\frac{1}{2}}
	\]
	则
	\[
	\sigma^2|\vec{x} \sim \Gamma^{-1}(-\frac{n}{2}+\frac{1}{2},\frac{\sum (x_i-\overline{X})^2}{2})
	\]
	最后
	\[
	\hat{\sigma^2}_{E}=\frac{\sum\limits_{i=1}^{n} (x_i-\overline{X})^2}{2(-\frac{n}{2}-\frac{1}{2})}=480
	\]
	\textbf{RK}:在矩存在的条件下有\\
	\[
	X \sim \Gamma(\alpha,\beta) \quad \text{有} \forall n \in \mathbb{Z} \quad \mathbb{E}[X^n]=\frac{\Gamma(\alpha+n)}{\Gamma(\alpha)\beta^n}
	\]
	\[
	Y \sim \Gamma^{-1}(\alpha,\beta) \quad \text{有} \forall n \in \mathbb{Z} \quad \mathbb{E}[Y^n]=\frac{\Gamma(\alpha-n)\beta^n}{\Gamma(\alpha)}
	\]
	事实上,$\Gamma$分布的$n$阶矩就是$\Gamma^{-1}$分布的$-n$阶矩\\
	(2)\\
	先解释一下置信区间与可信区间的区别:\\
	置信区间:经过多次重复实验,$\mu$落在置信区间的\textbf{频率}趋于$95\%$\\
	可信区间:相当于把$\mu$视为\textbf{随机变量},$\mu$落在可信区间的\textbf{概率}为$95\%$\\
	构造置信区间:\\
	未知$\mu$,$\sigma^2$\\
	取\\
	\[
	\frac{\sqrt{n}(\overline{X}-\mu)}{S} \sim t_{n-1}
	\]
	那么置信区间为:
	\[
	[\overline{X}-\frac{t_{n-1}(\frac{\alpha}{2})S}{\sqrt{n}},\overline{X}+\frac{t_{n-1}(\frac{\alpha}{2})S}{\sqrt{n}}]
	\]
	带入数值为$[171.76,188.24]$\\
	构造可信区间:\\
	首先要求出$\mu$的后验分布:\\
	\begin{gather}
		\begin{aligned}
			\pi(\mu|\vec{x})&=\int_{0}^{+\infty} \pi(\mu,\sigma^2|\vec{x})d\sigma^2\\
			&=\int_{0}^{+\infty}(\sigma^2)^{-\frac{n}{2}-1}e^{-\frac{\sum (x_i-\overline{X})^2 +n(\overline{X}-\mu)^2}{2\sigma^2}}d\sigma^2\\
			&\overset{t=\frac{1}{\sigma^2}}=\int_{0}^{+\infty}(t)^{\frac{n}{2}-1}e^{-\frac{t(\sum (x_i-\overline{X})^2 +n(\overline{X}-\mu)^2)}{2}}dt
		\end{aligned}
	\end{gather}
	设$\alpha=\frac{n}{2}$,$\beta=\frac{\sum_{i=1}^n (x_i-\overline{X})^2 +n(\overline{X}-\mu)^2}{2}$\\
	有$\Gamma(\alpha,\beta)$的密度的积分为$1$\\
	即\\
	\[
	\int_{0}^{+\infty}\frac{\beta^{\alpha}}{\Gamma(\alpha)}x^{\alpha-1}e^{-\beta x}=1
	\]
	则
	\begin{gather}
		\begin{aligned}
			\pi(\mu|\vec{x}) & \propto \frac{\Gamma(\alpha)}{\beta^{\alpha}}\\
			&=\frac{\Gamma(\frac{n}{2})}{(\frac{\sum_{i=1}^n  (x_i-\overline{X})^2-n(\overline{X}-\mu)^2}{2})^{\frac{n}{2}}}
		\end{aligned}
	\end{gather}
	接下来的计算意义不大,因为考试时没有提供非标准$t$分布的密度\\
	\[
	t_v(\mu,\sigma^2) \sim f(x)=\frac{\Gamma(\frac{v+1}{2})}{\sigma \sqrt{v\pi} \Gamma(\frac{v}{2})}(1+\frac{(x+\mu)^2}{v\sigma^2})^{-\frac{v+1}{2}}
	\]
	带入数值可以得到
	\[
	\mu|\vec{x} \sim t_{24}(180,16)
	\]
	\[
	\frac{\mu|\vec{x}-180}{4} \sim t_{24}
	\]
	取其上下$\frac{\alpha}{2}$分位数,得到的可信区间也为$[171.76,188.24]$\\
	
	
	
	\noindent 附表:上分位数 $u_{0.025}=1.960, u_{0.05}=1.645, t_{24}(0.025)=2.06, t_{24}(0.05)=1.71$\\
	伽马分布,逆伽马分布与 $t$ 分布概率密度函数:
	
	$$
	\begin{aligned}
		& G a(\alpha, \beta): f(x)=\frac{\beta^{\alpha}}{\Gamma(\alpha)} x^{\alpha-1} e^{-\beta x}, \alpha, \beta, x>0 \\
		& \operatorname{Inv} G a(\alpha, \beta): f(x)=\frac{\beta^{\alpha}}{\Gamma(\alpha)} x^{-\alpha-1} e^{-\frac{\beta}{x}}, \alpha, \beta, x>0 \\
		& t_{n}: f(x)=\frac{\Gamma\left(\frac{n+1}{2}\right)}{\Gamma\left(\frac{n}{2}\right) \sqrt{n \pi}}\left(1+\frac{x^{2}}{n}\right)^{-(n+1) / 2},-\infty<x<\infty
	\end{aligned}
	$$
	\newpage
	\section{\centering 24数理统计期中}
	\noindent 一.(20分)设从总体
	$$
	\begin{tabular}{c|ccc}
		X & 0 & 1 & 2 \\
		\hline$P$ & $(1-\theta) / 3$ & $1 / 3$ & $(1+\theta) / 3$
	\end{tabular}
	$$
	(其中 $-1<\theta<1$ 为未知参数)中抽取的一个简单样本 $X_1, \ldots, X_n$ \\
	(1)求 $\theta$ 的充分统计量,其是否为完全统计量?\\
	(2)求 $\theta$ 的矩估计 $\tilde{\theta}$ 和最大似然估计 $\hat{\theta}$ ,是否为无偏估计?\\
	解:(1)设 $n_0, n_1, n_2$ 分别为 $\{x_n\}$ 中为取值为 $1,2,3$ 的个数,则 
	$$ n=n_0+n_1+n_2$$
	样本联合密度为
	$$
	f\left(x_1 \cdots, x_n ; \theta\right)=\left(\frac{1-\theta}{3}\right)^{n_0}\left(\frac{1}{3}\right)^{n_1}\left(\frac{1+\theta}{3}\right)^{n_2}
	$$
	注意这是指数族,改写为
	$$
	f\left(x_1 \cdots, x_n ; \theta\right)=\left(\frac{1}{3}\right)^{n_1} e^{n_0 \ln \frac{1-\theta}{3}+n_2 \ln \frac{1+\theta}{3}}
	$$
	即 $(\ln\frac{1-\theta}{3}, \ln \frac{1+\theta}{3})$ 作为自然参数,显然其在 $\mathbb{R}^2$ 中有内点\\
	则$T(\boldsymbol{X})=\left(n_0, n_2\right)$ 是充分完全统计量\\
	(2)矩估计:
	$$
	\mathbb{E}[X]=1+\frac{2 \theta}{3} 
	$$
	令 $$\hat{\theta}_M=\frac{3}{2}(\overline{X}-1)$$
	即可,显然无偏。\\
	MLE:
	$$
	\ln f\left(x_1, \cdots, x_n ; \theta\right)=n_1 \ln \frac{1}{3}+n_0 \ln \frac{1-\theta}{3}+n_2 \ln \frac{1+\theta}{3} 
	$$
	$$
	\frac{\partial \ln f}{\partial \theta}=\frac{-n_0}{1-\theta}+\frac{n_2}{1+\theta}=0 
	$$
	$$
	\hat{\theta}_{MLE}=\frac{n_2-n_0}{n_2+n_0} 
	$$
	用 $n=1 \quad \theta=0.5$ 验证知非无偏。\\
	
	
	
	\noindent 二.(20分)一个移动通讯公司随机抽取了其 900 个包月客户,计算得知他们一个月平均使用时间是 220 分钟,样本标准差是 90 分钟.假设使用时间服从正态分布.\\
	(1)求包月客户平均使用时间和标准差的 $95 \%$ 置信区间,并解释所得区间的含义.\\
	(2)如果要求客户平均使用时间的 $95 \%$ 置信区间的长度不超过 5 分钟,应至少抽取多少个客户?该公司的抽样规模是否满足要求?\\
	解:(1)$\mu, \sigma^2$ 均未知\\
	利用 
	$$\frac{\sqrt{n}(\overline{X}-\mu)}{S} \rightarrow t_{n-1}$$
	估计 $\mu$ ,进而有置信区间
	\[
	\left[ \overline{X} - t_{n-1}\left(\frac{\alpha}{2}\right) \frac{S}{\sqrt{n}},\ 
	\overline{X} + t_{n-1}\left(\frac{\alpha}{2}\right) \frac{S}{\sqrt{n}} \right]
	\]
	代值有
	$$[214.12,225.885]$$ 
	这里因为$t_n\rightarrow N(0,1)$ ,用$u_{\frac{\alpha}{2}}$代替$t_{n-1}\left(\frac{\alpha}{2}\right)$\\
	再利用 
	$$\frac{(n-1) S^2}{\sigma^2} \rightarrow \chi^2_{(n-1)}$$
	$\sigma^2$ 有置信区间 
	$$\left[\frac{(n-1) S^2}{\chi^2_{(n-1)}\left(\frac{\alpha}{2}\right)}, \frac{(n-1) S^2}{\chi^2_{(n-1)}\left(1-\frac{\alpha}{2}\right)}\right]$$
	开方有 $\sigma$ 的置信区间
	$$
	\left[\sqrt{\frac{(n-1) S^2}{\chi^2_{(n-1)}\left(\frac{\alpha}{2}\right)}},\sqrt{\frac{(n-1) S^2}{\chi^2_{(n-1)}\left(1-\frac{\alpha}{2}\right)}}\right]
	$$
	代值为
	\[
	[88.15,92.02]
	\]
	区间的含义:使用这个区间充分大次数后,落在置信区间的频率接近于置信系数\\
	(2)即需要
	\[
	2\times u_{\frac{\alpha}{2}}\frac{S}{\sqrt{n}}\le 5
	\]
	这里同样因为$t_n\rightarrow N(0,1)$ ,用$u_{\frac{\alpha}{2}}$代替$t_{n-1}\left(\frac{\alpha}{2}\right)$解得
	\[
	n\ge 4979
	\]
	不满足要求\\
	
	
	
	
	\noindent 三.(20分)下表统计了某铁路局 122 个扳道员五年内由于操作失误引起的严重事故情况,其中 $r$ 表示一扳道员某五年内引起严重事故的次数,$s$ 表示扳道员人数.假设扳道员由于操作失误在五年内所引起的严重事故的次数服从Poisson分布.求
	$$
	\begin{tabular}{c|ccccccc}
		\hline$r$ & 0 & 1 & 2 & 3 & 4 & 5 & $\geq 6$ \\
		\hline$s$ & 44 & 42 & 21 & 9 & 4 & 2 & 0 \\
		\hline
	\end{tabular}
	$$
	(1)一个扳道员在五年内未引起严重事故的概率 $p$ 的最小方差无偏估计 $\hat{p}_1$ 和最大似然估计 $\hat{p}_2$ .\\
	(2)$p$ 的一个(渐近) $95 \%$ 水平的置信上界.\\
	解:(1)下面记Poisson分布的参数为$\lambda$\\
	MLE:样本联合密度为
	\[
	f\left(x_1, \cdots, x_n ; \lambda\right)=\frac{e^{-n \lambda} \lambda^{x_1+\cdots+x_n}}{x_{1}!\cdots x_{n}!}
	\]
	进而
	\[
	\ln f\left(x_1, \cdots, x_n ; \lambda\right) \propto \left(\sum_{i=1}^n x_i  \right) \ln \lambda -n\lambda
	\]
	则令
	\[
	\frac{\partial \ln f}{\partial \lambda}=\frac{\sum\limits_{i=1}^n x_i}{\lambda} -n=0
	\]
	解得
	\[
	\hat{\lambda}_{MLE}=\frac{\sum\limits_{i=1}^n x_i}{n}
	\]
	由于MLE的不变性有
	\[
	\hat{p_2}=\hat{e^{-\lambda}}_{MLE}=e^{-\frac{\sum\limits_{i=1}^n x_i}{n}}=0.325
	\]
	UMVUE:先找一个无偏估计,又由于Poisson分布是指数族,充分完全统计量是明显的,即$T(\boldsymbol{X})=\sum\limits_{i=1}^n X_i$,对无偏估计取充分完全统计量的条件期望即得UMVUE\\
	选取
	\[
	\mathbb{I}_{\{X_1=0\}}
	\]
	作为无偏估计,因为$\mathbb{E}\left[\mathbb{I}_{\{X_1=0\}} \right]=P(X_1=0) $\\
	下一步取条件期望
	\begin{align*}
		\mathbb{E}\left[\mathbb{I}_{\{X_1=0\}}\mid T(\boldsymbol{X})=t \right]&=P\left(X_1=0\mid T(\boldsymbol{X})=t \right) \\
		&=\frac{P\left(X_1=0,\sum\limits_{i=2}^n X_i=t \right) }{P(T(\boldsymbol{X})=t)}
	\end{align*}
	利用Poisson分布的可加性,有
	\[
	\sum\limits_{i=2}^n X_i \sim \mathrm{P}((n-1)\lambda) \quad \sum\limits_{i=1}^n X_i \sim \mathrm{P}(n\lambda)
	\]
	则
	\[
	\mathbb{E}\left[\mathbb{I}_{\{X_1=0\}}\mid T(\boldsymbol{X})=t \right]=\left(\frac{n-1}{n} \right)^t 
	\]
	即UMVUE为
	\[
	\hat{p_1}=\left(\frac{n-1}{n} \right)^{\sum\limits_{i=1}^n X_i}
	\]
	代值为
	\[
	\hat{p_1}=0.325
	\]
	(2)解一:利用MLE的渐进正态性
	\[
	\sqrt{n}\left(\hat{\lambda}_{MLE}-\lambda \right) \rightarrow N\left(0,\frac{1}{I(\lambda)} \right) 
	\]
	其中$I(\lambda)$为总体(单个样本)的Fisher信息量,为
	\[
	I(\lambda)=-\mathbb{E}\left[\frac{\partial^2 \ln f(x,\lambda)}{\partial \lambda^2} \right] =\frac{1}{\lambda}
	\]
	这里
	\[
	f(x,\lambda)=P(X=x,\lambda)=\frac{\lambda^x}{x!}e^{-\lambda}
	\]
	得到
	\[
	\sqrt{n}\left(\hat{\lambda}_{MLE}-\lambda \right) \rightarrow N\left(0,\lambda \right) 
	\]
	法一:因为$e^{-\lambda}$单调递减,因此只需要求出$\lambda$的置信下界即可
	\[
	\frac{\sqrt{n}\left(\hat{\lambda}_{MLE}-\lambda \right)}{\sqrt{\lambda}}\le u_{\alpha}
	\]
	处理一:直接解一元二次方程,较复杂\\
	处理二:利用MLE做二次近似\\
	用$\sqrt{\overline{X}}$代替$\sqrt{\lambda}$
	\[
	\frac{\sqrt{n}\left(\hat{\lambda}_{MLE}-\lambda \right)}{\sqrt{\overline{X}}}\le u_{\alpha}
	\]
	得到
	\[
	\lambda \ge \overline{X}-\frac{\sqrt{\overline{X}}u_{\alpha}}{\sqrt{n}}
	\]
	则$e^{\lambda}$的置信上界为
	\[
	e^{-\overline{X}+\frac{\sqrt{\overline{X}}u_{\alpha}}{\sqrt{n}}}
	\]
	法二:使用$\Delta$方法,取$g(x)=e^{-x}$,则
	\[
	\sqrt{n}\left(\hat{e^{-\lambda}}_{MLE}-e^{-\lambda} \right) \rightarrow N\left(0,e^{-2\lambda}\lambda \right) 
	\]
	即
	\[
	\frac{\sqrt{n}\left(\hat{e^{-\lambda}}_{MLE}-e^{-\lambda} \right)}{\sqrt{\lambda e^{-2\lambda}}}\rightarrow N(0,1)
	\]
	仍类似处理二,利用MLE做二次近似
	\[
	\frac{\sqrt{n}\left(\hat{e^{-\lambda}}_{MLE}-e^{-\lambda} \right)}{\sqrt{\overline{X} e^{-2\overline{X}}}}\ge u_\alpha
	\]
	解得置信上界为
	\[
	e^{-\overline{X}}-\frac{\sqrt{\overline{X} e^{-2 \overline{X}}} u_\alpha}{\sqrt{n}}
	\]
	解二:利用CLT
	\[
	\sqrt{n}\left(\overline{X}-\lambda \right)\rightarrow N(0,\lambda) 
	\]
	余下解法同解一\\
	解三:把$p$看作成功概率,则问题变为求两点分布的参数$p$的置信上界
	\[
	Y_i=\mathbb{I}_{\{X_1=0\}}\sim \mathrm{B}(1,p)
	\]
	则有
	\[
	\frac{\sqrt{n}(\overline{Y}-p)}{\sqrt{\overline{Y}(1-\overline{Y})}} \xrightarrow{D} N(0,1)
	\]
	直接可以得到$p$的置信上界\\
	
	
	
	
	\noindent 四.(15分)设 $X_1, \ldots, X_n$ 为来自正态总体 $N\left(1, \sigma^2\right)$ 一组简单样本,$\sigma^2>0$ 为参数.试\\
	(1)求 $\sigma^2$ 的最小方差无偏估计 $\hat{\sigma}^2$ ,其是否达到Cramer-Rao下界?\\
	(2)给出一个比最小方差无偏估计 $\hat{\sigma}^2$ 在均方误差准则下更优的估计.\\
	解:(1)$N\left(1, \sigma^2\right)$是指数族,样本联合密度为
	\[
	f\left(x_1, \cdots, x_n ; \sigma^2\right)=\left(\frac{1}{\sqrt{2 \pi \sigma^2}}\right)^n \cdot e^{-\frac{\sum\limits_{i=1}^n \left(x_{i}-1\right)^2}{2 \sigma^2}}
	\]
	显然自然参数空间有内点,进而
	\[
	T(\boldsymbol{X})=\sum\limits_{i=1}^n \left(X_{i}-1\right)^2
	\]
	为$\sigma^2$的充分完全统计量\\
	注意当$\mu=1$已知的时候,$\sigma^2$有无偏估计
	\[
	\frac{\sum\limits_{i=1}^n\left(X_i-\mu\right)^2}{n}=\frac{\sum\limits_{i=1}^n\left(X_i-1\right)^2}{n}
	\]
	它正是充分完全统计量的函数,因此UMVUE就是
	\[
	\hat{\sigma^2}=\frac{\sum\limits_{i=1}^n\left(X_i-1\right)^2}{n}
	\]
	对C-R下界,注意$N\left(1, \sigma^2\right)$是单参数指数族,且UMVUE为充分完全统计量的线性函数,那么UMVUE可以达到C-R下界\\
	下面进行数值验证:$n$个样本的Fisher信息量为
	\[
	\begin{aligned}
		I\left(\sigma^2\right) & =-\mathbb{E}\left[\frac{\partial^2}{\partial \left(\sigma^2\right)^2} \ln f\left(x_1, \cdots, x_n ; \sigma^2\right)\right] \\
		& =\frac{n}{2 \sigma^4} 
	\end{aligned}
	\]
	因此C-R下界为
	\[
	\frac{1}{I\left(\sigma^2\right)}=\frac{2 \sigma^4}{n} 
	\]
	另外一方面
	\[
	\operatorname{Var}\left(\hat{\sigma^2}\right)=\operatorname{Var}\left(\frac{T(\boldsymbol{X})}{n}\right)=\frac{1}{n^2} \operatorname{Var}(T(\boldsymbol{X}))
	\]
	注意
	\[
	\frac{\sum\limits_{i=1}^n\left(X_i-\mu\right)^2}{\sigma^2} \sim \chi^2(n)
	\]
	即
	\[
	\operatorname{Var}\left(\frac{T(\boldsymbol{X})}{\sigma^2}\right)=2n
	\]
	也就是
	\[
	\operatorname{Var}\left(\hat{\sigma^2}\right)=\frac{2 \sigma^4}{n} 
	\]
	达到C-R下界\\
	(2)这时我们不要求无偏性\\
	对估计$\hat{\theta}$,均方误差
	\[
	\begin{aligned}
		\operatorname{MSE}(\hat{\theta}) & =\mathbb{E}\left[(\hat{\theta}-\mathbb{E}[\hat{\theta}])^2\right]+(\mathbb{E}[\hat{\theta}]-\theta)^2 \\
		& =\operatorname{Var}(\hat{\theta})+(\mathbb{E}[\hat{\theta}]-\theta)^2
	\end{aligned}
	\]
	考虑
	\[
	\tilde{\sigma^2}=c\hat{\sigma^2}
	\]
	则
	\[
	\begin{aligned}
		\operatorname{MSE}\left(\widetilde{\sigma}^2\right) & =\operatorname{Var}\left(c \hat{\sigma^2}
		\right)+\left(c \sigma^2-\sigma^2\right)^2 \\
		& =c^2 \cdot \frac{2 \sigma^2}{4}+\left(c \sigma^2-\sigma^2\right)^2
	\end{aligned}
	\]
	对$c$求导数并令其为0
	\[
	\frac{4 c \sigma^4}{n}+2(c-1) \sigma^4=0
	\]
	得到
	\[
	c=\frac{n}{n+2}
	\]
	即
	\[
	\tilde{\sigma^2}=\frac{1}{n+2} \sum_{i=1}^n\left(X_i-1\right)^2 
	\]
	且满足$\operatorname{MSE}\left(\tilde{\sigma^2}\right)<\operatorname{MSE}\left(\hat{\sigma^2}\right)$\\
	
	
	
	
	
	
	
	\noindent 五.(25分)设 $X_1, \ldots, X_n$ 为来自如下指数总体的简单样本,总体密度函数为
	$$
	f(x , a)=e^{-(x-a)} I(x \geq a),-\infty<a<1
	$$
	其中 $a$ 为未知参数.试\\
	(1)求 $a$ 的最大似然估计,并讨论其相合性和极限分布.\\
	(2)证明 $T=X_{(1)}$ 为 $a$ 的充分统计量但不是完全统计量.\\
	(3)求 $a$ 的最小方差无偏估计.\\
	解:(1)样本联合密度为
	\[
	f\left(x_1, \cdots x_n ; a\right)=e^{-\sum_{i=1}^n x_i} \cdot e^{n a} \cdot \mathbb{I}_{\left\{x_{(1)} \geq a\right\}}
	\]
	由单调性
	\[
	\hat{a}_{MLE}=X_{(1)}
	\]
	$X_{(1)}$有密度
	\[
	f(x)=ne^{-n(x-a)},x\ge a
	\]
	有分布函数
	\[
	F(x)=1-e^{-n(x-a)}
	\]
	进一步
	\[
	P\left(\left|\hat{a}_{M L E}-a\right| \geqslant \varepsilon\right)=P(X_{(1)} \geqslant a+\varepsilon)=e^{-n \varepsilon} \rightarrow 0
	\]
	这就说明了弱收敛\\
	\textbf{RK}:进一步的
	\[
	\sum\limits_{i=1}^{\infty} P\left(\left|\hat{a}_{M L E}-a\right| \geqslant \varepsilon\right)<\infty
	\]
	由B-C引理可知强收敛\\
	极限分布:
	\[
	X_i-a\sim \mathrm{Exp}(1)\triangleq Y_i
	\]
	而
	\[
	Y_{(1)}\sim \mathrm{Exp}(n)
	\]
	则
	\[
	n(X_{(1)}-a)\sim \mathrm{Exp}(1)
	\]
	(2)由因子分解定理知$T(\boldsymbol{X})=X_{(1)}$充分\\
	下面证明它不是完全的,即存在$\phi(T)$使得$\mathbb{E}[\phi(T)]=0$但是$\phi(T)$不恒为0\\
	条件
	\[
	\begin{aligned}
		\mathbb{E}[\phi(T)]&=\int_a^{+\infty} \phi(t) \cdot n e^{-n(t-a)} d t\\
		&=\int_a^{1} \phi(t) \cdot n e^{-n(t-a)} d t+\int_1^{+\infty} \phi(t) \cdot n e^{-n(t-a)} d t\\
		&=0
	\end{aligned}
	\]
	求导有
	\[
	\phi(t)=0\quad \forall t<1
	\]
	下面构造$t\ge 1$的部分,把积分分段成有限和无限的两段,这两段都不能为0,且两段的积分之和为0,去掉常数部分,只需要满足
	\[
	\int_1^{c} \phi(t) \cdot  e^{-nt} d t=-\int_c^{+\infty} \phi(t) \cdot  e^{-nt} d t
	\]
	不妨就设$c=2$且$\phi(t)=1,t\ge 2$,则
	\[
	\int_2^{+\infty} \phi(t) \cdot  e^{-nt} d t=\frac{e^{-2n}}{n}
	\]
	另一方面
	\[
	\int_1^{2}    e^{-nt} d t=\frac{-e^{-2n}+e^{-n}}{n}
	\]
	那么只需要令
	\[
	\phi(t)=
	\begin{cases}
		0 & a\le t<1\\
		1-e^{nt}\cdot \frac{e^{-n}}{n}   &  1\le t\le 2\\
		1  &  t>2
	\end{cases}
	\]
	就构造出了这个反例,进而说明$T(\boldsymbol{X})=X_{(1)}$ 为 $a$ 的充分统计量但不是完全统计量\\
	\textbf{RK}:构造的$\phi(T)$不应该与未知参数有关\\
	另外地,对于包含无穷长区间的分布都可以类似地使用这个办法,将有限部分和无限部分分成两段再构造反例(这么做是为了让任何无限长的区间内$\phi(t)$不为0,否则$\phi(t)$仍然有可能以概率1地为0,如平均地从$\mathbb{R}$中取得区间$(0,1)$中的实数的概率为0)\\
	(3)对于充分但不完全的统计量,用零无偏法,注意参数取值范围$a<1$\\
	设$\mathbb{E}[\delta(T)]=0$,且$h(T)$为所求的UMVUE,则有
	\[
	\mathbb{E}[\delta(T)]=\int_a^{+\infty} \delta(t) \cdot n e^{-n(t-a)} d t=0 
	\]
	也就是
	\[
	\int_a^{+\infty} \delta(t) \cdot  e^{-nt} d t=0 
	\]
	即
	\[
	\int_a^{1} \delta(t) \cdot  e^{-nt} d t+\int_1^{+\infty} \delta(t) \cdot  e^{-nt} d t=0 
	\]
	求导有
	\[
	\delta(t)=0\quad \forall t \le 1
	\]
	另一方面$\mathbb{E}\left[\delta(T)h(T) \right]=0 $,即
	\[
	\int_a^{1} \delta(t)h(t)f(t)d t+\int_1^{+\infty} \delta(t)h(t)f(t) d t=0
	\]
	也就是
	\[
	\int_1^{+\infty} \delta(t)h(t)f(t) d t=0
	\]
	为了满足这个要求,待定$h(T)=c,T>1$\\
	又需要无偏性,即
	\[
	\mathbb{E}[h(T)]=a
	\]
	待定
	\[
	h(t)=
	\begin{cases}
		bt+d  &  a\le t \le 1\\
		c  &  t>1
	\end{cases}
	\]
	对积分逐项计算得到
	\[
	b=1\quad d=-\frac{1}{n}\quad c=-1
	\]
	最后
	\[
	h(t)=
	\begin{cases}
		t-\frac{1}{n}  &  a\le t \le 1\\
		-1  &  t>1
	\end{cases}
	\]\\
	
	
	
	
	
	
	
	\noindent 附表:上分位数
	$$
	u_{0.025}=1.960, \quad u_{0.05}=1.645, \quad \chi_{899}^2(0.025)=984, \quad \chi_{899}^2(0.975)=817.8
	$$
	\newpage
	\section{\centering 23数理统计期中残卷}
	\noindent 1.设从总体
	$$
	\begin{tabular}{c|cccc}
		$X$ & 0 & 1 & 2 & 3 \\
		\hline $\mathbb{P}$ & $3 \theta / 4$ & $\theta / 4$ & $2 \theta$ & $1-3 \theta$
	\end{tabular}
	$$
	(其中 $0<\theta<1 / 3$ )中抽取的一个简单样本 $X_1, \ldots, X_n$ ,试\\
	(1)将样本分布表示为指数族自然形式,指出自然参数及自然参数空间。\\
	(2)求 $\theta$ 的最大似然估计量 $\hat{\theta}$ ,其是否为一致最小方差无偏估计?\\
	(3)证明 $\hat{\theta}$ 具有相合性和渐近正态性。\\
	解:(1)样本联合密度为
	\[
	f\left(x_1 \cdots, x_n ; \theta\right)=\left(\frac{3 \theta}{4}\right)^{n_0}\left(\frac{\theta}{4}\right)^{n_1} \cdot(2 \theta)^{n_2} \cdot(1-3 \theta)^{n_3}
	\]
	这里$n_i$为$n$个样本中取值为$i$的个数,把样本联合密度写成指数族的形式为
	\[
	f\left(x_1, \cdots, x_n ; \theta\right)=\left(\frac{3}{4}\right)^{n_0}\left(\frac{1}{4}\right)^{n_1} \cdot 2^{n_2} \cdot e^{\left(n-n_3\right) \ln \theta+n_3 \ln (1-3 \theta)}
	\]
	其中
	\[
	\begin{aligned}
		& h(\boldsymbol{X})=\left(\frac{3}{4}\right)^{n_0}\left(\frac{1}{4}\right)^{n_1} 2^{n_2} \\
		& C(\theta)=e^{n \ln \theta}
	\end{aligned}
	\]
	自然参数为$\varphi=\ln \frac{1-3\theta}{\theta}$\\
	又$0<\theta<\frac{1}{3}$,则自然参数空间为$\mathbb{R}$\\
	由因子分解定理和自然参数空间有内点,充分完全统计量为$T(\boldsymbol{X})=n_3$\\
	(2)
	\[
	\ln f\left(x_1, \cdots, x_n ; \theta\right) \propto \left(n-n_3\right) \ln \theta+n_3 \ln (1-3 \theta)
	\]
	令
	\[
	\frac{\partial \ln f}{\partial \theta}=\frac{n-n_3}{\theta}-\frac{3 n_3}{1-3 \theta}=0
	\]
	解得
	\[
	\hat{\theta}_{MLE}=\frac{n-n_3}{3n}
	\]
	又$n_3\sim \mathrm{B}(n,1-3\theta)$
	\[
	\mathbb{E}[n_3]=n(1-3\theta) \quad \mathbb{E}\left[\hat{\theta}_{MLE} \right] =\theta
	\]
	也就是MLE为无偏估计量,且它是充分完全统计量的函数,则其为UMVUE\\
	(3)相合性:\\
	强相合:注意
	\[
	n_3\sim \mathrm{B}(n,1-3\theta)
	\]
	为$n$个$\mathrm{B}(1,1-3\theta)$的独立和\\
	由SLLN
	\[
	\frac{n_3}{n} \xrightarrow{a.s.}1-3\theta
	\]
	则
	\[
	\frac{n-n_3}{3n} \xrightarrow{a.s.}\theta
	\]
	即强相合\\
	弱相合:\\
	i.直接由强相合得到\\
	ii.由Chebyshev不等式
	\[
	P\left(\mid \hat{\theta}_{MLE}-\theta \mid \ge \epsilon \right)\le \frac{\operatorname{Var}\left( \hat{\theta}_{MLE}\right) }{\epsilon^2} 
	\]
	另一方面
	\[
	\begin{aligned}
		\operatorname{Var}(\hat{\theta}_{MLE})&=\operatorname{Var}\left(\frac{n-n_3}{3 n}\right)  &=\operatorname{Var}\left(\frac{n_3}{3 n}\right)\\
		&=\frac{1}{9 n^2}\operatorname{Var}\left(n_3\right)&=\frac{\theta(1-3 \theta)}{3 n}
	\end{aligned}
	\]
	则
	\[
	\lim _{n \rightarrow \infty} P(|\hat{\theta}_{MLE}-\theta| \geq \varepsilon) \leqslant \frac{\theta(1-3 \theta)}{3 n \varepsilon^2} \rightarrow 0
	\]
	就说明了弱收敛\\
	渐进正态性:\\
	i.MLE的渐近正态性
	\[
	\sqrt{n}\left(\hat{\theta}_{MLE}-\theta \right)\xrightarrow{D} N\left(0,\frac{1}{I(\theta)} \right)  
	\]
	这里总体(单个样本)的Fisher信息量为
	\[
	I(\theta)=-\mathbb{E}[\frac{\partial^2 \ln f(x, \theta)}{\partial \theta^2}]
	\]
	其中$f(x, \theta)$为总体的密度
	\[
	\mathbb{E}[\frac{\partial^2 \ln f(x, \theta)}{\partial \theta^2}]=\ln ^{\prime \prime} f(x=0, \theta) P(X=0)+\cdots+\ln ^{\prime \prime} f(x=3, \theta) P(X=3)
	\]
	得到
	\[
	I(\theta)=\frac{3}{\theta}+\frac{9}{1-3\theta}
	\]
	则
	\[
	\sqrt{n}\left(\hat{\theta}_{MLE}-\theta \right)\xrightarrow{D} N\left(0,\frac{\theta(1-3\theta)}{3} \right) 
	\]
	ii.CLT
	\[
	\sqrt{n}\left(\frac{n_3}{n}-(1-3 \theta)\right) \xrightarrow{D} N(0,(1-3 \theta) 3 \theta) 
	\]
	则
	\[
	\sqrt{n}\left(\hat{\theta}_{MLE}-\theta \right)\xrightarrow{D} N\left(0,\frac{\theta(1-3\theta)}{3} \right) 
	\]
	
	
	\noindent 附表:$u_{0.025}=1.960, u_{0.05}=1.645, \Phi(1.36)=0.9131, \Phi(1.4)=0.9192$.
	\newpage
	\section{\centering 21数理统计期中}
	\noindent 一、(20分)假设 $X_1, X_2, \cdots, X_n$ 为来自Poisson总体 $\mathrm{P}(\lambda)$ 的一组简单样本,试\\
	(1)将抽样分布表示为指数族的自然形式,并给出自然参数空间;\\
	(2)证明统计量 $T=\sum_{i=1}^n X_i$ 为充分完备统计量。\\
	(3)证明 $g(\lambda)=e^{-\lambda}$ 不存在可以达到 C-R不等式下界的无偏估计.\\
	解:(1)样本联合密度
	\[
	f\left(x_1, \cdots, x_n ; \lambda\right)=e^{-n \lambda} \frac{\lambda^{\sum\limits_{i=1}^n x_i}}{x_{1}!\cdots x_{n}!}
	\]
	写成指数族的自然形式为
	\[
	f\left(x_1, \cdots, x_n ; \lambda\right)=e^{-n \lambda} \cdot \frac{1}{x_{1}!\cdots x_{n}!} \cdot e^{\sum\limits_{i=1}^{n} x_i \ln \lambda}
	\]
	这里
	\[
	C(\theta)=e^{-n \lambda}\quad h(\boldsymbol{X})=\frac{1}{x_{1}!\cdots x_{n}!}
	\]
	自然参数为$\ln \lambda$,又$\lambda \in (0,+\infty)$\\
	则自然参数空间为
	\[
	\Theta=\{\ln \lambda:\ln \lambda \in \mathbb{R}\}
	\]
	(2)由因子分解定理和自然参数空间有内点,则$T(\boldsymbol{X})=\sum\limits_{i=1}^n X_i$为充分完全统计量\\
	(3)无偏估计能达到C-R下界$\iff$分布为单参数指数族且UMVUE为充分完全统计量$T(\boldsymbol{X})$的线性函数\\
	先求UMVUE\\
	解一:注意到
	\[
	e^{-\lambda}=P(X_1=0)
	\]
	再仿照24期中3(1)可以得到UMVUE
	\[
	\hat{e^{-\lambda}}=\hat{g}(T)=\left(\frac{n-1}{n} \right)^T 
	\]
	解二:首先
	\[
	T(\boldsymbol{X})\sim \mathrm{P}(n\lambda)
	\]
	设UMVUE为$\hat{g}(T)$,则
	\[
	\mathbb{E}\left[\hat{g}(T) \right] =\sum_{k=0}^{\infty} \hat{g}(k)\frac{(nk)^k}{k!}\cdot e^{-n\lambda}=e^{-\lambda}
	\]
	也就是
	\[
	\sum_{k=0}^{\infty} \hat{g}(k) \frac{(n\lambda)^k}{k!}=e^{(n-1) \lambda}=\sum_{k=0}^{\infty} \frac{(n-1)^k \lambda^k}{k!}
	\]
	就能得到
	\[
	\hat{g}(T)=\left(\frac{n-1}{n} \right)^T 
	\]
	显然UMVUE不是充分完全统计量的线性函数,因此不能达到C-R下界\\
	数值验证:\\
	下面求总体(单个样本)的Fisher信息量
	\[
	I(\lambda)=-\mathbb{E}\left[\frac{\partial \ln f(x ; \lambda)^2}{\partial^2 \lambda}\right]=-\mathbb{E}\left[-\frac{X}{\lambda^2} \right] =\frac{1}{\lambda}
	\]
	这里$X$为总体分布,且密度为
	\[
	f(x ; \lambda)=\frac{\lambda^x}{x!}e^{-\lambda}
	\]
	因此C-R下界为
	\[
	\frac{\left( \frac{\partial\left(e^{-\lambda}\right)}{\partial \lambda}\right)^2}{n I(\lambda)}=\frac{\lambda e^{-2 \lambda}}{n} 
	\]
	再计算UMVUE的方差
	\[
	\begin{aligned}
		\operatorname{Var}(\hat{g}(T) ) & =\mathbb{E}\left[\left(\frac{n-1}{n}\right)^{2 T}\right]-(\mathbb{E}[\hat{g}(T)])^2 \\
		& =\mathbb{E}\left[\left(\frac{n-1}{n}\right)^{2 T}\right]-e^{-2 \lambda}
	\end{aligned}
	\]
	其中
	\[
	\begin{aligned}
		E\left[\left(\frac{n-1}{n}\right)^{2 T}\right] & =\sum_{k=0}^{\infty}\left(\frac{n-1}{n}\right)^{2 k} \frac{(n \lambda)^k}{k!} e^{-n \lambda} \\
		& =e^{-n \lambda} \sum_{k=0}^{\infty} \frac{\left( \frac{(n-1)^2 \lambda}{ n}\right) ^k}{k!}\\
		& =e^{-n \lambda} \cdot e^{\frac{(n-1)^2 \lambda}{n}} \\
		& =e^{-2 \lambda+\frac{\lambda}{n}}
	\end{aligned}
	\]
	最后
	\[
	\operatorname{Var}(\hat{g}(T))=e^{-2 \lambda+\frac{\lambda}{n}}-e^{-2 \lambda}
	\]
	不能达到C-R下界\\
	
	
	\noindent 二.(30分)设 $X_1, \ldots, X_n$ 为来自0-1分布 $\mathrm{B}(1, p), 0<p<1$ 的一组简单样本,试\\
	(1)求 $g(p)=(1-p)^2$ 的矩估计量和极大似然估计量,并说明是否为无偏估计。\\
	(2)求 $g(p)$ 的UMVUE,其方差是否达到C-R不等式的下界?\\
	(3)证明 $g(p)$ 的极大似然估计量具有渐近正态性.\\
	解:(1)由于$\mathbb{E}[X]=p$,矩估计量为
	\[
	\hat{g({p})}_M=(1-\overline{X})^2
	\]
	又样本联合密度
	\[
	f\left(x_1, \cdots, x_n ; p\right)=p^{\sum_{i=1}^n x_i}(1-p)^{n-\sum_{i=1}^n x_i}
	\]
	求导有
	\[
	\frac{\partial \ln f}{\partial p}=\frac{\sum_{i=1}^{n} x_i}{p}-\frac{n-\sum_{i=1}^n x_i}{1-p}=0 .
	\]
	解得
	\[
	\hat{p}_{MLE}=\overline{X}
	\]
	再由MLE的不变性,有
	\[
	\hat{g(p)}_{MLE}=\left(1-\overline{X} \right)^2 
	\]
	无偏性:
	\[
	\begin{aligned}
		\mathbb{E}\left[(1-\overline{X})^2\right] & =\mathbb{E}\left[1-2 \overline{X}+\overline{X}^2\right] \\
		& =1-2 \mathbb{E}[\overline{X}]+\frac{1}{n^2} \mathbb{E}\left[\left(\sum_{i=1}^n X_i\right)^2 \right] \\
		& =1-2 p+p^2+\frac{p(1-p)}{n} \\
		& \neq(1-p)^2
	\end{aligned}
	\]
	这就说明两个估计均不无偏\\
	(2)注意二项分布为指数族,且自然参数空间有内点,易知 $T(\boldsymbol{X})=\sum\limits_{i=1}^n X_i$ 为充分完全统计量\\
	先找一个无偏估计
	\[
	\mathbb{I}_{\left\{X_1=0, X_2=0\right\}}
	\]
	则UMVUE为
	\[
	\begin{aligned}
		\hat{g}(p)_{UMVUE} &=\mathbb{E}\left[\mathbb{I}_{\left\{X_1=0, X_2=0\right\}} \mid \sum_{i=1}^n x_i=t\right] \\
		& =P\left(X_1=0, X_2=0 \mid \sum_{i=1}^n X_i=t \right) \\
		& =\frac{P\left(X_1=0, X_2=0, \sum_{i=3}^n X_i=t\right)}{P\left(\sum_{i=1}^n X_i=t\right)}\\
		&=\frac{(n-t)(n-t-1)}{n(n-1) }
	\end{aligned}
	\]
	也就是
	\[
	\hat{g}(p)_{UMVUE}=\frac{(n-T)(n-T-1)}{n(n-1) }
	\]
	它不是$T(\boldsymbol{X})=\sum\limits_{i=1}^n X_i$的线性函数,因此不能达到C-R下界\\
	数值验证:先求总体(单个样本)的Fisher信息量
	\[
	\begin{aligned}
		I\left(p\right) & =-\mathbb{E}\left[\frac{\partial^2}{\partial p^2} \ln f\left(x ; p\right)\right] \\
		& =\frac{1}{p(1-p)} 
	\end{aligned}
	\]
	这里总体的密度为
	\[
	f(x ; p)=p^x(1-p)^{1-x}
	\]
	因此C-R下界为
	\[
	\frac{\frac{\partial g(p)}{\partial p}}{n I(p)}=\frac{4 p(1-p)^3}{n}
	\]
	又
	\[
	\operatorname{Var}(\hat{g}(p)_{UMVUE} )=\operatorname{Var}\left(\frac{(n-T)(n-T-1)}{n(n-1)}\right) =\frac{\operatorname{Var}\left(n^2-n+(1-2n)T+T^2\right)}{n^2(n-1)^2}
	\]
	其中
	\begin{align*}
		\operatorname{Var}\left(n^2-n+(1-2n)T+T^2\right) & =\operatorname{Var}\left((1-2n)T+T^2\right) \\
		& =\mathbb{E}\left[\left[(1-2 n) T+T^2\right]^2\right]-\left(\mathbb{E}\left[(1-2 n) T+T^2\right]\right)^2
	\end{align*}
	这里由于$T\sim \mathrm{B}(n,p)$
	\begin{align*}
		\mathbb{E}\left[\left[(1-2 n) T+T^2\right]^2\right] & =\mathbb{E}\left[(1-2n)^2 T^2 +(2-4n)T^3+T^4 \right]  \\
		& =\mathbb{E}\left[(1-2n)^2 T^2 \right] +\mathbb{E}\left[(2-4n)T^3 \right] +\mathbb{E}\left[T^4 \right] \\
		&=(1-2n)^2 np(1-p+np)+(2-4n)(1-3p+3np+(2-3n+n^2)p^2)\\
		&+np(1-p)(1-6p+6p^2)+7np(1-p)(n-1)p+6n(n-1)(n-2)p^3+n^3p^4\\
	\end{align*}
	另一方面
	\begin{align*}
		\mathbb{E}\left[(1-2 n) T+T^2\right] & =(1-2n)p+np(1-p+np) 
	\end{align*}
	这就说明了UMVUE不能达到C-R下界\\
	(3)解一:直接利用MLE的渐进正态性
	\[
	\sqrt{n}\left(\hat{p}_{MLE}-p \right)\xrightarrow{D} N \left(0,\frac{1}{I(p)} \right) 
	\]
	也就是
	\[
	\sqrt{n}\left(\hat{p}_{MLE}-p \right)\xrightarrow{D} N \left(0,p(1-p) \right) 
	\]
	再使用$\Delta$方法
	\[
	\sqrt{n}\left(\hat{g(p)}_{MLE}-g(p) \right)\xrightarrow{D} N \left(0,4p(1-p)^3 \right) 
	\]
	解二:$\overline{X}$有CLT
	\[
	\sqrt{n}\left(\overline{X}-p \right)\xrightarrow{D} N \left(0,p(1-p) \right) 
	\]
	再使用$\Delta$方法
	\[
	\sqrt{n}\left(\hat{g(p)}_{MLE}-g(p) \right)\xrightarrow{D} N \left(0,4p(1-p)^3 \right) 
	\]\\
	
	
	
	
	\noindent 三.(15分)设 $X_1, \ldots, X_n$ 来自正态总体 $N(\mu, 1)$ 的一组简单随机样本,试 $(0<\alpha<1)$\\
	(1)证明样本平均值 $\overline{X}$ 与统计量 $X_1-\overline{X}$ 相互独立.\\
	(2)求概率 $P\left(X_1 \leq 0\right)$ 的UMVUE以及其置信水平为 $1-\alpha$ 的置信区间.\\
	解:(1)注意方差已知,由 $N(\mu,1)$ 为指数族,容易得到$T(\boldsymbol{X})=\overline{X}$为充分完全统计量\\
	设 
	$$Y_i=X_i-\mu \sim N(0,1)$$
	则 $$X_1-\overline{X}=\left(X_1-\mu\right)-\frac{\sum\limits_{i=1}^n \left(X_i-\mu\right)}{n}=Y_1-\overline{Y}$$
	分布与 $\mu$ 无关,是辅助量,由Basu定理知二者独立\\
	(2)显然$\mathbb{I}_{\{X_1\le 0\}}$就是无偏估计,设要求的UMVUE为$h(T)$
	\[
	\begin{aligned}
		h(T) & =\mathbb{E}\left[\mathbb{I}_{\{X_1\le 0\}} \mid T(X)\right] \\
		& =P\left(X_1 \leqslant 0 \mid T(X)\right) \\
		&=P\left(X_1-\overline{X} \leqslant -\overline{X} \mid \overline{X}\right) \\
		&\stackrel{\text{独立性}}{=} P\left(X_1-\overline{X} \leqslant-\overline{X}\right) 
	\end{aligned}
	\]
	注意$Y_1$与$\overline{Y}$不独立,有
	$$X_1-\overline{X}\sim Y_1-\overline{Y}\sim N\left(1,1-\frac{1}{n} \right) $$
	则
	\[
	h(T)=\Phi\left(-\frac{\overline{X}}{\sqrt{1-\frac{1}{n}}} \right) 
	\]
	这里$\Phi(x)$为$N(0,1)$的分布函数,又
	\[
	\overline{X} \sim N\left(\mu, \frac{1}{n}\right) \quad \sqrt{n}(\overline{X}-\mu) \sim N(0,1) 
	\]
	则
	\[
	P(X_1\le 0)=\Phi(-\mu)
	\]
	即$P(X_1\le 0)$关于$\mu$单调递减,只需要求$\mu$的置信区间即可
	另外$\mu$有置信系数为$1-\alpha$的置信区间
	\[
	\left[\overline{X}-\frac{u_{\frac{\alpha}{2}}}{\sqrt{n}},\overline{X}+\frac{u_{\frac{\alpha}{2}}}{\sqrt{n}} \right] 
	\]
	则$P(X_1\le 0)$有置信区间
	\[
	\left[\Phi\left(-\left(\overline{X}+\frac{u_{\frac{\alpha}{2}}}{\sqrt{n}}\right)\right), \Phi\left(-\left(\overline{X}-\frac{u_{\frac{\alpha}{2}}}{\sqrt{n}}\right)\right)\right]
	\]\\
	
	
	
	
	\noindent 四.(15分)设 $X_1, X_2, \cdots, X_n$ 为来自如下分布的一组简单样本
	$$
	\begin{tabular}{|c|ccc|}
		\hline$X$ & 0 & 1 & 2 \\
		\hline$P$ & $\frac{1}{2}\left[(1-\theta)^2+\theta^2\right]$ & $2 \theta(1-\theta)$ & $\frac{1}{2}\left[(1-\theta)^2+\theta^2\right]$ \\
		\hline
	\end{tabular}
	$$
	其中 $0<\theta<1 / 2$ 为参数。试利用重参数化方法和极大似然估计的不变性\\
	(1)求 $\theta$ 的极大似然估计量,并求其渐近分布.\\
	(2)由此给出 $\theta$ 的一个(渐近)置信水平为 $1-\alpha$ 的置信区间 $(0<\alpha<1)$ .\\
	解:(1)重参数化
	\[
	P(X=1)=P(X=3) \triangleq \mu=\frac{1}{2}\left[(1-\theta)^2+\theta^2\right] 
	\]
	则
	\[
	P(X=2)=1-2\mu
	\]
	同时令$n_0,n_1,n_2$为$n$个样本中取值为0,1,2的样本个数,有$n_0+n_1+n_2=n$,则样本联合密度为
	\[
	f\left(x_1, \cdots, x_n ; \mu\right)=\mu^{n_0}(1-2 \mu)^{n_1} \mu^{n_2}
	\]
	进而
	\[
	\ln f\left(x_1, \cdots, x_n ; \mu\right)=n_0 \ln \mu+n_1 \ln (1-2 \mu)+n_2 \ln \mu
	\]
	\[
	\frac{\partial  \ln f}{\partial \mu}=\frac{n_0}{\mu}+\frac{n_2}{\mu}-\frac{2 n_1}{1-2 \mu}=0
	\]
	得到
	\[
	\hat{\mu}_{M L E}=\frac{n_0+n_2}{2 n}
	\]
	注意$\theta<\frac{1}{2}$,反解$\theta$有
	\[
	\hat{\mu}_{MLE}=\frac{1}{2}\left[(1-\hat{\theta}_{MLE})^2+\hat{\theta}_{MLE}\right] 
	\]
	得到
	\[
	\hat{\theta}_{MLE}=\frac{1-\sqrt{1-\frac{2 n_1}{n}}}{2}
	\]
	由MLE的渐进正态性
	\[
	\sqrt{n}\left(\hat{\theta}_{M L E}-\theta\right) \xrightarrow{D} N\left( 0,\frac{1}{I(\theta)} \right) 
	\]
	这里总体(单个样本)的Fisher信息量为
	\[
	\begin{aligned}
		I(\theta)= & -\mathbb{E}\left[\ln ^{\prime \prime} f(x ; \theta)\right] . \\
		= & -\left[P(x=0) \cdot \ln ^{\prime \prime} f(x=0 ; \theta)+P(x=1) \cdot \ln ^{\prime \prime} f(x=1 ; \theta)\right. \\
		& \left.+P(x=2) \cdot \ln ^{\prime \prime} f\left(x=2 ; \theta\right)\right]
	\end{aligned}
	\]
	对$X=0$或2
	\[
	f(x ; \theta)=\frac{(1-\theta)^2+\theta^2}{2}
	\]
	进而
	\[
	\frac{\partial^2}{\partial \theta^2} \ln f=\frac{4\left(2 \theta^2-2 \theta+1\right)-2(2 \theta-1)(4 \theta-2)}{\left(2 \theta^2-2 \theta+1\right)^2}
	\]
	而对$X=1$
	\[
	\frac{\partial^2}{\partial \theta^2} \ln f(x=1 ; \theta)=-\frac{1}{\theta^2}-\frac{1}{(1-\theta)^2}
	\]
	则
	\[
	I(\theta)=\frac{2}{2 \theta^2-2 \theta+1}+\frac{1}{\theta(1-\theta)}
	\]
	就有
	\[
	\sqrt{n}\left(\hat{\theta}_{M L E}-\theta\right) \xrightarrow{D} N\left( 0,\frac{1}{\frac{2}{2 \theta^2-2 \theta+1}+\frac{1}{\theta(1-\theta)}} \right) 
	\]
	(2)置信区间为
	\[
	\left[\hat{\theta}_{M L E}-\frac{u_{\frac{\alpha}{2}}}{\sqrt{n I(\theta)}}, \hat{\theta}_{M L E}+\frac{{u_{\frac{\alpha}{2}}}}{\sqrt{n I(\theta)}}\right]
	\]\\
	
	
	
	\noindent 五.(20分)设 $X_1, \ldots, X_n, i . i . d \sim U(0, \theta)$ ,其中 $\theta>1$ 为未知参数.记 $X_{(n)}=\max\limits_{1 \leq i \leq n} X_i$ ,试\\
	(1)证明 $X_{(n)}$ 是充分但不完备的统计量.\\
	(2)求 $\theta$ 的UMVUE.\\
	解:(1)样本联合密度为
	\[
	f\left(x_1, \cdots, x_n ; \theta\right)=\frac{1}{\theta^n} \mathbb{I}_{\left\{X_{(n)} < \theta\right\}}
	\]
	取 $ g(T(\boldsymbol{X}) ; \theta)=\theta^{-n} \cdot I_{\left\{x_{(n)}<\theta\right\}}, h(\boldsymbol{X})=1$,由因子分解定理知道$T(\boldsymbol{X})=X_{(n)}$为充分统计量\\
	注意$\theta>1$,为了证明 $T(\boldsymbol{X})=X_{(n)}$ 不是 $\theta$ 的完全统计量.只需寻找 $t$ 的某一实函数 $\varphi(t)$ ,满足 $\mathbb{E}_\theta[\varphi(T)]=0$ ,即
	$$
	\int_0^\theta \varphi(t) t^{n-1} \mathrm{~d} t=0, \quad \theta>1
	$$
	但 $\mathbb{P}_\theta\left(\varphi\left(X_{(n)}\right)=0\right)<1$ \\
	注意到上式对 $\theta$ 求导后只能得到 $\varphi(t)=0, t>1$ .因此我们只要构造合适的 $\varphi(t), t \leq 1$ ,使得
	$$
	\int_0^1 \varphi(t) t^{n-1} \mathrm{~d} t=0
	$$
	即可,例如 
	\[
	\varphi(t)=
	\begin{cases}
		t^{1-n}  &  t \leq \frac{1}{2} \\
		-t^{1-n}  &  \frac{1}{2}<t\le 1\\
		0 & 1<t\le \theta
	\end{cases}
	\]
	(2)统计量只充分不完全,考虑零无偏法\\
	$X_{(n)}$ 有密度  
	\[
	g(t)=n \theta^{-n} t^{n-1} \mathbb{I}_{(0, \theta)}(t) 
	\]
	设$\mathbb{E}[\delta(t)]=0$,即
	\[
	\int_0^\theta \delta(t) n \theta^{-n} t^{n-1} d t=0 
	\]
	也就是
	\[
	\int_0^\theta \delta(t)  t^{n-1} d t=0 
	\]
	进而
	\[
	\int_0^1 \delta(t) t^{n-1} d t+\int_1^\theta \delta(t) t^{n-1} d t=0
	\]
	求导有
	\[
	\delta(t) \equiv 0 \quad \forall t>1
	\]
	设要求的UMVUE为$h(T)$,它要满足以下条件
	\[
	\mathbb{E}[h(T) \delta(T)]=0
	\]
	\[
	\int_0^\theta \delta(t) h(t) n\theta^{-n} t^{n-1} d t=0
	\]
	注意$\delta(t) \equiv 0 \quad \forall t>1$,上式也就说明
	\[
	\int_0^1 \delta(t) h(t) n\theta^{-n} t^{n-1} d t=0
	\]
	又$ h(T)$ 无偏,即  $\mathbb{E}[h(T)]=\theta$ 
	\[
	\theta=\int_{0}^{\theta} h(t) n \theta^{-n} t^{n-1} d t
	\]
	待定$h(T)=c,0\le T<1$,有
	\[
	c \int_0^1 \frac{n t^{n-1}}{\theta^n} d t=c \theta^{-n}
	\]
	想要$$\int_1^\theta h(t) \frac{n t^{n-1}}{\theta^n} d t $$ 含有 $ \theta^{-n} $ 及 $ \theta $ 项, 
	待定剩下部分为
	\[
	h(T)=
	\begin{cases}
		c  &  0\le T<1 \\
		bT+d  &  T\ge 1
	\end{cases}
	\]
	代入积分的值则有
	\[
	c=1\quad b=
	\frac{n+1}{n} \quad d=0
	\]
	最后
	\[
	h(T)=
	\begin{cases}
		1  &  0\le T<1 \\
		\frac{n+1}{n}T  &  T\ge 1
	\end{cases}
	\]
	\newpage
	\chapter{\centering 往年期末试卷}
	\section{\centering 25数理统计期末}
	\noindent 一、填空及判断题(15分)\\
	1.设$X_1, \cdots, X_n, X_{n+1} \stackrel{\text{i.i.d.}}{\sim} N(\mu, \sigma^2)$,若
	\[
	\frac{\overline{X_n}-X_{n+1}}{\sqrt{b_n \sum_{i=1}^n X_i^2 - c_n (\overline{X_n})^2}} \sim t_{n-1}
	\]
	其中$\overline{X_n}=\frac{1}{n}\sum_{i=1}^n X_i$,则$b_n=$\underline{\hspace{2cm}},$c_n=$\underline{\hspace{2cm}}。\\
	答案:$b_n=\frac{n+1}{n(n-1)} \quad c_n=\frac{n+1}{n-1}$\\
	先将分子归一化为$N(0,1)$
	\[
	\overline{X_n}\sim N\left( \mu,\frac{\sigma^2}{n}\right)  \quad \overline{X_n}-X_{n+1}\sim N\left( 0,\frac{(n+1)\sigma^2}{n}\right) 
	\]
	则
	\[
	\frac{\overline{X_n}-X_{n+1}}{\sqrt{\frac{(n+1)\sigma^2}{n}}} \sim N(0,1)
	\]
	再处理分母
	\[
	\frac{\sum\limits_{i=1}^{n} \left( X_i-\overline{X_n}\right)^2}{\sigma^2}\sim \chi^2(n-1)
	\]
	其中
	\begin{align*}
		\sum\limits_{i=1}^{n} \left( X_i-\overline{X_n}\right)^2 &=\sum\limits_{i=1}^{n}X_i^2-2\overline{X_n}\sum\limits_{i=1}^{n}X_i+n\overline{X_n}^2  \\
		& =\sum\limits_{i=1}^{n}X_i^2-n\overline{X_n}^2
	\end{align*}
	则
	\[
	\frac{\sum\limits_{i=1}^{n}X_i^2-n\overline{X_n}^2}{(n-1)\sigma^2}\sim \frac{\chi^2(n-1)}{n-1}
	\]
	也就是
	\[
	\frac{\frac{\overline{X_n}-X_{n+1}}{\sqrt{\frac{(n+1)\sigma^2}{n}}}}{\sqrt{\frac{\sum\limits_{i=1}^{n}X_i^2-n\overline{X_n}^2}{(n-1)\sigma^2}}}\sim t_{n-1}
	\]
	其中分子分母的独立性来自正态总体的$\overline{X_n}$与$S^2$独立\\
	化简得到
	\[
	\frac{\overline{X_n}-X_{n+1}}{\sqrt{\frac{n+1}{n(n-1)} \sum_{i=1}^n X_i^2 - \frac{n+1}{n-1} (\overline{X_n})^2}}\sim t_{n-1}
	\]
	就有
	\[
	b_n=\frac{n+1}{n(n-1)} \quad c_n=\frac{n+1}{n-1}
	\]
	2.设$X \sim N(\mu, 2)$,求$\mu$的95\%置信区间长度小于1所需的最小样本量\underline{\hspace{2cm}}。\\
	答案:$n \geq 31$\\
	已知方差,则置信区间为
	\[
	\left[\overline{X}-\sqrt{\frac{2}{n}}u_{\alpha/2},\overline{X}+\sqrt{\frac{2}{n}}u_{\alpha/2} \right] 
	\]
	代入数值即得\\
	3.$\times$\\
	4.$\times$\\
	5.$\times$\\
	6.$\checkmark$;$\checkmark$;$\times$\\
	7.样本量小导致有些格子计数小于5\\
	
	
	
	\noindent 二、(15 分)证明来自总体分布$U(\theta, 2\theta)$的$n$个样本$(X_1,...,X_n)$的统计量$(X_{(1)}, X_{(n)})$是极小充分统计量但不是完全统计量。\\
	解:均匀总体下的充分完全统计量.\\
	通过因子分解定理说明充分统计量:根据题意有样本联合密度
	$$
	f(\boldsymbol{x} ; \theta)=\frac{\mathbb{I}_{\left\{\theta<x_i<2 \theta, i=1,2, \ldots, n\right\}}}{\theta^n}=\theta^{-n} \mathbb{I}_{\left\{x_{(n)} / 2<\theta<x_{(1)}\right\}} .
	$$
	因此由因子分解定理知 $T=\left( X_{(1)}, X_{(n)}\right) $ 是 $\theta$ 的充分统计量.\\
	通过定理 2.6.2 说明极小充分统计量:对任意 $\boldsymbol{x}$ 和 $\boldsymbol{y}$ ,
	$$
	\frac{f(\boldsymbol{x} ; \theta)}{f(\boldsymbol{y} ; \theta)}=\frac{\mathbb{I}_{\left\{x_{(n)} / 2<\theta<x_{(1)}\right\}}}{\mathbb{I}_{\left\{y_{(n)} / 2<\theta<y_{(1)}\right\}}}
	$$
	要使得上式与 $\theta$ 无关,当且仅当 $$\left(x_{(1)}, x_{(n)}\right)=\left(y_{(1)}, y_{(n)}\right)$$
	因此由定理2.6.2知 $T=\left(X_{(1)}, X_{(n)}\right)$ 是 $\theta$的极小充分统计量.\\
	构造函数或利用辅助统计量说明不完全性:\\
	法一:注意到
	$$
	\mathbb{E}_\theta\left[X_{(1)}\right]=\theta+\frac{1}{n+1} \theta=\frac{n+2}{n+1} \theta, \quad \mathbb{E}_\theta\left[X_{(n)}\right]=\theta+\frac{n}{n+1} \theta=\frac{2 n+1}{n+1} \theta,
	$$
	上式来自$n$个独立同分布的$\mathrm{U}(0,1)$的次序统计量满足
	\[
	U_{(1)}\sim \beta(1,n) \quad U_{(n)}\sim \beta(n,1)
	\]
	取 $g\left(x_{(1)}, x_{(n)}\right)=(2 n+1) x_{(1)}-(n+2) x_{(n)}$ ,则
	$$
	\mathbb{E}_\theta\left[g\left(X_{(1)}, X_{(n)}\right)\right]=0, \quad \forall \theta>0,
	$$
	但函数 $g$ 并不几乎处处为零。因此 $\left(X_{(1)}, X_{(n)}\right)$ 不是 $\theta$ 的完全统计量.\\
	法二:注意到 $X_i / \theta \sim \mathrm{U}(1,2)$ ,构造辅助统计量
	$$
	T=\frac{X_{(n)}}{X_{(1)}}=\frac{X_{(n)} / \theta}{X_{(1)} / \theta},
	$$
	于是 $T$ 的分布与 $\theta$ 无关,是一个辅助统计量.因此 $\left(X_{(1)}, X_{(n)}\right)$ 不是 $\theta$ 的完全统计量.\\
	
	
	
	
	
	\noindent 三、(15分)设$X$的概率分布为:
	\[
	\begin{array}{c|ccc}
		X & 1 & 2 & 3 \\ \hline
		P(X) & 2\theta & 3\theta & 1-5\theta \\
	\end{array}
	\]
	取出$20$个样本,其中有$6$个取值为$1$,有$8$个取值为$2$,有$6$个取值为$3$\\
	(1) 求$\theta$的最大似然估计和矩估计,并判断该估计是否无偏;\\
	(2)求$\theta$的一致最小方差无偏估计(UMVUE),并比较其方差与C-R下界。\\
	解:(1)矩估计:根据题意有
	$$
	\mathbb{E}[X]=1 \times 2 \theta+2 \times 3 \theta+3 \times(1-5 \theta)=3-7 \theta .
	$$
	因此反解得 $\theta$ 的矩估计量为
	$$\hat{\theta}_M(\boldsymbol{X})=\frac{3-\overline{X}}{7}$$
	由 $\overline{x}=\frac{1 \times 6+2 \times 8+3 \times 6}{20}=2$ 知 $\theta$ 的矩估计值为 
	$$\hat{\theta}_M(\boldsymbol{x})=1 / 7$$ 而
	$$
	\mathbb{E}_\theta\left[\hat{\theta}_M(\boldsymbol{X})\right]=\mathbb{E}_\theta\left(\frac{3-\overline{X}}{7}\right)=\frac{3-\mathbb{E}(\overline{X})}{7}=\theta .
	$$
	所以矩估计 $\hat{\theta}_M(\boldsymbol{X})$ 是 $\theta$ 的无偏估计.\\
	最大似然估计:记 $n_j=\sum_{i=1}^n \mathbb{I}_{\left(X_i=j\right)}, j=1,2,3$ 。由分布列知 $\theta$ 的似然函数为
	$$
	L(\theta)=(2 \theta)^{n_1}(3 \theta)^{n_2}(1-5 \theta)^{n_3}=2^{n_1} 3^{n_2} \theta^{n-n_3}(1-5 \theta)^{n_3} .
	$$
	由对数似然方程
	$$
	\frac{\partial \log L(\theta)}{\partial \theta}=\frac{n-n_3}{\theta}-\frac{5 n_3}{1-5 \theta}=0 \Rightarrow \theta=\frac{n-n_3}{5 n} .
	$$
	经微分检验其确为似然函数的最大值点,因此 $\theta$ 的最大似然估计为 $$\hat{\theta}_L(\boldsymbol{X})=\frac{1}{5 n} \sum_{i=1}^n \mathbb{I}_{\left(X_i \neq 3\right)}$$
	由样本观测值知其估计值为 $$\hat{\theta}_L(\boldsymbol{x})=0.14$$
	而
	$$
	\mathbb{E}_\theta\left[\hat{\theta}_L(\boldsymbol{X})\right]=\mathbb{E}_\theta\left(\frac{1}{5 n} \sum_{i=1}^n \mathbb{I}_{\left(X_i \neq 3\right)}\right) =\frac{1}{5 n} \sum_{i=1}^n(5 \theta)=\theta .
	$$
	所以最大似然估计 $\hat{\theta}_L(\boldsymbol{X})$ 也是 $\theta$ 的无偏估计.\\
	(2)一致最小方差无偏估计:样本联合密度函数为
	$$
	f(\boldsymbol{x} ; \theta)=(2 \theta)^{n_1}(3 \theta)^{n_2}(1-5 \theta)^{n_3}=2^{n_1} 3^{n_2} \theta^n \exp \left\{n_3 \log \frac{1-5 \theta}{\theta}\right\}
	$$
	自然参数空间 $\Theta^*=\{\eta:-\infty<\eta<\infty\}$ 有内点,于是 $T(\boldsymbol{X})=n_3$ 是 $\theta$ 的充分完全统计量.从而由 Lehmann-Scheffé 定理知,$\hat{\theta}_L(\boldsymbol{X})$ 作为基于 $n_3$ 的无偏估计是 $\theta$ 的 UMVUE,其方差
	$$
	\operatorname{Var}_\theta\left[\hat{\theta}_L(\boldsymbol{X})\right]=\frac{\operatorname{Var}_\theta\left(I\left(X_1 \neq 3\right)\right)}{25 n}=\frac{5 \theta(1-5 \theta)}{25 n}=\frac{\theta(1-5 \theta)}{5 n} .
	$$
	而 $n$个样本的Fisher 信息量
	$$
	I(\theta)=-\mathbb{E}_\theta\left[\frac{\partial^2 \log L(\theta)}{\partial \theta^2}\right]=\mathbb{E}_\theta\left[\frac{n-n_3}{\theta^2}-\frac{25 n_3}{(1-5 \theta)^2}\right]=\frac{5 n \theta}{\theta^2}-\frac{25 n(1-5 \theta)}{(1-5 \theta)^2}=\frac{5 n}{\theta(1-5 \theta)} .
	$$
	因而 Cramér-Rao 下界为 
	$$1 / I(\theta)=\operatorname{Var}_\theta\left[\hat{\theta}_L(\boldsymbol{X})\right]$$
	即一致最小方差无偏估计 $\hat{\theta}_L(\boldsymbol{X})$ 的方差达到 $\theta$ 的无偏估计方差的下界.\\
	
	
	
	\noindent 四、(12分)设$\left( X_1,...,X_m\right)$来自某总体分布;$\left( Y_1,...,Y_n\right)$来自另外的总体分布;
	设$\overline{X} = 70$,$\overline{Y} = 80$,样本容量$n = m = 26$,样本方差$S_1^2 = 10$,$S_2^2 = 12$。\\
	(1)假设总体服从正态分布:判断两总体方差是否相同;判断两总体均值是否相同。\\
	(2)若去掉正态性假设,检验两总体均值是否相同\\
	解:(1)\\
	i.两正态总体方差齐性检验:假设检验问题为
	$$
	H_0: \sigma_1^2=\sigma_2^2 \longleftrightarrow H_1: \sigma_1^2 \neq \sigma_2^2 .
	$$
	取检验统计量为
	$$
	F(\boldsymbol{X}, \boldsymbol{Y})=\frac{S_2^2}{S_1^2} \stackrel{H_0}{\sim} F_{n-1, m-1}
	$$
	由 $F$ 检验知水平 $\alpha=0.2$ 的检验拒绝域为
	$$
	D=\left\{(\boldsymbol{X}, \boldsymbol{Y}): F(\boldsymbol{X}, \boldsymbol{Y})>F_{25,25}(0.1) \text { 或 } F(\boldsymbol{X}, \boldsymbol{Y})<F_{25,25}(0.9)=\frac{1}{F_{25,25}(0.1)}\right\} \text {. }
	$$
	现有 
	$$F(\boldsymbol{x}, \boldsymbol{y})=s_2^2 / s_1^2=1.2 \in(1 / 1.68,1.68)$$
	所以不能拒绝原假设,可认为方差齐性.\\
	ii.两正态总体均值差的检验:假设检验问题为
	$$
	H_0: \mu_1=\mu_2 \longleftrightarrow H_1: \mu_1 \neq \mu_2 .
	$$
	由于我们认为方差齐性,所以可取检验统计量为
	$$
	T(\boldsymbol{X}, \boldsymbol{Y})=\frac{\overline{Y}-\overline{X}}{S_w \sqrt{1 / m+1 / n}} \stackrel{H_0}{\sim} t_{m+n-2}
	$$
	其中 $(m+n-2) S_w^2=(m-1) S_1^2+(n-1) S_2^2$ .由 $t$ 检验知水平 $\alpha=0.05$ 的检验拒绝域为
	$$
	D=\left\{(\boldsymbol{X}, \boldsymbol{Y}):|T(\boldsymbol{X}, \boldsymbol{Y})|>t_{50}(0.025)\right\}
	$$
	现有 
	$$T(\boldsymbol{x}, \boldsymbol{y})=\frac{80-70}{\sqrt{11} \times \sqrt{1 / 13}}=10.9>2.01$$
	所以拒绝原假设,认为两班学生成绩均值有显著差异.\\
	(2)两一般总体均值差的检验:假设检验问题仍为
	$$
	H_0: \mu_1=\mu_2 \longleftrightarrow H_1: \mu_1 \neq \mu_2 .
	$$
	此时可用大样本检验,取检验统计量为
	$$
	Z(\boldsymbol{X}, \boldsymbol{Y})=\frac{\overline{Y}-\overline{X}}{\sqrt{S_1^2 / m+S_2^2 / n}} \xrightarrow{H_0} N(0,1), \quad \text { as } m, n \rightarrow \infty .
	$$
	由 $z$ 检验知水平 $\alpha=0.05$ 的检验拒绝域为
	$$
	D=\left\{(\boldsymbol{X}, \boldsymbol{Y}):|Z(\boldsymbol{X}, \boldsymbol{Y})|>u_{0.025}\right\} .
	$$
	现有 
	$$Z(\boldsymbol{x}, \boldsymbol{y})=\frac{80-70}{\sqrt{10 / 26+12 / 26}}=10.9>1.96$$
	所以拒绝原假设,认为两班学生成绩均值有显著差异.\\
	
	
	
	
	\noindent 五、(15分)设有$n$个样本$(X_1,...,X_n)$来自$\mathrm{Beta}(2,\beta)$分布。
	检验假设:
	\[
	H_0: \beta = 1 \quad \leftrightarrow \quad H_1: \beta \neq 1
	\]
	求似然比检验,Wald检验和得分检验。\\
	解:伽马总体下的三大检验.\\
	似然比检验:由题意知似然函数
	$$
	L(\beta)=\prod_{i=1}^n \beta^2 x \exp \{-\beta x\}=\beta^{2 n} \exp \left\{-\beta \sum_{i=1}^n x_i\right\} \prod_{i=1}^n x_i, \quad \beta>0
	$$
	由对数似然方程
	$$
	\frac{\partial \log L(\beta)}{\partial \beta}=\frac{2 n}{\beta}-\sum_{i=1}^n x_i=0 \Rightarrow \beta=\frac{2 n}{\sum_{i=1}^n x_i}
	$$
	经微分检验其确为似然函数的最大值点,因此 $\beta$ 的最大似然估计为 $\hat{\beta}=2 / \overline{X}$ .于是似然比检验统计量
	$$
	\Lambda(\boldsymbol{X})=\frac{\sup _{\beta>0} L(\beta)}{\sup _{\beta=1} L(\beta)}=\frac{L(\hat{\beta})}{L(1)}=\frac{(2 / \overline{X})^{2 n} \exp \{-2 n\}}{\exp \{-\overline{X}\}}=(2 / \overline{X})^{2 n} \exp \{\overline{X}-2 n\}
	$$
	由似然比检验统计量的极限分布知
	$$
	2 \log \Lambda(\boldsymbol{X})=4 n \log \frac{2}{\overline{X}}+2 \overline{X}-4 n \xrightarrow{H_0} \chi_1^2, \quad \text { as } n \rightarrow \infty
	$$
	因此水平为 $\alpha$ 的似然比检验的拒绝域为
	$$
	D=\left\{\boldsymbol{X}: 4 n \log \frac{2}{\overline{X}}+2 \overline{X}-4 n>\chi_1^2(\alpha)\right\} .
	$$
	Wald 检验:由似然函数计算$n$个样本的 Fisher 信息量为
	$$
	I_n(\beta)=-\mathbb{E}_\beta\left[\frac{\partial^2 \log L(\beta)}{\partial \beta^2}\right]=\frac{2 n}{\beta^2}
	$$
	所以 Wald 检验统计量为
	$$
	W(\boldsymbol{X})=(\hat{\beta}-1)^2 I_n(\hat{\beta})=2 n(1-1 / \hat{\beta})^2=2 n(1-\overline{X} / 2)^2 \xrightarrow{H_0} \chi_1^2, \quad \text { as } n \rightarrow \infty .
	$$
	因此水平为 $\alpha$ 的 Wald 检验的拒绝域为 
	$$D=\left\{\boldsymbol{X}: 2 n(1-\overline{X} / 2)^2>\chi_1^2(\alpha)\right\}$$
	得分检验:得分函数为
	$$
	U_n(\beta)=\frac{\partial \log L(\beta)}{\partial \beta}=\frac{2 n}{\beta}-\sum_{i=1}^n x_i .
	$$
	所以得分检验统计量为
	$$
	S(\boldsymbol{X})=\left[U_n(1)\right]^2 I_n^{-1}(1)=\left(2 n-\sum_{i=1}^n X_i\right)^2 /(2 n)=n(2-\overline{X})^2 / 2 \xrightarrow{H_0} \chi_1^2, \quad \text { as } n \rightarrow \infty .
	$$
	因此水平为 $\alpha$ 的得分检验的拒绝域为 
	$$D=\left\{\boldsymbol{X}: n(2-\overline{X})^2 / 2>\chi_1^2(\alpha)\right\}$$
	该形式与 Wald 检验的形式是一致的。\\
	
	
	
	
	
	\noindent 六、(10分)男女舒张压检测数据如下:共检测男性$16$人,其中舒张压$<60$的有$4$人,$>90$的有$2$人;共检测女性$21$人,其中舒张压$<60$的有$5$人,$>90$的有$2$人。
	
	是否能认为男女舒张压分布相同?\\
	解:未合并列的齐一性检验(酌情扣分):首先根据题目描述写出列联表如下:
	$$
	\begin{tabular}{c|ccc|c}
		\hline 舒张压 & $<60$ & $60 \sim 90$ & $>90$ & 合计 \\
		\hline 男性人数 & 4 & 10 & 2 & 16 \\
		女性人数 & 5 & 14 & 2 & 21 \\
		\hline 合计 & 9 & 24 & 4 & 37 \\
		\hline
	\end{tabular}
	$$
	要检验的假设为 $H_0$ :男女性舒张压分布没有显著差异.取检验统计量为
	$$
	K(\boldsymbol{X})=n \sum_{i=1}^r \sum_{j=1}^s \frac{\left(n_{i j}-n_{i \cdot} n_{\cdot j} / n\right)^2}{n_{i \cdot} n_{\cdot j}} \xrightarrow{H_0} \chi_{(r-1)(s-1)}^2, \quad \text { as } n \rightarrow \infty .
	$$
	因此水平 $\alpha$ 的检验拒绝域为 
	$$D=\left\{\boldsymbol{X}: K(\boldsymbol{X})>\chi_{(r-1)(s-1)}^2(\alpha)\right\}$$
	由样本观测值计算检验统计量值时,可以考虑等价形式
	$$
	K(\boldsymbol{x})=n\left(\sum_{i=1}^r \sum_{j=1}^s \frac{n_{i j}^2}{n_{i \cdot} n_{\cdot j}}-1\right)=0.1040<\chi_2^2(0.2)=3.22 .
	$$
	因此不拒绝原假设,认为男女性舒张压没有显著差异.\\
	合并列的齐一性检验:由于有格子点计数过小,需要合并首尾两列得列联表如下:
	$$
	\begin{tabular}{c|ccc}
		\hline 舒张压 & 正常 & 过低或过高 & 合计 \\
		\hline 男性人数 & 10 & 6 & 16 \\
		女性人数 & 14 & 7 & 21 \\
		\hline 合计 & 24 & 13 & 37 \\
		\hline
	\end{tabular}
	$$
	要检验的假设为 $H_0$ :男女性舒张压分布没有显著差异.取检验统计量为
	$$
	K(\boldsymbol{X})=n \sum_{i=1}^r \sum_{j=1}^s \frac{\left(n_{i j}-n_i \cdot n_{\cdot j} / n\right)^2}{n_i \cdot n_{\cdot j}} \xrightarrow{H_0} \chi_{(r-1)(s-1)}^2, \quad \text { as } n \rightarrow \infty .
	$$
	因此水平 $\alpha$ 的检验拒绝域为 
	$$D=\left\{\boldsymbol{X}: K(\boldsymbol{X})>\chi_{(r-1)(s-1)}^2(\alpha)\right\}$$
	由样本观测值计算检验统计量值时,可以考虑 $2 \times 2$ 列联表的等价形式
	$$
	K(\boldsymbol{x})=\frac{n\left(n_{11} n_{22}-n_{12} n_{21}\right)^2}{n_{1\cdot} n_{\cdot1} n_{2\cdot} n_{\cdot 2}}=0.0692<\chi_1^2(0.2)=1.64 .
	$$
	因此不拒绝原假设,认为男女性舒张压没有显著差异.\\
	
	
	
	\noindent 七、(18分)$n$个样本$(X_1,...,X_n)$来自总体分布$X\sim \Gamma\left(\alpha, \frac{1}{\theta}\right)$,其中$\alpha$已知,先验密度为$\pi(\theta) = \frac{1}{\theta} $ ,
	损失函数为 
	$$L^2(d,\theta) = \frac{1}{\theta^2}(d-\theta)^2$$
	
	求贝叶斯解并证明其为Minimax解\\
	解:后验分布的计算:样本 $\boldsymbol{X}=\left(X_1, \ldots, X_n\right)$ 的联合概率密度函数为
	$$
	f(\boldsymbol{x} \mid \theta)=\prod_{i=1}^n \frac{1}{\theta^\alpha \Gamma(\alpha)} x_i^{\alpha-1} \exp \left\{-\frac{x_i}{\theta}\right\}=\frac{\theta^{-n \alpha}}{[\Gamma(\alpha)]^n} \exp \left\{-\frac{1}{\theta} \sum_{i=1}^n x_i\right\} \prod_{i=1}^n x_i^{\alpha-1}
	$$
	因此在无信息先验 $\pi(\theta)=(1 / \theta) I_{(0, \infty)}(\theta)$ 下,$\theta$ 的后验密度为
	$$
	\pi(\theta \mid \boldsymbol{x}) \propto f(\boldsymbol{x} \mid \theta) \pi(\theta) \propto \theta^{-n \alpha-1} \exp \left\{-\frac{1}{\theta} \sum_{i=1}^n x_i\right\}, \quad \theta>0
	$$
	添加归一化常数后可知 $\theta$ 的后验分布为
	$$
	\theta \mid \boldsymbol{X}=\boldsymbol{x} \sim \Gamma^{-1}\left(n \alpha, \sum_{i=1}^n x_i\right)
	$$
	加权平方损失下的 Bayes 估计:在加权平方损失下,$\theta$ 的 Bayes 估计为
	$$
	\hat{\theta}_B(\boldsymbol{X})=\frac{\mathbb{E}\left(\theta^{-1} \mid \boldsymbol{X}\right)}{\mathbb{E}\left(\theta^{-2} \mid \boldsymbol{X}\right)}=\frac{\frac{n \alpha}{n \overline{X}}}{\frac{n \alpha(n \alpha+1)}{(n \overline{X})^2}}=\frac{n \overline{X}}{n \alpha+1}
	$$
	\textbf{RK}:在矩存在的条件下有\\
	\[
	X \sim \Gamma(\alpha,\beta) \quad \text{有} \forall n \in \mathbb{Z} \quad \mathbb{E}[X^n]=\frac{\Gamma(\alpha+n)}{\Gamma(\alpha)\beta^n}
	\]
	\[
	Y \sim \Gamma^{-1}(\alpha,\beta) \quad \text{有} \forall n \in \mathbb{Z} \quad \mathbb{E}[Y^n]=\frac{\Gamma(\alpha-n)\beta^n}{\Gamma(\alpha)}
	\]
	事实上,$\Gamma$分布的$n$阶矩就是$\Gamma^{-1}$分布的$-n$阶矩\\
	验证该 Bayes 估计为 Minimax 估计:Bayes 估计 $\hat{\theta}_B(\boldsymbol{X})$ 的风险函数,注意$n\overline{X}=\sum\limits_{i=1}^n X_i \sim \Gamma\left(n\alpha,\frac{1}{\theta} \right) $
	$$
	\begin{aligned}
		R\left(\hat{\theta}_B(\boldsymbol{X}), \theta\right) & =\mathbb{E}\left[\frac{(n \overline{X} /(n \alpha+1)-\theta)^2}{\theta^2}\right] \\
		& =\frac{1}{\theta^2} \mathbb{E}\left(\frac{n \overline{X}+\theta}{n \alpha+1}-\theta-\frac{\theta}{n \alpha+1}\right)^2 \\
		& =\frac{1}{\theta^2}\left[\operatorname{Var}\left(\frac{n \overline{X}+\theta}{n \alpha+1}\right)+\frac{\theta^2}{(n \alpha+1)^2}\right]\\
		&=\frac{1}{n \alpha+1}
	\end{aligned}
	$$
	为常数,所以 $\theta$ 的 Bayes 估计 $\hat{\theta}_B(\boldsymbol{X})=n \overline{X} /(n \alpha+1)$ 是 $\theta$ 的 Minimax 估计.\\
	
	
	\noindent 八、(20分,附加题)
	将Hardy-Weinberg 定律简化如下:$n$个样本$(X_1,...,X_n)$来自总体分布$X$\\
	设$X$的概率分布为:
	\[
	\begin{array}{c|ccc}
		X & 1 & 2 & 3 \\ \hline
		P(X) & p^2 & 2p(1-p) & (1-p)^2 \\
	\end{array}
	\]
	对于检验问题:
	\[
	H_0: p \leq p_0 \quad \leftrightarrow \quad H_1: p > p_0
	\]
	求UMPT。\\
	解:先说明样本分布族是单参数指数族:样本联合概率质量函数为
	$$
	\begin{aligned}
		f(\boldsymbol{x} ; p) & =\left(p^2\right)^{n_0}[2 p(1-p)]^{n_1}\left[(1-p)^2\right]^{n_2} \\
		& =2^{n_1} p^{2 n_0+n_1}(1-p)^{n_1+2 n_2}\\
		&=2^{n_1}(1-p)^{2 n} \exp \left\{\left(2 n_0+n_1\right) \log \frac{p}{1-p}\right\}
	\end{aligned}
	$$
	这是单参数指数族,且 $Q(p)=\log \frac{p}{1-p}$ 为 $p$ 的严格单调增函数,$T=T(\boldsymbol{X})=2 n_0+n_1$ .\\
	根据推论 5.4.2 给出检验的 UMPT:注意到原检验问题等价于
	$$
	H_0: \theta \leq \theta_0 \longleftrightarrow H_1: \theta>\theta_0
	$$
	其中 $\theta_0=\log \frac{p_0}{1-p_0}$ .由推论 5.4.2 知,离散型要补上随机化常数,假设检验的 UMPT 可取为
	$$
	\phi(T)= \begin{cases}1, & T>c \\ r, & T=c \\ 0, & T<c\end{cases}
	$$
	其中常数 $c$ 和 $r$ 满足条件
	$$
	c=\underset{c^{\prime}}{\arg \min }\left\{\mathrm{P}_{p_0}(T>c) \leq \alpha\right\}, \quad r=\frac{\alpha-\mathrm{P}_{p_0}(T>c)}{\mathrm{P}_{p_0}(T=c)} .
	$$
	确定检验函数中的常数 $c$ 和 $r$ :记
	$$
	p_t=\sum_{1 \leq i, j \leq n, i+j \leq n, 2 i+j=t} \frac{n!}{i!j!(n-i-j)!} 2^i p_0^{2 i+j}\left(1-p_0\right)^{j+2(n-i-j)},
	$$
	则
	
	$$
	\mathrm{P}_{p_0}(T=c)=p_c, \quad \mathrm{P}_{p_0}(T>c)=\sum_{t=c+1}^n p_t .
	$$
	\newpage
	\section{\centering 24数理统计期末}
	\noindent 一.(20 分)单项选择填空题(每题 2 分)\\
	1.设 $X_1, \ldots, X_n$ 为来自均匀分布 $\mathrm{U}(-\theta, \theta)$ 的一组样本,$\theta$ 为未知参数,则下述量为统计量的是\underline{\hspace{2cm}}\\
	(A) $\overline{X}-\theta$\\
	(B) $\max _{1 \leq i \leq n}\left(X_i-\theta\right)-\min _{1 \leq i \leq n}\left(X_i-\theta\right)$\\
	(C) $\max _{1 \leq i \leq n}\left(X_i-\theta\right)$\\
	(D) $\min _{1 \leq i \leq n}\left(X_i-\theta\right)$\\
	答案:B\\
	统计量要与未知参数无关\\
	2.设 $\hat{\theta}_n$ 为末知参数 $\theta$ 的一个估计量,如果 $\lim _{n \rightarrow \infty} \mathbb{E}\left[ \hat{\theta}_n-\theta\right] =0$ ,则 $\hat{\theta}_n$ 为 $\theta$ 的\underline{\hspace{2cm}}\\
	(A)无偏估计\\
	(B)有效估计\\
	(C)相合估计\\
	(D)渐近正态估计\\
	答案:C\\
	3.假设样本 $X$ 的密度为 $f_\theta(x)$ ,其中 $\theta$ 为参数,则下列表述不正确的是\underline{\hspace{2cm}}\\
	(A)固定 $x$ 时 $f_\theta(x)$ 为似然函数\\
	(B)固定 $\theta$ 时 $f_\theta(x)$ 为似然函数\\
	(C)固定 $\theta$ 时 $f_\theta(x)$ 为密度函数\\
	(D)$f_\theta(x)$ 衡量了不同 $\theta$ 下观测到值 $x$ 的可能性大小\\
	答案:B\\
	4.一个参数 $\theta$ 的 $95 \%$ 区间估计为 $[0.1,0.3]$ ,则下列表述正确的是\underline{\hspace{2cm}}\\
	(A)若该区间为置信区间,则表明 $\theta$ 位于该区间的概率是 0.95\\
	(B)该区间的边际误为 0.2\\
	(C)对假设 $H_0: \theta=0.2 \leftrightarrow H_1: \theta \neq 0.2$ ,会在 0.05 水平下拒绝原假设\\
	(D)若该区间为贝叶斯可信区间,则表明 $\theta$ 位于该区间的概率是 0.95\\
	答案:D\\
	5.下列表述错误的是\underline{\hspace{2cm}}\\
	(A)矩估计量一般不唯一\\
	(B)无偏估计总是优于有偏估计\\
	(C)相合性是一个估计量的基本性质\\
	(D)最大似然估计可以不存在\\
	答案:B\\
	6.若 $\delta(X)$ 是一个损失下的 Bayes 法则,则下列表述正确的是\underline{\hspace{2cm}}\\
	(A)$\delta(X)$ 的贝叶斯风险不超过 Minimax 风险\\
	(B)$\delta(X)$ 不可能是一个 Minimax 法则\\
	(C)$\delta(X)$ 是可容许的\\
	(D)$\delta(X)$ 的风险为常数\\
	答案:A\\
	7.下述对一个显著性检验方法的描述错误的是\underline{\hspace{2cm}}\\
	(A)原假设与对立假设地位不均等,原假设被保护起来\\
	(B)$p$ 值越显著表明原假设成立的依据越强烈\\
	(C)在一个检验结果是不能拒绝零假设时,检验只可能会犯第二类错误\\
	(D)双边假设的接受域等价于参数的置信区间\\
	答案:B\\
	8.设 $X_1, \ldots, X_n$ 为来自正态总体 $N(\mu, 1)$ 的简单样本,考虑假设检验问题 $H_0: \mu=0 \leftrightarrow H_1$ : $\mu=0.5$ 。如果要求检验的第一类和第二类错误均不超过 $\alpha(0<\alpha<1)$ ,则样本量 $n$ 应满足$\underline{n \geq\left\lceil 16 u_\alpha^2\right\rceil}$ (结果用分位数表示).\\
	解:两点假设的拒绝域形如 $R=\{\boldsymbol{X}: \overline{X}>c\}$ .按要求
	$$
	\begin{aligned}
		& \alpha \geq \mathrm{P}\left(\boldsymbol{X} \in R \mid H_0\right)=\mathrm{P}_{\mu=0}(\overline{X}>c)=1-\Phi(\sqrt{n} c), \\
		& \alpha \geq \mathrm{P}\left(\boldsymbol{X} \notin R \mid H_1\right)=\mathrm{P}_{\mu=0.5}(\overline{X} \leq c)=\Phi(\sqrt{n}(c-0.5)) .
	\end{aligned}
	$$
	于是我们有
	$$
	\left\{\begin{array}{l}
		\sqrt{n} c \geq u_\alpha, \\
		\sqrt{n}(c-0.5) \leq-u_\alpha, \quad \Longrightarrow \quad 0.5 \sqrt{n} \geq 2 u_\alpha, \quad \Longrightarrow \quad n \geq\left\lceil 16 u_\alpha^2\right\rceil .
	\end{array}\right.
	$$
	9.设某种产品的质量等级可以划分为"优"、"合格"和"不合格",为了判断生产此产品的三家工厂的产品是否有差异,使用拟合优度检验方法时的原假设为 $\underline{\text{三家工厂生产的产品质量无差异}} $,渐近卡方分布的自由度为 $\underline{4}$.\\
	10.设 $X_1, \ldots, X_n$ 为来自均匀分布 $\mathrm{U}(0, \theta), \theta>0$ 的一组简单样本,$\theta$ 的先验密度为 $\pi(\theta)=$ $1 /\left(2 \theta^2\right), \theta \geq 1 / 2$ 。 考虑假设检验问题 $H_0: \theta \leq 1 \leftrightarrow H_1: \theta>1$ ,则其 Bayes 因子 $\mathrm{BF}_{01}$为 $\underline{\left[\left(x_{(n)} \vee 0.5\right)^{-n-1}-1\right] \vee 0}$ \\
	解:样本联合密度与先验分别为
	$$
	f(\boldsymbol{x} \mid \theta)=\theta^{-n} \cdot \mathbb{I}_{\left\{0<x_{(n)}<\theta\right\}}, \quad \pi(\theta)=0.5 \theta^{-2} \cdot \mathbb{I}_{\{\theta \geq 0.5\}}
	$$
	因此 $\theta$ 的后验密度为
	$$
	\pi(\theta \mid \boldsymbol{x}) \propto \theta^{-n-2} \cdot \mathbb{I}_{\left\{\theta>x_{(n)} \vee 0.5\right\}}
	$$
	归一化后可得后验密度,进而求得后验分布函数为
	$$
	\Pi(\theta \mid \boldsymbol{x})= \begin{cases}1-\left(\frac{\theta_*}{\theta}\right)^{n+1}, & \theta \geq \theta_* \\ 0, & \theta<\theta_*\end{cases}
	$$
	其中 $\theta_*=x_{(n)} \vee 0.5$ .于是当 $\theta_*<1$ ,即 $x_{(n)}<1$ 时,
	$$
	\alpha_0=\mathrm{P}(\theta \leq 1 \mid \boldsymbol{x})=\Pi(1 \mid \boldsymbol{x})=1-\theta_*^{n+1}, \quad \alpha_1=1-\alpha_0=\theta_*^{n+1}
	$$
	当 $\theta_* \geq 1$ ,即 $x_{(n)} \geq 1$ 时,$\alpha_0=0, \alpha_1=1$ .又因为 $\pi_0=\mathrm{P}(\theta \leq 1)=0.5, \pi_1=\mathrm{P}(\theta>1)=0.5$ ,所以贝叶斯因子为
	$$
	\mathrm{BF}_{01}=\frac{\alpha_0 / \alpha_1}{\pi_0 / \pi_1}=\left\{\begin{array}{ll}
		\theta_*^{-n-1}-1, & \theta_*<1, \\
		0, & \theta_* \geq 1
	\end{array}=\left[\left(x_{(n)} \vee 0.5\right)^{-n-1}-1\right] \vee 0\right.
	$$\\
	
	
	
	\noindent 二.(20 分)设从总体
	$$
	\begin{tabular}{c|ccc}
		$X$ & 0 & 1 & 2 \\
		\hline P & $p_1$ & $p_2$ & $p_3$
	\end{tabular}
	$$
	(其中 $0<p_1, p_2, p_3<1, p_1+p_2+p_3=1$ 为末知参数)中抽取的一个简单样本 $X_1, \ldots, X_n$ ,试\\
	(1)求 $p_1-p_2$ 的最大似然估计,并证明其为最小方差无偏估计。\\
	(2)求检验问题 $H_0: p_1=p_2 \leftrightarrow H_1: p_1 \neq p_2$ 的一个(渐近)水平 $\alpha$ 检验。\\
	解:(1)似然函数为
	$$
	L\left(p_1, p_2 ; \boldsymbol{x}\right)=p_1^{n_0} p_2^{n_1}\left(1-p_1-p_2\right)^{n-n_0-n_1}
	$$
	其中 $n_i=\sum_{j=1}^n \mathbb{I}_{\left\{X_j=i\right\}}, i=0,1$ .由对数似然方程
	$$
	\left\{\begin{array}{l}
		\frac{\partial l\left(p_1, p_2 ; \boldsymbol{x}\right)}{\partial p_1}=\frac{n_0}{p_1}-\frac{n-n_0-n_1}{1-p_1-p_2}=0 \\
		\frac{\partial l\left(p_1, p_2 ; \boldsymbol{x}\right)}{\partial p_2}=\frac{n_1}{p_2}-\frac{n-n_0-n_1}{1-p_1-p_2}=0
	\end{array}\right.
	$$
	解得 $p_1, p_2$ 的最大似然估计分别为
	$$
	\hat{p}_1=\frac{n_0}{n}, \quad \hat{p}_2=\frac{n_1}{n}
	$$
	进一步由最大似然估计的不变性可知 $p_1-p_2$ 的最大似然估计为 $\hat{p}_1-\hat{p}_2=\left(n_0-n_1\right) / n$ .\\
	最小方差无偏估计:将样本联合密度函数写成指数族形式如下
	$$
	f\left(\boldsymbol{x} ; p_1, p_2\right)=\exp \left\{n_0 \log \frac{p_1}{1-p_1-p_2}+n_1 \log \frac{p_2}{1-p_1-p_2}\right\} \cdot\left(1-p_1-p_2\right)^n
	$$
	令 
	$$\eta_1=\log \frac{p_1}{1-p_1-p_2}, \eta_2=\log \frac{p_2}{1-p_1-p_2}$$
	于是自然参数空间 
	$$\Theta^*=\left\{\left(\eta_1, \eta_2\right):-\infty<\eta_1<\infty,-\infty<\eta_2<\infty\right\}$$
	有内点,因此 $T=\left(n_0, n_1\right)$ 是 $\left(p_1, p_2\right)$ 的充分完全统计量.又注意到 $\hat{p}_1$ 和 $\hat{p}_2$ 分别是 $p_1$ 和 $p_2$ 的无偏估计,因此由 Lehmann-Scheffé 定理知 $p_1-p_2$ 的最大似然估计是最小方差无偏估计.\\
	(2)法一:拟合优度检验,取检验统计量为
	$$
	K(\boldsymbol{X})=\sum_{r=1}^3 \frac{\left(n_{r-1}-n \hat{p}_r\right)^2}{n \hat{p}_r} \xrightarrow{H_0} \chi_{3-1-1}^2,
	$$
	其中 $\hat{p}_r$ 为 $H_0$ 下的极大似然估计.注意到当 $p_1=p_2=p$ 时,样本的似然函数为
	$$
	L(p ; \boldsymbol{x})=p^{n_0+n_1}(1-2 p)^{n_2} .
	$$
	由对数似然方程可得 $\hat{p}=\left(n_0+n_1\right) /(2 n)$ .代入检验统计量表达式得
	$$
	K(\boldsymbol{X})=\frac{\left(n_0-n_1\right)^2}{n_0+n_1} .
	$$
	因此检验问题渐近水平 $\alpha$ 检验为
	$$
	\phi(\boldsymbol{X})= \begin{cases}1, & \text { 当 }\left(n_0-n_1\right)^2>\left(n_0+n_1\right) \chi_1^2(\alpha), \\ 0, & \text { 当 }\left(n_0-n_1\right)^2 \leq\left(n_0+n_1\right) \chi_1^2(\alpha) .\end{cases}
	$$
	法二:似然比检验,注意到似然比
	$$
	\lambda(\boldsymbol{X})=\frac{\sup _{\theta \in \Theta} L(\theta)}{\sup _{\theta \in \Theta_0} L(\theta)}=\frac{\hat{p}_1^{n_0} \hat{p}_2^{n_1}\left(1-\hat{p}_1-\hat{p}_2\right)^{n_2}}{\hat{p}^{\left(n_0+n_1\right)}(1-2 \hat{p})^{n_2}}
	$$
	在大样本下,我们有 $2 \log \lambda(\boldsymbol{X}) \xrightarrow{H_0} \chi_1^2$ .代入检验统计量表达式得
	$$
	2 \log \lambda(\boldsymbol{X})=2 \log \frac{\left(n_0 / n\right)^{n_0} \cdot\left(n_1 / n\right)^{n_1}}{\left(n_0+n_1\right)^{n_0+n_1} /(2 n)^{n_0+n_1}}=2 n_0 \log \frac{2 n_0}{n_0+n_1}+2 n_1 \log \frac{2 n_1}{n_0+n_1}
	$$
	因此检验问题渐近水平 $\alpha$ 检验为
	$$
	\phi(\boldsymbol{X})= \begin{cases}1, & \text { 当 } 2 n_0 \log \frac{2 n_0}{n_0+n_1}+2 n_1 \log \frac{2 n_1}{n_0+n_1}>\chi_1^2(\alpha), \\ 0, & \text { 当 } 2 n_0 \log \frac{2 n_0}{n_0+n_1}+2 n_1 \log \frac{2 n_1}{n_0+n_1} \leq \chi_1^2(\alpha) .\end{cases}
	$$
	法三:利用渐近正态检验,注意到
	$$
	\hat{p}_1-\hat{p}_2=\frac{n_0-n_1}{n}=\frac{1}{n} \sum_{j=1}^n\left[\mathbb{I}_{\left\{X_j=0\right\}}-\mathbb{I}_{\left\{X_j=1\right\}}\right]
	$$
	是独立随机变量之平均,于是由中心极限定理知
	$$
	\frac{\sqrt{n}\left(\hat{p}_1-\hat{p}_2-\mathbb{E}\left[\mathbb{I}_{\left\{X_j=0\right\}}-\mathbb{I}_{\left\{X_j=1\right\}}\right]\right)}{\sqrt{\operatorname{Var}\left[\mathbb{I}_{\left\{X_j=0\right\}}-\mathbb{I}_{\left\{X_j=1\right\}}\right]}} \xrightarrow{D} N(0,1)
	$$
	其中 $$\mathbb{E}\left[\mathbb{I}_{\left\{X_1=0\right\}}-\mathbb{I}_{\left\{X_1=1\right\}}\right]=p_1-p_2\quad \operatorname{Var}\left[\mathbb{I}_{\left\{X_1=0\right\}}-\mathbb{I}_{\left\{X_1=1\right\}}\right]=p_1+p_2-\left(p_1-p_2\right)^2$$
	结合 Slutsky 定理,因此考虑取检验统计量为
	$$
	U(\boldsymbol{X})=\frac{\sqrt{n}\left(\hat{p}_1-\hat{p}_2\right)}{\sqrt{\hat{p}_1+\hat{p}_2-\left(\hat{p}_1-\hat{p}_2\right)^2}}
	$$
	在 $H_0$ 下,我们有 $U(\boldsymbol{X}) \xrightarrow{D} N(0,1)$ .于是检验问题渐近水平 $\alpha$ 检验为
	$$
	\phi(\boldsymbol{X})= \begin{cases}1, & \text { 当 }|U(\boldsymbol{X})|>u_{\alpha / 2}, \\ 0, & \text { 当 }|U(\boldsymbol{X})| \leq u_{\alpha / 2} .\end{cases}
	$$
	法四:利用 Wald 检验,记 $\boldsymbol{\theta}=\left(p_1, p_2\right)^{T}, \hat{\boldsymbol{\theta}}=\left(\hat{p}_1, \hat{p}_2\right)^{T}$ ,于是由中心极限定理有
	$$
	\sqrt{n}(\hat{\boldsymbol{\theta}}-\boldsymbol{\theta}) \xrightarrow{D} N\left(0, \boldsymbol{I}^{-1}(\boldsymbol{\theta})\right),
	$$
	其中总体(单个样本)的 Fisher 信息阵为
	$$
	\boldsymbol{I}(\theta)=\left(\begin{array}{cc}
		p_1^{-1}+p_3^{-1} & p_3^{-1} \\
		p_3^{-1} & p_2^{-1}+p_3^{-1}
	\end{array}\right)
	$$
	注意 $h(\boldsymbol{\theta})=p_1-p_2, \boldsymbol{B}=\partial h / \partial \boldsymbol{\theta}=(1,-1)$ ,因此取检验统计量为
	$$
	W_n=n h(\hat{\boldsymbol{\theta}})\left[\boldsymbol{B}(\hat{\boldsymbol{\theta}}) \boldsymbol{I}^{-1}(\hat{\boldsymbol{\theta}}) \boldsymbol{B}^{T}(\hat{\boldsymbol{\theta}})\right]^{-1} h(\hat{\boldsymbol{\theta}})=\frac{n\left(\hat{p}_1-\hat{p}_2\right)^2}{\hat{p}_1+\hat{p}_2-\left(\hat{p}_1-\hat{p}_2\right)^2}
	$$
	在 $H_0$ 下,我们有 $W_n \xrightarrow{D} \chi_1^2$ .于是检验问题渐近水平 $\alpha$ 检验为
	$$
	\phi(\boldsymbol{X})= \begin{cases}1, & \text { 当 } W_n>\chi_1^2(\alpha), \\ 0, & \text { 当 } W_n \leq \chi_1^2(\alpha) .\end{cases}
	$$\\
	
	
	
	
	\noindent 三.(30 分)设 $X_1, \ldots, X_n$ i.i.d.$\sim N(\mu, 1)$ ,其中 $\mu$ 为参数.对水平 $\alpha$ ,试\\
	(1)求 $\mathrm{P}\left(X_1>0\right)$ 的最大似然估计,并求其渐近方差。\\
	(2)证明检验问题 $H_0: \mu=\mu_0 \leftrightarrow H_1: \mu \neq \mu_0$ 不存在 UMPT,其中 $\mu_0$ 为一已知数.\\
	(3)若参数 $\mu$ 在 $\mu=\mu_0$ 上的先验概率为 0.6 ,在 $\mu \neq \mu_0$ 上的先验分布为 $N\left(\mu_0, 4\right)$ ,损失函数取为 0-1 损失,求(2)中的假设检验问题的 Bayes 决策。\\
	解:(1)似然函数
	$$
	L(\mu ; \boldsymbol{x})=(2 \pi)^{-\frac{n}{2}} \exp \left\{-\frac{1}{2} \sum_{i=1}^n\left(x_i-\mu\right)^2\right\}=(2 \pi)^{-\frac{n}{2}} \exp \left\{-\frac{1}{2} \sum_{i=1}^n\left(x_i-\overline{x}\right)^2-\frac{n(\mu-\overline{x})^2}{2}\right\}
	$$
	因此 $\mu$ 的最大似然估计为 $\hat{\mu}=\overline{X}$ 。\\
	由最大似然估计的不变性可知,$p=\mathrm{P}\left(X_1>0\right)=\Phi(\mu)$ 的最大似然估计为 $\hat{p}=\Phi(\hat{\mu})=\Phi(\overline{X})$ 。\\
	注意到 $\overline{X} \sim N(\mu, 1 / n)$ ,由 Delta 方法可知 $\hat{p}$ 的渐近方差为
	$$
	[\phi(\mu)]^2 \cdot \frac{1}{n}=\frac{\exp \left\{-\mu^2\right\}}{2 n \pi}
	$$
	(2)首先注意到正态分布族(方差已知,均值为未知参数)关于 $T=\overline{X}$ 是单调似然比族。因而对检验问题 $H_0: \mu=\mu_0 \leftrightarrow H_1^{\prime}: \mu>\mu_0$ ,存在 UMPT 形如
	$$
	\phi_1(\boldsymbol{x})= \begin{cases}1, & \text { 当 } \overline{x}>\mu_0+u_\alpha / \sqrt{n}, \\ 0, & \text { 其他. }\end{cases}
	$$
	对检验问题 $H_0: \mu=\mu_0 \leftrightarrow H_1^{\prime \prime}: \mu<\mu_0$ ,存在 UMPT 形如
	$$
	\phi_2(\boldsymbol{x})= \begin{cases}1, & \text { 当 } \overline{x}<\mu_0-u_\alpha / \sqrt{n}, \\ 0, & \text { 其他. }\end{cases}
	$$
	显然 $\phi_1$ 和 $\phi_2$ 都是检验问题 $H_0 \leftrightarrow H_1$ 的水平 $\alpha$ 检验。\\
	假设检验问题 $H_0 \leftrightarrow H_1$ 的 UMPT 存在,令其为 $\phi_0$ 。对固定 $\mu_1>\mu_0$ 和 $\mu_2<\mu_0$ ,检验 $\phi_0$ 也是简单假设 $H_0: \mu=\mu_0 \leftrightarrow K_1: \mu=\mu_1$ 和 $H_0: \mu=\mu_0 \leftrightarrow K_2: \mu=\mu_2$ 的UMPT.因此由 Neyman-Pearson 引理可知 $\phi_0$ 有形式
	$$
	\begin{aligned}
		& \phi_0(\boldsymbol{x})= \begin{cases}1, & \text { 当 } f\left(\boldsymbol{x} ; \mu_1\right)>k_1 f\left(\boldsymbol{x} ; \mu_0\right), \\
			0, & \text { 当 } f\left(\boldsymbol{x} ; \mu_1\right) \leq k_1 f\left(\boldsymbol{x} ; \mu_0\right),\end{cases} \\
		& \phi_0(\boldsymbol{x})= \begin{cases}1, & \text { 当 } f\left(\boldsymbol{x} ; \mu_2\right)>k_2 f\left(\boldsymbol{x} ; \mu_0\right), \\
			0, & \text { 当 } f\left(\boldsymbol{x} ; \mu_2\right) \leq k_2 f\left(\boldsymbol{x} ; \mu_0\right) .\end{cases}
	\end{aligned}
	$$
	法一:考虑 $\boldsymbol{x} \in\left\{\boldsymbol{x}: \phi_0(\boldsymbol{x})=1\right\}$ ,由单调似然比的性质可知
	\begin{enumerate}[label=\textbf{\textbullet}]
		\item 如果 $T(\boldsymbol{y})>T(\boldsymbol{x})$,则由第一个检验形式知 $\phi_0(\boldsymbol{y})=1$。
		\item 如果 $T(\boldsymbol{y})<T(\boldsymbol{x})$,则由第二个检验形式知 $\phi_0(\boldsymbol{y})=1$。
	\end{enumerate}
	于是要么 $\phi_0(\boldsymbol{y})=1$ 对所有 $\boldsymbol{y}$ 成立,要么 $\phi_0(\boldsymbol{x}) \neq 1$ 对所有 $\boldsymbol{x}$ 成立.这时 $\phi_0$ 的功效比 $\phi_1$ 和 $\phi_2$ 在各自的检验问题 $H_0 \leftrightarrow K_1$ 和 $H_0 \leftrightarrow K_2$ 都要小,导出矛盾。\\
	法二:由唯一性可知,在 $\mu_1>\mu_0$ 上, $\phi_0=\phi_1$ ,a.e.;在 $\mu_2<\mu_0$ 上,$\phi_0=\phi_2$ ,a.e.由 $\phi_1$ 和 $\phi_2$ 的形式知这不可能成立。\\
	(3)由题意知两个假设的后验概率分别为
	$$
	\alpha_0=\mathrm{P}\left(\mu=\mu_0 \mid \boldsymbol{x}\right)=\frac{\pi_0 f\left(\boldsymbol{x} \mid \mu_0\right)}{m(\boldsymbol{x})}, \quad \alpha_1=\mathrm{P}\left(\mu \neq \mu_0 \mid \boldsymbol{x}\right)=\frac{\pi_1 m_1(\boldsymbol{x})}{m(\boldsymbol{x})},
	$$
	或者直接注意到简单假设对复杂假设的贝叶斯因子有形式
	$$
	\mathrm{BF}_{01}(\boldsymbol{x})=\frac{f\left(\boldsymbol{x} \mid \mu_0\right)}{m_1(\boldsymbol{x})}, \Longrightarrow \frac{\alpha_0}{\alpha_1}=\frac{\pi_0 f\left(\boldsymbol{x} \mid \mu_0\right)}{\pi_1 m_1(\boldsymbol{x})},
	$$
	其中 
	$$m(\boldsymbol{x})=\pi_0 f\left(\boldsymbol{x} \mid \mu_0\right)+\pi_1 m_1(\boldsymbol{x}), m_1(\boldsymbol{x})=\int_{\mu \neq \mu_0} f(\boldsymbol{x} \mid \mu) \pi(\mu) \mathrm{d} \mu$$
	下面计算 $m_1(\boldsymbol{x})$ 如下
	$$
	\begin{aligned}
		m_1(\boldsymbol{x}) & =\int_{\mu \neq \mu_0} f(\boldsymbol{x} \mid \mu) \pi(\mu) \mathrm{d} \mu \\
		& =\int_{\mu \neq \mu_0}(2 \pi)^{-\frac{n}{2}} \exp \left\{-\frac{1}{2} \sum_{i=1}^n\left(x_i-\overline{x}\right)^2-\frac{n(\mu-\overline{x})^2}{2}\right\} \cdot(8 \pi)^{-\frac{1}{2}} \exp \left\{-\frac{\left(\mu-\mu_0\right)^2}{8}\right\} \mathrm{d} \mu \\
		& =\frac{(2 \pi)^{-\frac{n+1}{2}}}{2} \exp \left\{-\frac{(n-1) s^2}{2}\right\} \int_{\mu \neq \mu_0} \exp \left\{-\frac{A \mu^2-2 B \mu+C}{2}\right\} \mathrm{d} \mu \\
		& =\frac{(2 \pi)^{-\frac{n+1}{2}}}{2} \exp \left\{-\frac{(n-1) s^2}{2}\right\} \cdot(2 \pi / A)^{\frac{1}{2}} \exp \left\{-\frac{1}{2}\left(C-\frac{B^2}{A}\right)\right\} \\
		& =\frac{(2 \pi)^{-\frac{n}{2}}}{\sqrt{4 n+1}} \exp \left\{-\frac{(n-1) s^2}{2}-\frac{n\left(\overline{x}-\mu_0\right)^2}{2(4 n+1)}\right\}
	\end{aligned}
	$$
	其中 
	$$A=n+\frac{1}{4}, B=n \overline{x}+\frac{\mu_0}{4}, C=n \overline{x}^2+\frac{\mu_0^2}{4}$$
	因此
	$$
	\frac{\alpha_0}{\alpha_1}=\frac{0.6(2 \pi)^{-\frac{n}{2}} \exp \left\{-\frac{(n-1) s^2}{2}-\frac{n\left(\overline{x}-\mu_0\right)^2}{2}\right\}}{0.4 \frac{(2 \pi)^{-\frac{n}{2}}}{\sqrt{4 n+1}} \exp \left\{-\frac{(n-1) s^2}{2}-\frac{n\left(\overline{x}-\mu_0\right)^2}{2(4 n+1)}\right\}}=\frac{3 \sqrt{4 n+1}}{2} \exp \left\{-\frac{2 n^2\left(\overline{x}-\mu_0\right)^2}{4 n+1}\right\}
	$$
	在 0-1 损失下,该检验问题的贝叶斯决策为
	$$
	\delta(\boldsymbol{x})= \begin{cases}a_0, & \text { 当 }\left(\overline{x}-\mu_0\right)^2 \leq \frac{4 n+1}{2 n^2} \log \frac{3 \sqrt{4 n+1}}{2}, \\ a_1, & \text { 其他 },\end{cases}
	$$
	其中 $a_0$ 表示接受假设 $H_0, a_1$ 表示接受假设 $H_1$ .\\
	
	
	
	
	\noindent 四.(30 分)设 $X_1, \ldots, X_m$ i.i.d.$\sim \operatorname{Exp}\left(\lambda_1\right)$(期望是 $1 / \lambda_1$ 的指数分布),$Y_1, \ldots, Y_n$ i.i.d.$\sim \operatorname{Exp}\left(\lambda_2\right)$ ,且样本 $X_1, \ldots, X_m$ 和 $Y_1, \ldots, Y_n$ 独立,其中 $\lambda_1, \lambda_2$ 为正参数。记 $\overline{X}$ 和 $\overline{Y}$ 分别为两组样本的样本均值.试\\
	(1)求 $\mathbb{E}\left[\left(X_1-Y_1\right)^2 \mid \overline{X}, \overline{Y}\right]$ 。\\
	(2)求 $\lambda_1 / \lambda_2$ 的置信系数为 $1-\alpha$ 的置信区间.\\
	(3)求检验问题 $H_0: \lambda_1=c \lambda_2 \leftrightarrow H_1: \lambda_1 \neq c \lambda_2$ 的水平 $\alpha$ 似然比检验。\\
	解:(1)法一:由指数分布的性质知
	$$
	\mathbb{E}\left[\left(X_1-Y_1\right)^2\right]=\mathbb{E}\left(X_1^2\right)-2 \mathbb{E}\left(X_1 Y_1\right)+\mathbb{E}\left(Y_1^2\right)=\frac{2}{\lambda_1^2}-\frac{2}{\lambda_1 \lambda_2}+\frac{2}{\lambda_2^2} .
	$$
	注意到 $(\overline{X}, \overline{Y})$ 是 $\left(\lambda_1, \lambda_2\right)$ 的充分完全统计量,由 UMVUE 的唯一性可知
	$$
	\mathbb{E}\left[\left(X_1-Y_1\right)^2 \mid \overline{X}, \overline{Y}\right]=\left(\frac{2}{\lambda_1^2}-\frac{2}{\lambda_1 \lambda_2}+\frac{2}{\lambda_2^2}\right)_{\mathrm{UMVUE}}
	$$
	显然地,我们有
	$$
	\mathbb{E}\left(\overline{X}^2\right)=\operatorname{Var}(\overline{X})+[\mathbb{E}(\overline{X})]^2=\frac{m+1}{m \lambda_1^2}, \quad \mathbb{E}\left(\overline{Y}^2\right)=\operatorname{Var}(\overline{Y})+[\mathbb{E}(\overline{Y})]^2=\frac{n+1}{n \lambda_2^2} .
	$$
	因此
	$$
	\mathbb{E}\left[\left(X_1-Y_1\right)^2 \mid \overline{X}, \overline{Y}\right]=\frac{2 m \overline{X}^2}{m+1}-2 \overline{X} \overline{Y}+\frac{2 n \overline{Y}^2}{n+1} .
	$$
	法二:由条件期望的线性性及独立性可知
	$$
	\mathbb{E}\left[\left(X_1-Y_1\right)^2 \mid \overline{X}, \overline{Y}\right]=\mathbb{E}\left(X_1^2 \mid \overline{X}\right)-2 \mathbb{E}\left(X_1 Y_1 \mid \overline{X}, \overline{Y}\right)+\mathbb{E}\left(Y_1^2 \mid \overline{Y}\right) .
	$$
	注意到 
	$$X_1 /(m \overline{X}) \sim \operatorname{Be}(1, m-1), Y_1 /(n \overline{Y}) \sim \operatorname{Be}(1, n-1)$$
	且 $(\overline{X}, \overline{Y})$ 是 $\left(\lambda_1, \lambda_2\right)$ 的充分完全统计量,由 Basu 定理可知
	$$
	\begin{aligned}
		& \mathbb{E}\left(X_1^2 \mid \overline{X}\right)=\overline{X}^2 \cdot \mathbb{E}\left(\left.\frac{X_1^2}{\overline{X}^2} \right\rvert\, \overline{X}\right)=\overline{X}^2 \cdot \mathbb{E}\left(\frac{X_1^2}{\overline{X}^2}\right), \quad \mathbb{E}\left(Y_1^2 \mid \overline{Y}\right)=\overline{Y}^2 \cdot \mathbb{E}\left(\frac{Y_1^2}{\overline{Y}^2}\right), \\
		& \mathbb{E}\left(X_1 Y_1 \mid \overline{X}, \overline{Y}\right)=\overline{X} \overline{Y} \mathbb{E}\left(\left.\frac{X_1}{\overline{X}} \cdot \frac{Y_1}{\overline{Y}} \right\rvert\, \overline{X}, \overline{Y}\right)=\overline{X} \overline{Y} \mathbb{E}\left(\frac{X_1}{\overline{X}}\right) \cdot \mathbb{E}\left(\frac{Y_1}{\overline{Y}}\right) .
	\end{aligned}
	$$
	由贝塔分布性质知
	$$
	\mathbb{E}\left(\frac{X_1}{m \overline{X}}\right)=\frac{1}{m}, \quad \mathbb{E}\left(\frac{X_1^2}{m^2 \overline{X}^2}\right)=\frac{2}{m(m+1)}, \quad \mathbb{E}\left(\frac{Y_1}{n \overline{Y}}\right)=\frac{1}{n}, \quad \mathbb{E}\left(\frac{Y_1^2}{n^2 \overline{Y}^2}\right)=\frac{2}{n(n+1)} .
	$$
	于是代入可得
	$$
	\mathbb{E}\left[\left(X_1-Y_1\right)^2 \mid \overline{X}, \overline{Y}\right]=\frac{2 m \overline{X}^2}{m+1}-2 \overline{X} \overline{Y}+\frac{2 n \overline{Y}^2}{n+1}
	$$
	(2)注意 $2 m \lambda_1 \overline{X} \sim \chi_{2 m}^2, 2 n \lambda_2 \overline{Y} \sim \chi_{2 n}^2$ ,取枢轴变量为
	$$
	\frac{\lambda_1 \overline{X}}{\lambda_2 \overline{Y}} \sim F_{2 m, 2 n}
	$$
	由 $\mathrm{P}\left(F_{2 m, 2 n}(1-\alpha / 2) \leq \frac{\lambda_1 \overline{X}}{\lambda_2 \overline{Y}} \leq F_{2 m, 2 n}(\alpha / 2)\right)=1-\alpha$ ,反解得到 $\lambda_1 / \lambda_2$ 的置信系数为 $1-\alpha$ 的置信区间为
	$$
	\left[F_{2 m, 2 n}(1-\alpha / 2) \cdot \frac{\overline{Y}}{\overline{X}}, F_{2 m, 2 n}(\alpha / 2) \cdot \frac{\overline{Y}}{\overline{X}}\right] .
	$$
	(3)似然函数
	$$
	L\left(\lambda_1, \lambda_2 ; \boldsymbol{x}, \boldsymbol{y}\right)=\lambda_1^m \exp \left\{-\lambda_1 \sum_{i=1}^m x_i\right\} \cdot \lambda_2^n \exp \left\{-\lambda_2 \sum_{j=1}^n y_j\right\} .
	$$
	由似然比检验的思想,取检验统计量为
	\[
	\Lambda(\boldsymbol{x}, \boldsymbol{y})=\frac{\sup _{\lambda_1>0, \lambda_2>0} L\left(\lambda_1, \lambda_2 ; \boldsymbol{x}, \boldsymbol{y}\right)}{\sup _{\lambda_1 / \lambda_2=c} L\left(\lambda_1, \lambda_2 ; \boldsymbol{x}, \boldsymbol{y}\right)} .
	\]
	注意到全空间下 $\hat{\lambda}_1=1 / \overline{X}, \hat{\lambda}_2=1 / \overline{Y}$ .在原假设空间下,记 $\lambda_1=c \lambda, \lambda_2=\lambda$ ,考虑 $\lambda$ 的似然函数为
	$$
	L(\lambda ; \boldsymbol{x}, \boldsymbol{y})=c^m \lambda^{m+n} \exp \{-\lambda(c m \overline{x}+n \overline{y})\}
	$$
	此时最大似然估计 $\hat{\lambda}=(m+n) /(c m \overline{x}+n \overline{y})$ .于是似然比可化简为
	$$
	\begin{aligned}
		\Lambda(\boldsymbol{x}, \boldsymbol{y})=\frac{L\left(\hat{\lambda}_1, \hat{\lambda}_2 ; \boldsymbol{x}, \boldsymbol{y}\right)}{L(c \hat{\lambda}, \hat{\lambda} ; \boldsymbol{x}, \boldsymbol{y})} & =\frac{(c m \overline{x}+n \overline{y})^{m+n}}{c^m(m+n)^{m+n} \overline{x}^m \cdot \overline{y}^n} \\
		& =\frac{1}{(m+n)^{m+n}}\left(m+n \frac{\overline{y}}{c \overline{x}}\right)^m\left(n+m \frac{c \overline{x}}{\overline{y}}\right)^n \\
		& \triangleq \frac{1}{(m+n)^{m+n}}\left(m+n F^{-1}\right)^m(n+m F)^n
	\end{aligned}
	$$
	其中 $F:=F(\boldsymbol{x}, \boldsymbol{y})=c \overline{x} / \overline{y}$ .注意到 $\Lambda$ 关于 $F$ 先递减后递增,因此检验的拒绝域的形式为
	$$
	R=\left\{(\boldsymbol{X}, \boldsymbol{Y}): F(\boldsymbol{X}, \boldsymbol{Y})<c_1 \text { 或 } F(\boldsymbol{X}, \boldsymbol{Y})>c_2\right\} \text {. }
	$$
	注意到 $2 m \lambda_1 \overline{X} \sim \chi_{2 m}^2, 2 n \lambda_2 \overline{Y} \sim \chi_{2 n}^2$ ,所以在 $H_0$ 下,
	$$
	F(\boldsymbol{X}, \boldsymbol{Y})=\frac{c \overline{X}}{\overline{Y}} \sim F_{2 m, 2 n}
	$$
	由显著性水平 $\alpha$ 要求知 $c_1=F_{2 m, 2 n}(1-\alpha / 2), c_2=F_{2 m, 2 n}(\alpha / 2)$ .	
	\newpage
	\section{\centering 23数理统计期末残卷}
	\noindent 1.一个移动通讯公司随机抽取了其 900 个包月客户,计算得知他们一个月平均使用时间是 220 分钟,样本标准差是 90 分钟.假设使用时间服从正态分布 $N\left(\mu, \sigma^2\right)$ .\\
	(1)求包月客户平均使用时间和标准差的 $95 \%$ 置信区间。\\
	(2)如果要求客户平均使用时间的 $95 \%$ 置信区间的长度不超过 5 分钟,应至少抽取多少个客户?该公司的抽样规模是否满足要求?\\
	(3)假设总体标准差为 90 分钟,取 $\mu$ 的先验分布为无信息先验,求 $\mu$ 的后验分布并据此给出 $\mu$ 的可信系数为 $95 \%$ 的可信区间。\\
	(4)解释(1)和(3)中关于平均使用时间所得区间的含义与区别.\\
	解:(1)$\mu, \sigma^2$ 均未知\\
	利用 
	$$\frac{\sqrt{n}(\overline{X}-\mu)}{S} \rightarrow t_{n-1}$$
	估计 $\mu$ ,进而有置信区间
	\[
	\left[ \overline{X} - t_{n-1}\left(\frac{\alpha}{2}\right) \frac{S}{\sqrt{n}},\ 
	\overline{X} + t_{n-1}\left(\frac{\alpha}{2}\right) \frac{S}{\sqrt{n}} \right]
	\]
	代值有
	$$[214.12,225.885]$$ 
	这里因为$t_n\rightarrow N(0,1)$ ,用$u_{\frac{\alpha}{2}}$代替$t_{n-1}\left(\frac{\alpha}{2}\right)$\\
	再利用 
	$$\frac{(n-1) S^2}{\sigma^2} \rightarrow \chi^2_{(n-1)}$$
	$\sigma^2$ 有置信区间 
	$$\left[\frac{(n-1) S^2}{\chi^2_{(n-1)}\left(\frac{\alpha}{2}\right)}, \frac{(n-1) S^2}{\chi^2_{(n-1)}\left(1-\frac{\alpha}{2}\right)}\right]$$
	开方有 $\sigma$ 的置信区间
	$$
	\left[\sqrt{\frac{(n-1) S^2}{\chi^2_{(n-1)}\left(\frac{\alpha}{2}\right)}},\sqrt{\frac{(n-1) S^2}{\chi^2_{(n-1)}\left(1-\frac{\alpha}{2}\right)}}\right]
	$$
	代值为
	\[
	[88.15,92.02]
	\]
	(2)即需要
	\[
	2\times u_{\frac{\alpha}{2}}\frac{S}{\sqrt{n}}\le 5
	\]
	这里同样因为$t_n\rightarrow N(0,1)$ ,用$u_{\frac{\alpha}{2}}$代替$t_{n-1}\left(\frac{\alpha}{2}\right)$解得
	\[
	n\ge 4979
	\]
	不满足要求\\
	(3)位置参数的无信息先验为
	\[
	\pi(\mu)=1
	\]
	样本联合密度为
	\[
	\begin{aligned}
		f(x_1,\cdots,x_n;\mu) & =\left(2 \pi \sigma^2\right)^{-\frac{n}{2}} \exp \left\{-\frac{1}{2 \sigma^2}\sum_{i=1}^{n}\left(x_i-\mu\right)^2\right\} \\
		& =\left(2 \pi \sigma^2\right)^{-\frac{n}{2}} \exp \left\{-\frac{1}{2 \sigma^2} \sum_{i=1}^n\left(x_i-\overline{X}\right)^2+n(\overline{X}-\mu)^2\right\} 
	\end{aligned}
	\]
	则
	\[
	\pi(\mu \mid \vec{X})\propto \exp\left\{-\frac{n}{2\sigma^2}\left(\overline{X}-\mu \right)^2 \right\}
	\]
	也就是
	\[
	\mu \mid \vec{X}\sim N\left(\overline{X},\frac{\sigma^2}{n} \right) 
	\]
	则可信区间为
	\[
	\left[\overline{X}-\frac{\sigma}{\sqrt{n}} u_{\alpha / 2}, \overline{X}+\frac{\sigma}{\sqrt{n}} u_{\alpha / 2} \right] 
	\]
	代入数值为
	\[
	[214.12,225.88]
	\]
	(4)置信区间:经过多次重复实验,$\mu$落在置信区间的\textbf{频率}趋于$95\%$,参数是一个真实的固定值\\
	可信区间:相当于把$\mu$视为\textbf{随机变量},一次试验后$\mu$落在可信区间的\textbf{概率}为$95\%$\\
	
	
	
	
	\noindent 2.设 $X_1, \ldots, X_n$ i.i.d.$\sim N(\theta, 1), Y_1, \ldots, Y_n$ i.i.d.$\sim N(2 \theta, 1), \theta$ 为未知参数.又设合样本 $X_1, \ldots, X_n, Y_1, \ldots, Y_n$ 独立.试\\
	(1)求 $\theta$ 的 UMVUE.\\
	(2)如果要求 $\theta$ 的置信系数为 $1-\alpha$ 的置信区间宽不超过指定的 $d$ ,则样本量 $n$应该至少多大?\\
	解:(1)样本联合密度为
	\[
	\begin{aligned}
		f\left(X_1, \cdots, X_n, Y_1, \cdots, Y_n ; \theta\right) 
		&=\left(\frac{1}{\sqrt{2 \pi}}\right)^{-2 n} \cdot e^{\frac{-\sum_{i=1}^n\left(X_i-\theta\right)^2-\sum_{i=1}^n\left(Y_i-2 \theta\right)^2}{2}} \\
		& =\left(\frac{1}{2 \pi}\right)^n \cdot e^{-\frac{5 n \theta^2}{2}} \cdot e^{ \theta \cdot\left(\sum_{i=1}^n\left(X_i+2 Y_i\right)\right)} \cdot e^{-\frac{1}{2} \sum_{i=1}^n\left(X_i^2+Y_i^2\right)}
	\end{aligned}
	\]
	显然自然参数空间有内点,则$T(\boldsymbol{X})=\sum\limits_{i=1}^n \left(X_i+2Y_i \right) $为充分完全统计量\\
	注意独立性,有
	\[
	T(\boldsymbol{X})\sim N(5n\theta,5n)
	\]
	则有无偏估计
	\[
	\frac{T(\boldsymbol{X})}{5n}=\frac{\overline{X}+2\overline{Y}}{5}
	\]
	它也是充分完全统计量的函数,则其为UMVUE\\
	(2)取
	\[
	\frac{\frac{\overline{X}+2\overline{Y}}{5}-\theta}{\frac{1}{\sqrt{5n}}} \sim N(0,1)
	\]
	作为枢轴变量\\
	则区间长度
	\[
	\frac{2u_{\alpha/2}}{\sqrt{5n}}\le d
	\]
	得到
	\[
	n \geq\left\lceil\frac{4 u_{\alpha/2}^2}{5 d^2}\right\rceil
	\]\\
	
	
	
	
	\noindent 3.随机地从一批铁钉中抽取 16 枚,测得它们的平均长度(单位: cm$) \bar{x}=3.035$ .已知铁钉长度 $X$ 服从正态分布 $N\left(\mu, 0.1^2\right)$ .考虑假设检验问题
	\begin{equation}
		H_0: \mu \leq 3 \longleftrightarrow H_1: \mu > 3 \tag{$\star$}
	\end{equation}
	(1)在 $\alpha=0.05$ 水平下对检验问题 $(\star)$ 进行检验,并给出检验的 $p$ 值。\\
	(2)如果行动 $a=0$ 和 $a=1$ 分别表示接受 $H_0$ 和拒绝 $H_0, ~ \mu$ 的先验分布为 $N\left(3,0.1^2\right)$ ,损失函数取为
	$$
	L(a, \mu)= \begin{cases}11, & a=1, \mu \leq 3, \\ 1, & a=0, \mu>3, \\ 0, & \text { 其他. }\end{cases}
	$$
	基于后验风险最小原则给出检验问题$(\star)$的最优决策行动,并与(1)中检验结论进行对比。\\
	解:(1)给出检验统计量
	$$U(\boldsymbol{X})=\frac{\bar{X}-\mu_0}{\sigma / \sqrt{n}} \stackrel{H_0}{\sim} N(0,1)$$
	由检验问题的形式知拒绝域为 
	$$D=\left\{\boldsymbol{X}: U(\boldsymbol{X})>u_\alpha\right\}$$
	现有 $\bar{x}=3.035, \mu_0=3, \sigma=0.1, n=16$ ,查表得 $u_{0.05}=1.645$ ,因此
	$$
	u(\boldsymbol{x})=\frac{3.035-3}{0.1 / \sqrt{16}}=1.4<1.645
	$$
	因此不能拒绝原假设.\\
	检验的 $p$ 值为 $$p=P(U(\boldsymbol{X})>u(\boldsymbol{x}))=1-\Phi(1.4)=0.0808$$
	(2)在先验分布 $\mu \sim N\left(\mu_0, \tau^2\right)$ 下,正确计算出 $\mu$ 的后验分布
	$$
	\mu \left\lvert\, \boldsymbol{x} \sim N\left(\frac{n \tau^2 \bar{x}+\sigma^2 \mu_0}{n \tau^2+\sigma^2}, \frac{\tau^2 \sigma^2}{n \tau^2+\sigma^2}\right) \stackrel{\text { 代入数据 }}{=} N\left(\frac{1289}{425}, \frac{1}{1700}\right) .\right.
	$$
	后验风险 
	$$R\left(a_0, \mu\right)=\mathbb{P}(\mu>3 \mid \boldsymbol{x})=0.9131\quad R\left(a_1, \mu\right)=11 \mathbb{P}(\mu \leq 3 \mid \boldsymbol{x})=0.9559$$
	结论:按后验风险最小原则,应采取行动 $a_0$ ,即接受原假设 $H_0$ 。\\
	这与(1)中检验结果一致,这是因为我们对拒绝 $H_0$ 赋予了较大的惩罚。\\
	
	
	
	\noindent 4.设总体 $X$ 的密度函数为
	$$
	f(x ; \theta)= \begin{cases}\theta, & 0<x<1, \\ 1-\theta, & 1 \leq x<2, \\ 0, & \text { 其他 },\end{cases}
	$$
	其中 $0<\theta<1$ 为未知参数.现从此总体中抽出一样本量为 $n$ 的样本 $X_1, \ldots, X_n$ .\\
	(1)试求参数 $\theta$ 的充分统计量,并说明它是否为完全统计量?\\
	(2)求假设检验问题 $H_0: \theta=\theta_0 \leftrightarrow H_1: \theta \neq \theta_0$ 的水平 $\alpha$ 似然比检验,其中 $0<\theta_0, \alpha<1$ 已知。\\
	(3)上述假设是否存在 UMPT?为什么?\\
	解:(1)样本联合密度
	\[
	f(x_1,\cdots,x_n;\theta)=\theta^{n_1}\cdot (1-\theta)^{n_2}\mathbb{I}_{\left\{0<X_{(1)}<X_{(n)}<2\right\}}
	\]
	其中
	\[
	n_1=\sum_{i=1}^{n} \mathbb{I}_{\left\{0<X_i<1 \right\}}\quad n_2=n-n_1
	\]
	则
	\[
	n_1\sim \mathrm{B}(n,\theta)
	\]
	将联合密度改写成指数族形式
	\begin{align*}
		f(x_1,\cdots,x_n;\theta)&=e^{n_1\ln \theta +n_2\ln (1-\theta)}\mathbb{I}_{\left\{0<X_{(1)}<X_{(n)}<2\right\}}  \\
		& =(1-\theta)^n \cdot e^{n_1\ln \frac{\theta}{1-\theta}}\cdot h(\boldsymbol{X})
	\end{align*}
	由因子分解定理和自然参数空间有内点,$T(\boldsymbol{X})=n_1$为充分完全统计量\\
	(2)对全空间$0<\theta<1$,易得
	\[
	\hat{\theta}_{MLE}=\frac{n_1}{n}
	\]
	则
	\[
	\sup\limits_{0<\theta<1} f(x_1,\cdots,x_n;\theta) = \left(\frac{n_1}{n}\right)^{n_1}\left(1-\frac{n_1}{n}\right)^{n-n_1}\mathbb{I}_{\left\{0<X_{(1)}<X_{(n)}<2\right\}}
	\]
	另一方面
	\[
	\sup\limits_{\theta=\theta_0} f(x_1,\cdots,x_n;\theta)=f(x_1,\cdots,x_n;\theta_0)
	\]
	则似然比为
	\[
	\lambda(\boldsymbol{X})=\frac{\sup\limits_{0<\theta<1} f(x_1,\cdots,x_n;\theta)}{\sup\limits_{\theta=\theta_0} f(x_1,\cdots,x_n;\theta)}=\left(\frac{n_1}{\theta_0 n}\right)^{n_1}\left(\frac{n-n_1}{\left(1-\theta_0\right) n}\right)^{n-n_1}
	\]
	则
	\[
	2\log \lambda(\boldsymbol{X})\xrightarrow{H_0}\chi^2(1)
	\]
	拒绝域为
	\[
	\left\{(X_1,\cdots,X_n):2\log \lambda(\boldsymbol{X})>\chi^2_1(\alpha) \right\}
	\]
	(3)设$\varphi_1$为假设检验问题
	\[
	H_0:\theta=\theta_0 \leftrightarrow H_1^{\prime}:\theta=\theta_1
	\]
	的UMPT,这里$\theta_1>\theta_0$\\
	则由N-P引理
	\[
	\varphi_1(\boldsymbol{x})= 
	\begin{cases}
		1 & f\left(x, \theta_1\right) / f\left(x, \theta_0\right)>c_1 \\ 
		r_1 & f\left( x, \theta_1 \right) /f\left(x, \theta_0\right)=c_1 \\ 
		0 & f\left(x, \theta_1\right) / f\left(x, \theta_0\right) <c_1
	\end{cases}
	\]
	同理对假设检验问题
	\[
	H_0:\theta=\theta_0 \leftrightarrow H_1^{\prime \prime}:\theta=\theta_2
	\]
	这里$\theta_2<\theta_0$\\
	有UMPT
	\[
	\varphi_2(\boldsymbol{x})= 
	\begin{cases}
		1 & f\left(x, \theta_2\right) / f\left(x, \theta_0\right)>c_2 \\ 
		r_2 & f\left( x, \theta_2 \right) /f\left(x, \theta_0\right)=c_2 \\ 
		0 & f\left(x, \theta_2\right) / f\left(x, \theta_0\right) <c_2
	\end{cases}
	\]
	另一方面
	\[
	\begin{aligned}
		f(\vec{x},\theta_1)/f(\vec{x},\theta_0)&=\left(\frac{\theta_1}{\theta_0} \right)^{n_1} \left(\frac{1-\theta_1}{1-\theta_0}\right)^{n-n_1}\\
		&=\left(\frac{1-\theta_1}{1-\theta_0}\right)^n \quad\left(\frac{\theta_1\left(1-\theta_0\right)}{\theta_0\left(1-\theta_{1}\right)}\right)^{n_1} 
	\end{aligned}
	\]
	关于$n_1$单调递增\\
	则$\varphi_1(\boldsymbol{x})$可以改写为
	\[
	\varphi_1(\boldsymbol{x})= \begin{cases}1 & n_1>c_1 \\ r_1 & n_1=c_1 \\ 0 & n_1<c_1\end{cases}
	\]
	同理
	\[
	f(\vec{x},\theta_2)/f(\vec{x},\theta_0)=\left(\frac{1-\theta_2}{1-\theta_0}\right)^n \cdot\left(\frac{\theta_2-\theta_2 \theta_0}{\theta_0-\theta_0 \theta_2}\right)^{n_1}
	\]
	关于$n_1$单调递减\\
	则$\varphi_1(\boldsymbol{x})$可以改写为
	\[
	\varphi_2(\boldsymbol{x})= \begin{cases}1 & n_1<c_2 \\ r_2 & n_1=c_2 \\ 0 & n_1>c_2\end{cases}
	\]
	如果假设检验问题
	\[
	H_0: \theta=\theta_0 \leftrightarrow H_1: \theta \neq \theta_0
	\]
	的UMPT存在,记为$\varphi_0(\boldsymbol{x})$\\
	则由N-P引理的唯一性\\
	\[
	\begin{array}{ll}
		\theta_1>\theta_0 \text { 时} & \varphi_1=\varphi_0 \quad a.e.\\
		\theta_2<\theta_0 \text { 时 } & \varphi_2=\varphi_0 \quad a.e.
	\end{array}
	\]
	但这与$\varphi_1,\varphi_2$的形式矛盾,也就不存在假设检验问题 $H_0: \theta=\theta_0 \leftrightarrow H_1: \theta \neq \theta_0$的UMPT\\
	
	
	
	
	\noindent 附表:$u_{0.025}=1.960, u_{0.05}=1.645, \chi_{899}^2(0.025)=984, \chi_{899}^2(0.975)=817.8$ .
\end{document}
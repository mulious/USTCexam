\documentclass[UTF8]{ctexart}
%\documentclass{article}
\usepackage{graphicx,amsfonts,amsmath,mathrsfs,amssymb,amsthm,url,color}
\usepackage{fancyhdr,indentfirst,bm,enumerate,natbib, float,tikz,graphicx}
\usepackage{caption}
\usepackage{subcaption}
\usepackage{calligra}

\title{24数理统计期中}
\author{\textcalligra{NULIOUS}}
\date{}

\textheight 23cm
\textwidth 16.5cm
\topmargin -1.2cm
\oddsidemargin 0cm
\evensidemargin 0cm

\begin{document}

\maketitle
\noindent 一.(20分)设从总体
$$
\begin{tabular}{c|ccc}
	X & 0 & 1 & 2 \\
	\hline$P$ & $(1-\theta) / 3$ & $1 / 3$ & $(1+\theta) / 3$
\end{tabular}
$$
(其中 $-1<\theta<1$ 为未知参数)中抽取的一个简单样本 $X_1, \ldots, X_n$ \\
(1)求 $\theta$ 的充分统计量,其是否为完全统计量?\\
(2)求 $\theta$ 的矩估计 $\tilde{\theta}$ 和最大似然估计 $\hat{\theta}$ ,是否为无偏估计?\\
解:(1)设 $n_0, n_1, n_2$ 分别为 $\{x_n\}$ 中为取值为 $1,2,3$ 的个数,则 
$$ n=n_0+n_1+n_2$$
样本联合密度为
$$
f\left(x_1 \cdots, x_n ; \theta\right)=\left(\frac{1-\theta}{3}\right)^{n_0}\left(\frac{1}{3}\right)^{n_1}\left(\frac{1+\theta}{3}\right)^{n_2}
$$
注意这是指数族,改写为
$$
f\left(x_1 \cdots, x_n ; \theta\right)=\left(\frac{1}{3}\right)^{n_1} e^{n_0 \ln \frac{1-\theta}{3}+n_2 \ln \frac{1+\theta}{3}}
$$
即 $(\ln\frac{1-\theta}{3}, \ln \frac{1+\theta}{3})$ 作为自然参数,显然其在 $\mathbb{R}^2$ 中有内点\\
则$T(\boldsymbol{X})=\left(n_0, n_2\right)$ 是充分完全统计量\\
(2)矩估计:
$$
\mathbb{E}[X]=1+\frac{2 \theta}{3} 
$$
令 $$\hat{\theta}_M=\frac{3}{2}(\overline{X}-1)$$
即可,显然无偏。\\
MLE:
$$
	 \ln f\left(x_1, \cdots, x_n ; \theta\right)=n_1 \ln \frac{1}{3}+n_0 \ln \frac{1-\theta}{3}+n_2 \ln \frac{1+\theta}{3} 
$$
$$
	 \frac{\partial \ln f}{\partial \theta}=\frac{-n_0}{1-\theta}+\frac{n_2}{1+\theta}=0 
$$
$$
	 \hat{\theta}_{MLE}=\frac{n_2-n_0}{n_2+n_0} 
$$
用 $n=1 \quad \theta=0.5$ 验证知非无偏。\\



\noindent 二.(20分)一个移动通讯公司随机抽取了其 900 个包月客户,计算得知他们一个月平均使用时间是 220 分钟,样本标准差是 90 分钟.假设使用时间服从正态分布.\\
(1)求包月客户平均使用时间和标准差的 $95 \%$ 置信区间,并解释所得区间的含义.\\
(2)如果要求客户平均使用时间的 $95 \%$ 置信区间的长度不超过 5 分钟,应至少抽取多少个客户?该公司的抽样规模是否满足要求?\\
解:(1)$\mu, \sigma^2$ 均未知\\
利用 
$$\frac{\sqrt{n}(\overline{X}-\mu)}{S} \rightarrow t_{n-1}$$
估计 $\mu$ ,进而有置信区间
\[
\left[ \overline{X} - t_{n-1}\left(\frac{\alpha}{2}\right) \frac{S}{\sqrt{n}},\ 
\overline{X} + t_{n-1}\left(\frac{\alpha}{2}\right) \frac{S}{\sqrt{n}} \right]
\]
代值有
$$[214.12,225.885]$$ 
这里因为$t_n\rightarrow N(0,1)$ ,用$u_{\frac{\alpha}{2}}$代替$t_{n-1}\left(\frac{\alpha}{2}\right)$\\
再利用 
$$\frac{(n-1) S^2}{\sigma^2} \rightarrow \chi^2_{(n-1)}$$
$\sigma^2$ 有置信区间 
$$\left[\frac{(n-1) S^2}{\chi^2_{(n-1)}\left(\frac{\alpha}{2}\right)}, \frac{(n-1) S^2}{\chi^2_{(n-1)}\left(1-\frac{\alpha}{2}\right)}\right]$$
开方有 $\sigma$ 的置信区间
$$
\left[\sqrt{\frac{(n-1) S^2}{\chi^2_{(n-1)}\left(\frac{\alpha}{2}\right)}},\sqrt{\frac{(n-1) S^2}{\chi^2_{(n-1)}\left(1-\frac{\alpha}{2}\right)}}\right]
$$
代值为
\[
[88.15,92.02]
\]
区间的含义:使用这个区间充分大次数后,落在置信区间的频率接近于置信系数\\
(2)即需要
\[
2\times u_{\frac{\alpha}{2}}\frac{S}{\sqrt{n}}\le 5
\]
这里同样因为$t_n\rightarrow N(0,1)$ ,用$u_{\frac{\alpha}{2}}$代替$t_{n-1}\left(\frac{\alpha}{2}\right)$解得
\[
n\ge 4979
\]
不满足要求\\




\noindent 三.(20分)下表统计了某铁路局 122 个扳道员五年内由于操作失误引起的严重事故情况,其中 $r$ 表示一扳道员某五年内引起严重事故的次数,$s$ 表示扳道员人数.假设扳道员由于操作失误在五年内所引起的严重事故的次数服从Poisson分布.求
$$
\begin{tabular}{c|ccccccc}
	\hline$r$ & 0 & 1 & 2 & 3 & 4 & 5 & $\geq 6$ \\
	\hline$s$ & 44 & 42 & 21 & 9 & 4 & 2 & 0 \\
	\hline
\end{tabular}
$$
(1)一个扳道员在五年内未引起严重事故的概率 $p$ 的最小方差无偏估计 $\hat{p}_1$ 和最大似然估计 $\hat{p}_2$ .\\
(2)$p$ 的一个(渐近) $95 \%$ 水平的置信上界.\\
解:(1)下面记Poisson分布的参数为$\lambda$\\
MLE:样本联合密度为
\[
f\left(x_1, \cdots, x_n ; \lambda\right)=\frac{e^{-n \lambda} \lambda^{x_1+\cdots+x_n}}{x_{1}!\cdots x_{n}!}
\]
进而
\[
\ln f\left(x_1, \cdots, x_n ; \lambda\right) \propto \left(\sum_{i=1}^n x_i  \right) \ln \lambda -n\lambda
\]
则令
\[
\frac{\partial \ln f}{\partial \lambda}=\frac{\sum\limits_{i=1}^n x_i}{\lambda} -n=0
\]
解得
\[
\hat{\lambda}_{MLE}=\frac{\sum\limits_{i=1}^n x_i}{n}
\]
由于MLE的不变性有
\[
\hat{p_2}=\hat{e^{-\lambda}}_{MLE}=e^{-\frac{\sum\limits_{i=1}^n x_i}{n}}=0.325
\]
UMVUE:先找一个无偏估计,又由于Poisson分布是指数族,充分完全统计量是明显的,即$T(\boldsymbol{X})=\sum\limits_{i=1}^n X_i$,对无偏估计取充分完全统计量的条件期望即得UMVUE\\
选取
\[
\mathbb{I}_{\{X_1=0\}}
\]
作为无偏估计,因为$\mathbb{E}\left[\mathbb{I}_{\{X_1=0\}} \right]=P(X_1=0) $\\
下一步取条件期望
\begin{align*}
	\mathbb{E}\left[\mathbb{I}_{\{X_1=0\}}\mid T(\boldsymbol{X})=t \right]&=P\left(X_1=0\mid T(\boldsymbol{X})=t \right) \\
	&=\frac{P\left(X_1=0,\sum\limits_{i=2}^n X_i=t \right) }{P(T(\boldsymbol{X})=t)}
\end{align*}
利用Poisson分布的可加性,有
\[
\sum\limits_{i=2}^n X_i \sim \mathrm{P}((n-1)\lambda) \quad \sum\limits_{i=1}^n X_i \sim \mathrm{P}(n\lambda)
\]
则
\[
	\mathbb{E}\left[\mathbb{I}_{\{X_1=0\}}\mid T(\boldsymbol{X})=t \right]=\left(\frac{n-1}{n} \right)^t 
\]
即UMVUE为
\[
\hat{p_1}=\left(\frac{n-1}{n} \right)^{\sum\limits_{i=1}^n X_i}
\]
代值为
\[
\hat{p_1}=0.325
\]
(2)解一:利用MLE的渐进正态性
\[
\sqrt{n}\left(\hat{\lambda}_{MLE}-\lambda \right) \rightarrow N\left(0,\frac{1}{I(\lambda)} \right) 
\]
其中$I(\lambda)$为总体(单个样本)的Fisher信息量,为
\[
I(\lambda)=-\mathbb{E}\left[\frac{\partial^2 \ln f(x,\lambda)}{\partial \lambda^2} \right] =\frac{1}{\lambda}
\]
这里
\[
f(x,\lambda)=P(X=x,\lambda)=\frac{\lambda^x}{x!}e^{-\lambda}
\]
得到
\[
\sqrt{n}\left(\hat{\lambda}_{MLE}-\lambda \right) \rightarrow N\left(0,\lambda \right) 
\]
法一:因为$e^{-\lambda}$单调递减,因此只需要求出$\lambda$的置信下界即可
\[
\frac{\sqrt{n}\left(\hat{\lambda}_{MLE}-\lambda \right)}{\sqrt{\lambda}}\le u_{\alpha}
\]
处理一:直接解一元二次方程,较复杂\\
处理二:利用MLE做二次近似\\
用$\sqrt{\overline{X}}$代替$\sqrt{\lambda}$
\[
\frac{\sqrt{n}\left(\hat{\lambda}_{MLE}-\lambda \right)}{\sqrt{\overline{X}}}\le u_{\alpha}
\]
得到
\[
\lambda \ge \overline{X}-\frac{\sqrt{\overline{X}}u_{\alpha}}{\sqrt{n}}
\]
则$e^{\lambda}$的置信上界为
\[
e^{-\overline{X}+\frac{\sqrt{\overline{X}}u_{\alpha}}{\sqrt{n}}}
\]
法二:使用$\Delta$方法,取$g(x)=e^{-x}$,则
\[
\sqrt{n}\left(\hat{e^{-\lambda}}_{MLE}-e^{-\lambda} \right) \rightarrow N\left(0,e^{-2\lambda}\lambda \right) 
\]
即
\[
\frac{\sqrt{n}\left(\hat{e^{-\lambda}}_{MLE}-e^{-\lambda} \right)}{\sqrt{\lambda e^{-2\lambda}}}\rightarrow N(0,1)
\]
仍类似处理二,利用MLE做二次近似
\[
\frac{\sqrt{n}\left(\hat{e^{-\lambda}}_{MLE}-e^{-\lambda} \right)}{\sqrt{\overline{X} e^{-2\overline{X}}}}\ge u_\alpha
\]
解得置信上界为
\[
 e^{-\overline{X}}-\frac{\sqrt{\overline{X} e^{-2 \overline{X}}} u_\alpha}{\sqrt{n}}
\]
解二:利用CLT
\[
\sqrt{n}\left(\overline{X}-\lambda \right)\rightarrow N(0,\lambda) 
\]
余下解法同解一\\
解三:把$p$看作成功概率,则问题变为求两点分布的参数$p$的置信上界
\[
Y_i=\mathbb{I}_{\{X_1=0\}}\sim \mathrm{B}(1,p)
\]
则有
\[
\frac{\sqrt{n}(\overline{Y}-p)}{\sqrt{\overline{Y}(1-\overline{Y})}} \xrightarrow{D} N(0,1)
\]
直接可以得到$p$的置信上界\\




\noindent 四.(15分)设 $X_1, \ldots, X_n$ 为来自正态总体 $N\left(1, \sigma^2\right)$ 一组简单样本,$\sigma^2>0$ 为参数.试\\
(1)求 $\sigma^2$ 的最小方差无偏估计 $\hat{\sigma}^2$ ,其是否达到Cramer-Rao下界?\\
(2)给出一个比最小方差无偏估计 $\hat{\sigma}^2$ 在均方误差准则下更优的估计.\\
解:(1)$N\left(1, \sigma^2\right)$是指数族,样本联合密度为
\[
f\left(x_1, \cdots, x_n ; \sigma^2\right)=\left(\frac{1}{\sqrt{2 \pi \sigma^2}}\right)^n \cdot e^{-\frac{\sum\limits_{i=1}^n \left(x_{i}-1\right)^2}{2 \sigma^2}}
\]
显然自然参数空间有内点,进而
\[
T(\boldsymbol{X})=\sum\limits_{i=1}^n \left(X_{i}-1\right)^2
\]
为$\sigma^2$的充分完全统计量\\
注意当$\mu=1$已知的时候,$\sigma^2$有无偏估计
\[
\frac{\sum\limits_{i=1}^n\left(X_i-\mu\right)^2}{n}=\frac{\sum\limits_{i=1}^n\left(X_i-1\right)^2}{n}
\]
它正是充分完全统计量的函数,因此UMVUE就是
\[
\hat{\sigma^2}=\frac{\sum\limits_{i=1}^n\left(X_i-1\right)^2}{n}
\]
对C-R下界,注意$N\left(1, \sigma^2\right)$是单参数指数族,且UMVUE为充分完全统计量的线性函数,那么UMVUE可以达到C-R下界\\
下面进行数值验证:$n$个样本的Fisher信息量为
\[
\begin{aligned}
	I\left(\sigma^2\right) & =-\mathbb{E}\left[\frac{\partial^2}{\partial \left(\sigma^2\right)^2} \ln f\left(x_1, \cdots, x_n ; \sigma^2\right)\right] \\
	& =\frac{n}{2 \sigma^4} 
\end{aligned}
\]
因此C-R下界为
\[
\frac{1}{I\left(\sigma^2\right)}=\frac{2 \sigma^4}{n} 
\]
另外一方面
\[
\operatorname{Var}\left(\hat{\sigma^2}\right)=\operatorname{Var}\left(\frac{T(\boldsymbol{X})}{n}\right)=\frac{1}{n^2} \operatorname{Var}(T(\boldsymbol{X}))
\]
注意
\[
\frac{\sum\limits_{i=1}^n\left(X_i-\mu\right)^2}{\sigma^2} \sim \chi^2(n)
\]
即
\[
\operatorname{Var}\left(\frac{T(\boldsymbol{X})}{\sigma^2}\right)=2n
\]
也就是
\[
\operatorname{Var}\left(\hat{\sigma^2}\right)=\frac{2 \sigma^4}{n} 
\]
达到C-R下界\\
(2)这时我们不要求无偏性\\
对估计$\hat{\theta}$,均方误差
\[
\begin{aligned}
	\operatorname{MSE}(\hat{\theta}) & =\mathbb{E}\left[(\hat{\theta}-\mathbb{E}[\hat{\theta}])^2\right]+(\mathbb{E}[\hat{\theta}]-\theta)^2 \\
	& =\operatorname{Var}(\hat{\theta})+(\mathbb{E}[\hat{\theta}]-\theta)^2
\end{aligned}
\]
考虑
\[
\tilde{\sigma^2}=c\hat{\sigma^2}
\]
则
\[
\begin{aligned}
	\operatorname{MSE}\left(\widetilde{\sigma}^2\right) & =\operatorname{Var}\left(c \hat{\sigma^2}
	\right)+\left(c \sigma^2-\sigma^2\right)^2 \\
	& =c^2 \cdot \frac{2 \sigma^2}{4}+\left(c \sigma^2-\sigma^2\right)^2
\end{aligned}
\]
对$c$求导数并令其为0
\[
\frac{4 c \sigma^4}{n}+2(c-1) \sigma^4=0
\]
得到
\[
c=\frac{n}{n+2}
\]
即
\[
\tilde{\sigma^2}=\frac{1}{n+2} \sum_{i=1}^n\left(X_i-1\right)^2 
\]
且满足$\operatorname{MSE}\left(\tilde{\sigma^2}\right)<\operatorname{MSE}\left(\hat{\sigma^2}\right)$\\







\noindent 五.(25分)设 $X_1, \ldots, X_n$ 为来自如下指数总体的简单样本,总体密度函数为
$$
f(x , a)=e^{-(x-a)} I(x \geq a),-\infty<a<1
$$
其中 $a$ 为未知参数.试\\
(1)求 $a$ 的最大似然估计,并讨论其相合性和极限分布.\\
(2)证明 $T=X_{(1)}$ 为 $a$ 的充分统计量但不是完全统计量.\\
(3)求 $a$ 的最小方差无偏估计.\\
解:(1)样本联合密度为
\[
f\left(x_1, \cdots x_n ; a\right)=e^{-\sum_{i=1}^n x_i} \cdot e^{n a} \cdot \mathbb{I}_{\left\{x_{(1)} \geq a\right\}}
\]
由单调性
\[
\hat{a}_{MLE}=X_{(1)}
\]
$X_{(1)}$有密度
\[
f(x)=ne^{-n(x-a)},x\ge a
\]
有分布函数
\[
F(x)=1-e^{-n(x-a)}
\]
进一步
\[
P\left(\left|\hat{a}_{M L E}-a\right| \geqslant \varepsilon\right)=P(X_{(1)} \geqslant a+\varepsilon)=e^{-n \varepsilon} \rightarrow 0
\]
这就说明了弱收敛\\
\textbf{RK}:进一步的
\[
\sum\limits_{i=1}^{\infty} P\left(\left|\hat{a}_{M L E}-a\right| \geqslant \varepsilon\right)<\infty
\]
由B-C引理可知强收敛\\
极限分布:
\[
X_i-a\sim \mathrm{Exp}(1)\triangleq Y_i
\]
而
\[
Y_{(1)}\sim \mathrm{Exp}(n)
\]
则
\[
n(X_{(1)}-a)\sim \mathrm{Exp}(1)
\]
(2)由因子分解定理知$T(\boldsymbol{X})=X_{(1)}$充分\\
下面证明它不是完全的,即存在$\phi(T)$使得$\mathbb{E}[\phi(T)]=0$但是$\phi(T)$不恒为0\\
条件
\[
\begin{aligned}
	\mathbb{E}[\phi(T)]&=\int_a^{+\infty} \phi(t) \cdot n e^{-n(t-a)} d t\\
	&=\int_a^{1} \phi(t) \cdot n e^{-n(t-a)} d t+\int_1^{+\infty} \phi(t) \cdot n e^{-n(t-a)} d t\\
	&=0
\end{aligned}
\]
求导有
\[
\phi(t)=0\quad \forall t<1
\]
下面构造$t\ge 1$的部分,把积分分段成有限和无限的两段,这两段都不能为0,且两段的积分之和为0,去掉常数部分,只需要满足
\[
\int_1^{c} \phi(t) \cdot  e^{-nt} d t=-\int_c^{+\infty} \phi(t) \cdot  e^{-nt} d t
\]
不妨就设$c=2$且$\phi(t)=1,t\ge 2$,则
\[
\int_2^{+\infty} \phi(t) \cdot  e^{-nt} d t=\frac{e^{-2n}}{n}
\]
另一方面
\[
\int_1^{2}    e^{-nt} d t=\frac{-e^{-2n}+e^{-n}}{n}
\]
那么只需要令
\[
\phi(t)=
\begin{cases}
	0 & a\le t<1\\
	1-e^{nt}\cdot \frac{e^{-n}}{n}   &  1\le t\le 2\\
	1  &  t>2
\end{cases}
\]
就构造出了这个反例,进而说明$T(\boldsymbol{X})=X_{(1)}$ 为 $a$ 的充分统计量但不是完全统计量\\
\textbf{RK}:构造的$\phi(T)$不应该与未知参数有关\\
另外地,对于包含无穷长区间的分布都可以类似地使用这个办法,将有限部分和无限部分分成两段再构造反例(这么做是为了让任何无限长的区间内$\phi(t)$不为0,否则$\phi(t)$仍然有可能以概率1地为0,如平均地从$\mathbb{R}$中取得区间$(0,1)$中的实数的概率为0)\\
(3)对于充分但不完全的统计量,用零无偏法,注意参数取值范围$a<1$\\
设$\mathbb{E}[\delta(T)]=0$,且$h(T)$为所求的UMVUE,则有
\[
\mathbb{E}[\delta(T)]=\int_a^{+\infty} \delta(t) \cdot n e^{-n(t-a)} d t=0 
\]
也就是
\[
\int_a^{+\infty} \delta(t) \cdot  e^{-nt} d t=0 
\]
即
\[
\int_a^{1} \delta(t) \cdot  e^{-nt} d t+\int_1^{+\infty} \delta(t) \cdot  e^{-nt} d t=0 
\]
求导有
\[
\delta(t)=0\quad \forall t \le 1
\]
另一方面$\mathbb{E}\left[\delta(T)h(T) \right]=0 $,即
\[
\int_a^{1} \delta(t)h(t)f(t)d t+\int_1^{+\infty} \delta(t)h(t)f(t) d t=0
\]
也就是
\[
\int_1^{+\infty} \delta(t)h(t)f(t) d t=0
\]
为了满足这个要求,待定$h(T)=c,T>1$\\
又需要无偏性,即
\[
\mathbb{E}[h(T)]=a
\]
待定
\[
h(t)=
\begin{cases}
	bt+d  &  a\le t \le 1\\
	c  &  t>1
\end{cases}
\]
对积分逐项计算得到
\[
b=1\quad d=-\frac{1}{n}\quad c=-1
\]
最后
\[
h(t)=
\begin{cases}
	t-\frac{1}{n}  &  a\le t \le 1\\
	-1  &  t>1
\end{cases}
\]\\







\noindent 附表:上分位数
$$
u_{0.025}=1.960, \quad u_{0.05}=1.645, \quad \chi_{899}^2(0.025)=984, \quad \chi_{899}^2(0.975)=817.8
$$

\end{document}
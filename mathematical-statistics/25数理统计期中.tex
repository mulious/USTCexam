\documentclass[12pt]{article}
\usepackage{amsmath, amssymb}
\usepackage[utf8]{inputenc}
\usepackage{ctex}
\usepackage{graphicx} % 需要添加这个包来插入图片
\usepackage{mathrsfs}
\usepackage{calligra} % 添加书法字体包

\title{25数理统计期中}
\author{\textcalligra{NULIOUS}} % 改用书法体
\date{}
\begin{document}

\maketitle

\noindent 一.填空选择题(每空两分)\\
(1)设 $X_{1}, X_{2}, \ldots, X_{n}, X_{n+1}$ 为来自同一正态总体的一组简单随机样本,且记 $\overline{X}=$ $\frac{1}{n} \sum_{i=1}^{n} X_{i}$ 及 $S^{2}=\frac{1}{n-1} \sum_{i=1}^{n}\left(X_{i}-\overline{X}\right)^{2}$ 。若统计量 $c_{n}\left(X_{n+1}-\overline{X}\right) / S$ 服从 $t$ 分布,则常数 $c_{n}=$\underline{\hspace{1cm}} ,$t$ 分布的自由度为\underline{\hspace{1cm}},且与 $\sum_{i=1}^{n+1} X_{i}$ 的相关系数为\underline{\hspace{1cm}}\\
答案:$(\pm) \sqrt{\frac{n}{n+1}} ; n-1 ; 0$\\
首先对于正态分布,$\overline{X}$与$S^2$是独立的,这说明了$t$分布的分子分母的独立性\\
由$t$分布的定义:
\begin{gather}
t_{n-1}=\frac{N(0,1)}{\sqrt{\frac{\chi^2(n-1)}{n-1}}}
\end{gather}
再有\\
\begin{gather}
\frac{(n-1) s^2}{\sigma^2} \sim \chi^2(n-1) \\
X_{n+1} - \overline{X} \sim N\left(\mu, \sigma^2\right) - N\left(\mu, \frac{\sigma^2}{n}\right) = N\left(0, \frac{n+1}{n} \sigma^2\right)
\end{gather}
(3)式来自两个变量的独立性\\
则\\
\begin{gather}
\sqrt{\frac{n}{n+1}}\left(X_{n+1}-\overline{X}\right) / S \sim t_{n-1}
\end{gather}
特别的,正负号来自$t$分布是对称的(没有负号也没算错)\\
数理统计涉及独立性的几乎只有$Basu$定理一个,猜测它们独立,即相关系数为$0$\\
把$\left(X_1, \cdots, X_{n+1}\right)$ 视为样本,则由于指数族的性质,关于 $\lambda$ 有充分完全统计量$T=\sum_{i=1}^{n+1} X_i$\\
我们不考虑常数部分,则\\
\begin{gather}
\frac{\left(X_{n+1}\right)-\left(\overline{X}\right)}{S}=\frac{\left(X_{n+1}-\mu\right)-\left(\overline{X}-\mu\right)}{S}
\end{gather}
是与$\mu$无关的统计量(即辅助量),因此由$Basu$定理它们独立,进而相关系数为$0$\\
(2)设统计量 $\hat{\theta}$ 为总体参数 $\theta$ 的一个点估计,下列说法一般不成立的是\underline{\hspace{1cm}}\\
(A)若 $\hat{\theta}$ 为 $\theta$ 的矩估计,则 $\hat{\theta}^{2}$ 为 $\theta^{2}$ 的矩估计\\
(B)若 $\hat{\theta}$ 为 $\theta$ 的最大似然估计,则 $\hat{\theta}^{2}$ 为 $\theta^{2}$ 的最大似然估计\\
(C)若 $\hat{\theta}$ 为 $\theta$ 的无偏估计,则 $\hat{\theta}^{2}$ 为 $\theta^{2}$ 的无偏估计\\
(D)若 $\hat{\theta}$ 为 $\theta$ 的相合估计,则 $\hat{\theta}^{2}$ 为 $\theta^{2}$ 的相合估计\\
答案:C\\
一般一个随机变量的二阶矩不等于其一阶矩的平方,因此$C$错误\\
(3)如果极小充分统计量存在,那么充分完全统计量必是极小充分统计量,但是极小充分统计量不一定是完全的。这种说法\underline{\hspace{1cm}}\\
(A)正确\\
(B)错误\\
答案:A\\
(4)设 $X_{1}, \cdots, X_{n}$ 为来自于正态总体 $N(\mu, 1)$ 的简单随机样本,若要求参数 $\mu$ 的置信系数为 $95 \%$ 的置信区间长度不超过 1 ,则至少需要抽取的样本量 $n$ 为 \underline{\hspace{1cm}}\\
(A) 14\\
(B) 16\\
(C) 18\\
(D) 20\\
答案:B\\
注意方差已知。则置信区间为$\left[\overline{X}-\frac{\sigma}{\sqrt{n}} u_{\frac{\alpha}{2}}, \overline{X}+\frac{\sigma}{\sqrt{n}} u_{\frac{\alpha}{2}}\right]$,带入数值$\sigma=1$即可\\
(5)在给定一组样本值和先验下,采用后验期望作为感兴趣参数 $\theta$ 的估计,得到估计值 $\hat{\theta}=5$ .下述说法正确的是\underline{\hspace{1cm}}\\
(A)在重复抽取样本意义下 $\theta$ 的无偏估计值为 1.5\\
(B)$\hat{\theta}=1.5$ 是 $\theta$ 的有效估计\\
(C)估计值 1.5 是最小后验均方误差估计\\
(D)估计值 1.5 是 $\theta$ 的相合估计\\
答案:C\\


\noindent 二。(16 分)随机调查了某保险公司 $n$ 个独立的车险索赔额 $X_{1}, \ldots, X_{n}$(单位:千元),得到如下样本直方图和正态 Q-Q 图。据此回答\\
(1)该样本来自的总体分布有何特点?可以选择什么分布作为总体分布?给出理由.\\
(2)试选择合适的参数统计模型,并讨论参数的充分完全统计量.\\
\begin{figure}[h]  
\centering  
\includegraphics[width=0.5\textwidth]{25期中.png}  
\label{fig:my_label}  
\end{figure}
\\
解:\\
(1)总体分布为单峰且峰偏左(右偏分布),且正态q-q图在第一象限对角线$y=x$的下方;我们可以选择$\Gamma$分布,卡方分布(也是一种$\Gamma$分布)等符合要求的分布\\
(2)以总体分布为$\Gamma$分布:$X \sim \Gamma(\alpha,\beta)$为例\\
样本$(X_1,...,X_n)$有联合密度:
\begin{gather}
	f(\vec{x};\alpha,\beta)=\frac{\beta^{n\alpha}}{\Gamma(\alpha)^n} \prod_{i=1}^{n} x_i^{\alpha-1} e^{-\beta \sum_{i=1}^{n} x_i}
\end{gather}
把密度写成指数族的形式:\\
\begin{gather}
f(\vec{x};\alpha,\beta)=\frac{\beta^{n\alpha}}{\Gamma(\alpha)^n} e^{(\alpha-1)\sum_{i=1}^n \ln(x_i)-\beta\sum_{i=1}^n x_i}
\end{gather}
自然参数分别为$\alpha-1$,$\beta$,自然空间显然有内点,同时由因子分解定理,有充分完全统计量$T=\left( \sum\limits_{i=1}^n \ln(X_i),\sum\limits_{i=1}^n X_i\right) $\\
特别的,若选择了卡方分布,则充分完全统计量为$T=\sum\limits_{i=1}^n \ln(X_i)$\\


\noindent 三.(20 分)设 $X_{1}, \ldots, X_{n}$ 为来自均匀总体 $\mathrm{U}(\theta, \theta+1)$ 的简单样本,其中 $\theta \in R$ 为未知参数.试\\
(1)证明 $T=\left(X_{(1)}, X_{(n)}\right)$ 为 $\theta$ 的极小充分统计量但不是完全统计量.\\
(2)求 $\theta$ 的最大似然估计,并讨论其相合性.\\
解:\\
(1)首先要利用因子分解定理证明$(X_{(1)},X_{(n)})$是充分统计量\\
样本联合密度:\\
\begin{gather}
f(\vec{x};\theta) = \mathbb{I}_{\theta < X_{(1)} < X_{(n)} < \theta + 1}
\end{gather}
则$h(\boldsymbol{X})=1$,$g(X_{(1)},X_{(n)};\theta)=\mathbb{I}_{\theta < X_{(1)} < X_{(n)} < \theta + 1}$\\
由因子分解定理可以知道$(X_{(1)},X_{(n)})$是充分统计量\\
下面再取相同总体中的$n$个样本$(Y_1,\dots,Y_n)$,构造似然比:
\begin{gather}
\begin{aligned}
	\frac{f(\vec{x};\theta)}{f(\vec{y};\theta)}=C(\vec{x},\vec{y})& \iff \frac{\mathbb{I}_{\theta < X_{(1)} < X_{(n)} < \theta + 1}}{\mathbb{I}_{\theta < Y_{(1)} < Y_{(n)} < \theta + 1}}=C(\vec{x},\vec{y}) \\
	&\iff (X_{(1)},X_{(n)})=(Y_{(1)},Y_{(n)})
\end{aligned}
\end{gather}
其中$C(\vec{x},\vec{y})$表示仅与$\vec{x}$,$\vec{y}$有关的常数,这就说明了$(X_{(1)},X_{(n)})$是极小充分统计量\\
下面通过充分统计量来构造辅助量(与$\theta$无关的统计量)来说明$T=(X_{(1)},X_{(n)})$不是完全统计量\\
设$Z_i=X_i-\theta \sim U(0,1)$,则
$$Z_{(n)}-Z_{(1)}=((X_{(n)}-\theta)-(X_{(1)}-\theta)) \sim \beta(n-1,2)$$
与$\theta$无关\\
上面的结论来自$U(0,1)$的极差分布为$\beta(n-1,2)$\\
取$a$,$b$使得
$$P(Z_{(n)}-Z_{(1)}>a)=P(Z_{(n)}-Z_{(1)}<b)>0$$
再取
$$\varphi(x)= \begin{cases}1, & x>a \\ 1, & x<b \\ 0, & \text { 其他 }\end{cases}$$
则$\mathbb{E}[\varphi(T)]=0$但是$\varphi(T)$显然不处处为$0$\\
则$T$不是完全统计量\\
\textbf{RK}:对于二元的充分统计量要说明其不是完全的,往往通过相减和相除构造辅助量,再取如$\varphi$这样的函数进行说明\\
(2)接下来求$\theta$的最大似然估计\\
由式(8)可以看出\\
\begin{gather}
f(\vec{x};\theta) = \begin{cases}
1, & X_{(n)}-1<\theta<X_{(1)}, \\
0, & \text{其他}.
\end{cases}
\end{gather}
则$\theta$的最大似然估计$\hat{\theta}_{MLE}$为$(X_{(n)}-1, X_{(1)})$中的任何值\\
下面利用$Markov$不等式证明其弱相合性\\
只需说明$tX_{(1)}+(1-t)(X_{(n)}-1)$对于$\theta$的相合性即可,其中$0<t<1$\\
$$
	\begin{aligned}
		P\Bigl(\bigl|tX_{(1)}+(1-t)(X_{(n)}-1)-\theta\bigr| \geq \epsilon\Bigr) \leq \frac{\mathbb{E}\bigl[\bigl|tX_{(1)}+(1-t)(X_{(n)}-1)-\theta\bigr|\bigr]}{\epsilon}& \\
		\leq \frac{\mathbb{E}\bigl[|tX_{(1)}-t\theta|\bigr] + \mathbb{E}\bigl[|(1-t)(X_{(n)}-1)-(1-t)\theta|\bigr]}{\epsilon}&
	\end{aligned}
$$
而
\begin{gather}
X_{(1)}-\theta \sim \beta(1,n), \\
X_{(n)}-\theta \sim \beta(n,1).
\end{gather}
则$\mathbb{E}[X_{(1)}] = \theta + \frac{1}{n+1}$,$\mathbb{E}[X_{(n)}] = \theta + \frac{n}{n+1}$\\
注意关系$\theta < X_{(1)} < X_{(n)} < \theta + 1$\\
代入$Markov$不等式后令$n\rightarrow \infty$即证弱收敛\\


\noindent 四.(25 分)某厂生产的产品分为三个质量等级 $(X=1,2,3)$ ,各等级产品的分布如下\\
\begin{center}
\begin{tabular}{c|ccc}
\hline
$X$ & 1 & 2 & 3 \\
\hline
$P$ & $\theta$ & $2 \theta$ & $1-3 \theta$ \\
\hline
\end{tabular}
\end{center}

其中 $\theta \in(0,1 / 3)$ 未知.为了解该厂产品的质量分布情况,从该厂产品中随机有放回抽取 20 件产品检测后发现一等品有 5 件,二等品有 7 件,三等品有 8 件.试\\
(1)求 $\theta$ 的矩估计和最大似然估计量,是否都为无偏估计?给出估计值.\\
(2)求 $\theta$ 的最小方差无偏估计量,其方差是否达到了 Cramér-Rao下界?\\
解:\\
(1)$\mathbb{E}[X]=3-4\theta$\\
则$\theta$的矩估计为$\hat{\theta}_{M}=\frac{3-\overline{X}}{4}$,它自然是无偏的(因为就是拿期望算出来的)\\
记$n_k=\sum_{i=1}^{n} \mathbb{I}_{X_i=k}$\\
则$(X_1,...,X_n)$有联合密度$$f(\vec{x};\theta)=\theta^{n_1}(2\theta)^{n_2}(1-3\theta)^{n_3}$$
则\\
\begin{gather}
\ln f(\vec{x};\theta)=n_1 \ln \theta+n_2\ln (2\theta) +n_3\ln(1-3\theta)\\
\frac{\partial \ln f\left(\vec{x} ; \theta\right)}{\partial \theta}=\frac{n_1}{\theta}+\frac{n_2}{\theta}-\frac{3n_3}{1-3\theta}=0
\end{gather}
有$$\hat{\theta}_{MLE}=\frac{n-n_3}{3n}$$
又$n_3 \sim B(n,1-3\theta)$\\
则$$\mathbb{E}[\hat{\theta}_{MLE}]=\frac{1}{3}-\frac{\mathbb{E}[n_3]}{3n}=\frac{1}{3}-\frac{n(1-3\theta)}{3n}=\theta$$
即$\hat{\theta}_{MLE}$为无偏估计\\
带入数值有$\hat{\theta}_{M}=0.2125$,$\hat{\theta}_{MLE}=0.2$\\
(2)化为自然指数族的形式\\
\begin{gather}
f(\vec{x};\theta)=e^{n_2\ln2}e^{n_1\ln\theta+n_2\ln\theta+n_3\ln(1-3\theta)}\\
=e^{n_2\ln2}e^{\ln\theta}e^{n_3\ln(\frac{1-3\theta}{\theta})}
\end{gather}
其中$h(\boldsymbol{X})=e^{n_2\ln2} \quad C(\theta)=e^{\ln\theta}=\theta$,自然参数为$\ln(\frac{1-3\theta}{\theta})$\\
又$\theta \in (0,\frac{1}{3})$,则自然参数空间有内点,且由因子分解定理,$T=n_3$为$\theta$的充分完全统计量\\
同时注意到$\hat{\theta}_{MLE}$无偏且为充分完全统计量的函数,它也就是$\theta$的$UMVUE$\\
注意$UMVUE$能达到$C-R$下界$\iff$分布为单参数指数族且$UMVUE$为充分完全统计量$T(\vec{x})$的线性函数\\
所以本题的$UMVUE$能达到$C-R$下界\\
下面额外给出数值验证:\\
\begin{gather}
\text{Var}(\hat{\theta}_{\text{MLE}}) = \frac{\text{Var}(n_3)}{9n^2} = \frac{3n\theta(1-3\theta)}{9n^2} = \frac{\theta(1-3\theta)}{3n}, \\
\begin{aligned}
	n \text{个样本的Fisher信息量:} I(\theta) &= -\mathbb{E}\left[ \frac{\partial^2}{\partial\theta^2} \ln f(\vec{x};\theta) \right]\\
	&=-\mathbb{E}[-\frac{n_1}{\theta^2}-\frac{n_2}{\theta^2}-\frac{9n_3}{(1-3\theta)^2}]\\
	&=\frac{3n}{\theta(1-3\theta)}
\end{aligned}
\end{gather}
对于$\theta$和$n$个样本的$Fisher$信息量$I(\theta)$,其$C-R$下界为$$\frac{1}{I(\theta)}=\frac{\theta(1-3\theta)}{3n}$$这就说明了$UMVUE$的方差能达到$C-R$下界\\
\textbf{RK}:对于$n$个样本的$Fisher$信息量$I(\theta)$,$g(\theta)$的$C-R$下界为$\frac{[g'(\theta)]^2}{I(\theta)}$\\
对于单个样本(总体)的$Fisher$信息量$\widetilde{I(\theta)}$,$g(\theta)$的$C-R$下界为$\frac{[g'(\theta)]^2}{n\widetilde{I(\theta)}}$\\
这实际上是因为$I(\theta)$是$\widetilde{I(\theta)}$的$n$倍,另外对于$Fisher$信息阵仍有相同的规律\\


\noindent 五.(25 分)调查发现人们每天使用手机的时间(单位:分钟)服从正态分布 $N\left(\mu, \sigma^{2}\right)$ ,其中 $\mu \in R, \sigma^{2}>0$ 为未知参数。现随机调查了 25 个人每天使用手机时间,得到样本均值 $\overline{X}=180$ 分钟,样本标准差 $S=20$ 分钟。若取先验分布为 $\pi\left(\mu, \sigma^{2}\right) \propto \sigma^{-2}$ .试\\
(1)求 $\sigma^{2}$ 的边际后验分布,并给出 $\sigma^{2}$ 的后验期望估计值.\\
(2)求一个人每天平均使用手机时长 $\mu$ 的 $95 \%$ 置信区间和可信区间,两者的解释有何不同?\\
解:\\
(1)先解释一下$\pi\left(\mu, \sigma^{2}\right) \propto \sigma^{-2}$的含义,他表示先验分布与$\mu$成常数倍的关系,但不意味着先验分布给出的\textbf{只有}$\sigma$的信息,因此按照这个先验分布算出来的后验分布是$\mu$与$\sigma$的联合分布,如果要求某一个参数的分布还需要对另一个参数进行积分\\
样本联合密度为:
\begin{gather}
f(\vec{x};\mu,\sigma^2)=(2\pi \sigma^2)^{-\frac{n}{2}}e^{-\frac{\sum_{i=1}^{n} (x_i-\mu)^2}{2\sigma^2}}
\end{gather}
后验联合密度为:
\begin{gather}
\pi(\mu,\sigma^2|\vec{x}) \propto f(\vec{x};\mu,\sigma^2)\cdot\pi\left(\mu, \sigma^{2}\right) \propto (\sigma^2)^{-\frac{n}{2}-1}e^{-\frac{\sum (x_i-\overline{X})^2 +n(\overline{X}-\mu)^2}{2\sigma^2}}
\end{gather}
其中用到了
$$\sum_{i=1}^{n} (x_i-\mu)^2=\sum_{i=1}^{n} (x_i-\overline{X})^2 +n(\overline{X}-\mu)^2$$
接下来对参数$\mu$做积分来得到$\sigma^2$的后验密度
\begin{gather}
	\begin{aligned}
		\pi(\sigma^2|\vec{x})&=\int_{-\infty}^{+\infty} \pi(\mu , \sigma^2|\vec{x})d\mu\\
		&=(\sigma^2)^{-\frac{n}{2}-1}e^{-\frac{\sum_{i=1}^n (x_i-\overline{X})^2}{2\sigma^2}} \int_{-\infty}^{+\infty}e^{-\frac{n(\overline{X}-\mu)^2}{2\sigma^2}}d\mu\\
		&=(\sigma^2)^{-\frac{n}{2}-\frac{1}{2}}e^{-\frac{\sum_{i=1}^n (x_i-\overline{X})^2}{2\sigma^2}} 
	\end{aligned}
\end{gather}
式(26)来自$N(\overline{X},\frac{\sigma^2}{n})$的密度的积分为1\\
即
\[
\int_{-\infty}^{+\infty}e^{-\frac{n(\overline{X}-\mu)^2}{2\sigma^2}}d\mu=(\frac{2\pi \sigma^2}{n})^{\frac{1}{2}}
\]
则
\[
\sigma^2|\vec{x} \sim \Gamma^{-1}(-\frac{n}{2}+\frac{1}{2},\frac{\sum (x_i-\overline{X})^2}{2})
\]
最后
\[
\hat{\sigma^2}_{E}=\frac{\sum\limits_{i=1}^{n} (x_i-\overline{X})^2}{2(-\frac{n}{2}-\frac{1}{2})}=480
\]
\textbf{RK}:在矩存在的条件下有\\
\[
X \sim \Gamma(\alpha,\beta) \quad \text{有} \forall n \in \mathbb{Z} \quad \mathbb{E}[X^n]=\frac{\Gamma(\alpha+n)}{\Gamma(\alpha)\beta^n}
\]
\[
Y \sim \Gamma^{-1}(\alpha,\beta) \quad \text{有} \forall n \in \mathbb{Z} \quad \mathbb{E}[Y^n]=\frac{\Gamma(\alpha-n)\beta^n}{\Gamma(\alpha)}
\]
事实上,$\Gamma$分布的$n$阶矩就是$\Gamma^{-1}$分布的$-n$阶矩\\
(2)\\
先解释一下置信区间与可信区间的区别:\\
置信区间:经过多次重复实验,$\mu$落在置信区间的\textbf{频率}趋于$95\%$\\
可信区间:相当于把$\mu$视为\textbf{随机变量},$\mu$落在可信区间的\textbf{概率}为$95\%$\\
构造置信区间:\\
未知$\mu$,$\sigma^2$\\
取\\
\[
\frac{\sqrt{n}(\overline{X}-\mu)}{S} \sim t_{n-1}
\]
那么置信区间为:
\[
[\overline{X}-\frac{t_{n-1}(\frac{\alpha}{2})S}{\sqrt{n}},\overline{X}+\frac{t_{n-1}(\frac{\alpha}{2})S}{\sqrt{n}}]
\]
带入数值为$[171.76,188.24]$\\
构造可信区间:\\
首先要求出$\mu$的后验分布:\\
\begin{gather}
	\begin{aligned}
		\pi(\mu|\vec{x})&=\int_{0}^{+\infty} \pi(\mu,\sigma^2|\vec{x})d\sigma^2\\
		&=\int_{0}^{+\infty}(\sigma^2)^{-\frac{n}{2}-1}e^{-\frac{\sum (x_i-\overline{X})^2 +n(\overline{X}-\mu)^2}{2\sigma^2}}d\sigma^2\\
		&\overset{t=\frac{1}{\sigma^2}}=\int_{0}^{+\infty}(t)^{\frac{n}{2}-1}e^{-\frac{t(\sum (x_i-\overline{X})^2 +n(\overline{X}-\mu)^2)}{2}}dt
	\end{aligned}
\end{gather}
设$\alpha=\frac{n}{2}$,$\beta=\frac{\sum_{i=1}^n (x_i-\overline{X})^2 +n(\overline{X}-\mu)^2}{2}$\\
有$\Gamma(\alpha,\beta)$的密度的积分为$1$\\
即\\
\[
\int_{0}^{+\infty}\frac{\beta^{\alpha}}{\Gamma(\alpha)}x^{\alpha-1}e^{-\beta x}=1
\]
则
\begin{gather}
	\begin{aligned}
		\pi(\mu|\vec{x}) & \propto \frac{\Gamma(\alpha)}{\beta^{\alpha}}\\
		&=\frac{\Gamma(\frac{n}{2})}{(\frac{\sum_{i=1}^n  (x_i-\overline{X})^2-n(\overline{X}-\mu)^2}{2})^{\frac{n}{2}}}
	\end{aligned}
\end{gather}
接下来的计算意义不大,因为考试时没有提供非标准$t$分布的密度\\
\[
t_v(\mu,\sigma^2) \sim f(x)=\frac{\Gamma(\frac{v+1}{2})}{\sigma \sqrt{v\pi} \Gamma(\frac{v}{2})}(1+\frac{(x+\mu)^2}{v\sigma^2})^{-\frac{v+1}{2}}
\]
带入数值可以得到
\[
\mu|\vec{x} \sim t_{24}(180,16)
\]
\[
\frac{\mu|\vec{x}-180}{4} \sim t_{24}
\]
取其上下$\frac{\alpha}{2}$分位数,得到的可信区间也为$[171.76,188.24]$\\



\noindent 附表:上分位数 $u_{0.025}=1.960, u_{0.05}=1.645, t_{24}(0.025)=2.06, t_{24}(0.05)=1.71$\\
伽马分布,逆伽马分布与 $t$ 分布概率密度函数:

$$
\begin{aligned}
& G a(\alpha, \beta): f(x)=\frac{\beta^{\alpha}}{\Gamma(\alpha)} x^{\alpha-1} e^{-\beta x}, \alpha, \beta, x>0 \\
& \operatorname{Inv} G a(\alpha, \beta): f(x)=\frac{\beta^{\alpha}}{\Gamma(\alpha)} x^{-\alpha-1} e^{-\frac{\beta}{x}}, \alpha, \beta, x>0 \\
& t_{n}: f(x)=\frac{\Gamma\left(\frac{n+1}{2}\right)}{\Gamma\left(\frac{n}{2}\right) \sqrt{n \pi}}\left(1+\frac{x^{2}}{n}\right)^{-(n+1) / 2},-\infty<x<\infty
\end{aligned}
$$



\end{document} 
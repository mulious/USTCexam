\documentclass[UTF8]{ctexart}
%\documentclass{article}
\usepackage{graphicx,amsfonts,amsmath,mathrsfs,amssymb,amsthm,url,color}
\usepackage{fancyhdr,indentfirst,bm,enumerate,natbib, float,tikz,graphicx}
\usepackage{caption}
\usepackage{subcaption}
\usepackage{calligra} 

\title{21数理统计期中}
\author{\textcalligra{NULIOUS}} 
\date{}

\textheight 23cm
\textwidth 16.5cm
\topmargin -1.2cm
\oddsidemargin 0cm
\evensidemargin 0cm
\begin{document}
\maketitle
\noindent 一、(20分)假设 $X_1, X_2, \cdots, X_n$ 为来自Poisson总体 $\mathrm{P}(\lambda)$ 的一组简单样本,试\\
(1)将抽样分布表示为指数族的自然形式,并给出自然参数空间;\\
(2)证明统计量 $T=\sum_{i=1}^n X_i$ 为充分完备统计量。\\
(3)证明 $g(\lambda)=e^{-\lambda}$ 不存在可以达到 C-R不等式下界的无偏估计.\\
解:(1)样本联合密度
\[
f\left(x_1, \cdots, x_n ; \lambda\right)=e^{-n \lambda} \frac{\lambda^{\sum\limits_{i=1}^n x_i}}{x_{1}!\cdots x_{n}!}
\]
写成指数族的自然形式为
\[
f\left(x_1, \cdots, x_n ; \lambda\right)=e^{-n \lambda} \cdot \frac{1}{x_{1}!\cdots x_{n}!} \cdot e^{\sum\limits_{i=1}^{n} x_i \ln \lambda}
\]
这里
\[
C(\theta)=e^{-n \lambda}\quad h(\boldsymbol{X})=\frac{1}{x_{1}!\cdots x_{n}!}
\]
自然参数为$\ln \lambda$,又$\lambda \in (0,+\infty)$\\
则自然参数空间为
\[
\Theta=\{\ln \lambda:\ln \lambda \in \mathbb{R}\}
\]
(2)由因子分解定理和自然参数空间有内点,则$T(\boldsymbol{X})=\sum\limits_{i=1}^n X_i$为充分完全统计量\\
(3)无偏估计能达到C-R下界$\iff$分布为单参数指数族且UMVUE为充分完全统计量$T(\boldsymbol{X})$的线性函数\\
先求UMVUE\\
解一:注意到
\[
e^{-\lambda}=P(X_1=0)
\]
再仿照24期中3(1)可以得到UMVUE
\[
\hat{e^{-\lambda}}=\hat{g}(T)=\left(\frac{n-1}{n} \right)^T 
\]
解二:首先
\[
T(\boldsymbol{X})\sim \mathrm{P}(n\lambda)
\]
设UMVUE为$\hat{g}(T)$,则
\[
\mathbb{E}\left[\hat{g}(T) \right] =\sum_{k=0}^{\infty} \hat{g}(k)\frac{(nk)^k}{k!}\cdot e^{-n\lambda}=e^{-\lambda}
\]
也就是
\[
\sum_{k=0}^{\infty} \hat{g}(k) \frac{(n\lambda)^k}{k!}=e^{(n-1) \lambda}=\sum_{k=0}^{\infty} \frac{(n-1)^k \lambda^k}{k!}
\]
就能得到
\[
\hat{g}(T)=\left(\frac{n-1}{n} \right)^T 
\]
显然UMVUE不是充分完全统计量的线性函数,因此不能达到C-R下界\\
数值验证:\\
下面求总体(单个样本)的Fisher信息量
\[
I(\lambda)=-\mathbb{E}\left[\frac{\partial \ln f(x ; \lambda)^2}{\partial^2 \lambda}\right]=-\mathbb{E}\left[-\frac{X}{\lambda^2} \right] =\frac{1}{\lambda}
\]
这里$X$为总体分布,且密度为
\[
f(x ; \lambda)=\frac{\lambda^x}{x!}e^{-\lambda}
\]
因此C-R下界为
\[
\frac{\left( \frac{\partial\left(e^{-\lambda}\right)}{\partial \lambda}\right)^2}{n I(\lambda)}=\frac{\lambda e^{-2 \lambda}}{n} 
\]
再计算UMVUE的方差
\[
\begin{aligned}
	\operatorname{Var}(\hat{g}(T) ) & =\mathbb{E}\left[\left(\frac{n-1}{n}\right)^{2 T}\right]-(\mathbb{E}[\hat{g}(T)])^2 \\
	& =\mathbb{E}\left[\left(\frac{n-1}{n}\right)^{2 T}\right]-e^{-2 \lambda}
\end{aligned}
\]
其中
\[
\begin{aligned}
	E\left[\left(\frac{n-1}{n}\right)^{2 T}\right] & =\sum_{k=0}^{\infty}\left(\frac{n-1}{n}\right)^{2 k} \frac{(n \lambda)^k}{k!} e^{-n \lambda} \\
	& =e^{-n \lambda} \sum_{k=0}^{\infty} \frac{\left( \frac{(n-1)^2 \lambda}{ n}\right) ^k}{k!}\\
	& =e^{-n \lambda} \cdot e^{\frac{(n-1)^2 \lambda}{n}} \\
	& =e^{-2 \lambda+\frac{\lambda}{n}}
\end{aligned}
\]
最后
\[
\operatorname{Var}(\hat{g}(T))=e^{-2 \lambda+\frac{\lambda}{n}}-e^{-2 \lambda}
\]
不能达到C-R下界\\


\noindent 二.(30分)设 $X_1, \ldots, X_n$ 为来自0-1分布 $\mathrm{B}(1, p), 0<p<1$ 的一组简单样本,试\\
(1)求 $g(p)=(1-p)^2$ 的矩估计量和极大似然估计量,并说明是否为无偏估计。\\
(2)求 $g(p)$ 的UMVUE,其方差是否达到C-R不等式的下界?\\
(3)证明 $g(p)$ 的极大似然估计量具有渐近正态性.\\
解:(1)由于$\mathbb{E}[X]=p$,矩估计量为
\[
\hat{g({p})}_M=(1-\overline{X})^2
\]
又样本联合密度
\[
f\left(x_1, \cdots, x_n ; p\right)=p^{\sum_{i=1}^n x_i}(1-p)^{n-\sum_{i=1}^n x_i}
\]
求导有
\[
\frac{\partial \ln f}{\partial p}=\frac{\sum_{i=1}^{n} x_i}{p}-\frac{n-\sum_{i=1}^n x_i}{1-p}=0 .
\]
解得
\[
\hat{p}_{MLE}=\overline{X}
\]
再由MLE的不变性,有
\[
\hat{g(p)}_{MLE}=\left(1-\overline{X} \right)^2 
\]
无偏性:
\[
\begin{aligned}
	\mathbb{E}\left[(1-\overline{X})^2\right] & =\mathbb{E}\left[1-2 \overline{X}+\overline{X}^2\right] \\
	& =1-2 \mathbb{E}[\overline{X}]+\frac{1}{n^2} \mathbb{E}\left[\left(\sum_{i=1}^n X_i\right)^2 \right] \\
	& =1-2 p+p^2+\frac{p(1-p)}{n} \\
	& \neq(1-p)^2
\end{aligned}
\]
这就说明两个估计均不无偏\\
(2)注意二项分布为指数族,且自然参数空间有内点,易知 $T(\boldsymbol{X})=\sum\limits_{i=1}^n X_i$ 为充分完全统计量\\
先找一个无偏估计
\[
\mathbb{I}_{\left\{X_1=0, X_2=0\right\}}
\]
则UMVUE为
\[
\begin{aligned}
	\hat{g}(p)_{UMVUE} &=\mathbb{E}\left[\mathbb{I}_{\left\{X_1=0, X_2=0\right\}} \mid \sum_{i=1}^n x_i=t\right] \\
	& =P\left(X_1=0, X_2=0 \mid \sum_{i=1}^n X_i=t \right) \\
	& =\frac{P\left(X_1=0, X_2=0, \sum_{i=3}^n X_i=t\right)}{P\left(\sum_{i=1}^n X_i=t\right)}\\
	&=\frac{(n-t)(n-t-1)}{n(n-1) }
\end{aligned}
\]
也就是
\[
\hat{g}(p)_{UMVUE}=\frac{(n-T)(n-T-1)}{n(n-1) }
\]
它不是$T(\boldsymbol{X})=\sum\limits_{i=1}^n X_i$的线性函数,因此不能达到C-R下界\\
数值验证:先求总体(单个样本)的Fisher信息量
\[
\begin{aligned}
	I\left(p\right) & =-\mathbb{E}\left[\frac{\partial^2}{\partial p^2} \ln f\left(x ; p\right)\right] \\
	& =\frac{1}{p(1-p)} 
\end{aligned}
\]
这里总体的密度为
\[
f(x ; p)=p^x(1-p)^{1-x}
\]
因此C-R下界为
\[
\frac{\frac{\partial g(p)}{\partial p}}{n I(p)}=\frac{4 p(1-p)^3}{n}
\]
又
\[
\operatorname{Var}(\hat{g}(p)_{UMVUE} )=\operatorname{Var}\left(\frac{(n-T)(n-T-1)}{n(n-1)}\right) =\frac{\operatorname{Var}\left(n^2-n+(1-2n)T+T^2\right)}{n^2(n-1)^2}
\]
其中
\begin{align*}
	\operatorname{Var}\left(n^2-n+(1-2n)T+T^2\right) & =\operatorname{Var}\left((1-2n)T+T^2\right) \\
	 & =\mathbb{E}\left[\left[(1-2 n) T+T^2\right]^2\right]-\left(\mathbb{E}\left[(1-2 n) T+T^2\right]\right)^2
\end{align*}
这里由于$T\sim \mathrm{B}(n,p)$
\begin{align*}
	\mathbb{E}\left[\left[(1-2 n) T+T^2\right]^2\right] & =\mathbb{E}\left[(1-2n)^2 T^2 +(2-4n)T^3+T^4 \right]  \\
	& =\mathbb{E}\left[(1-2n)^2 T^2 \right] +\mathbb{E}\left[(2-4n)T^3 \right] +\mathbb{E}\left[T^4 \right] \\
	&=(1-2n)^2 np(1-p+np)+(2-4n)(1-3p+3np+(2-3n+n^2)p^2)\\
	&+np(1-p)(1-6p+6p^2)+7np(1-p)(n-1)p+6n(n-1)(n-2)p^3+n^3p^4\\
\end{align*}
另一方面
\begin{align*}
	\mathbb{E}\left[(1-2 n) T+T^2\right] & =(1-2n)p+np(1-p+np) 
\end{align*}
这就说明了UMVUE不能达到C-R下界\\
(3)解一:直接利用MLE的渐进正态性
\[
\sqrt{n}\left(\hat{p}_{MLE}-p \right)\xrightarrow{D} N \left(0,\frac{1}{I(p)} \right) 
\]
也就是
\[
\sqrt{n}\left(\hat{p}_{MLE}-p \right)\xrightarrow{D} N \left(0,p(1-p) \right) 
\]
再使用$\Delta$方法
\[
\sqrt{n}\left(\hat{g(p)}_{MLE}-g(p) \right)\xrightarrow{D} N \left(0,4p(1-p)^3 \right) 
\]
解二:$\overline{X}$有CLT
\[
\sqrt{n}\left(\overline{X}-p \right)\xrightarrow{D} N \left(0,p(1-p) \right) 
\]
再使用$\Delta$方法
\[
\sqrt{n}\left(\hat{g(p)}_{MLE}-g(p) \right)\xrightarrow{D} N \left(0,4p(1-p)^3 \right) 
\]\\




\noindent 三.(15分)设 $X_1, \ldots, X_n$ 来自正态总体 $N(\mu, 1)$ 的一组简单随机样本,试 $(0<\alpha<1)$\\
(1)证明样本平均值 $\overline{X}$ 与统计量 $X_1-\overline{X}$ 相互独立.\\
(2)求概率 $P\left(X_1 \leq 0\right)$ 的UMVUE以及其置信水平为 $1-\alpha$ 的置信区间.\\
解:(1)注意方差已知,由 $N(\mu,1)$ 为指数族,容易得到$T(\boldsymbol{X})=\overline{X}$为充分完全统计量\\
设 
$$Y_i=X_i-\mu \sim N(0,1)$$
则 $$X_1-\overline{X}=\left(X_1-\mu\right)-\frac{\sum\limits_{i=1}^n \left(X_i-\mu\right)}{n}=Y_1-\overline{Y}$$
分布与 $\mu$ 无关,是辅助量,由Basu定理知二者独立\\
(2)显然$\mathbb{I}_{\{X_1\le 0\}}$就是无偏估计,设要求的UMVUE为$h(T)$
\[
\begin{aligned}
	h(T) & =\mathbb{E}\left[\mathbb{I}_{\{X_1\le 0\}} \mid T(X)\right] \\
	& =P\left(X_1 \leqslant 0 \mid T(X)\right) \\
	&=P\left(X_1-\overline{X} \leqslant -\overline{X} \mid \overline{X}\right) \\
	&\stackrel{\text{独立性}}{=} P\left(X_1-\overline{X} \leqslant-\overline{X}\right) 
\end{aligned}
\]
注意$Y_1$与$\overline{Y}$不独立,有
$$X_1-\overline{X}\sim Y_1-\overline{Y}\sim N\left(1,1-\frac{1}{n} \right) $$
则
\[
h(T)=\Phi\left(-\frac{\overline{X}}{\sqrt{1-\frac{1}{n}}} \right) 
\]
这里$\Phi(x)$为$N(0,1)$的分布函数,又
\[
\overline{X} \sim N\left(\mu, \frac{1}{n}\right) \quad \sqrt{n}(\overline{X}-\mu) \sim N(0,1) 
\]
则
\[
P(X_1\le 0)=\Phi(-\mu)
\]
即$P(X_1\le 0)$关于$\mu$单调递减,只需要求$\mu$的置信区间即可
另外$\mu$有置信系数为$1-\alpha$的置信区间
\[
\left[\overline{X}-\frac{u_{\frac{\alpha}{2}}}{\sqrt{n}},\overline{X}+\frac{u_{\frac{\alpha}{2}}}{\sqrt{n}} \right] 
\]
则$P(X_1\le 0)$有置信区间
\[
\left[\Phi\left(-\left(\overline{X}+\frac{u_{\frac{\alpha}{2}}}{\sqrt{n}}\right)\right), \Phi\left(-\left(\overline{X}-\frac{u_{\frac{\alpha}{2}}}{\sqrt{n}}\right)\right)\right]
\]\\




\noindent 四.(15分)设 $X_1, X_2, \cdots, X_n$ 为来自如下分布的一组简单样本
$$
\begin{tabular}{|c|ccc|}
	\hline$X$ & 0 & 1 & 2 \\
	\hline$P$ & $\frac{1}{2}\left[(1-\theta)^2+\theta^2\right]$ & $2 \theta(1-\theta)$ & $\frac{1}{2}\left[(1-\theta)^2+\theta^2\right]$ \\
	\hline
\end{tabular}
$$
其中 $0<\theta<1 / 2$ 为参数。试利用重参数化方法和极大似然估计的不变性\\
(1)求 $\theta$ 的极大似然估计量,并求其渐近分布.\\
(2)由此给出 $\theta$ 的一个(渐近)置信水平为 $1-\alpha$ 的置信区间 $(0<\alpha<1)$ .\\
解:(1)重参数化
\[
P(X=1)=P(X=3) \triangleq \mu=\frac{1}{2}\left[(1-\theta)^2+\theta^2\right] 
\]
则
\[
P(X=2)=1-2\mu
\]
同时令$n_0,n_1,n_2$为$n$个样本中取值为0,1,2的样本个数,有$n_0+n_1+n_2=n$,则样本联合密度为
\[
f\left(x_1, \cdots, x_n ; \mu\right)=\mu^{n_0}(1-2 \mu)^{n_1} \mu^{n_2}
\]
进而
\[
\ln f\left(x_1, \cdots, x_n ; \mu\right)=n_0 \ln \mu+n_1 \ln (1-2 \mu)+n_2 \ln \mu
\]
\[
\frac{\partial  \ln f}{\partial \mu}=\frac{n_0}{\mu}+\frac{n_2}{\mu}-\frac{2 n_1}{1-2 \mu}=0
\]
得到
\[
\hat{\mu}_{M L E}=\frac{n_0+n_2}{2 n}
\]
注意$\theta<\frac{1}{2}$,反解$\theta$有
\[
\hat{\mu}_{MLE}=\frac{1}{2}\left[(1-\hat{\theta}_{MLE})^2+\hat{\theta}_{MLE}\right] 
\]
得到
\[
\hat{\theta}_{MLE}=\frac{1-\sqrt{1-\frac{2 n_1}{n}}}{2}
\]
由MLE的渐进正态性
\[
\sqrt{n}\left(\hat{\theta}_{M L E}-\theta\right) \xrightarrow{D} N\left( 0,\frac{1}{I(\theta)} \right) 
\]
这里总体(单个样本)的Fisher信息量为
\[
\begin{aligned}
	I(\theta)= & -\mathbb{E}\left[\ln ^{\prime \prime} f(x ; \theta)\right] . \\
	= & -\left[P(x=0) \cdot \ln ^{\prime \prime} f(x=0 ; \theta)+P(x=1) \cdot \ln ^{\prime \prime} f(x=1 ; \theta)\right. \\
	& \left.+P(x=2) \cdot \ln ^{\prime \prime} f\left(x=2 ; \theta\right)\right]
\end{aligned}
\]
对$X=0$或2
\[
f(x ; \theta)=\frac{(1-\theta)^2+\theta^2}{2}
\]
进而
\[
\frac{\partial^2}{\partial \theta^2} \ln f=\frac{4\left(2 \theta^2-2 \theta+1\right)-2(2 \theta-1)(4 \theta-2)}{\left(2 \theta^2-2 \theta+1\right)^2}
\]
而对$X=1$
\[
\frac{\partial^2}{\partial \theta^2} \ln f(x=1 ; \theta)=-\frac{1}{\theta^2}-\frac{1}{(1-\theta)^2}
\]
则
\[
I(\theta)=\frac{2}{2 \theta^2-2 \theta+1}+\frac{1}{\theta(1-\theta)}
\]
就有
\[
\sqrt{n}\left(\hat{\theta}_{M L E}-\theta\right) \xrightarrow{D} N\left( 0,\frac{1}{\frac{2}{2 \theta^2-2 \theta+1}+\frac{1}{\theta(1-\theta)}} \right) 
\]
(2)置信区间为
\[
\left[\hat{\theta}_{M L E}-\frac{u_{\frac{\alpha}{2}}}{\sqrt{n I(\theta)}}, \hat{\theta}_{M L E}+\frac{{u_{\frac{\alpha}{2}}}}{\sqrt{n I(\theta)}}\right]
\]\\



\noindent 五.(20分)设 $X_1, \ldots, X_n, i . i . d \sim U(0, \theta)$ ,其中 $\theta>1$ 为未知参数.记 $X_{(n)}=\max\limits_{1 \leq i \leq n} X_i$ ,试\\
(1)证明 $X_{(n)}$ 是充分但不完备的统计量.\\
(2)求 $\theta$ 的UMVUE.\\
解:(1)样本联合密度为
\[
f\left(x_1, \cdots, x_n ; \theta\right)=\frac{1}{\theta^n} \mathbb{I}_{\left\{X_{(n)} < \theta\right\}}
\]
取 $ g(T(\boldsymbol{X}) ; \theta)=\theta^{-n} \cdot I_{\left\{x_{(n)}<\theta\right\}}, h(\boldsymbol{X})=1$,由因子分解定理知道$T(\boldsymbol{X})=X_{(n)}$为充分统计量\\
注意$\theta>1$,为了证明 $T(\boldsymbol{X})=X_{(n)}$ 不是 $\theta$ 的完全统计量.只需寻找 $t$ 的某一实函数 $\varphi(t)$ ,满足 $\mathbb{E}_\theta[\varphi(T)]=0$ ,即
$$
\int_0^\theta \varphi(t) t^{n-1} \mathrm{~d} t=0, \quad \theta>1
$$
但 $\mathbb{P}_\theta\left(\varphi\left(X_{(n)}\right)=0\right)<1$ \\
注意到上式对 $\theta$ 求导后只能得到 $\varphi(t)=0, t>1$ .因此我们只要构造合适的 $\varphi(t), t \leq 1$ ,使得
$$
\int_0^1 \varphi(t) t^{n-1} \mathrm{~d} t=0
$$
即可,例如 
\[
\varphi(t)=
\begin{cases}
	t^{1-n}  &  t \leq \frac{1}{2} \\
	-t^{1-n}  &  \frac{1}{2}<t\le 1\\
	0 & 1<t\le \theta
\end{cases}
\]
(2)统计量只充分不完全,考虑零无偏法\\
$X_{(n)}$ 有密度  
\[
g(t)=n \theta^{-n} t^{n-1} \mathbb{I}_{(0, \theta)}(t) 
\]
设$\mathbb{E}[\delta(t)]=0$,即
\[
\int_0^\theta \delta(t) n \theta^{-n} t^{n-1} d t=0 
\]
也就是
\[
\int_0^\theta \delta(t)  t^{n-1} d t=0 
\]
进而
\[
\int_0^1 \delta(t) t^{n-1} d t+\int_1^\theta \delta(t) t^{n-1} d t=0
\]
求导有
\[
\delta(t) \equiv 0 \quad \forall t>1
\]
设要求的UMVUE为$h(T)$,它要满足以下条件
\[
\mathbb{E}[h(T) \delta(T)]=0
\]
\[
\int_0^\theta \delta(t) h(t) n\theta^{-n} t^{n-1} d t=0
\]
注意$\delta(t) \equiv 0 \quad \forall t>1$,上式也就说明
\[
\int_0^1 \delta(t) h(t) n\theta^{-n} t^{n-1} d t=0
\]
又$ h(T)$ 无偏,即  $\mathbb{E}[h(T)]=\theta$ 
\[
\theta=\int_{0}^{\theta} h(t) n \theta^{-n} t^{n-1} d t
\]
待定$h(T)=c,0\le T<1$,有
\[
c \int_0^1 \frac{n t^{n-1}}{\theta^n} d t=c \theta^{-n}
\]
想要$$\int_1^\theta h(t) \frac{n t^{n-1}}{\theta^n} d t $$ 含有 $ \theta^{-n} $ 及 $ \theta $ 项, 
待定剩下部分为
\[
h(T)=
\begin{cases}
	c  &  0\le T<1 \\
	bT+d  &  T\ge 1
\end{cases}
\]
代入积分的值则有
\[
c=1\quad b=
\frac{n+1}{n} \quad d=0
\]
最后
\[
h(T)=
\begin{cases}
	1  &  0\le T<1 \\
	\frac{n+1}{n}T  &  T\ge 1
\end{cases}
\]




\end{document}
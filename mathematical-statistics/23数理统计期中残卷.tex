\documentclass[UTF8]{ctexart}
%\documentclass{article}
\usepackage{graphicx,amsfonts,amsmath,mathrsfs,amssymb,amsthm,url,color}
\usepackage{fancyhdr,indentfirst,bm,enumerate,natbib, float,tikz,graphicx}
\usepackage{caption}
\usepackage{subcaption}
\usepackage{calligra}

\title{23数理统计期中残卷}
\author{\textcalligra{NULIOUS}}
\date{}

\textheight 23cm
\textwidth 16.5cm
\topmargin -1.2cm
\oddsidemargin 0cm
\evensidemargin 0cm

\begin{document}
	
\maketitle
\noindent 1.设从总体
$$
\begin{tabular}{c|cccc}
	$X$ & 0 & 1 & 2 & 3 \\
	\hline $\mathbb{P}$ & $3 \theta / 4$ & $\theta / 4$ & $2 \theta$ & $1-3 \theta$
\end{tabular}
$$
(其中 $0<\theta<1 / 3$ )中抽取的一个简单样本 $X_1, \ldots, X_n$ ,试\\
(1)将样本分布表示为指数族自然形式,指出自然参数及自然参数空间。\\
(2)求 $\theta$ 的最大似然估计量 $\hat{\theta}$ ,其是否为一致最小方差无偏估计?\\
(3)证明 $\hat{\theta}$ 具有相合性和渐近正态性。\\
解:(1)样本联合密度为
\[
f\left(x_1 \cdots, x_n ; \theta\right)=\left(\frac{3 \theta}{4}\right)^{n_0}\left(\frac{\theta}{4}\right)^{n_1} \cdot(2 \theta)^{n_2} \cdot(1-3 \theta)^{n_3}
\]
这里$n_i$为$n$个样本中取值为$i$的个数,把样本联合密度写成指数族的形式为
\[
f\left(x_1, \cdots, x_n ; \theta\right)=\left(\frac{3}{4}\right)^{n_0}\left(\frac{1}{4}\right)^{n_1} \cdot 2^{n_2} \cdot e^{\left(n-n_3\right) \ln \theta+n_3 \ln (1-3 \theta)}
\]
其中
\[
\begin{aligned}
	& h(\boldsymbol{X})=\left(\frac{3}{4}\right)^{n_0}\left(\frac{1}{4}\right)^{n_1} 2^{n_2} \\
	& C(\theta)=e^{n \ln \theta}
\end{aligned}
\]
自然参数为$\varphi=\ln \frac{1-3\theta}{\theta}$\\
又$0<\theta<\frac{1}{3}$,则自然参数空间为$\mathbb{R}$\\
由因子分解定理和自然参数空间有内点,充分完全统计量为$T(\boldsymbol{X})=n_3$\\
(2)
\[
\ln f\left(x_1, \cdots, x_n ; \theta\right) \propto \left(n-n_3\right) \ln \theta+n_3 \ln (1-3 \theta)
\]
令
\[
\frac{\partial \ln f}{\partial \theta}=\frac{n-n_3}{\theta}-\frac{3 n_3}{1-3 \theta}=0
\]
解得
\[
\hat{\theta}_{MLE}=\frac{n-n_3}{3n}
\]
又$n_3\sim \mathrm{B}(n,1-3\theta)$
\[
\mathbb{E}[n_3]=n(1-3\theta) \quad \mathbb{E}\left[\hat{\theta}_{MLE} \right] =\theta
\]
也就是MLE为无偏估计量,且它是充分完全统计量的函数,则其为UMVUE\\
(3)相合性:\\
强相合:注意
\[
n_3\sim \mathrm{B}(n,1-3\theta)
\]
为$n$个$\mathrm{B}(1,1-3\theta)$的独立和\\
由SLLN
\[
\frac{n_3}{n} \xrightarrow{a.s.}1-3\theta
\]
则
\[
\frac{n-n_3}{3n} \xrightarrow{a.s.}\theta
\]
即强相合\\
弱相合:\\
i.直接由强相合得到\\
ii.由Chebyshev不等式
\[
P\left(\mid \hat{\theta}_{MLE}-\theta \mid \ge \epsilon \right)\le \frac{\operatorname{Var}\left( \hat{\theta}_{MLE}\right) }{\epsilon^2} 
\]
另一方面
\[
\begin{aligned}
	\operatorname{Var}(\hat{\theta}_{MLE})&=\operatorname{Var}\left(\frac{n-n_3}{3 n}\right)  &=\operatorname{Var}\left(\frac{n_3}{3 n}\right)\\
	&=\frac{1}{9 n^2}\operatorname{Var}\left(n_3\right)&=\frac{\theta(1-3 \theta)}{3 n}
\end{aligned}
\]
则
\[
\lim _{n \rightarrow \infty} P(|\hat{\theta}_{MLE}-\theta| \geq \varepsilon) \leqslant \frac{\theta(1-3 \theta)}{3 n \varepsilon^2} \rightarrow 0
\]
就说明了弱收敛\\
渐进正态性:\\
i.MLE的渐近正态性
\[
\sqrt{n}\left(\hat{\theta}_{MLE}-\theta \right)\xrightarrow{D} N\left(0,\frac{1}{I(\theta)} \right)  
\]
这里总体(单个样本)的Fisher信息量为
\[
I(\theta)=-\mathbb{E}[\frac{\partial^2 \ln f(x, \theta)}{\partial \theta^2}]
\]
其中$f(x, \theta)$为总体的密度
\[
\mathbb{E}[\frac{\partial^2 \ln f(x, \theta)}{\partial \theta^2}]=\ln ^{\prime \prime} f(x=0, \theta) P(X=0)+\cdots+\ln ^{\prime \prime} f(x=3, \theta) P(X=3)
\]
得到
\[
I(\theta)=\frac{3}{\theta}+\frac{9}{1-3\theta}
\]
则
\[
\sqrt{n}\left(\hat{\theta}_{MLE}-\theta \right)\xrightarrow{D} N\left(0,\frac{\theta(1-3\theta)}{3} \right) 
\]
ii.CLT
\[
\sqrt{n}\left(\frac{n_3}{n}-(1-3 \theta)\right) \xrightarrow{D} N(0,(1-3 \theta) 3 \theta) 
\]
则
\[
\sqrt{n}\left(\hat{\theta}_{MLE}-\theta \right)\xrightarrow{D} N\left(0,\frac{\theta(1-3\theta)}{3} \right) 
\]


\noindent 附表:$u_{0.025}=1.960, u_{0.05}=1.645, \Phi(1.36)=0.9131, \Phi(1.4)=0.9192$.

\end{document}
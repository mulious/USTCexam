\documentclass[UTF8]{ctexart}
%\documentclass{article}
\usepackage{graphicx,amsfonts,amsmath,mathrsfs,amssymb,amsthm,url,color}
\usepackage{fancyhdr,indentfirst,bm,enumerate,natbib, float,tikz,graphicx}
\usepackage{caption}
\usepackage{subcaption}
\usepackage{calligra}

\title{23数理统计期末残卷}
\author{\textcalligra{NULIOUS}}
\date{}

\textheight 23cm
\textwidth 16.5cm
\topmargin -1.2cm
\oddsidemargin 0cm
\evensidemargin 0cm

\begin{document}
	
\maketitle
\noindent 1.一个移动通讯公司随机抽取了其 900 个包月客户,计算得知他们一个月平均使用时间是 220 分钟,样本标准差是 90 分钟.假设使用时间服从正态分布 $N\left(\mu, \sigma^2\right)$ .\\
(1)求包月客户平均使用时间和标准差的 $95 \%$ 置信区间。\\
(2)如果要求客户平均使用时间的 $95 \%$ 置信区间的长度不超过 5 分钟,应至少抽取多少个客户?该公司的抽样规模是否满足要求?\\
(3)假设总体标准差为 90 分钟,取 $\mu$ 的先验分布为无信息先验,求 $\mu$ 的后验分布并据此给出 $\mu$ 的可信系数为 $95 \%$ 的可信区间。\\
(4)解释(1)和(3)中关于平均使用时间所得区间的含义与区别.\\
解:(1)$\mu, \sigma^2$ 均未知\\
利用 
$$\frac{\sqrt{n}(\overline{X}-\mu)}{S} \rightarrow t_{n-1}$$
估计 $\mu$ ,进而有置信区间
\[
\left[ \overline{X} - t_{n-1}\left(\frac{\alpha}{2}\right) \frac{S}{\sqrt{n}},\ 
\overline{X} + t_{n-1}\left(\frac{\alpha}{2}\right) \frac{S}{\sqrt{n}} \right]
\]
代值有
$$[214.12,225.885]$$ 
这里因为$t_n\rightarrow N(0,1)$ ,用$u_{\frac{\alpha}{2}}$代替$t_{n-1}\left(\frac{\alpha}{2}\right)$\\
再利用 
$$\frac{(n-1) S^2}{\sigma^2} \rightarrow \chi^2_{(n-1)}$$
$\sigma^2$ 有置信区间 
$$\left[\frac{(n-1) S^2}{\chi^2_{(n-1)}\left(\frac{\alpha}{2}\right)}, \frac{(n-1) S^2}{\chi^2_{(n-1)}\left(1-\frac{\alpha}{2}\right)}\right]$$
开方有 $\sigma$ 的置信区间
$$
\left[\sqrt{\frac{(n-1) S^2}{\chi^2_{(n-1)}\left(\frac{\alpha}{2}\right)}},\sqrt{\frac{(n-1) S^2}{\chi^2_{(n-1)}\left(1-\frac{\alpha}{2}\right)}}\right]
$$
代值为
\[
[88.15,92.02]
\]
(2)即需要
\[
2\times u_{\frac{\alpha}{2}}\frac{S}{\sqrt{n}}\le 5
\]
这里同样因为$t_n\rightarrow N(0,1)$ ,用$u_{\frac{\alpha}{2}}$代替$t_{n-1}\left(\frac{\alpha}{2}\right)$解得
\[
n\ge 4979
\]
不满足要求\\
(3)位置参数的无信息先验为
\[
\pi(\mu)=1
\]
样本联合密度为
\[
\begin{aligned}
	f(x_1,\cdots,x_n;\mu) & =\left(2 \pi \sigma^2\right)^{-\frac{n}{2}} \exp \left\{-\frac{1}{2 \sigma^2}\sum_{i=1}^{n}\left(x_i-\mu\right)^2\right\} \\
	& =\left(2 \pi \sigma^2\right)^{-\frac{n}{2}} \exp \left\{-\frac{1}{2 \sigma^2} \sum_{i=1}^n\left(x_i-\overline{X}\right)^2+n(\overline{X}-\mu)^2\right\} 
\end{aligned}
\]
则
\[
\pi(\mu \mid \vec{X})\propto \exp\left\{-\frac{n}{2\sigma^2}\left(\overline{X}-\mu \right)^2 \right\}
\]
也就是
\[
\mu \mid \vec{X}\sim N\left(\overline{X},\frac{\sigma^2}{n} \right) 
\]
则可信区间为
\[
\left[\overline{X}-\frac{\sigma}{\sqrt{n}} u_{\alpha / 2}, \overline{X}+\frac{\sigma}{\sqrt{n}} u_{\alpha / 2} \right] 
\]
代入数值为
\[
[214.12,225.88]
\]
(4)置信区间:经过多次重复实验,$\mu$落在置信区间的\textbf{频率}趋于$95\%$,参数是一个真实的固定值\\
可信区间:相当于把$\mu$视为\textbf{随机变量},一次试验后$\mu$落在可信区间的\textbf{概率}为$95\%$\\




\noindent 2.设 $X_1, \ldots, X_n$ i.i.d.$\sim N(\theta, 1), Y_1, \ldots, Y_n$ i.i.d.$\sim N(2 \theta, 1), \theta$ 为未知参数.又设合样本 $X_1, \ldots, X_n, Y_1, \ldots, Y_n$ 独立.试\\
(1)求 $\theta$ 的 UMVUE.\\
(2)如果要求 $\theta$ 的置信系数为 $1-\alpha$ 的置信区间宽不超过指定的 $d$ ,则样本量 $n$应该至少多大?\\
解:(1)样本联合密度为
\[
\begin{aligned}
	f\left(X_1, \cdots, X_n, Y_1, \cdots, Y_n ; \theta\right) 
	&=\left(\frac{1}{\sqrt{2 \pi}}\right)^{-2 n} \cdot e^{\frac{-\sum_{i=1}^n\left(X_i-\theta\right)^2-\sum_{i=1}^n\left(Y_i-2 \theta\right)^2}{2}} \\
	& =\left(\frac{1}{2 \pi}\right)^n \cdot e^{-\frac{5 n \theta^2}{2}} \cdot e^{ \theta \cdot\left(\sum_{i=1}^n\left(X_i+2 Y_i\right)\right)} \cdot e^{-\frac{1}{2} \sum_{i=1}^n\left(X_i^2+Y_i^2\right)}
\end{aligned}
\]
显然自然参数空间有内点,则$T(\boldsymbol{X})=\sum\limits_{i=1}^n \left(X_i+2Y_i \right) $为充分完全统计量\\
注意独立性,有
\[
T(\boldsymbol{X})\sim N(5n\theta,5n)
\]
则有无偏估计
\[
\frac{T(\boldsymbol{X})}{5n}=\frac{\overline{X}+2\overline{Y}}{5}
\]
它也是充分完全统计量的函数,则其为UMVUE\\
(2)取
\[
\frac{\frac{\overline{X}+2\overline{Y}}{5}-\theta}{\frac{1}{\sqrt{5n}}} \sim N(0,1)
\]
作为枢轴变量\\
则区间长度
\[
\frac{2u_{\alpha/2}}{\sqrt{5n}}\le d
\]
得到
\[
n \geq\left\lceil\frac{4 u_{\alpha/2}^2}{5 d^2}\right\rceil
\]\\




\noindent 3.随机地从一批铁钉中抽取 16 枚,测得它们的平均长度(单位: cm$) \bar{x}=3.035$ .已知铁钉长度 $X$ 服从正态分布 $N\left(\mu, 0.1^2\right)$ .考虑假设检验问题
\begin{equation}
	H_0: \mu \leq 3 \longleftrightarrow H_1: \mu > 3 \tag{$\star$}
\end{equation}
(1)在 $\alpha=0.05$ 水平下对检验问题 $(\star)$ 进行检验,并给出检验的 $p$ 值。\\
(2)如果行动 $a=0$ 和 $a=1$ 分别表示接受 $H_0$ 和拒绝 $H_0, ~ \mu$ 的先验分布为 $N\left(3,0.1^2\right)$ ,损失函数取为
$$
L(a, \mu)= \begin{cases}11, & a=1, \mu \leq 3, \\ 1, & a=0, \mu>3, \\ 0, & \text { 其他. }\end{cases}
$$
基于后验风险最小原则给出检验问题$(\star)$的最优决策行动,并与(1)中检验结论进行对比。\\
解:(1)给出检验统计量
$$U(\boldsymbol{X})=\frac{\bar{X}-\mu_0}{\sigma / \sqrt{n}} \stackrel{H_0}{\sim} N(0,1)$$
由检验问题的形式知拒绝域为 
$$D=\left\{\boldsymbol{X}: U(\boldsymbol{X})>u_\alpha\right\}$$
现有 $\bar{x}=3.035, \mu_0=3, \sigma=0.1, n=16$ ,查表得 $u_{0.05}=1.645$ ,因此
$$
u(\boldsymbol{x})=\frac{3.035-3}{0.1 / \sqrt{16}}=1.4<1.645
$$
因此不能拒绝原假设.\\
检验的 $p$ 值为 $$p=P(U(\boldsymbol{X})>u(\boldsymbol{x}))=1-\Phi(1.4)=0.0808$$
(2)在先验分布 $\mu \sim N\left(\mu_0, \tau^2\right)$ 下,正确计算出 $\mu$ 的后验分布
$$
\mu \left\lvert\, \boldsymbol{x} \sim N\left(\frac{n \tau^2 \bar{x}+\sigma^2 \mu_0}{n \tau^2+\sigma^2}, \frac{\tau^2 \sigma^2}{n \tau^2+\sigma^2}\right) \stackrel{\text { 代入数据 }}{=} N\left(\frac{1289}{425}, \frac{1}{1700}\right) .\right.
$$
后验风险 
$$R\left(a_0, \mu\right)=\mathbb{P}(\mu>3 \mid \boldsymbol{x})=0.9131\quad R\left(a_1, \mu\right)=11 \mathbb{P}(\mu \leq 3 \mid \boldsymbol{x})=0.9559$$
结论:按后验风险最小原则,应采取行动 $a_0$ ,即接受原假设 $H_0$ 。\\
这与(1)中检验结果一致,这是因为我们对拒绝 $H_0$ 赋予了较大的惩罚。\\



\noindent 4.设总体 $X$ 的密度函数为
$$
f(x ; \theta)= \begin{cases}\theta, & 0<x<1, \\ 1-\theta, & 1 \leq x<2, \\ 0, & \text { 其他 },\end{cases}
$$
其中 $0<\theta<1$ 为未知参数.现从此总体中抽出一样本量为 $n$ 的样本 $X_1, \ldots, X_n$ .\\
(1)试求参数 $\theta$ 的充分统计量,并说明它是否为完全统计量?\\
(2)求假设检验问题 $H_0: \theta=\theta_0 \leftrightarrow H_1: \theta \neq \theta_0$ 的水平 $\alpha$ 似然比检验,其中 $0<\theta_0, \alpha<1$ 已知。\\
(3)上述假设是否存在 UMPT?为什么?\\
解:(1)样本联合密度
\[
f(x_1,\cdots,x_n;\theta)=\theta^{n_1}\cdot (1-\theta)^{n_2}\mathbb{I}_{\left\{0<X_{(1)}<X_{(n)}<2\right\}}
\]
其中
\[
n_1=\sum_{i=1}^{n} \mathbb{I}_{\left\{0<X_i<1 \right\}}\quad n_2=n-n_1
\]
则
\[
n_1\sim \mathrm{B}(n,\theta)
\]
将联合密度改写成指数族形式
\begin{align*}
	f(x_1,\cdots,x_n;\theta)&=e^{n_1\ln \theta +n_2\ln (1-\theta)}\mathbb{I}_{\left\{0<X_{(1)}<X_{(n)}<2\right\}}  \\
	& =(1-\theta)^n \cdot e^{n_1\ln \frac{\theta}{1-\theta}}\cdot h(\boldsymbol{X})
\end{align*}
由因子分解定理和自然参数空间有内点,$T(\boldsymbol{X})=n_1$为充分完全统计量\\
(2)对全空间$0<\theta<1$,易得
\[
\hat{\theta}_{MLE}=\frac{n_1}{n}
\]
则
\[
\sup\limits_{0<\theta<1} f(x_1,\cdots,x_n;\theta) = \left(\frac{n_1}{n}\right)^{n_1}\left(1-\frac{n_1}{n}\right)^{n-n_1}\mathbb{I}_{\left\{0<X_{(1)}<X_{(n)}<2\right\}}
\]
另一方面
\[
\sup\limits_{\theta=\theta_0} f(x_1,\cdots,x_n;\theta)=f(x_1,\cdots,x_n;\theta_0)
\]
则似然比为
\[
\lambda(\boldsymbol{X})=\frac{\sup\limits_{0<\theta<1} f(x_1,\cdots,x_n;\theta)}{\sup\limits_{\theta=\theta_0} f(x_1,\cdots,x_n;\theta)}=\left(\frac{n_1}{\theta_0 n}\right)^{n_1}\left(\frac{n-n_1}{\left(1-\theta_0\right) n}\right)^{n-n_1}
\]
则
\[
2\log \lambda(\boldsymbol{X})\xrightarrow{H_0}\chi^2(1)
\]
拒绝域为
\[
\left\{(X_1,\cdots,X_n):2\log \lambda(\boldsymbol{X})>\chi^2_1(\alpha) \right\}
\]
(3)设$\varphi_1$为假设检验问题
\[
H_0:\theta=\theta_0 \leftrightarrow H_1^{\prime}:\theta=\theta_1
\]
的UMPT,这里$\theta_1>\theta_0$\\
则由N-P引理
\[
\varphi_1(\boldsymbol{x})= 
\begin{cases}
	1 & f\left(x, \theta_1\right) / f\left(x, \theta_0\right)>c_1 \\ 
	r_1 & f\left( x, \theta_1 \right) /f\left(x, \theta_0\right)=c_1 \\ 
	0 & f\left(x, \theta_1\right) / f\left(x, \theta_0\right) <c_1
\end{cases}
\]
同理对假设检验问题
\[
H_0:\theta=\theta_0 \leftrightarrow H_1^{\prime \prime}:\theta=\theta_2
\]
这里$\theta_2<\theta_0$\\
有UMPT
\[
\varphi_2(\boldsymbol{x})= 
\begin{cases}
	1 & f\left(x, \theta_2\right) / f\left(x, \theta_0\right)>c_2 \\ 
	r_2 & f\left( x, \theta_2 \right) /f\left(x, \theta_0\right)=c_2 \\ 
	0 & f\left(x, \theta_2\right) / f\left(x, \theta_0\right) <c_2
\end{cases}
\]
另一方面
\[
\begin{aligned}
	 f(\vec{x},\theta_1)/f(\vec{x},\theta_0)&=\left(\frac{\theta_1}{\theta_0} \right)^{n_1} \left(\frac{1-\theta_1}{1-\theta_0}\right)^{n-n_1}\\
	 &=\left(\frac{1-\theta_1}{1-\theta_0}\right)^n \quad\left(\frac{\theta_1\left(1-\theta_0\right)}{\theta_0\left(1-\theta_{1}\right)}\right)^{n_1} 
\end{aligned}
\]
关于$n_1$单调递增\\
则$\varphi_1(\boldsymbol{x})$可以改写为
\[
\varphi_1(\boldsymbol{x})= \begin{cases}1 & n_1>c_1 \\ r_1 & n_1=c_1 \\ 0 & n_1<c_1\end{cases}
\]
同理
\[
f(\vec{x},\theta_2)/f(\vec{x},\theta_0)=\left(\frac{1-\theta_2}{1-\theta_0}\right)^n \cdot\left(\frac{\theta_2-\theta_2 \theta_0}{\theta_0-\theta_0 \theta_2}\right)^{n_1}
\]
关于$n_1$单调递减\\
则$\varphi_1(\boldsymbol{x})$可以改写为
\[
\varphi_2(\boldsymbol{x})= \begin{cases}1 & n_1<c_2 \\ r_2 & n_1=c_2 \\ 0 & n_1>c_2\end{cases}
\]
如果假设检验问题
\[
H_0: \theta=\theta_0 \leftrightarrow H_1: \theta \neq \theta_0
\]
的UMPT存在,记为$\varphi_0(\boldsymbol{x})$\\
则由N-P引理的唯一性\\
\[
\begin{array}{ll}
	\theta_1>\theta_0 \text { 时} & \varphi_1=\varphi_0 \quad a.e.\\
	\theta_2<\theta_0 \text { 时 } & \varphi_2=\varphi_0 \quad a.e.
\end{array}
\]
但这与$\varphi_1,\varphi_2$的形式矛盾,也就不存在假设检验问题 $H_0: \theta=\theta_0 \leftrightarrow H_1: \theta \neq \theta_0$的UMPT\\




\noindent 附表:$u_{0.025}=1.960, u_{0.05}=1.645, \chi_{899}^2(0.025)=984, \chi_{899}^2(0.975)=817.8$ .

\end{document}
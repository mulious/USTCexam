\documentclass[UTF8]{ctexart}
\usepackage{graphicx,amsfonts,amsmath,mathrsfs,amssymb,amsthm,url,color}
\usepackage{fancyhdr,indentfirst,bm,enumerate,natbib,float,tikz}
\usepackage{caption,subcaption,calligra}
\usepackage{graphicx}
\usepackage{enumitem}
\usepackage{bm}

\title{25实分析期中}
\author{}
\date{}

\textheight 23cm
\textwidth 16.5cm
\topmargin -1.2cm
\oddsidemargin 0cm
\evensidemargin 0cm

\begin{document}
\maketitle
\noindent 1.\\
(1)简述函数$f$可测的定义\\
(2)求证:$f$可测且集合 $\{f>0\}$ 可测 $\Rightarrow f$ 可测\\
2.\\
(1)课本中的不可测集$N$的可测子集一定为零测集\\
(2)$m(E)=0 \Leftrightarrow E$ 的任何可测子集都是零测集 \\
3.设$B=\{\|x\|<1\}$ 为$R^d$中的开球且$f$ 非负可测,有 $\int_{B} f d x=1$\\
求证: 
$$\int_{B} f(x)\|x\| \ d x<1$$ 
4.求解 
$$\lim _{k \rightarrow \infty} \int_{0}^{+\infty} \frac{x+\sin ^{k} x}{1-e^{-k x}+x^{k}} d x$$
5.$f, f_{1}, \cdots, f_n$ 均在 $[0, ~ 1]$ 上可测\\
(1)$f_{n} \stackrel{L_1}{\longrightarrow} f$是否能推出$f_n \stackrel{L_2}{\longrightarrow} f$, 证明或者给出反例\\
(2)$f_{n} \stackrel{L_2}{\longrightarrow} f$是否能推出$f_n \stackrel{L_1}{\longrightarrow} f$ ,证明或者给出反例\\
(3)$f_{n} \stackrel{m}{\longrightarrow} f$是否能推出$\lim _{n \rightarrow \infty} m\left(\left\{\left|f_n-f\right|>0\right\}\right)=0$ ,证明或者给出反例\\
6.设 $E_k=\{|f| \geqslant k\}$且$f$可积,求证:\\
(1) $\lim _{k \rightarrow \infty} m(E_k)=0$\\
(2) $\sum_{k=1}^{\infty} m(E_k)<+\infty$\\
7.$g$ 为周期为 $1$ 的光滑函数且 $\int_{0}^{1} g(x) d x=0$ 。\\
(1)求证 :对任意闭区间$[a, b]$都有 
$$\lim _{n \rightarrow \infty} \int_{a}^{b} g(n x) d x=0 $$
(2)对任意可积函数$f$, 都有
$$\lim _{n \rightarrow \infty} \int_{R}f(x) g(n x) d x=0$$
 成立


\end{document} 
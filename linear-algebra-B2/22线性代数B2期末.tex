\documentclass[UTF8]{ctexart}
\usepackage{graphicx,amsfonts,amsmath,mathrsfs,amssymb,amsthm,url,color}
\usepackage{fancyhdr,indentfirst,bm,enumerate,natbib,float,tikz}
\usepackage{caption,subcaption,calligra}
\usepackage{graphicx}
\usepackage{enumitem}
\usepackage{bm}

\title{22线性代数B2期末}
\author{\calligra{NULIOUS}}
\date{}

\textheight 23cm
\textwidth 16.5cm
\topmargin -1.2cm
\oddsidemargin 0cm
\evensidemargin 0cm

\begin{document}
\maketitle
\noindent 一.填空\\
1.$A=\left(\begin{array}{cccc}1 & 0 & 1 & 0 \\ 0 & 1 & 0 & -1 \\ 0 & 0 & 1 & 0 \\ 0 & 0 & 0 & 1\end{array}\right)$ 的 Jordan 标准型为 $\underline{\hspace{1cm}}$ ,最小多项式为$\underline{\hspace{1cm}}$ ,奇异值为 $\underline{\hspace{1cm}}$\\
解: $\operatorname{diag}\left(J_{2}(1), J_{2}(1)\right) ;(x-1)^{2}$ ;略\\
特征值显然均为 1\\
且
 $$\operatorname{rank}(A-I)=2 \quad \operatorname{rank}(A-I)^{2}=0$$
那么 
$$J=\operatorname{diag}\left(J_{2}(1), J_{2}(1)\right)$$
最小多项式显然为 
$$d_{A}(x)=(x-1)^{2}$$
又
$$A^{T} A=\left(\begin{array}{cccc}1 & 0 & 1 & 0 \\ 0 & 1 & 0 & -1 \\ 1 & 0 & 2 & 0 \\ 0 & -1 & 0 & 2\end{array}\right)$$
$$\varphi_{A^{T} A}(\lambda)=\lambda^{4}-6 \lambda^{3}+11 \lambda^{2}-6 \lambda+1$$
给出近似解:
$$\lambda_{1}=0.382\quad \text{(二重)}\quad \lambda_{2}=2.62\quad  \lambda_{3}=\frac{2184}{987}$$
最后别忘记开根号\\



\noindent 2.$A=\left(\begin{array}{ccc}2 & 2 & -1 \\ 2 & -1 & 2 \\ 1 & -2 & -2\end{array}\right)$ 的正交相似标准型为 $\underline{\hspace{1cm}}$\\
解:$\left(\begin{array}{ccc}3 & 0 & 0 \\ 0 & -2 & \sqrt{5} \\ 0 & -\sqrt{5} & -2\end{array}\right)$\\
注意 $A$ 不是反对称阵,但是经检验他是规范阵 $$A^{T} A=A A^{T}=9 I$$
那么只需要求其特征值即可\\
有 $\lambda=3$ 或 $-2 \pm \sqrt{5} i$\\
则正交相似标准型为 
$$\left(\begin{array}{ccc}3 & 0 & 0 \\ 0 & -2 & \sqrt{5} \\ 0 & -\sqrt{5} & -2\end{array}\right)$$\\



\noindent 3.酉空间 $\mathbb{C}^{3}$ 中内积为标准内积,即 $(x, y)=x^{*} y$ ,对向量组 $\left(\begin{array}{l}1 \\ i \\ 1\end{array}\right),\left(\begin{array}{l}0 \\ 1 \\ i\end{array}\right),\left(\begin{array}{c}-1 \\ 0 \\ 1\end{array}\right)$ 做正交化得到的一组标准正交基为$\underline{\hspace{1cm}}$\\
解:$\frac{\sqrt{3}}{3}\left(\begin{array}{l}1 \\ i \\ 1\end{array}\right) \quad \frac{\sqrt{2}}{2}\left(\begin{array}{l}0 \\ 1 \\ i\end{array}\right) \quad \frac{\sqrt{6}}{3}\left(\begin{array}{c}-1 \\ -\frac{1}{2} i \\ \frac{1}{2} i\end{array}\right)$\\
在西空间中我们一定要注意内积定义,西空间内积有两种定义方式\\
(1)$(u, v)=u^{*} G v$ 时,这里 $G$ 为复正定阵,对一组基 $\left\{\alpha_{1}, \ldots, \alpha_{n}\right\}$ 做 Schmidt 正交化\\
先有 $$\beta_{i}=\alpha_{i}-\frac{\left(\beta_{1}, \alpha_{i}\right)}{\left(\beta_{1}, \beta_{1}\right)} \beta_{1}-\cdots-\frac{\left(\beta_{i-1}, \alpha_{i}\right)}{\left(\beta_{i-1}, \beta_{i-1}\right)} \beta_{i-1}$$
再对 $\beta_{n}$ 做标准化即可\\
(2)$(u, v)=u G v^{*}$ 时,这里 $G$ 为复正定阵,对一组基 $\left\{\alpha_{1}, \ldots, \alpha_{n}\right\}$ 做Schmidt 正交化\\
先有 $$\beta_{i}=\alpha_{i}-\frac{\left(\alpha_{i}, \beta_{1}\right)}{\left(\beta_{1}, \beta_{1}\right)} \beta_{1}-\cdots-\frac{\left(\alpha_{i}, \beta_{i-1}\right)}{\left(\beta_{i-1}, \beta_{i-1}\right)} \beta_{i-1}$$
\textbf{RK}:这是我们在欧氏空间常常写作的样子,但是在欧氏空间内积是对称的\\
再对 $\beta_{n}$ 做标准化即可\\
\textbf{RK}:或者我们可以边做正交化边标准化,这里我们拿欧氏空间举例(酉空间形式是相似的,不过要注意内积定义),对一组基 $\left\{\alpha_{1}, \ldots, \alpha_{n}\right\}$ 做 Schmidt 正交化\\
$\left\{\gamma_{1}, \cdots, \gamma_{n-1}\right\}$ 是 $\left\{\beta_{1}, \cdots, \beta_{n-1}\right\}$ 标准化得到的\\
则 $$\beta_{n}=\alpha_{n}-\sum_{i=1}^{n-1}\left(\alpha_{n}, \gamma_{i}\right) \gamma_{i}$$
再对 $\beta_{n}$ 标准化即可\\



\noindent 二.设 $\alpha$ 是欧氏空间 $\mathbb{R}^{n}$ 中的一个单位向量,记矩阵 $H=I_{n}-2 \alpha \alpha^{T}$定义映射 $\mathcal{H}: x \in \mathbb{R}^{n} \rightarrow H x \in \mathbb{R}^{n}$ ,求证: $\mathcal{H}$ 是(关于某个 $\mathrm{n}-1$ 维超平面的)反射变换\\
证:首先求其特征值\\
由 
$$\left|I_{m}-A B\right|=\lambda^{m-n}\left|I_{n}-B A\right|$$
$\mathcal{H}$ 的特征多项式为 $$\operatorname{det}\left((\lambda-1) I_{n}+2 \alpha \alpha^{T}\right)=(\lambda-1)^{n-1}(\lambda-1+2)=(\lambda-1)^{n-1}(\lambda+1)$$
则其特征值为1( $n-1$ 重)和-1(一重)\\
下面再证明 $\mathcal{H}$ 是正交变换
$$
\begin{aligned}
	H H^{T} & =\left(I_{n}-2 \alpha \alpha^{T}\right)\left(I_{n}-2 \alpha \alpha^{T}\right) \\
	& =I_{n}-4 \alpha \alpha^{T}+4 \alpha\left(\alpha^{T} \alpha\right) \alpha^{T} \\
	& =I_{n}
\end{aligned}
$$
又正交阵 $H$ 有正交相似标准型,且其特征值无虚数,那么它可以相似对角化,也就是说 $\mathbb{R}^{n}$ 可以分解成 $H$ 的特征子空间的直和\\
注意 $H$ 为正交阵,不同特征值对应的特征向量正交,我们可以取特征值为 1 的特征子空间一组标准正交基(不然可以标准正交化),再取特征值为 -1 的特征子空间的一个单位向量,把这两个向量组合并,这样我们就取得了 $\mathbb{R}^{n}$ 一组标准正交基,也就是说映射 $\mathcal{H}$ 把 $\mathbb{R}^{n}$ 中一个方向上的向量变为反向,其他方向的向量不变,特征值为 -1 的特征向量就是题目中 $n-1$ 维超平面的法向量\\



\noindent 三.给定矩阵 $A \in F^{m \times n}, ~ B \in F^{n \times p}$ 。定义线性映射 $\mathcal{A}: x \in F^{n} \rightarrow A x \in$ $F^{m}$ 与 $\mathcal{B}$ :$y \in F^{p} \rightarrow B y \in F^{n}$ 。记 $U=\operatorname{ker}(\mathcal{A} \mathcal{B})$ ,求证: $\operatorname{Im}(\mathcal{B} \mid u) \subset \operatorname{ker}(\mathcal{A})$并由此证明 $r(A B) \geq r(A)+r(B)-n$\\
证:考虑限制 
$$\mathcal{B}: \operatorname{ker}(\mathcal{A} \mathcal{B}) \rightarrow F^{n}$$
设 $\alpha \in \operatorname{ker}(\mathcal{A} \mathcal{B})$ .则 
$$\mathcal{A} \mathcal{B} \alpha=0$$
那么 $\alpha \in \operatorname{ker}(\mathcal{A})$\\
即 
$$\operatorname{Im}(\mathcal{B} \mid u) \subset \operatorname{ker}(\mathcal{A})$$
有 
$$\operatorname{dim} \operatorname{ker}(\mathcal{A}) \geq \operatorname{dim} \operatorname{Im}(\mathcal{B} \mid u)$$
分别考虑以下的维数公式:\\
对 $\mathcal{A}$ 有 
$$\operatorname{dim} \operatorname{ker} \mathcal{A}+\operatorname{dim} \operatorname{lm} \mathcal{A}=n$$
对限制 $\mathcal{B}: \operatorname{ker}(\mathcal{A} \mathcal{B}) \rightarrow F^{n}$ 有
$$\operatorname{dim} \operatorname{Im}(\mathcal{B})|\operatorname{ker}(\mathcal{A} \mathcal{B})=\operatorname{dimker}(\mathcal{A} \mathcal{B})-\operatorname{dim} \operatorname{ker}(\mathcal{B})| \operatorname{ker}(\mathcal{A} \mathcal{B})$$
对 $\mathcal{A B}$ 有 
$$\operatorname{dim} \operatorname{ker}(\mathcal{A B})=p-r(A B)$$
又 
$$\operatorname{ker}(\mathcal{B}) \subset \operatorname{ker}(\mathcal{A} \mathcal{B}) \quad \operatorname{ker}(\mathcal{B}) \mid \operatorname{ker}(\mathcal{A} \mathcal{B})=\operatorname{ker}(\mathcal{B})$$
对 $\mathcal{B}$ 有 
$$\operatorname{dim} \operatorname{ker}(\mathcal{B})=p-r(B)$$
逐个代入就有结果 
$$r(A B) \geq r(A)+r(B)-n$$\\



\noindent 四.设 $\mathcal{A}$ 是欧氏空间 $V$ 上的规范变换,$W \subset V$ 是 $\mathcal{A}$ 的不变子空间,求证 $W^{\perp}$ 也是 $\mathcal{A}$ 的不变子空间\\
证:设 $\alpha_{1}, \alpha_{2}, \ldots, \alpha_{r}$ 是 $W$ 的一组标准正交基,将其扩充为 $V$ 的一组标准正交基 $M:=\left\{\alpha_{1}, \alpha_{2}, \ldots, \alpha_{n}\right\}$ 。由于 $W$ 为 $\mathcal{A}$ 的不变子空间,所以
$$
\mathcal{A}\left(\alpha_{1}, \alpha_{2}, \ldots, \alpha_{n}\right)=\left(\alpha_{1}, \alpha_{2}, \ldots, \alpha_{n}\right)\left(\begin{array}{cc}
	A_{11} & A_{12} \\
	O & A_{22}
\end{array}\right)
$$
其中 $A_{11} \in \mathbb{R}^{r \times r}, A_{2} \in \mathbb{R}^{(n-r) \times(n-r)}$ 。由于 $\mathcal{A}$ 为规范变换,故 $\mathcal{A}$ 在 $M$ 下的矩阵为规范矩阵,从而 $A_{12}=0$ 。于是
$$
\mathcal{A}\left(\alpha_{r+1}, \alpha_{r+2}, \ldots, \alpha_{n}\right)=\left(\alpha_{r+1}, \alpha_{r+2}, \ldots, \alpha_{n}\right) A_{22}
$$
因此,$W^{\perp}=\left\langle\alpha_{r+1}, \ldots, \alpha_{n}\right\rangle$ 也是 $\mathcal{A}$ 的不变子空间.\\
\textbf{RK}:这里我们用到了一个引理:\\
设实分块方阵 $A=\left(\begin{array}{cc}A_{11} & A_{12} \\ 0 & A_{22}\end{array}\right)$ 或 $A=\left(\begin{array}{cc}A_{11} & 0 \\ A_{21} & A_{22}\end{array}\right)$ .则 $A$ 为规范方阵当且仅当 $A_{12}=0$(或 $A_{21}=0$ ),且 $A_{11}, A_{22}$ 为规范方阵\\
证:比较 $A^{T} A$ 与 $A A^{T}$ 两边元素易得\\



\noindent 五.设 $A$ 为 n 阶实对称半正定方阵,且 $A$ 的对角元全为 0 ,求证:$A=0$\\
证:半正定 $\Rightarrow$ 各阶主子式非负\\
先考虑一个对角元 $a_{i i}$\\
取二阶主子式,并考虑对称性 $$\left|\begin{array}{ll}a_{i i} & a_{i j} \\ a_{j i} & a_{j j}\end{array}\right|=0-a_{i j}^{2} \geq 0$$
则
$$ a_{i j}=a_{j i}=0 \quad \forall j \quad \text{成立}$$
又每个对角元均为 0\\
则 $$a_{i j}=0 \quad \forall i, j \quad \text{成立}$$
就有 $A=0$\\


\noindent 六.$A$ 为 n 阶复方阵,记 $Z(A)=\{B \mid A B=B A\}$ ,求证:$Z(A)$ 是复线性空间,且下列条件等价\\
(1) $\operatorname{dim} Z(A)=n$\\
(2)$A$ 的最小多项式 $d_{A}(x)$ 等于其特征多项式 $\varphi_{A}(x)$\\
(3)$Z(A)$ 中任意矩阵 $B$ 都可以写成 $A$ 的多项式的形式\\
证:先证明复线性空间\\
首先 $0, I \in Z(A)$\\
又对 $\forall X, Y \in Z(A) \quad \lambda, \mu \in \mathbb{C}$有
$$A(\lambda X+\mu Y)=\lambda A X+\mu A Y=\lambda X A+\mu Y A=(\lambda X+\mu Y) A$$
即$Z(A)$ 是复线性空间\\
这三条等价关系的核心是第二条\\
先给出要使用的书上有理标准型的定理,证明过程略\\
定理 4.4.4: 设线性变换 $\mathcal{A}: V \rightarrow V$ 在基 $\left\{\alpha_{1}, \cdots, \alpha_{n}\right\}$ 下的矩阵是 $A$ ,并设
$$
P(x)\left(x I_{n}-A\right) Q(x)=D(x):=\operatorname{diag}\left(1, \cdots, 1, f_{t+1}, \cdots, f_{n}\right)
$$
其中 $P(x), Q(x)$ 是可逆多项式矩阵,$f_{i}(x)$ 是首一多项式。令
$$
\left(\beta_{1}, \cdots, \beta_{n}\right)=\left(\alpha_{1}, \cdots, \alpha_{n}\right) P(x)^{-1}
$$
则下列结果成立:\\
(1)$\beta_{1}=\cdots=\beta_{t}=\mathbf{0}$ ;\\
(2)对于 $i>t, \beta_{i}$ 的最小多项式是 $f_{i}(x)$ ;\\
(3)$V$ 是循环子空间 $F[x] \beta_{i}(t+1 \leq i \leq n)$ 的直和.\\
定理 4.4.5(有理标准形):设 $A$ 是域 $F$ 上的 $n$ 阶方阵,设多项式方阵 $x I_{n}-A$ 的 Smith 标准形为:
$$S(x)=\operatorname{diag}\left(1, \cdots, 1, d_{t+1}(x), \cdots, d_{n}(x)\right)$$
则:\\
(1)$A$ 的特征多项式 $\varphi_{A}(x)=d_{t+1}(x) \cdots d_{n}(x)$ ,最小多项式 $d_{A}(x)=$ $d_{n}(x)$\\
(2)设 $B_{i}(i=t+1, \cdots, n)$ 是不变因子 $d_{i}(x)$ 的友阵,则 $A$ 相似于准对角阵 $\operatorname{diag}\left(B_{t+1}, \cdots, B_{n}\right)$ \\
推论 4.4.6 :设 $d_{i}(x)(i=t+1, \cdots, n)$ 的初等因子组是 $\left\{p_{i j}^{e_{i j}}(x) \mid 1 \leq j \leq\right.$ $\left.j_{i}\right\}$ ,并设 $B_{i j}$ 是 $p_{i j}^{e_{i j}}(x)$ 的友阵,则 $A$ 相似于以 $B_{i j}(t+1 \leq i \leq$ $\left.n, 1 \leq j \leq j_{i}\right)$ 为准对角阵块的准对角阵。\\
此条推论同时说明了特征方阵的初等因子与 Jordan 标准型的 Jordan块是一一对应的\\
先证明 $(2) \Rightarrow(3)$\\


\noindent 由定理 4.4.4 我们得到(2)的一个等价条件:存在一个向量 $\beta$ s.t.
$$V=\mathbb{C}^{n \times 1}=\mathbb{C}[\mathcal{A}] \beta$$
证:设 $V=\mathbb{C}^{n \times 1}$ 和 $\mathcal{A}: X \rightarrow A X, V \rightarrow V$\\
$A$ 的特征方阵 $\lambda I-A$ 柤抵 于 Smith 标 准 型 
$$S(\lambda)=\left(I(s), d_{s+1}(\lambda), \cdots, d_{n}(\lambda)\right)$$
又 $\varphi_{A}(x)=d_{A}(x)$ 等价于 $d_{s+1}(\lambda)=\cdots=d_{n}(\lambda)=1$ 即得(定理 4.4.4)\\


\noindent 设 $A$ 与 $B$ 可交换,那么 $V$ 上线性变换 $\mathcal{B}: y \rightarrow B y$ 与 $\mathcal{A}: X \rightarrow A X$ 可交换\\
$$\mathcal{B} \beta \in V=\mathbb{C}[A] \beta$$
 即存在 $g(x)$ s.t. $$\mathcal{B} \beta=g(\mathcal{A}) \beta$$
又 $\forall v \in V$ 可以写成 $v=h(\mathcal{A}) \beta$ ,其中 $h$ 为多项式\\
就有 
$$\mathcal{B} v=\mathcal{B} h(\mathcal{A}) \beta=h(\mathcal{A}) \mathcal{B} \beta=h(\mathcal{A}) g(\mathcal{A}) \beta=g(\mathcal{A}) h(\mathcal{A}) \beta=g(\mathcal{A}) v$$
也就是 
$$\mathcal{B}=g(\mathcal{A})$$
即 $B=g(A)$\\
(1)与(2)等价的证明\\
我们不妨直接考虑 Jordan 标准型\\
引理1:与单个 Jordan 块可交换的矩阵 $B$ 构成的线性空间维数等于此Jordan 块的阶数\\
证:$N_{n}=J_{n}(\lambda)-\lambda I_{n}$ 也与 $B$ 可交换,比较两边元素即可,$B$ 形如 $\left(\begin{array}{cccc}b_{11} & b_{12} & \ldots & b_{1 n} \\ & \ddots & \ddots & \vdots \\ & & & b_{12} \\ & & & b_{11}\end{array}\right)$\\
引理2:与两个及以上相同特征值 Jordan 块构成的 Jordan 标准型 $J$ 可交换的矩阵构成的线性空间维数比 $J$ 的阶数大\\
证:考虑两个相同特征值 Jordan 块构成的 Jordan 标准型 $J$ 即可,$J-\lambda I_{n}$也与 $B$ 可交换,比较两边元素立得\\
再给出一个(2)的等价条件:$A$ 的 Jordan 标准型的每个特征值对应的 Jordan 块都只有一个(这是推论 4.4.6 的直接结果)\\
结合上面的引理和等价条件就说明(1)与(2)等价\\
\textbf{RK}:事实上,利用此方法配合中国剩余定理可以得到(2)$\Rightarrow$(3)的另外一种证法\\
$(3) \Rightarrow(1)$\\
先由引理二我们知道 
$$\operatorname{dim} Z(A) \geq n$$
又特征多项式零化 $A$ ,那么
$$\operatorname{dim} Z(A) \leq n$$
则 
$$\operatorname{dim} Z(A)=n$$\\


\noindent 至此我们证明了这三个命题是等价的\\



\noindent \textbf{RK}:给出扩充版的等价条件:\\
$A$ 为 n 阶复方阵,记 $Z(A)=\{B \mid A B=B A\}$ ,则 $Z(A)$ 是复线性空间,且下列条件等价\\
(1) $\operatorname{dim} Z(A)=n$\\
(2)$A$ 的最小多项式 $d_{A}(x)$ 等于其特征多项式 $\varphi_{A}(x)$\\
(3)$Z(A)$ 中任意矩阵 $B$ 都可以写成 $A$ 的多项式的形式\\
(4)存在一个向量 $\beta$ s.t.$V=\mathbb{C}^{n \times 1}=\mathbb{C}[\mathcal{A}] \beta$\\
(5) $\mathcal{A}$ 是循环变换\\
(6)$A$ 的特征方阵的行列式因子为 $1, \cdots, 1, f(x)$\\
(7)$A$ 的特征方阵的不变因子为 $1, \cdots, 1, f(x)$\\
(8)$A$ 任意一个特征值的特征子空间维数为 $1\left(r\left(A-\lambda_{i} I\right)=n-1\right)$\\
(9)$A$ 的特征方阵的初等因子组为 $p_{1}(x)^{r_{1}}, \cdots, p_{k}(x)^{r_{k}}$ ,其中 $p_{1}(x), \cdots, p_{k}(x)$ 为数域 $F$ 上互异的首一多项式\\
(10)$A$ 的 Jordan 标准型的每个特征值对应的 Jordan 块都只有一个\\
(11) $\mathcal{A}$ 在一组基下的矩阵为友阵\\
其余等价条件均是书上定理的简单推论,在这里不再证明了

\end{document}
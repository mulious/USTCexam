\documentclass[UTF8]{ctexart}
\usepackage{graphicx,amsfonts,amsmath,mathrsfs,amssymb,amsthm,url,color}
\usepackage{fancyhdr,indentfirst,bm,enumerate,natbib,float,tikz}
\usepackage{caption,subcaption,calligra}
\usepackage{graphicx}
\usepackage{enumitem}
\usepackage{bm}

\title{21线性代数B2期末}
\author{\calligra{NULIOUS}}
\date{}

\textheight 23cm
\textwidth 16.5cm
\topmargin -1.2cm
\oddsidemargin 0cm
\evensidemargin 0cm

\begin{document}
\maketitle
\noindent 一.填空\\
1.设矩阵 $A=\left(\begin{array}{lll}1 & 0 & 1 \\ 0 & 1 & 0 \\ 0 & 0 & 1\end{array}\right)$ .则 $A$ 的 Jordan 标准形是 $\underline{\hspace{1cm}}$ ,$A$的最小多项式是 $\underline{\hspace{1cm}}$ ,$A$ 的奇异值是 $\underline{\hspace{1cm}}$ . $\mathbb{R}^{3}$ 上的线性变换 $\mathcal{A}: x \rightarrow A x$ 的二维不变子空间为 $\underline{\hspace{1cm}}$\\
解: $\operatorname{diag}\left(J_{2}(1), 1\right) ;(x-1)^{2} ; 1, \sqrt{\frac{3 \pm \sqrt{5}}{2}}$ ;$\left\langle\left(\begin{array}{l}1 \\ 0 \\ 0\end{array}\right),\left(\begin{array}{l}0 \\ 0 \\ 1\end{array}\right)\right\rangle$ 或 $\left\langle\left(\begin{array}{l}1 \\ 0 \\ 0\end{array}\right),\left(\begin{array}{l}0 \\ 1 \\ 0\end{array}\right)\right\rangle$\\
显然特征值为 $\lambda=1$(三重)\\
又 
$$r(A-I)=1 \quad r(A-I)^{2}=0$$
则
$$ J=\operatorname{diag}\left(J_{2}(1), 1\right)$$
由有理标准型知识知道最小多项式为 $$d_{A}(x)=(x-1)^{2}$$
另一方面
$$A A^{T}=\left(\begin{array}{lll}2 & 0 & 1 \\ 0 & 1 & 0 \\ 1 & 0 & 1\end{array}\right)$$
其特征值为 $1, \frac{3 \pm \sqrt{5}}{2}$\\
则$A$的奇异值为 $1, \sqrt{\frac{3 \pm \sqrt{5}}{2}}$\\
$\mathcal{A}$ 的二维不变子空间必然包含其特征子空间,具体证明请仿照 23 年期末第四题(3)\\
逐个检验二维线性空间就有 $\mathcal{A}$ 的二维不变子空间
$$
\left\langle\left(\begin{array}{l}
	1 \\
	0 \\
	0
\end{array}\right),\left(\begin{array}{l}
	0 \\
	0 \\
	1
\end{array}\right)\right\rangle \text { 或 }\left\langle\left(\begin{array}{l}
	1 \\
	0 \\
	0
\end{array}\right),\left(\begin{array}{l}
	0 \\
	1 \\
	0
\end{array}\right)\right\rangle
$$\\



\noindent 2.在四维欧式空间 $\mathbb{R}^{4}$ 中,$W$ 是由 $\alpha_{1}=(1,0,1,0)$ 与 $\alpha_{2}=(0,1,0,1)$生成的子空间。向量 $\alpha=(1,1,-1,-1)$ 在 $W$ 中的正交投影向量 $\beta$(即满足 $\alpha-$ $\beta \in W^{\perp}$ 且在 $W$ 中的向量 $\beta$ )是 $\underline{\hspace{1cm}}$\\
解:( $0,0,0,0$ )\\
先标准正交化 $\alpha_{1}, \alpha_{2}$ ,注意其已经正交,只需单位化\\
又注意到 $\alpha$ 实际上已经正交于 $\alpha_{1}, \alpha_{2}$\\
则 
$$\beta=(0,0,0,0)$$
\textbf{RK}:补充不同内积定义下的西空间上求正交投影的方法:\\
(1)$(u, v)=u^{*} G v$ 时,这里 $G$ 为复正定阵,给定 $W$ 一组标准正交基 $\left\{\alpha_{1}, \ldots, \alpha_{n}\right\}$
对任意 $\alpha \in V, \alpha$ 在 $W$ 上的正交投影存在唯一且 
$$P \alpha=\left(\alpha_{1}, \alpha\right) \alpha_{1}+\left(\alpha_{2}, \alpha\right) \alpha_{2}+\ldots+\left(\alpha_{n}, \alpha\right) \alpha_{n}$$
(2)$(u, v)=u G v^{*}$ 时,这里 $G$ 为复正定阵,给定 $W$ 一组标准正交基 $\left\{\alpha_{1}, \ldots, \alpha_{n}\right\}$对任意 $\alpha \in V, \alpha$ 在 $W$ 上的正交投影存在唯一且 
$$P \alpha=\left(\alpha, \alpha_{1}\right) \alpha_{1}+\left(\alpha, \alpha_{2}\right) \alpha_{2}+\ldots+\left(\alpha, \alpha_{n}\right) \alpha_{n}$$\\



\noindent 3.复方阵 $A=\left(\begin{array}{ccc}0 & i & 1 \\ -i & 0 & i \\ 1 & -i & 0\end{array}\right)$ 的西相似标准形为 $\underline{\hspace{1cm}}$\\
解: $\operatorname{diag}(1,1,-2)$\\
$A$有特征值$\lambda_{1}=1$(二重)$\lambda_{2}=-2$\\
又 $$\mathrm{r}(A-I)=1$$
则
$$ J=\operatorname{diag}(1,1,-2)$$\\



\noindent 二.给定数域 $\mathbb{F}$ 的 $n$ 阶方阵 $A$ ,定义 $V=\mathbb{F}^{n \times n}$ 上的线性变换 $\mathcal{A}: X \rightarrow A X-X A$.如果 $A$ 可以对角化, $\mathcal{A}$ 是否也可以对角化?请说明理由。\\
证:可以对角化\\
$A$ 可以对角化,则存在可逆阵 $P$ 使得
 $$P^{-1} A P=D=\operatorname{diag}\left(\lambda_{1}, \cdots, \lambda_{n}\right)$$
我们考虑构造一组基使得 $\mathcal{A}$ 在这组基下为对角阵,注意一组基经过可逆线性变换之后还是一组基\\
先取 $\mathbb{F}^{n \times n}$ 自然基 $$E=\left\{E_{i j} \mid 1 \leq i, j \leq n\right\}$$
构造 
$$M=\left\{X_{i j} \mid 1 \leq i, j \leq n\right\}$$
 其中 $X_{i j}=P E_{i j} P^{-1}$\\
 则
$$
\begin{aligned}
	\mathcal{A}\left(X_{i j}\right)&=A  X_{i j}-X_{i j} A\\
	&=P D P^{-1} P E_{i j} P^{-1}-P E_{i j} P^{-1} P D P^{-1} \\
	& =P\left(D E_{i j}-E_{i j} D\right) P^{-1}\\
	&=P\left(\lambda_{i} E_{i j}-E_{i j} \lambda_{j}\right) P^{-1} \\
	& =\left(\lambda_{i}-\lambda_{j}\right) X_{i j}
\end{aligned}
$$
这就说明了 $\mathcal{A}$ 在 $M=\left\{X_{i j} \mid 1 \leq i, j \leq n\right\}$ 下的矩阵为对角阵\\



\noindent 三.设 $A$ 为 $n$ 阶复方阵,$k$ 为正整数。用 Jordan 标准形证明:
$$
\operatorname{rank} A^{k}-\operatorname{rank} A^{k+1} \geq \operatorname{rank} A^{k+1}-\operatorname{rank} A^{k+2}
$$
证:相似的矩阵秩相同,不妨考虑 Jordan 标准型\\
首先可逆阵自然满足等号成立(特征值均非零)\\
或考虑 $\lambda \neq 0$ 时有 $$r(J(\lambda))=r\left(J^{k}(\lambda)\right)$$
再考虑特征值为 0 的 Jordan 块:\\
一阶 Jordan 块有 
$$J_{1}^{k}(0)=J_{1}(0)$$
秩不变\\
$J_{n}(0)=N_{n}$ 有性质
$$r\left(N_{n}^{k}\right)=\left\{\begin{array}{l}n-k, k<n \\ 0, k \geq n\end{array}\right.$$
即 $$r\left(N_{n}^{k}\right)-r\left(N_{n}^{k+1}\right)= \begin{cases}1, & k<n \\ 0, & k \geq n\end{cases}$$
综合以上结果就有 $\operatorname{rank} A^{k}-\operatorname{rank} A^{k+1} \geq \operatorname{rank} A^{k+1}-\operatorname{rank} A^{k+2}$\\



\noindent 四.设矩阵 $A=\left(\begin{array}{ccc}\frac{2}{3} & \frac{2}{3} & -\frac{1}{3} \\ \frac{2}{3} & -\frac{1}{3} & \frac{2}{3} \\ \frac{1}{3} & -\frac{2}{3} & -\frac{2}{3}\end{array}\right)$\\
1.求矩阵 $A$ 的正交相似标准形.\\
2.设 $\mathbb{R}^{3}$ 上的线性变换 $\mathcal{A}: x \rightarrow A x \in \mathbb{R}^{3}$ .证明: $\mathcal{A}$ 是绕过原点的直线 $l$ 的旋转变换,并求变换的轴 $l$ 及旋转角度 $\theta$\\
解:1. $\operatorname{det}(A)=1$\\
有特征值 
$$\lambda_{1}=1 \quad \lambda_{2}=-\frac{2}{3}+\frac{\sqrt{5}}{3} i \quad \lambda_{3}=-\frac{2}{3}-\frac{\sqrt{5}}{3} i$$
那么其正交相似标准型为 $$\left(\begin{array}{ccc}1 & 0 & 0 \\ 0 & -\frac{2}{3} & \frac{\sqrt{5}}{3} \\ 0 & -\frac{\sqrt{5}}{3} & \frac{2}{3}\end{array}\right)$$
2.先给出一般性方法\\
对三阶正交阵 $A$ 且 $\operatorname{det}(A)=1$ ,存在正交阵 $T$ 使得
$$T^{T} A T=\left(\begin{array}{lll}1 & 0 & 0 \\ 0 & \cos \theta & -\sin \theta \\ 0 & \sin \theta & \cos \theta\end{array}\right)$$
即其正交相似标准型为旋转变换\\
旋转轴为特征值 1 对应特征向量的所在直线\\
旋转角有 
$$\cos \theta=\frac{\operatorname{tr}(A)-1}{2}$$
由 $A$ 与 $\left(\begin{array}{lll}1 & 0 & 0 \\ 0 & \cos \theta & -\sin \theta \\ 0 & \sin \theta & \cos \theta\end{array}\right)$ 相似立得\\
对本题来说\\
$\lambda_{1}=1$ 对应特征向量为 $\left(\begin{array}{l}2 \\ 1 \\ 0\end{array}\right)$\\
旋转轴为直线 $$l:\left(\begin{array}{l}2 \\ 1 \\ 0\end{array}\right) t \quad$$
这里 $t \in \mathbb{R}$\\
旋转角为 
$$\arccos \left(-\frac{2}{3}\right)$$\\



\noindent 五.设 $\mathcal{A}$ 是有限维欧式空间 $V$ 上的线性变换。证明:
$$
V=\operatorname{Im} \mathcal{A} \oplus \operatorname{Ker} \mathcal{A}
$$
证:本题题目有误,反例可以由 23 年期末第三题给出\\
如:
$$\mathcal{A}: x \rightarrow\left(\begin{array}{ll}0 & 1 \\ 0 & 0\end{array}\right) x . x \in \mathbb{R}^{2}$$
则$$\binom{1}{0} \in \operatorname{Im} \mathcal{A}\quad \binom{1}{0} \in \operatorname{ker} \mathcal{A}$$ 
因为 $$\mathcal{A}\binom{0}{1}=\binom{1}{0}$$
则 $\operatorname{Im} \mathcal{A}+\operatorname{ker} \mathcal{A}$ 不是直和\\



\noindent 六.设 $A, B$ 为同阶实对称方阵,且 $A \geq B \geq 0$(即 $A \geq 0, B \geq 0$ 且 $B-A \geq 0$ ),证明:$\sqrt{A} \geq \sqrt{B}$\\
证:\\
证法 1:$\sqrt{A}-\sqrt{B}$ 仍然是对称阵,有正交相似标准型,即存在正交矩阵 $P$ ,使得
$$
P^{T}(\sqrt{A}-\sqrt{B}) P=\operatorname{diag}\left(\lambda_{1}, \ldots, \lambda_{n}\right)
$$
设 $P^{T} \sqrt{A} P=\left(a_{i j}\right), P^{T} \sqrt{B} P=\left(b_{i j}\right)$ .于是
$$
a_{i j}=b_{i j}, i \neq j ; a_{i i}-b_{i i}=\lambda_{i}, 1 \leq i \leq n
$$
设
$$
C=\left(c_{i j}\right)=P^{T}(A-B) P=\left(a_{i j}\right)^{2}-\left(b_{i j}\right)^{2}
$$
则
 $$c_{i i}=\sum_{j=1}^{n} a_{i j}^{2}-\sum_{j=1}^{n} b_{i j}^{2}=\lambda_{i}\left(a_{i i}+b_{i i}\right)$$
因为 
$$P^{T} \sqrt{A} P, P^{T} \sqrt{B} P \geq 0, C \geq 0$$
所以 
$$a_{i i}, b_{i i} \geq 0, \lambda_{i}\left(a_{i i}+b_{i i}\right) \geq 0,1 \leq i \leq n$$
如果 $a_{i i}>0$ ,则有 $\lambda_{i} \geq 0$ ,否则 $a_{i i}=b_{i i}=\lambda_{i}=0$.\\
证法 2:设 $\sqrt{A}-\sqrt{B}$ 特征值 $\lambda$ 的特征向量为 $\alpha$ ,则
$$
(\sqrt{A}-\sqrt{B}) \alpha=\lambda \alpha
$$
即 
$$\sqrt{A} \alpha=(\sqrt{B}+\lambda I) \alpha,(\sqrt{A}-\lambda I) \alpha=\sqrt{B} \alpha$$
左乘转置就可以得到
$$
\begin{aligned}
	\alpha^{T}(A-B) \alpha & =2 \lambda \alpha^{T} \sqrt{B} \alpha+\lambda^{2} \alpha^{T} \alpha \\
	\alpha^{T}(A-B) \alpha & =2 \lambda \alpha^{T} \sqrt{A} \alpha-\lambda^{2} \alpha^{T} \alpha
\end{aligned}
$$
因此有
$$
\alpha^{T}(A-B) \alpha=\lambda \alpha^{T}(\sqrt{A}+\sqrt{B}) \alpha
$$
如果 $\alpha^{T}(\sqrt{A}+\sqrt{B}) \alpha>0$ ,则有 $\lambda \geq 0$ .否则 $\alpha^{T} \sqrt{A} \alpha=0, \alpha^{T} \sqrt{B} \alpha=$ 0 .根据定义也有 $\alpha^{T} A \alpha=0, \alpha^{T} B \alpha=0$ .由上式可以得到 $\lambda=0$ .综上所述 $\lambda \geq 0$ .\\
证法 3:设有相合规范型 $$P^{T}(\sqrt{A}+\sqrt{B}) P=\left(\begin{array}{cc}I_{r} & O \\ O & O\end{array}\right)$$
并设
$$
P^{T} \sqrt{A} P=\left(\begin{array}{ll}
	A_{1} & A_{2} \\
	A_{3} & A_{4}
\end{array}\right), P^{T} \sqrt{B} P=\left(\begin{array}{ll}
	B_{1} & B_{2} \\
	B_{3} & B_{4}
\end{array}\right)
$$
则
$$
A_{1}+B_{1}=I, A_{2}+B_{2}=A_{3}+B_{3}=O=A_{4}+B_{4}
$$
因为 $A_{4}, B_{4}$ 半正定,所以 $A_{4}=B_{4}=O$ 。再根据22期末第五题可以得到,$A_{2}=A_{3}=B_{2}=B_{3}=O$ 。存在 $r$ 阶正交矩阵 $P_{1}$ ,使得 $P_{1}^{T} A_{1} P_{2}=\operatorname{diag}\left(\lambda_{1}, \ldots, \lambda_{r}\right)$ ,取 $P_{2}=P \operatorname{diag}\left(P_{1}, I_{n-r}\right)$ ,有
$$
P_{2}^{T} \sqrt{A} P_{2}=\operatorname{diag}\left(\lambda_{1}, \ldots, \lambda_{r}, O\right), P_{2}^{T} \sqrt{B} P_{2}=\operatorname{diag}\left(1-\lambda_{1}, \ldots, 1-\lambda_{r}, O\right)
$$
所以
$$
P_{2}^{T}(\sqrt{A}-\sqrt{B}) P_{2}=\operatorname{diag}\left(2 \lambda_{1}-1, \ldots, 2 \lambda_{r}-1,0\right)
$$
设 $C=\sqrt{A}+\sqrt{B}$ 
$$
\begin{aligned}
	P_{2}^{T}(A-B) P_{2}= & P_{2}^{T}(\sqrt{A}-\sqrt{B})(\sqrt{A}+\sqrt{B}) P_{2}+P_{2}^{T}(\sqrt{B} \sqrt{A}-\sqrt{A} \sqrt{B}) P_{2} \\
	= & \operatorname{diag}\left(2 \lambda_{1}-1, \ldots, 2 \lambda_{r}-1, O\right) P_{2}^{-1} P_{2}^{-T} \operatorname{diag}\left(I_{r}, O\right)+P_{2}^{T}(\sqrt{B} \sqrt{A}-\sqrt{A} \sqrt{B}) P_{2}
\end{aligned}
$$
第二个式子是反对称矩阵,所以对角元素为 $0 . P_{2}^{-1} P_{2}^{-T}$ 是正定矩阵,所以对角元素大于 0 ,根据 $A-B$ 是半正定的,所以上式对角元素
$$
2 \lambda_{i}-1 \geq 0
$$
因此 $\sqrt{A}-\sqrt{B}$ 是半正定矩阵。
\end{document}
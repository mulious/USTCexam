\documentclass[UTF8]{ctexart}
%\documentclass{article}
\usepackage{graphicx,amsfonts,amsmath,mathrsfs,amssymb,amsthm,url,color}
\usepackage{fancyhdr,indentfirst,bm,enumerate,natbib, float,tikz,graphicx}
\usepackage{caption}
\usepackage{subcaption}
\usepackage{calligra} 

\title{24实用随机过程期中}
\author{\textcalligra{NULIOUS}} 
\date{}

\textheight 23cm
\textwidth 16.5cm
\topmargin -1.2cm
\oddsidemargin 0cm
\evensidemargin 0cm

\begin{document}
\maketitle
\noindent 1.(20 分)系统有两个编号为 1,2 的服务台,第 $i$ 号服务台给顾客提供的服务时间服从失效率为常数 $\lambda_{i}$ 的指数分布,其中 $i=1,2$ ,不同顾客的服务时间相互独立.采用先到先服务、后到排队的原则。当 A 到达系统时,发现 B 和 C 占据了两个服务台,求 A 在系统中滞留时间 $T$ 的期望.\\
解:利用指数分布的性质
\[
X_1\sim \mathrm{Exp}(\lambda_1) \quad X_2\sim \mathrm{Exp}(\lambda_2) 
\]
则有
\[
P(X_1<X_2)=\frac{\lambda_1}{\lambda_1+\lambda_2}
\]
和
\[
\min\{X_1,X_2\} \sim \mathrm{Exp}(\lambda_1+\lambda_2)
\]
设顾客$B$与$C$的服务时长分别为$X_1$与$X_2$,则
\begin{align*}
	\mathbb{E}\left[\text{顾客A在系统中滞留的时间} \right]  &= \mathbb{E}\left[\text{顾客A在系统中滞留的时间}\mid \text{B比C先走} \right]P(\text{B比C先走})\\
	&+\mathbb{E}\left[\text{顾客A在系统中滞留的时间}\mid \text{C比B先走} \right]P(\text{C比B先走}) \\
	& = \left(\frac{1}{\lambda_1+\lambda_2}+\frac{1}{\lambda_1} \right)\cdot \frac{\lambda_1}{\lambda_1+\lambda_2}+ \left(\frac{1}{\lambda_1+\lambda_2}+\frac{1}{\lambda_2} \right)\cdot \frac{\lambda_2}{\lambda_1+\lambda_2}\\
	&=\frac{3}{\lambda_1+\lambda_2}
\end{align*}\\


\noindent 2.(每小题 6 分,总 24 分)考虑一个 $M / G / \infty$ 系统,顾客到达系统的规律可以用齐次 Poisson 过程来描述,单位时间内平均到达的顾客数为 5 ,每个顾客需要服务员提供的服务时间相互独立,服从区间 $(1,3)$ 上的均匀分布,系统有无穷多个服务员(即顾客到达系统后立即能得到服务)\\
(1)求于 $(0,4]$ 到达,且于时刻 5 未被服务完毕的顾客人数 $M_{1}$ 服从的分布;\\
(2)求于 $(0,4]$ 到达,且于 $(4,5]$ 内被服务完毕的顾客人数 $M_{2}$ 服从的分布;\\
(3)求于 $(3,4]$ 到达,且于时刻 5 末被服务完毕的顾客人数 $M_{3}$ 服从的分布;\\
(4)判断 $M_{1}, M_{2}, M_{3}$ 两两之间的独立性.\\
解:(1)服务时间$G$的密度为
\[
g(x)=
\begin{cases}
	\frac{1}{2}  &  1<x<3 \\
	0  &  \text{其他}
\end{cases}
\]
其分布函数为
\[
G(x)=
\begin{cases}
	1  &  x<1 \\
	\frac{x-1}{2}  &  1\le x\le 2\\
	0  &  x>2 
\end{cases}
\]
则服务时间的生存函数为
\[
\bar{G}(x)=1-G(x)=
\begin{cases}
	0  &  x<1 \\
	\frac{3-x}{2}  &  1\le x\le 2\\
	1  &  x>2 
\end{cases}
\]
于 $(0,4]$ 中$s$时刻到达,且于时刻 5 未被服务完毕的概率为
\[
P_1(s)=
\begin{cases}
	\bar{G}(5-s) &  s\le 4 \\
	0  &  s > 4
\end{cases}
=
\begin{cases}
	\frac{s-2}{2} & 2\le s\le 4 \\
	0  &  \text{其他}
\end{cases}
\]
则由分类Poisson过程有
\[
M_1 \sim \mathrm{P}\left(\lambda \int_{0}^{5} P_1(s)ds  \right) =\mathrm{P}(5)
\]
(2)于 $(0,4]$ 中$s$时刻到达,且于$(4,5]$内被服务完毕的概率为
\[
P_2(s)=
\begin{cases}
	1  &  0<s\le 2 \\
	G(5-s)  &  2<s\le 4\\
	0 & s>4
\end{cases}
=
\begin{cases}
	1  &  0<s\le 2 \\
	\frac{4-s}{2}  &  2<s\le 4\\
	0 & s>4
\end{cases}
\]
则由分类Poisson过程有
\[
M_2 \sim \mathrm{P}\left(\lambda \int_{0}^{5} P_2(s)ds  \right) =\mathrm{P}(15)
\]
(3)于 $(3,4]$ 中$s$时刻 到达,且于时刻 5 末被服务完毕的概率为
\[
P_3(s)=
\begin{cases}
	0  &  s\le 3 \\
	\bar{G}(5-s)  &  3<s\le 4\\
	0 & s>4
\end{cases}
=
\begin{cases}
	0  &  s\le 3 \\
	\frac{s-2}{2}  &  3<s\le 4\\
	0 & s>4
\end{cases}
\]
则由分类Poisson过程有
\[
M_3 \sim \mathrm{P}\left(\lambda \int_{0}^{5} P_3(s)ds  \right) =\mathrm{P}\left(\frac{15}{4} \right) 
\]
(4)由分类Poisson过程,$M_1$和$M_2$是独立的,$M_2$和$M_3$也是独立的,而$M_1$和$M_3$不是独立的,因为原Poisson过程$N(t)$中的$N(4)$和$N(4)-N(3)$不是独立的\\





\noindent 3.(每小题 4 分,总 16 分)假设一个元件于时刻 0 开始投入使用,该元件易于受到外界的冲击,时间单位按小时计算.在时间段 $(0,3]$ 内冲击以每小时 2 个的泊松速率到达,在时间段 $(3,6]$ 内冲击以每小时 3 个的泊松速率到达,在其后的时间段 $(6,+\infty)$ 内冲击以每小时 1 个泊松速率到达.泊松速率到达是指不可能有两个或多个冲击同时到达.\\
(1)问冲击到达可以用什么样的过程来描述?\\
(2)求时间段 $(2,4]$ 有 1 个冲击发生的概率.\\
(3)求时间段 $(2,4]$ 和 $(4,6]$ 中各有 1 个冲击发生的概率.\\
(4)求前 10 个小时之内到达冲击期望个数\\
解:(1)这是一个非齐次Poisson过程$N(t)$,有强度函数
\[
\lambda(t)=
\begin{cases}
	2  &  0<t\le 3 \\
	3  &  3<t\le 6 \\
	1  &  t>6
\end{cases}
\]
(2)
\[
N(4)-N(2)\sim \mathrm{P}\left(\int_{2}^{4}\lambda(s)ds \right)=\mathrm{P}(5) 
\]
则
\[
P(N(4)-N(2)=1)=5e^{-5}
\]
(3)由独立增量性,两个事件是独立的,且
\[
N(6)-N(4)\sim \mathrm{P}\left(\int_{4}^{6}\lambda(s)ds \right)=\mathrm{P}(6) 
\]
则
\begin{align*}
	P(N(4)-N(2)=1,N(6)-N(4)=1) &= P(N(4)-N(2)=1) P(N(6)-N(4)=1)\\
	& =30e^{-11}
\end{align*}
(4)
\[
N(10)\sim \mathrm{P}\left(\int_{0}^{10}\lambda(s)ds \right)=\mathrm{P}(19) 
\]
则
\[
\mathbb{E}[N(10)]=19
\]\\



\noindent 4.(16 分)以 $A(t)$ 和 $Y(t)$ 记一个更新过程在时刻 $t$ 的年龄和剩余寿命,且假设更新间隔服从参数为 $\lambda$ 的指数分布,求 $\mathrm{P}(Y(t)>x \mid A(t+x)>s)$ ,其中 $s<t+x$ .\\
解:\\
i.$s<x$时\\
$A(t+x)>s$说明$(t+x-s,t+x)$时间段内不可能发生更新,则利用指数分布的无记忆性和生存函数$$\bar{F}(x)=e^{-\lambda x}$$
有
\begin{align*}
	\mathrm{P}(Y(t)>x \mid A(t+x)>s) & =P\left( (t,t+x)\text{时间段内不发生更新}\mid (t+x-s,t+x)\text{时间段内不发生更新}\right)  \\
	& = \frac{ P\left( (t,t+x)\text{时间段内不发生更新}, (t+x-s,t+x)\text{时间段内不发生更新}\right)}{ P\left(  (t+x-s,t+x)\text{时间段内不发生更新}\right)}\\
	&=\frac{ P\left( (t,t+x)\text{时间段内不发生更新}\right)}{ P\left(  (t+x-s,t+x)\text{时间段内不发生更新}\right)}\\
	&=\frac{\bar{F}(x)}{\bar{F}(s)}\\
	&=e^{-\lambda (x-s)}
\end{align*}
ii.$x\le s<t+x$时\\
$A(t+x)>s$就说明在时刻$t+s$距离上一次更新的时间间隔大于$s$,也就大于$x$,所以$(t,t+x)$时间段内不可能发生更新,这就说明$t$时刻距离下一次更新的时间间隔大于$x$,即$Y(t)>x$必然发生
\[
\mathrm{P}(Y(t)>x \mid A(t+x)>s)=1
\]
\textbf{RK}:更新间隔服从参数为 $\lambda$ 的指数分布的更新过程事实上就是参数为 $\lambda$ 的Poisson过程\\



\noindent 5.(每小题6分,总24分)观察一列独立同分布的离散随机变量序列 $\left\{W_{n}, n \geq 1\right\}$ ,已知
$$
\mathrm{P}\left(W_{1}=0\right)=\frac{1}{6}, \quad \mathrm{P}\left(W_{1}=1\right)=\frac{1}{3}, \quad \mathrm{P}\left(W_{1}=2\right)=\frac{1}{2}
$$
(1)分别求等待花样$"121"$和花样$"212"$首次发生所需要的期望时间.\\
(2)给定花样$"121"$已发生,求等待花样$"212"$首次发生所需要的额外期望时间.\\
(3)求等待花样$"121"$或花样$"212"$首次发生所需要的期望时间.\\
(4)求花样$"121"$于花样$"212"$之前发生的概率.\\
解:花样问题\\
(1)"121"有重叠"1","1"没有重叠;"212"有重叠"2","2"没有重叠,则
\[
\mathbb{E}[N_{121}]=\left( P^2\left(W_{1}=1\right)P\left(W_{1}=2\right) \right)^{-1} +\left(P\left(W_{1}=1\right) \right)^{-1}=21
\]
\[
\mathbb{E}[N_{212}]=\left(P\left(W_{1}=1\right)P^2\left(W_{1}=2\right) \right)^{-1} +\left(P\left(W_{1}=2\right) \right)^{-1}=14
\]
(2)花样$"121"$给花样$"212"$提供了花样$"21"$,并注意到"21"没有重叠,因此
\[
\mathbb{E}[N_{212\mid 121}]=\mathbb{E}[N_{212}]-\mathbb{E}[N_{21}]=21-6=15
\]
类似的,花样$"212"$给花样$"121"$提供了花样$"12"$,并注意到"12"没有重叠,因此
\[
\mathbb{E}[N_{121\mid 212}]=\mathbb{E}[N_{121}]-\mathbb{E}[N_{12}]=14-6=8
\]
(3)+(4)设$P_A=P(\text{花样$"121"$于花样$"212"$之前发生})$,$M=\min(N_{121},N_{212})$\\
对花样$"121"$是否于花样$"212"$之前发生取条件,有
\begin{align*}
	\mathbb{E}[N_{121}] & =\mathbb{E}[M]+\mathbb{E}[N_{121}-M] \\
	 & = \mathbb{E}[M]+\mathbb{E}[N_{121}-M\mid \text{花样$"121"$于花样$"212"$之前发生}](1-P_A)\\
	 &=\mathbb{E}[M]+\mathbb{E}[N_{121\mid 212}](1-P_A)
\end{align*}
类似的
\[
\mathbb{E}[N_{212}]=\mathbb{E}[M]+\mathbb{E}[N_{212\mid 121}]P_A
\]
最后
\[
P_A=\frac{\mathbb{E}[N_{212}]+\mathbb{E}[N_{121\mid 212}]-\mathbb{E}[N_{121}]}{\mathbb{E}[N_{212\mid 121}]+\mathbb{E}[N_{121\mid 212}]}=\frac{1}{23}
\]
\[
\mathbb{E}[M]=\mathbb{E}[N_{212}]-\mathbb{E}[N_{212\mid 121}]P_A=\frac{307}{23}
\]


\end{document}
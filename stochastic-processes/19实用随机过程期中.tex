\documentclass[UTF8]{ctexart}
\usepackage{graphicx,amsfonts,amsmath,mathrsfs,amssymb,amsthm,url,color}
\usepackage{fancyhdr,indentfirst,bm,enumerate,natbib,float,tikz}
\usepackage{caption,subcaption,calligra}

\title{19实用随机过程期中考试}
\author{\calligra{NULIOUS}}
\date{}

\textheight 23cm
\textwidth 16.5cm
\topmargin -1.2cm
\oddsidemargin 0cm
\evensidemargin 0cm

\begin{document}
\maketitle
\noindent 1.(16 分)一个盒子中 $n+m$ 个小球,其中 $n$ 个红球,$m$ 个黑球.现依次不放回地从盒子中取球,以 $X$记在首次取得黑球前取出的红球个数.求 $\mathbb{E} [X]$ .\\
解一:用条件期望递推\\
记有$n$个红球和$m$个黑球的盒子首次取得黑球前取出的红球个数为$X(n,m)$,对第一个球的颜色取条件\\
\begin{align*}
	\mathbb{E}[X(n,m)] & =\frac{n}{n+m}\left( \mathbb{E}[X(n-1,m)]+1 \right)+\frac{m}{n+m}\cdot 0 \\
	&=\frac{n}{n+m}\left( \mathbb{E}[X(n-1,m)]+1 \right)
\end{align*}
并且有边界条件
\[
\mathbb{E}[X(0,m)]=0
\]
递推可以得到
\[
\mathbb{E}[X(n,m)]=\frac{n}{m+1}
\]
解二:给$n$个红球编号$1,...,n$,记
\[
\mathbb{I}_k=
\begin{cases}
	1  &  \text{红球k在所有黑球前} \\
	0  &  \text{其他}
\end{cases}
\]
则$\mathbb{I}_k$同分布,且一个红球可以插空地放在每个黑球的两边,也就是有$m+1$个位置可以选择,则
\[
\mathbb{E}[\mathbb{I}_k]=P(\mathbb{I}_k=1)=\frac{1}{m+1}
\]
那么
\[
\mathbb{E}[X(n,m)]=\mathbb{E}\left[\sum_{k=1}^n  \mathbb{I}_k \right]=\frac{n}{m+1} 
\]\\





\noindent 2.(36 分)一个商店在上午 8:00 开门,下午 5:00 关门.从 8:00 到 10:00 顾客以每小时 4 人速率到达,从 10:00 到 12:00 顾客以每小时 8 人速率到达,从 12:00 到下午 2:00顾客到达率稳定地从 12:00 的每小时 8 人增加到下午 2:00 的每小时 10 人,而在下午2:00 到 5:00 顾客到达率稳定地从下午 2:00 的每小时 10 人下降到下午 5:00 的每小时 4 人。\\
(1)问顾客的到达规律可以用什么样的概率模型来描述?(要求详细描述该模型)\\
(2)求上午 8:30 到 9:00 之间没有顾客到达的概率.\\
(3)求上午 8:30 到 9:30 之间有 3 位顾客到达,而下午 1:30 至 2:30 之间有 6 位顾客到达的概率.\\
(4)求这家商店平均每天到达的顺客数.\\
(5)已知某天上午 8:00 到 12:00 之间有 20 位顾客到达,求该天下午 1:00 至 2:00 之间有 10 位顾客到达的概率.\\
(6)假定每位到达的顾客以概率 0.6 为男性,以概率 0.4 为女性.求某天上午 8:00 到 10:00 之间有 5 位男顾客到达,且上午10:00 至12:00 之间有 10 位女顾客到达的概率.\\
解:(1)这是一个非齐次Poisson过程$N(t)$,记上午8点为$t=0$,则强度函数为
\[
\lambda(t)=
\begin{cases}
	4  &  0\le t\le 2 \\
	8  &  8 <t\le 4 \\
	t+4 &  4<t\le 6\\
	-2t+22 & 6<t\le 9
\end{cases}
\]
(2)$N(1)-N(0.5)\sim \mathrm{P}(2)$,则
\[
P(N(1)-N(0.5)=0)=e^{-2}
\]
(3)由独立增量性,这两个事件是独立的,又
\begin{align*}
	N(6.5)-N(5.5) & \sim \mathrm{P}\left(\int_{5.5}^{6.5} \lambda(t)dt \right)  \\
	& = \mathrm{P}(9.625)\\
	N(1.5)-N(0.5)&\sim \mathrm{P}(4)
\end{align*}
最后
\begin{align*}
	P(N(1.5)-N(0.5)=3,N(6.5)-N(5.5)=6) & =P(N(1.5)-N(0.5)=3)P(N(6.5)-N(5.5)=6) \\
	& = \frac{4^3 e^{-4}}{3!}\cdot \frac{9.625^6 e^{-9.625}}{6!}\\
	&= e^{-13.625}\frac{2(9.625)^6}{135}
\end{align*}
(4)$N(9)\sim \mathrm{P}\left(\int_{0}^{9} \lambda(t)dt \right)=\mathrm{P}(63) $,则
\[
\mathbb{E}[N(9)]=63
\]
(5)由独立增量性,两个事件独立,又
\[
N(6)-N(5)\sim \mathrm{P}\left(\int_{5}^{6}\lambda(t)dt \right)=\mathrm{P}(9.5) 
\]
最后
\[
\text{所求}=P(N(6)-N(5)=10)=\frac{e^{-9.5}(9.5)^{10}}{10!}
\]
(6)由分类Poisson过程,两事件独立,并记$X_1$为前一事件,$X_2$为后一事件,则
\[
N(2)\sim \mathrm{P}(8) \quad X_1\sim \mathrm{P}(4.8)
\]
\[
N(4)-N(2)\sim \mathrm{P}(16) \quad X_2\sim \mathrm{P}(6.4)
\]
最后
\begin{align*}
	P(X_1=5,X_2=10) &= P(X_1=5)P(X_2=10) \\
	 & =\frac{(4.8)^5 (6.4)^10}{5!10!} e^{-11.2}
\end{align*}\\



\noindent 3.(14 分)设 $\left\{X_n, n \geq 1\right\}$ 为独立同分布的随机变量序列,共同分布为参数 $\lambda$ 的指数分布,$N$ 为几何分布随机变量,独立于 $\left\{X_n, n \geq 1\right\}$ ,其中 $\mathrm{P}(N=n)=p(1-p)^{n-1}$ , $n \geq 1$ .试基于 Poisson 过程的相关理论求 $S=\sum\limits_{k=1}^N X_k$ 的分布.\\
解:考虑几何分布的意义:第一次投掷出正面的硬币所需的总投掷数\\
设$X_k$为每次投掷的时间间隔,则$S=\sum\limits_{k=1}^N X_k$即为直到第一次投掷出正面所需的总时间\\
将Poisson过程分类,则投出正面的过程$N_1(t)$速率为$\lambda p$,
投出反面的过程$N_2(t)$速率为$\lambda (1-p)$\\
那么$S$就是$N_1(t)$中第一个时间间隔,即
\[
S \sim \mathrm{Exp}(\lambda p)
\]\\





\noindent 4.(14 分)设 $\{N(t), t \geq 0\}$ 是强度 $\lambda=10$ 的齐次 Poisson 过程,以 $S_i$ 记该过程的第 $i$ 个事件发生时刻,求 $\mathbb{E}\left[S_i \mid N(t)=n\right]$ .\\
解:i.$i \le n$时\\
\[
\mathbb{E}[S_i\mid N(t)=n]=\mathbb{E}[U_{(i)}]
\]
这里$U_{(i)}$为$U_1,...,U_n\stackrel{i.i.d.}{\sim} \mathrm{U}(0,t)$的次序统计量\\
则
\[
\mathbb{E}[S_i\mid N(t)=n]=\mathbb{E}[U_{(i)}]=\frac{i}{n+1}t
\]
\textbf{RK}:\\
$$
\mathbb{E}[U_{(i)}]=\int_{0}^{t} \dbinom{n}{1} \dbinom{n-1}{i-1} x\left(\frac{x}{t} \right)^{i-1} \left( 1-\frac{x}{t}\right)^{n-i} \frac{1}{t} dx=  \frac{i}{n+1}t
$$
ii.$i<n$时
\[
\mathbb{E}[S_i\mid N(t)=n]=\mathbb{E}[t+X_{n+1}+\cdots+X_i]
\]
其中$X_i$表示时间间隔,且
\[
X_{n+1},...,X_n \stackrel{i.i.d.}{\sim} \mathrm{Exp}(\lambda)\sim \mathrm{Exp}(10)
\]
则
\[
\mathbb{E}[S_i\mid N(t)=n]=t+\frac{i-n}{10}
\]\\




\noindent 5.(20 分)观察一列独立同分布的离散随机变量 $W_1, W_2, \ldots$ ,等待花样$" 010101 "$的发生.设
$$
\mathrm{P}\left(W_1=0\right)=\frac{1}{4}, \quad \mathrm{P}\left(W_1=1\right)=\frac{1}{2}, \quad \mathrm{P}\left(W_1=2\right)=\mathrm{P}\left(W_1=3\right)=\frac{1}{8}
$$
求等待花样$"010101"$首次发生所需要的期望时间.\\
解:花样问题\\
010101有重叠0101,0101有重叠01,01没有重叠,则\\
\begin{align*}
	\mathbb{E}[T_{010101}] & =P(010101)^{-1}+P(0101)^{-1}+P(01)^{-1} \\
	& =4^3 2^3+4^2 2^2+4 \cdot 2\\
	&=584
\end{align*}

\end{document}
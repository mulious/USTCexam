\documentclass[UTF8]{ctexart}
%\documentclass{article}
\usepackage{graphicx,amsfonts,amsmath,mathrsfs,amssymb,amsthm,url,color}
\usepackage{fancyhdr,indentfirst,bm,enumerate,natbib, float,tikz,graphicx}
\usepackage{caption}
\usepackage{subcaption}
\usepackage{calligra} % 添加书法字体包

\title{21实用随机过程期末}
\author{\textcalligra{NULIOUS}} % 改用书法体
\date{}

\textheight 23cm
\textwidth 16.5cm
\topmargin -1.2cm
\oddsidemargin 0cm
\evensidemargin 0cm

\begin{document}

\maketitle


\noindent 1.(总 8 分)试列表判断如下即几类过程是否具有独立增量性、平稳增量性、Markov 性质:(1)齐次 Poisson 过程;(2)非齐次 Poisson 过程;(3)标准更新过程;(4)布朗运动.\\
解:(1)\\
有独立增量性、平稳增量性;并且齐次 Poisson 过程也是一种连续Markov链,当然有Markov 性\\
(2)\\
非齐次Poisson 过程失去了平稳增量性,但还有独立增量性和Markov 性\\
(3)\\
三个性质都没有\\
(4)\\
三个性质都有\\



\noindent 2.(总 12 分,每小题4分)设 $\{N(t), t \geq 0\}$ 是以独立同分布的随机变量序列 $\left\{X_{n}, n \geq 1\right\}$为间隔的更新过程,其中 $\mathrm{P}\left(X_{1}=1\right)=p, \mathrm{P}\left(X_{1}=0\right)=1-p$ ,其中 $o<p<1$ .\\
(1)求于时刻 0 点发生的更新个数随机变量 $N(0)$ 的概率分布;\\
(2)求于时刻 2 点发生的更新个数随机变量的概率分布;\\
(3)求 $\lim _{t \rightarrow \infty} \mathbb{E}[N(t)] / t$ .\\
解:(1)\\
必须成功发生一次$X_i=1$过程才能到达时刻1,即为几何分布:
\[
N(0)\sim \mathrm{Ge}(p)
\]
(2)\\
必须成功发生三次$X_i=1$过程才能到达时刻3,即为负二项分布:
\[
N(3)\sim \mathrm{NB}(3,p)
\]
(3)\\
\[
\lim _{t \rightarrow \infty} \frac{\mathbb{E}[N(t)]}{t}=\frac{1}{\mathbb{E}[X_1]}=\frac{1}{p}
\]\\


\noindent 3.(总 20 分,每小题 10 分)连续拋掷一枚非均匀硬币,每次抛出正面的概率为 $p \in(0,1)$ ,抛出反面的概率为 $q=1-p$ .\\
(1)求直到出现花样"正、反、正、反、正、反、正"时抛掷次数的期望.\\
(2)求直到抛出上述花样时抛出正面的期望次数.\\
解:(1)\\
法一:花样问题\\
$HTHTHTH$有重叠$HTHTH$,$HTHTH$有重叠$HTH$,$HTH$有重叠$H$,$H$没有重叠,则\\
\[
\mathbb{E}\left[T_{THTHTHT} \right]=\frac{1}{p^4q^3}+\frac{1}{p^3q^2}+ \frac{1}{p^2q}+ \frac{1}{p}  
\]
法二:鞅\\
设每天都有一个新赌徒开始赌博,他或者输光所有财富或者连赌7天赢下赌局,每个人的开始财富都是1,为了保证赌局公平,如果赌徒赢了,那么他的财富变为原来的$\frac{1}{p}$倍(猜对硬币为正面)或者变为原来的$\frac{1}{q}$倍(猜对硬币为反面),则这个赌局构成一个鞅,记第$N$天第一个赌徒七局都赢,赌徒的盈亏情况如下\\
\[
\begin{tabular}{|c|c|}
	\hline
	第$i$个赌徒 & 赌徒的盈亏  \\
	\hline
	前$N-7$个 & $-(N-7)$(全输)  \\
	第$N-6$个 & $+(\frac{1}{p^4q^3}-1)$(赢7局)  \\
	第$N-5$个 & $-1$ (输) \\
	第$N-4$个 & $+(\frac{1}{p^3q^2}-1)$ (赢5局)  \\
	第$N-3$个 & $-1$(输)  \\
	第$N-2$个 & $+(\frac{1}{p^2q}-1)$(赢3局)   \\
	第$N-1$个 & $-1$ (输) \\
	第$N$个 & $+(\frac{1}{p}-1)$(赢1局)   \\
	\hline
\end{tabular}
\]
总盈亏的期望应为0
\[
\mathbb{E}\left[(\frac{1}{p^4q^3}-1)+(\frac{1}{p^3q^2}-1)+(\frac{1}{p^2q}-1)+(\frac{1}{p}-1)-(N-7+1+1+1) \right] =0
\]
即
\[
\mathbb{E}[N]=\frac{1}{p^4q^3}+\frac{1}{p^3q^2}+ \frac{1}{p^2q}+ \frac{1}{p}
\]
(2)\\
记$X_i$满足$P(X_i=1)=p=1-P(X_i=0)$,即$X_i$在抛出正面的时候为1,否则为0;注意到$N$是一个停时,则:
\[
\mathbb{E}\left[\sum_{i=1}^N X_i \right]=\mathbb{E}[N]\mathbb{E}[X_1]=p\mathbb{E}[N] 
\]
则抛出正面的次数的期望为
\[
\mathbb{E}\left[\sum_{i=1}^N X_i \right]=\frac{1}{p^3q^3}+\frac{1}{p^2q^2}+ \frac{1}{pq}+ 1
\]\\


\noindent 4.(总 24 分,每小题 6 分)设 $\mathrm{A} 、 \mathrm{~B}$ 两盒中共装有 $N$ 个编号分别为 $1 、 2 、 \cdots 、 N$ 的小球。考虑如下试验:先从 $N$ 个小球中随机地取出一个小球(每球被取出的概率等可能),再任意指定一个盒子( A 盒被指定的概率为 $p, \mathrm{~B}$ 盒被指定的概率为 $q=1-p$ ),然后把所取出的小球放入指定的盒子中.如此不停地重复试验.记 $X_{n}$ 为 $n$ 次试验后 A 盒中小球的个数,$X_{0}$ 表示试验之前 A 盒中小球的个数,则 $\left\{X_{n}, n \geq 0\right\}$ 构成一个 Markov 链。\\
(1)求该 Markov 链转移概率矩阵 $\mathbf{P}$;\\
(2)试判断此链是否可约?每个状态是否具有常返性?每个状态是否有周期?(其中假定 $0<p<1$ )\\
(3)当 $N=3, p=1 / 2$ 时,试求该 Markov 链的平稳分布 $\pi=\left(\pi_{0}, \pi_{1}, \pi_{2}, \pi_{3}\right)$ ;\\
(4)记 $\mathbf{P}^{(n)}$ 为该 Markov 链的 $n$ 步转移概率矩阵。当 $N=3, p=1 / 2$ 时,求 $\lim _{n \rightarrow \infty} \mathbf{P}^{(n)}$ ,并对结果做出解释。\\
解:(1)\\
因为等可能的指定编号,所以编号实际上相当于不存在,只是按照盒子中的小球个数为权重取小球,有转移概率\\
\[
\begin{aligned}
	P_{k,k+1}&=\frac{N-k}{N}p \\
	P_{k,k-1}&=\frac{k}{N}q \\
	P_{k,k}&=\frac{k}{N}p+\frac{N-k}{N}q
\end{aligned}
\]
\[
\mathbf{P}=
\begin{pmatrix}
	p & q & 0 & \cdots & 0 & 0 \\
	 \frac{1}{N}q & \frac{1}{N}p + \frac{N-1}{N}q & \frac{N-1}{N}p & \cdots & 0 & 0 \\
	\vdots & \vdots & \vdots & \ddots & \vdots & \vdots \\
	0 & 0 & 0 & \cdots & q & p
\end{pmatrix}
\]
(2)\\
所有状态都互通,因此不可约;有限状态不可约,则所有状态都正常返;所有状态都非周期$(P_{k,k}>0)$\\
(3)\\
此时转移矩阵为\\
\[
\mathbf{P'}=
\begin{pmatrix}
	\frac{1}{2}  &\frac{1}{2}&0&0\\
	\frac{1}{6} & \frac{1}{2} &\frac{1}{3} &0\\
	0&\frac{1}{3} &\frac{1}{2}&\frac{1}{6}\\
	0&0&\frac{1}{2}&\frac{1}{2}
\end{pmatrix}
\]
平稳方程为:
$$
\begin{aligned}
	\pi_0&=\frac{1}{2}\pi_0+\frac{1}{6}\pi_1\\
	\pi_1&=\frac{1}{2}\pi_0+\frac{1}{2}\pi_1+\frac{1}{3}\pi_2\\
	\pi_2&=\frac{1}{3}\pi_1+\frac{1}{2}\pi_2+\frac{1}{2}\pi_3\\
	\pi_3&=\frac{1}{6}\pi_2+\frac{1}{2}\pi_3
\end{aligned}
$$
\[
\pi_0+\pi_1+\pi_2\pi_3=1
\]
解得
\[
\pi_0=\frac{1}{8} \quad \pi_1=\frac{3}{8} \quad \pi_2=\frac{3}{8} \quad \pi_3=\frac{1}{8} \quad
\]
(4)\\
不可约正常返非周期,则极限概率存在,且极限概率等于平稳概率\\
\[
\forall i \quad \lim_{n\rightarrow \infty} P_{ij}^n=\frac{1}{\mu_{jj}}=\pi_j
\]
那么就有:
\[
\lim _{n \rightarrow \infty} \mathbf{P}^{(n)}=
\begin{pmatrix}
	\frac{1}{8} &\frac{3}{8}&\frac{3}{8}&\frac{1}{8}\\
	\frac{1}{8} &\frac{3}{8}&\frac{3}{8}&\frac{1}{8}\\
	\frac{1}{8} &\frac{3}{8}&\frac{3}{8}&\frac{1}{8}\\
	\frac{1}{8} &\frac{3}{8}&\frac{3}{8}&\frac{1}{8}
\end{pmatrix}
\]\\


\noindent 5.(总 24 分,前两小题各 10 分,第 3 小题 4 分)一个修理工照看机器 1 和机器\\
2.每次修复后,机器 $i$ 保持正常运行,运行时间服从参数(失效率)$\lambda_{i}$ 的指数分布, $i=1,2$ .当机器 $i$ 失效时,需要进行修理,修理时间服从参数 $\mu_{i}$ 的指数分布.机器 1的修理具有优先权,在机器 1 失效时总是先修理它。例如,若正在修理机器 2 时机器 1 突然失效,则修理工将立刻停止修理机器 2 ,而开始修理机器 1 。\\
(1)为该题建立有限状态的连续时间 Markov 链,写成相应的转移强调 Q 矩阵;\\
(2)设 $\lambda_{i}=\mu_{i}=1+i, i=1,2$ .若系统长时间运行下去,求机器 2 失效的时间占比;\\
(3)每当两台机器同时处于失效状态时,求同时处于失效状态持续的时长分布.\\
解:(1)\\
考虑状态$(x, y)$,$ x, y \in\{0,1\}$ ,0 表示失效状态,1 表示工作状态,这样一共有如下四个状态:$(1,1),(0,1),(1,0)$ 和 $(0,0)$ ,为简化分别记为状态 $0,1,2,3$ .构造 4 状态的连续时间 Markov 链 $\{X(t), t \geq 0\}$ ,其中 $X(t)$ 表示时刻 $t$ 系统两个机器所处的状态,相应的$\mathbf{Q}$矩阵(对角元为$v_i$,非对角元为$q_{ij}$)为
$$
\mathbf{Q}=
\begin{pmatrix}
	-\lambda_{1}-\lambda_{2} & \lambda_{1} & \lambda_{2} & 0 \\
	\mu_{1} & -\lambda_{2}-\mu_{1} & 0 & \lambda_{2} \\
	\mu_{2} & 0 & -\lambda_{1}-\mu_{2} & \lambda_{1} \\
	0 & 0 & \mu_{1} & -\mu_{1}
\end{pmatrix}
$$
(2)\\
极限概率 $\left(P_{0}, P_{1}, P_{2}, P_{3}\right)$ 满足极限概率方程$
\left(P_{0}, P_{1}, P_{2}, P_{3}\right)=\left(P_{0}, P_{1}, P_{2}, P_{3}\right) \cdot \mathbf{Q}
$,即\\
$$
\begin{aligned}
	P_0(\lambda_1+\lambda_2)&=P_1\mu_1+P_2\mu_2 \\
	P_1(\lambda_2+\mu_1)&=P_0\lambda_1 \\
	P_2(\lambda_1+\mu_2)&=P_0\lambda_2+P_3\mu_1\\
	P_3\mu_1&=P_1\lambda_2+P_2\lambda_1
\end{aligned}
$$
\[
P_0+P_1+P_2+P_3=1
\]
解得
\[
P_0=\frac{5}{24} \quad P_1=\frac{1}{12} \quad P_2=\frac{7}{24} \quad P_3=\frac{5}{12} 
\]
则
\[
\text{机器2失效的时间占比}= \lim_{t \rightarrow \infty}P(\text{机器2失效})=P_2+P_3=\frac{17}{24}
\]
(3)\\
状态3只能转移到状态2,则\\
\[
T\sim \mathrm{Exp}(\mu_1)
\]\\


\noindent 6.(总 12 分,每小题 6 分)设 $\{B(t), t \geq 0\}$ 是一个标准布朗运动,定义随机变量序列 $X_{n}=B^{2}(n)-n, n \geq 1$.\\
(1)证明 $\left\{X_{n}, n \geq 1\right\}$ 为一个鞅;\\
(2)求如下的概率
$$
P\left(\max _{0 \leq s \leq n} B(s) \geq u_{0.05} \sqrt{n}\right)
$$
解:(1)\\
$$
\begin{aligned}
	\mathbb{E}\left[X_{n+1}|X_1,...,X_n \right]&=\mathbb{E}\left[B^2(n+1)-(n+1)|X_1,...,X_n \right]  \\
	&=\mathbb{E}\left[B^2(n+1)|X_1,...,X_n \right]-(n+1)
\end{aligned}
$$
又
$$
\begin{aligned}
	B^2(n+1)&=\left(B(n+1)-B(n)+B(n) \right)^2\\
	&=\left(B(n+1)-B(n) \right)^2+B^2(n)+2 \left(B(n+1)-B(n) \right)B(n)
\end{aligned}
$$
由独立增量性和平稳增量性
\[
B(n+1)-B(n)\sim N(0,1) \quad B(n)\sim N(0,n)\quad \text{且它们相互独立} 
\]
注意给定$X_n$相当于给出了$B^2(n)$的信息
\[
\mathbb{E}\left[B^2(n+1)|X_1,...,X_n \right]=1+B^2(n)
\]
则
\[
\mathbb{E}\left[X_{n+1}|X_1,...,X_n \right]=X_n
\]
另外$\mathbb{E}[|X_n|]<\infty$\\
这就说明了$\left\{X_{n}, n \geq 1\right\}$ 为一个鞅\\
(2)\\
$$
\begin{aligned}
	P\left(\max _{0 \leq s \leq n} B(s) \geq u_{0.05} \sqrt{n}\right)&=2P\left(B(n)\ge u_{0.05} \sqrt{n} \right)\\
	&=2P\left(\frac{B(n)}{\sqrt{n}}\ge u_{0.05} \right)\\
	&=2P\left(\frac{B(n)}{\sqrt{n}}\ge u_{0.05} \right)\\
	&=  2P\left(N(0,1)\ge u_{0.05} \right)\\
	&= 0.1   
\end{aligned}
$$





\end{document}
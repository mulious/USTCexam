\documentclass[UTF8]{ctexart}
\usepackage{graphicx,amsfonts,amsmath,mathrsfs,amssymb,amsthm,url,color}
\usepackage{fancyhdr,indentfirst,bm,enumerate,natbib, float}

\title{2025年《实用随机过程》期中}
\author{华申瑞\ \ 苏煜庭}
\date{}

\textheight 23cm
\textwidth 16.5cm
\topmargin -1.2cm
\oddsidemargin 0cm
\evensidemargin 0cm
\begin{document}

\maketitle


\noindent 一、(18分)设 $\{N(t), t \geq 0\}$ 表示一个速率为 $\lambda$ 的 Poisson 过程,而 $S_n$ 为其第 $n$ 个事件的发生时刻,$n \geq 1$。\\
(1) 对任意 $s > 0$,求 $\mathbb{E}[S_2|N(s) = 1]$。\\
(2) 对任意 $s, t > 0$,求协方差 $\operatorname{Cov}(N(s), N(s + t))$。\\
(3) 对任意整数 $n \geq 1$,在 $N(t) = n$ 的条件下,求 $N(2t/3) - N(t/3)$ 的条件分布律。\\
解:\\
\qquad (1)由无记忆性,从 $s$ 时刻开始下次发生的平均时间为 $\displaystyle \frac{1}{\lambda}$,故
\begin{align*}
	\mathbb{E}[S_2 \mid N(s)=1]= s+\frac{1}{\lambda}.
\end{align*}
(2)由独立增量性和协方差的双线性性:
\begin{align*}
	\operatorname{Cov}(N(s),N(s+t)) &= \operatorname{Cov}(N(s),N(s))+\operatorname{Cov}(N(s),N(s+t)-N(s))\\
	&= \operatorname{Var}(N(s))\\
	&= \lambda s.
\end{align*}
(3)
由 $N(t)=n$ 知,$\{S_i,i=1,\dots,n\}$与 $\{U_{(i)},i=1,\dots,n\}$ 同分布,则
$$N(\frac{2}{3}t)-N(\frac{1}{3}t)=k \iff \text{有}\ k\ \text{个}U_i\in(\frac{1}{3}t,\frac{2}{3}t).$$
即为$\displaystyle \mathrm{B}(n,\frac{1}{3})$。\\


\noindent 二、(18分)设在某区间上,轿车,卡车和客车独立地分别以速率 $\lambda_1$, $\lambda_2$ 和 $\lambda_3$ 的 Poisson 过程到达且立即通过。\\
(1) 求所有汽车到达的时间间隔的分布。\\
(2) 设在 $t_0$ 时刻观察到一辆轿车通过,求下一辆通过的汽车仍为轿车的概率。\\
(3) 求在连续两辆轿车通过该路口期间非轿车通过数量的期望。\\
解:\\
(1)由Poisson过程可叠加的性质,汽车到达的过程为Poisson过程 $Poi(\lambda_1+\lambda_2+\lambda_3)$,因此间隔时间分布为 $\mathrm{Exp}(\lambda_1+\lambda_2+\lambda_3)$。\\
(2)设从 $t_0$ 之后下一辆轿车,卡车,客车到达的间隔时间分别为 $X,Y,Z$,由指数分布的无记忆性可得 $X\sim \mathrm{Exp}(\lambda_1),Y\sim \mathrm{Exp}(\lambda_2),Z\sim \mathrm{Exp}(\lambda_3)$,且 $X,Y,Z$ 相互独立,于是
$$
\begin{aligned}
	P(\mbox{下一辆为轿车})&=P(X=\min\{X,Y,Z\})\\&=\int_{0}^\infty \lambda_1e^{-\lambda_1x}e^{-\lambda_2x}e^{-\lambda_3x}dx\\&=\frac{\lambda_1}{\lambda_1+\lambda_2+\lambda_3}.
\end{aligned}
$$
(3)记两次轿车通过之间的时间为 $T\sim \mathrm{Exp}(\lambda_1)$,则由平稳增量性以及三类汽车相互独立可知两次轿车通过之间卡车通过的数量为 $N_2(T)$,客车为 $N_3(T)$,其中 $\{N_2(t)\}$ 和 $\{N_3(t)\}$ 为卡车和客车到达对应的Poisson过程。于是
$$
\begin{aligned}
	\mathbb{E}[N_2(T)+N_3(T)]&=\mathbb{E}[\mathbb{E}[N_2(T)|T]+\mathbb{E}[\mathbb{E}[N_3(T)|T]]\\
	&=\mathbb{E}[\lambda_2T]+\mathbb{E}[\lambda_3T]\\
	&=\frac{\lambda_2+\lambda_3}{\lambda_1}.
\end{aligned}
$$\\


\noindent 三、(12分)设顾客以 $\lambda = 2$ 人每小时的速度到达某个自动取款机,且每个顾客取走的金额相互独立,期望为 $\mu = 500$ 元,方差为 $\sigma^2 = 6000$。以 $X(t)$ 表示到 $t$ 为止顾客取走的总金额(忽略顾客取钱的时间)。求 $X(5)$ 的期望和标准差。\\
解:\\
设第 $i$ 个人取金额 $X_i$,则$$X(t)=\sum_{i=1}^{N(t)}X_i,$$
其中 $N(t) \sim \mathrm{P}(\lambda t),\mathbb{E}[X_i]=\mu,Var(X_i)=\sigma^2$,由复合Poisson过程知,
$$\mathbb{E}[X(t)]=\mathbb{E}[N(t)]\mathbb{E}[X_i],\operatorname{Var}(X(t))=\mathbb{E}[N(t)]\mathbb{E}[X_i^2],$$
故 $\mathbb{E}[X(5)]=\lambda t \cdot \mu=5000,\sqrt{\operatorname{Var}(X(5))}=\sqrt{\lambda t \cdot(\mu^2+\sigma^2)}=1600$。\\


\noindent 四、(12分)设 $\{N(t), t \geq 0\}$ 表示一个速率为 $\lambda$ 的 Poisson 过程,而 $X$ 为一独立于该过程的非负随机变量,且 $\mathbb{E}[X] = \mu$, $\operatorname{Var}(X) = \sigma^2$。求 $\mathbb{E}[N(X)]$ 和 $\operatorname{Var}[N(X)]$。\\
解:\\
利用条件期望平滑公式
$$
\mathbb{E}[N(X)]=\mathbb{E}[\mathbb{E}[N(X)|X]]=\mathbb{E}[\lambda X]=\lambda\mu.
$$
利用条件方差公式
$$
\begin{aligned}
	\operatorname{Var}(N(X))&=\operatorname{Var}(\mathbb{E}[N(X)|X])+\mathbb{E}[\operatorname{Var}(N(X)|X)]\\
	&=\operatorname{Var}(\lambda X)+\mathbb{E}[\lambda X]\\
	&=\lambda^2\sigma^2+\lambda\mu.
\end{aligned}
$$\\


\noindent 五、(20分)考虑一组独立可重复的随机试验,设每次只能出现 A, B 和 C 三种结果之一,且发生的概率依次为 0.5, 0.3 和 0.2。对花样 AAA 和 ABCAB,分别求它们首次出现所需平均试验次数。\\
解:\\
\begin{align*}
	\mathbb{E}[AAA]&=P(A)^{-3}+P(A)^{-2}+P(A)^{-1}\\
	&=14,\\
	\mathbb{E}[ABCAB]&=\mathbb{E}[AB]+P(A)^{-2}P(B)^{-2}P(C)^{-1}\\  &=P(A)^{-1}P(B)^{-1}+P(A)^{-2}P(B)^{-2}P(C)^{-1}\\
	&=\frac{2060}{9}.
\end{align*}\\



\noindent 六、(20分)设 $\{N(t), t \geq 0\}$ 为一更新过程,且其达到间隔时间 $X_1, X_2, \ldots$ 均服从均匀分布 $\mathrm{U}[1, 3]$。\\
(1) 求更新函数 $m(t)$ 的表达式,其中 $0 < t < 3$。\\
(2) 利用关键更新定理求 $\lim_{t \to \infty} \int_0^t e^{-(t-s)} dm(s)$。\\
(3) 设 $Y(t)$ 表示时刻 $t$ 的剩余寿命,求 $\lim_{t \to \infty} \int_0^t Y(s) ds/t$。\\
(4) 求 $\lim_{t \to \infty} \mathbb{E}[Y(t)]$。\\
解:(1)由于间隔时间大于1,于是 $m(t)=0,0<t<1$,当 $t\in[1,2)$ 时,将 $F$ 为 $\mathrm{U}(1,3)$ 的分布函数代入更新方程
$$
\begin{aligned}
	m(t)&=F(t)+\int_{0}^tm(t-x)dF(x)\\
	&=\frac{t-1}{2}+\frac{1}{2}\int_{1}^tm(t-x)dx\\
	&=\frac{t-1}{2}+\frac{1}{2}\int_{0}^{t-1}m(x)dx=\frac{t-1}{2}.
\end{aligned}
$$
当 $t\in[2,3)$ 时
$$
\begin{aligned}
	m(t)&=F(t)+\int_{0}^tm(t-x)dF(x)\\
	&=\frac{t-1}{2}+\frac{1}{2}\int_{1}^tm(t-x)dx\\
	&=\frac{t-1}{2}+\frac{1}{2}\int_{0}^{t-1}m(x)dx\\
	&=\frac{t-1}{2}+\frac{1}{2}\int_{1}^{t-1}\frac{x-1}{2}dx\\
	&=\frac{t-1}{2}+\frac{(t-2)^2}{8}=\frac{t^2}{8}.
\end{aligned}
$$
综上可得
$$
m(t)=
\begin{cases}
	0  &  0<t<1 \\
	\frac{t-1}{2}  &  1\leq t<2\\
	\frac{t^2}{8}  &	2\leq t<3
\end{cases}
$$
(2)令 $h(t)=e^{-t}$,可以验证 $h(t)$ 恒正递减且在 $(0,\infty)$ 上可积$\left( \int_{0}^{+\infty} h(s)ds=1 \right) $,因此满足直接Riemann可积,容易计算间隔时间的期望为2,从而由关键更新定理
$$
\lim_{t\to \infty}\int_{0}^t e^{-(t-s)}dm(s)=\lim_{t\to \infty}\frac{1}{2}\int_{0}^t e^{-s}ds=\frac{1}{2}.$$
(3)记间隔时间对应随机变量为 $X\sim U[1,3]$,由更新报酬过程(课本例3.6B)可得
$$
\lim_{t\to \infty}\frac{\int_{0}^t Y(s)ds}{t}=\frac{\mathbb{E}[X^2]}{2\mathbb{E}[X]}=\frac{13}{12}.
$$
(4)由课本命题3.4.6可得
$$
\lim_{t\to \infty}\mathbb{E}[Y(t)]=\frac{\mathbb{E}[X^2]}{2\mathbb{E}[X]}=\frac{13}{12}.
$$\\
\textbf{RK}:\\
(3)详细过程如下:\\
考虑更新报酬过程,设在$s$时刻收到$Y(s)$的钱
\[
\lim_{t\rightarrow \infty}\frac{\int_{0}^{t} Y(s)ds}{t}=\lim_{t\rightarrow \infty}\frac{\mathbb{E}\left[R(t) \right] }{t}=\frac{\mathbb{E}\left[ R\right] }{\mathbb{E}[X_i]}=\frac{\mathbb{E}\left[R\right] }{2}
\]
另外注意间隔时间$X_n$独立同分布
$$
\begin{aligned}
	\mathbb{E}[R]&=\mathbb{E}\left[\mathbb{E}\left[R\left| X_1=x\right. \right]  \right] \\
	&=\int_{\mathbb{R}} \mathbb{E}[R|X_1=x]f_{x_1}(x)dx\\
	&=\int_{1}^{3} \frac{1}{2}\mathbb{E}[R|X_1=x]dx\\
	&=\frac{1}{2}\int_{1}^{3} \int_{0}^{x}(x-t)dtdx\\
	&=\frac{13}{6}
\end{aligned}
$$
因此
\[
\lim_{t\to \infty}\mathbb{E}[Y(t)]=\frac{13}{12}
\]
(4)详细过程如下
\begin{align*}
	\mathbb{E}[Y(t)]&=\mathbb{E}\left[ Y(t)|S_{N(t)}=0 \right]\bar{F}(t) +
	\int_{0}^{t} \mathbb{E}\left[ Y(t)|S_{N(t)}=y \right]\bar{F}(t-y)dm(y)\\
	&=\mathbb{E}\left[X-t|X>t \right]\bar{F}(t)+ 	\int_{0}^{t} \mathbb{E}\left[X-(t-y)|X>t-y \right]\bar{F}(t-y)dm(y)\\
\end{align*}
设$h(t)=\mathbb{E}\left[X-t|X>t \right]\bar{F}(t)$,则其非负不增,且$\int_{0}^{+\infty} h(t)dt<\infty$(见下面计算过程)\\
由关键更新定理
\[
\lim_{t\to \infty}\mathbb{E}[Y(t)]=0+\frac{1}{\mu}\int_{0}^{+\infty}h(t)dt
\]
这里用到了$\lim_{t\to \infty}\bar{F}(t)=0$\\
由期望的定义(这里将密度$f_{X|X>t}(s)$改写为$P(X=s|X>t)$)
$$
\begin{aligned}
	\mathbb{E}\left[X-t|X>t \right]\bar{F}(t)&=\int_{t}^{+\infty}(s-t) f_{X|X>t}(s)P(X>t)ds\\
	&=\int_{t}^{+\infty}(s-t) P(X=s|X>t)P(X>t)ds\\
	&=\int_{t}^{+\infty} (s-t)P(X=s,X>t)ds\\
	&=\int_{t}^{+\infty} (s-t)dF(s)\\
	&\left( =\mathbb{E}\left[(X-t) \mathbb{I}_{\{X >t\}}\right]\right)  
\end{aligned}
$$
上式也可以由全概率公式(对$X$的取值取条件)得到
$$
\begin{aligned}
	\mathbb{E}\left[X-t|X>t \right]\bar{F}(t)&=\int_{t}^{+\infty}\mathbb{E}\left[X-t|X>t,X=s \right] P(X>t|X=s)P(X=s)ds\\
	&=\int_{t}^{+\infty} (s-t)P(X=s,X>t)ds\\
	&=\int_{t}^{+\infty} (s-t)dF(s)\\
	&\left( =\mathbb{E}\left[(X-t) \mathbb{I}_{\{X >t\}}\right]\right)  
\end{aligned}
$$
则
$$
\begin{aligned}
	\lim_{t\to \infty}\mathbb{E}[Y(t)]&=\frac{1}{\mu}\int_{0}^{+\infty} \int_{t}^{\infty}(s-t)dF(s)dt\\
	&\stackrel{Fubini\text{换序}}{=}\frac{1}{\mu}\int_{0}^{+\infty} \int_{0}^{s}(s-t)dtdF(s)\\
	&=\frac{\mathbb{E}[X_i^2]}{2\mu}\\
	&=\frac{13}{12}
\end{aligned}
$$\\



\noindent 七、(附加题, 10分)设 $\{N_D(t), t \geq 0\}$ 为一延迟更新过程,且其达到间隔时间 $X_1$ 和 $X_2$ 分别服从参数为 $\lambda$ 和 $\mu$ 的指数分布。试求更新函数 $m_d(t)$ 的精确表达式。\\
解:\\
我们先不看延迟更新的部分(当作$X_1=0$),先求解后半部分(普通更新过程$N(t)$)的更新函数$m(t)$\\
法一:\\
注意Poisson过程可以由如下定义:\\
设$\{X_n\}\stackrel{i.i.d.}{\sim} \mathrm{Exp}(\lambda)$,第$n$个事件发生的时刻$S_n$为
\[
S_n=X_1+X_2+\dots+X_n
\]
这样定义的计数过程为速率为$\lambda$的Poisson过程\\
则本题的延迟更新过程实际上是一个指数分布加上一个参数为$\mu$的Poisson过程$N(t)$\\
对于Poisson过程$N(t)$,它的更新函数为
\[
m(t)=\mathbb{E}[N(t)]=\mu t
\]
法二:\\
设 $X_1\sim F,X_2\sim G$,
记 $m(t)=\sum_{n=1}^{\infty}G_n(t)$,则有更新方程
\begin{align*}
	m(t)&=G(t) + \int_0^t m(t - x) dG(x)\\
	&=1-e^{-\mu t}+\int_0^t m(t - x)  \mu e^{-\mu x}dx\\
	&=1-e^{-\mu t}+\int_0^t m(x)  \mu e^{-\mu (t-x)}dx\\
	&=1-e^{-\mu t}+\mu e^{-\mu t}\int_0^t m(x)e^{\mu x}dx\\
	\Rightarrow m'(t)&=\mu e^{-\mu t}+\mu m(t)+\int_0^t m(x)e^{\mu x}dx\cdot(-\mu^2e^{-\mu t})\\
	&=\mu e^{-\mu t}+\mu m(t)+(-\mu)\cdot(m(t)-1+e^{-\mu t})\\
	&=\mu
\end{align*}
代入$\mathbb{E}[N(t)]=m(0)=0$,有
\[
m(t)=\mathbb{E}[N(t)]=\mu t
\]
再计算延迟更新过程的更新函数$m_d(t)$\\
法一:\\
利用延迟更新过程的更新方程
\begin{align*}
	m_d(t)&=F(t)+\sum_{n=1}^{\infty}G_n(t)*F(t)\\
	&=F(t)+\int_0^t m(t - x) dF(x)\\
	&=1-e^{-\lambda t}+\int_0^t \mu (t - x) \lambda e^{-\lambda x}dx\\
	&=1-e^{-\lambda t}+\mu t+\frac{\mu}{\lambda}e^{-\lambda t}-\frac{\mu}{\lambda}.
\end{align*}
法二:\\
对首次更新发生的时刻$s$取条件
$$
m_d(t)=\mathbb{E}[N_D(t)]=\mathbb{E}\left[\mathbb{E}[N_D(t)|X_1=s] \right] 
$$
且有
$$
\mathbb{E}[N_D(t)|X_1=s]=
\begin{cases}
	1+\mathbb{E}[N(t-s)]  &  t>s \\
0  &  t\le s
\end{cases}
$$
则
$$
\begin{aligned}
	m_d(t)=\mathbb{E}[N_D(t)]&=\int_{0}^{t} \left(1+\mathbb{E}[N(t-s)] \right) \lambda e^{-\lambda s}ds\\
	&=\int_{0}^{t} \left(1+\mu(t-s) \right) \lambda e^{-\lambda s}ds\\
	&=(1+\mu t)\left(-e^{-\lambda t}+1 \right) -\mu \int_{0}^{t} s\lambda e^{-\lambda s}ds \\
	&=(1+\mu t)\left(-e^{-\lambda t}+1 \right) -\frac{\mu}{\lambda}\left( -\lambda t e^{-\lambda t} -e^{-\lambda t}+1\right)\\
	&=1-e^{-\lambda t}+\mu t+\frac{\mu}{\lambda}e^{-\lambda t}-\frac{\mu}{\lambda}
\end{aligned}
$$


\end{document}
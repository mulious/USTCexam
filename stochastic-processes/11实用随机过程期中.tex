\documentclass[UTF8]{ctexart}
\usepackage{graphicx,amsfonts,amsmath,mathrsfs,amssymb,amsthm,url,color}
\usepackage{fancyhdr,indentfirst,bm,enumerate,natbib,float,tikz}
\usepackage{caption,subcaption,calligra}

\title{11实用随机过程期中考试}
\author{\calligra{NULIOUS}}
\date{}

\textheight 23cm
\textwidth 16.5cm
\topmargin -1.2cm
\oddsidemargin 0cm
\evensidemargin 0cm

\begin{document}
\maketitle
\noindent 1.(20分)设 $\{N(t), t \geq 0\}$ 是速率为 $\lambda$ 的Poisson过程,以 $S_{n}$ 记第 $n$ 个事件发生的时刻, $s, t>0$ .\\
(1)求 $\mathbb{E}\left[S_{4} \mid N(1)=2\right]$ .\\
(2)求 $\mathbb{E}[N(4)-N(2) \mid N(1)=3]$ .\\
(3)在条件 $N(s+t)=n$ 下,求 $N(s)$ 的分布律.\\
(4)求 $\operatorname{Cov}(N(s), N(s+t))$ .\\
解:(1)记$X_i$为事件发生的时间间隔,则$X_i\stackrel{i.i.d.}{\sim}\mathrm{Exp}(\lambda) \quad \forall i$,有
\[
\mathbb{E}\left[S_{4} \mid N(1)=2\right]=\mathbb{E}[1+X_3+X_4]=1+\frac{2}{\lambda}
\]
(2)由独立增量性
\[
\mathbb{E}[N(4)-N(2) \mid N(1)=3]=\mathbb{E}[N(4)-N(2) ]=2\lambda
\]
(3)设$U_1,...,U_n\stackrel{i.i.d.}{\sim}\mathrm{U}(0,s+t)$,$(U_{(1)},...,U_{(n)})$为其次序统计量,则
\[
S_1,...,S_n\mid N(s+t)=n  \sim (U_{(1)},...,U_{(n)})
\]
进而
\[
N(s)\mid N(s+t)=n \sim \mathrm{B}\left(n,\frac{s}{s+t} \right) 
\]
(4)只需证明
\[
\operatorname{Cov}\left(N(t),N(s) \right)=\lambda \min \{s,t\} 
\]
下面设$s\le t$\\
证一:
\begin{align*}
	\operatorname{Cov}\left(N(t),N(s) \right) &= \operatorname{Cov}\left(N(t)-N(s),N(s) \right)+Cov\left(N(s),N(s) \right)   \\
	&\stackrel{\text{独立增量性}}{=}0+\operatorname{Var}\left(N(s) \right) \\
	&=\lambda s
\end{align*}
证二:
\[
\operatorname{Cov}\left(N(t),N(s) \right)=\mathbb{E}\left[N(t)N(s) \right]-\mathbb{E}[N(t)]\mathbb{E}[N(s)] 
\]
且有
\[
\mathbb{E}[N(t)]=\lambda t \quad \mathbb{E}[N(s)]=\lambda s
\]
\begin{align*}
	\mathbb{E}\left[N(t)N(s) \right] & =\mathbb{E}\left[N(s)\left(N(s)+\left(N(t)-N(s) \right)  \right)  \right] \\
	& \stackrel{\text{独立增量性}}{=} \mathbb{E}\left[N^2(s) \right]+\mathbb{E}[N(s)]\mathbb{E}[N(t)-N(s)]\\
	&= \lambda^2 s^2+\lambda s + \lambda s \cdot \lambda(t-s)\\
	&=\lambda s +\lambda^2 st
\end{align*}
因此
\[
\operatorname{Cov}\left(N(t),N(s) \right)=\lambda s
\]
对于本题有
\[
\operatorname{Cov}(N(s), N(s+t))=\lambda s
\]\\


\noindent 2.(15分)设随机变量 $X$ 和 $Y$ 独立同分布于参数为$\lambda$的指数分布.\\
(1)问 $\max (X, Y)-\min (X, Y)$ 服从什么分布?\\
(2)问 $\max (X, Y)-\min (X, Y)$ 和 $\min (X, Y)$ 是否独立?请证明你的结论.\\
(3)试用Poisson过程的背景解释上面的结果.\\
(注:允许直接利用Poisson过程去求解(1)和(2))\\
解:(1)由指数分布无记忆性,超出量还是原指数分布,即
\[
\max (X, Y)-\min (X, Y)\sim \mathrm{Exp}(\lambda)
\]
(2)由于无记忆性
\[
\max (X, Y)-\min (X, Y) \mid \min (X, Y)=t \sim \mathrm{Exp}(\lambda)
\]
说明二者独立\\
(3)$\max (X, Y)$,$\min (X, Y)$可以视为Poisson过程的首次和第二次的事件发生时间,则(1)来自Poisson过程的时间间隔为指数分布,(2)来自Poisson过程的独立增量性\\





\noindent 3.(16分)以 $S_{n}$ 记速率为 $\lambda$ 的Poisson过程 $\{N(t), t \geq 0\}$ 的第 $n$ 个事件发生的时刻。对在意一元函数 $g_{i}$ 对任意 $t>0$ ,试求
$$
Y(t)=\sum_{i=1}^{N(t)} g\left(S_{i}\right)
$$
的期望和方差\\
解:解:设$U_1,...,U_n\stackrel{i.i.d.}{\sim }\mathrm{U}(0,t)$,$(U_{(1)},U_{(n)})$为其次序统计量,则对$N(t)=n$取条件有
\[
\mathbb{E}\left[\sum\limits_{k=1}^{N(t)}  g(S_k) \mid N(t)=n \right]=\mathbb{E}\left[\sum\limits_{k=1}^{n} g (U_{(k)}) \right]
\]
注意有限求和号内可以任意换序,就有
\[
\sum\limits_{k=1}^{n} g (U_{(k)})=\sum\limits_{k=1}^{n} g(U_{k}) 
\]
则
\[
\mathbb{E}\left[\sum\limits_{k=1}^{n} g (U_{(k)}) \right]=\mathbb{E}\left[\sum\limits_{k=1}^{n} g (U_{k}) \right]=\sum\limits_{k=1}^{n}\mathbb{E}\left[g (U_{k}) \right]
\]
则有
\[
\mathbb{E}\left[\sum\limits_{k=1}^{N(t)} g (S_k) \mid N(t)=n \right]=n\mathbb{E}[g(U_k)]
\]
进而
\[
\mathbb{E}\left[\sum\limits_{k=1}^{N(t)} g( S_k ) \right]=\mathbb{E}\left[\mathbb{E}\left[\sum\limits_{k=1}^{N(t)} g( S_k )\mid N(t)=n \right] \right] =\mathbb{E}[N(t)]\mathbb{E}[g(U_k)]=\lambda t\mathbb{E}[g(U_k)]
\]\\
另外
\[
\mathbb{E}[g(U_k)]=\int_{0}^{t} \frac{1}{t} g(u)du
\]
化简后
\[
\mathbb{E}\left[\sum\limits_{k=1}^{N(t)} g( S_k ) \right]=\lambda \int_{0}^{t}  g(u)du
\]
对于方差,由条件方差公式
\[
\operatorname{Var}(Y(t))=\mathbb{E}\left[\operatorname{Var}(Y(t)\mid N(t)) \right] +\operatorname{Var}\left(\mathbb{E}[Y(t)\mid N(t)] \right) 
\]
其中
\[
\operatorname{Var}\left(\mathbb{E}[Y(t)\mid N(t)] \right) =\operatorname{Var}(N(t))\left( \mathbb{E}[g(U_i)]\right)^2=\lambda t \left(\frac{1}{t} \int_{0}^{t} g(u)du \right)^2 
\]
另外由独立性
\[
\operatorname{Var}(Y(t)\mid N(t)=n)=n\operatorname{Var}(g(U_i))
\]
其中
\begin{align*}
	\operatorname{Var}(g(U_i))& = \mathbb{E}[g^2(U_i)]-\left(\mathbb{E}[g(U_i)] \right)^2  \\
	& = \frac{1}{t} \int_{0}^{t} g^2(u)du-\left(\frac{1}{t} \int_{0}^{t} g(u)du \right)^2 
\end{align*}
就有
\[
\mathbb{E}\left[\operatorname{Var}(Y(t)\mid N(t)) \right]=\lambda t \operatorname{Var}(g(U_i))
\]
最后
\[
\operatorname{Var}(Y(t))=\lambda \int_{0}^{t} g^2(u)du
\]\\





\noindent 4.(15分)设更新过程 $\{N(t), t \geq 0\}$ 的更新区间长度为一列独立同分布于 $\mathrm{U}(0,1)$ 的随机变量.\\
(1)对任意 $0<t \leq 1$ ,证明更新函数 $m(t)$ 满足函数方程
$$
m(t)=t+\int_{0}^{t} m(y) d y
$$
(2)利用(1)中的结论证明对任意 $0<t \leq 1, m(t)=e^{t}-1$ .\\
解:(1)即证明更新方程,用$\ast$表示卷积,$F_n$表示分布函数的$n$次卷积
\begin{align*}
m(t) & =\sum_{n=1}^{\infty} F_n(t) \\
& =F(t)+\sum_{n=2}^{\infty} F_n(t) \\
& =F(t)+F(t) \ast \sum_{n=1}^{\infty} F_n(t) \\
& =F(t)+F(t) \ast m(t) \\
& =F(t)+\int_0^t m(t-x) d F(x) 
\end{align*}
(2)对$\mathrm{U}(0,1)$有$F(t)=t$和$dF(x)=1$,对更新方程求导,有
\[
m'(t)=1+m(t)
\]
解微分方程就有
\[
m(t)=e^t-1
\]\\




\noindent 5.(16分)考虑连续地投掷一枚均匀硬币,以H和T分别记正面和反面,对花样TTHTT和HTHTHT,利用更新过程的知识分别求它们各自\\
(1)相继出现的平均间隔时间;\\
(2)首次出现的平均时间。\\
解:(1)出现过TTHTT相当于给后一个TTHTT提供了TT;出现过HTHTHT相当于后一个HTHTHT提供了HTHT,则
\[
\mathbb{E}[T_{TTHTT\mid TT}]=\mathbb{E}[T_{TTHTT}]-\mathbb{E}[T_{TT}]=\left(P(X=H)P^4(X=T)\right)^{-1}=2^5
\]
\[
\mathbb{E}[T_{HTHTHT\mid HTHT}]=\mathbb{E}[T_{THTHTH}]-\mathbb{E}[T_{THTH}]=\left(P^3(X=H)P^3(X=T)\right)^{-1}=2^6
\]
(2)TTHTT有重叠TT,TT又有重叠T,T没有重叠;HTHTHT有重叠HTHT,HTHT又有重叠HT,HT没有重叠,则
\[
\mathbb{E}[T_{TTHTT}]=\left(P(X=H)P^4(X=T)\right)^{-1}+\left(P^2(X=T)\right)^{-1}+\left(P(X=T)\right)^{-1}=38
\]
\[
\mathbb{E}[T_{HTHTHT}]=\left(P^3(X=H)P^3(X=T)\right)^{-1}+\left(P^2(X=H)P^2(X=T)\right)^{-1}+\left(P(X=H)P(X=T)\right)^{-1}=84
\]\\


\noindent 6.(18分)从数 $1,2, \cdots, N(N \geq 2)$ 中随机取一个数作为 $X_{1}$ ,然后依次对每个 $n \geq 2$ ,从数 $1,2, \cdots, X_{n-1}$ 中随机取一个数作为 $X_{n}$ 。则 $\left\{X_{n}, n \geq 1\right\}$ 为一个Markov链。\\
(1)试写出该Markov链的转移概率矩阵 $\mathbf{P}$ ,\\
(2)对该Markov链进行状态分类(讨论分几个等价类,周期性,是否常返,是否正常返),\\
(3)极限 $\lim\limits_{n \rightarrow \infty} \mathbf{P}^{(n)}$ 是否存在?为什么?\\
解:(1)
\[
\mathbf{P}=
\begin{pmatrix}
	1 & 0 & \cdots & 0 \\
	\frac{1}{2} & \frac{1}{2} & \ddots & \vdots \\
	\vdots & \vdots & \ddots & 0 \\
	\frac{1}{N}&\frac{1}{N} & \cdots & \frac{1}{N}
\end{pmatrix}
\]
(2)每个状态都不互通,因此每个状态都是一个等价类,即$N$个等价类;每个状态都是非周期的;只有状态1是正常返的,其他状态都是非常返\\
(3)状态1是吸收壁,其他状态在足够长的时间内都会转移到状态1
\[
\lim\limits_{n \rightarrow \infty} \mathbf{P}^{(n)}=
\begin{pmatrix}
	1 & 0 & \cdots & 0 \\
	1 & 0 & \ddots & \vdots \\
	\vdots & \vdots & \ddots & 0 \\
	1&0 & \cdots & 0
\end{pmatrix}
\]\\



\noindent 7.(附加题, 10 分)某个保险公司对参保人的收费率在 $r_{1}$ 和 $r_{0}$ 之中交替 $\left(r_{0}<r_{1}\right)$ 。一个新的参保人开始时收费率为每个单位时间 $r_{1}$ ,当一个收费率为 $r_{1}$ 的参保人在最近的 $s$ 个单位时间内没有理赔,那么他的收费率变成单位时间 $r_{0}$ ,收费率保持在 $r_{0}$ 直到作了一次理赔,这时收费率回转到 $r_{1}$ ,假定给定的一个参保人永远活着,而且按速率为$\lambda$的Poisson过程要求理赔。在很长一段时间内,求该参保人\\
(1)以收费率$r_{i}$ 付费的时间的比例 $P_{i}, i=0,1$ \\
(2)在单位时间所付的平均金额。\\
解:(1)设当参保人按 $r_1$ 付费时系统为开,每一次理赔作为一次更新,设 $X$ 为两次理赔之间的时间
$$
\begin{aligned}
	 \mathbb{E}[\text{开}]&=\mathbb{E}[\min \{X,s\}]\\
	 &=\int_0^s x\lambda e^{-\lambda x} d x+s e^{-\lambda s} \\
	& =\frac{1}{\lambda}\left(1-e^{-\lambda s}\right) 
\end{aligned}
$$
又
\[
\mathbb{E}[X]=\frac{1}{\lambda}
\]
则
\[
P_1=\frac{\mathbb{E}[\text{开}]}{\mathbb{E}[X]}=1-e^{-\lambda s}
\]
\[
P_0=1-P_1=e^{-\lambda s}
\]
(2)单位时间所付长程平均金额为
$$
r_0 p_0+r_1 p_1=r_1-\left(r_1-r_0\right) e^{-\lambda s}
$$


\end{document}
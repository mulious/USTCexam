\documentclass[UTF8]{ctexart}
\usepackage{graphicx,amsfonts,amsmath,mathrsfs,amssymb,amsthm,url,color}
\usepackage{fancyhdr,indentfirst,bm,enumerate,natbib,float,tikz}
\usepackage{caption,subcaption,calligra}

\title{15实用随机过程期中考试}
\author{\calligra{NULIOUS}}
\date{}

\textheight 23cm
\textwidth 16.5cm
\topmargin -1.2cm
\oddsidemargin 0cm
\evensidemargin 0cm

\begin{document}
\maketitle
\noindent 1.设 $x_{1}, \ldots, x_{n}$ 为正常数,请用概率的方法证明
$$
\frac{1}{n} \sum_{i=1}^{n} x_{i} \geq\left(\prod_{i=1}^{n} x_{i}\right)^{1 / n}
$$\\
证:设 $X$ 为一个随机变量,满足 $\mathrm{P}\left(X=\log x_{i}\right)=1 / n, i=1, \ldots, n$ .对 $\phi(x)=e^x$应用 Jensen 不等式 $\mathbb{E}_{\phi}[X] \geq \phi(\mathbb{E} [X])$ ,立得所欲证不等式,\\



\noindent 2.设 $X_{1} \sim \operatorname{Exp}\left(\lambda_{1}\right), X_{2} \sim \operatorname{Exp}\left(\lambda_{2}\right), X_{3} \sim \operatorname{Exp}\left(\lambda_{3}\right)$ 且三个随机变量相互独立,其中 $\lambda_{1}>0, \lambda_{2}>0, \lambda_{3}>0$ .求 $\mathrm{P}\left(X_{3}>X_{2}, X_{1}+X_{2}>X_{3}\right)$ .\\
解:
$$
\begin{aligned}
	P\left(X_{3}>X_{2}, X_{1}+X_{2}>X_{3}\right) & =\mathbb{E}\left[P\left(X_{3}>X_{2}, X_{1}+X_{2}>X_{3} \mid X_{1}, X_{2}\right)\right] \\
	& =\int_{0}^{\infty} \int_{0}^{\infty} \lambda_{1} \lambda_{2} e^{-\lambda_{1} x-\lambda_{2} y} \int_{y}^{x+y} \lambda_{3} e^{-\lambda_{3} z} d z \\
	& =\frac{\lambda_{2} \lambda_{3}}{\left(\lambda_{1}+\lambda_{3}\right)\left(\lambda_{2}+\lambda_{3}\right)}
\end{aligned}
$$\\




\noindent 3.考虑一个 $M / G / \infty$ 随机服务系统,顾客到达系统的规律可以用齐次 Poisson 过程米描述,单位时间内平均到达的顾客数为 1 ,每个顾客需要服务员提供的服务时间是独立同分布的,其共同分布的概率密度函数为 $g(u)=2(1+u)^{-3}, u \geq 0$ ,系统有无穷多个服务员(即顾客到达系统后立即能得到服务),以 $A_{t}$ 表示时刻 $t$ 系统中处于工作状态的服务员个数.\\
(1)已知时间段 $(1,10]$ 到达了 2 位顾客,求时间段 $(15,20]$ 到达 2 位顾客的概率;\\
(2)求 $A_{5}$ 的概率分布;\\
(3)求 $\operatorname{Cov}\left(A_{4}, A_{5}\right)$ .\\
解:顾客到达过程 $\{N(t)\}$ 为 $\operatorname{HPP}(1)$ ,即强度参数 $\lambda=1$ .\\
(1)利用 HPP 的独立增量性,得
$$
\mathrm{P}(N(20)-N(15)=2 \mid N(10)-N(1)=2)=\mathrm{P}(N(20)-N(15)=2)=12.5 e^{-5}
$$
(2)+(3)记服务时间生存函数为 $\bar{G}(u)=(1+u)^{-2}$ ,把到达的顾客分为以下 3 类:\\
I 型:于 $(0,4]$ 到达,且于时刻 5 未被服务完毕;\\
II型:于 $(0,4]$ 到达,且于 $(4,5]$ 内被服务完毕;\\
III 型:于 $(4,5]$ 到达,且于时刻 5 木被服务完毕.\\
具体分类如下:于任意时刻 $s$ 到达的顾客,以概率 $p_{i}(s)$ 被划入第 $i$ 型顾客,$i=1,2,3$ ,\\
其中
$$
\begin{gathered}
	p_{1}(s)=\left\{\begin{array}{ll}
		\bar{G}(5-s), & s \leq 4 \\
		0, & s>4
	\end{array} \quad p_{2}(s)= \begin{cases}\bar{G}(4-s)-\bar{G}(5-s), & s \leq 4 \\
		0, & s>4\end{cases} \right. \\
	p_{3}(s)= \begin{cases}\bar{G}(5-s), & 4<s \leq 5 \\
		0, & s \leq 4 \text { 或 } s>5\end{cases}
\end{gathered}
$$
以 $N_{i}$ 记 $(0,5]$ 时段第 $i$ 型顾客的总数,则 $A_{5}=N_{1}+N_{3}, A_{4}=N_{1}+N_{2}$ .由 Poisson过程的抽样性质知:$N_{1}, N_{2}, N_{3}$ 相耳独立,且皆服从 Poisson 分布,对应的 Poisson 参数分别为
$$
\begin{aligned}
	& \lambda_{1}=\lambda \int_{0}^{5} p_{1}(s) d s=1 / 3 \\
	& \lambda_{2}=\lambda \int_{0}^{5} p_{2}(s) d s=7 / 15 \\
	& \lambda_{3}=\lambda \int_{0}^{5} p_{3}(s) d s=1 / 2
\end{aligned}
$$
于是 $$A_{5} \sim \operatorname{Poisson(5/6)} \quad  \operatorname{Cov}\left(A_{i}, A_{5}\right)=\operatorname{Var}\left(N_{1}\right)=1 / 3$$\\




\noindent 4.假设顾客到达银行的规律可用强度参数 $\lambda=2$ 的齐次 Poisson 过程来描述,每位到达的顾客以概率 $1 / 2$ 为男性.已知在前 10 个单位时间里有 100 个顾客到达该银行,问在该时间段到达该银行的女性顾客平均有多少?\\
解:在该时间段到达该银行的女性顾客$N\sim \mathrm{B}\left(100,\frac{1}{2} \right) $\\
则
\[
\mathbb{E}[N]=50
\]\\




\noindent 5.设 $\left\{X_{n}, n \geq 1\right\}$ 为独立同分布的随机变量序列,共同分布为参数 $\lambda$ 的指数分布,$N$为几何分布随机变量,独立于 $\left\{X_{n}, n \geq 1\right\}$ ,其中
$$
\mathrm{P}(N=n)=p(1-p)^{n-1}, \quad n \geq 1
$$
求 $S=\sum\limits_{k=1}^{N} X_{k}$ 的分布,并基于齐次 Poisson 过程的相关理论加以解释\\
解:考虑几何分布的意义:第一次投掷出正面的硬币所需的总投掷数\\
设$X_k$为每次投掷的时间间隔,则$S=\sum\limits_{k=1}^N X_k$即为直到第一次投掷出正面所需的总时间\\
将Poisson过程分类,则投出正面的过程$N_1(t)$速率为$\lambda p$,
投出反面的过程$N_2(t)$速率为$\lambda (1-p)$\\
那么$S$就是$N_1(t)$中第一个时间间隔,即
\[
S \sim \mathrm{Exp}(\lambda p)
\]\\

	
\end{document}